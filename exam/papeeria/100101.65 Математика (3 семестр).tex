\documentclass[
	14pt,
	a4paper,
	]
	{scrartcl}

\usepackage{mysty}

\newcommand{\spec}{100101.65}
\newcommand{\disc}{Математика\\ \multicolumn{2}{r}{ (3 семестр) }}
\newcommand{\kafedra}{ЕМиТД}

\usepackage[
	papersize={210mm,99mm},
	top=.5cm,
	left=1.5cm,
	right=1.5cm,
	bottom=.5cm
	]{geometry}

\pagestyle{empty}

\begin{document}


\shapk
\bilet{Экзаменационный билет}
\setcounter{zad}{0}

\vfill
\z Функции двух и трех переменных, основные понятия. Предел и непрерывность.
 \vfill
\z Интегрирование систем дифференциальных уравнений. \vfill

\vfill

\newpage


\shapk
\bilet{Экзаменационный билет}
\setcounter{zad}{0}

\vfill
\z Частные и полное приращения функции. Частные производные I-го порядка.
 \vfill
\z Интегрирование линейных дифференциальных уравнений с постоянными коэффициентами.
 \vfill

\vfill

\newpage


\shapk
\bilet{Экзаменационный билет}
\setcounter{zad}{0}

\vfill
\z Полный дифференциал.
 \vfill
\z Сводная таблица оригиналов и изображений.
 \vfill

\vfill

\newpage


\shapk
\bilet{Экзаменационный билет}
\setcounter{zad}{0}

\vfill
\z Применение полного дифференциала в приближенных вычислениях.
 \vfill
\z Дифференцирование и интегрирование оригиналов.
 \vfill

\vfill

\newpage


\shapk
\bilet{Экзаменационный билет}
\setcounter{zad}{0}

\vfill
\z Производные высших порядков.
 \vfill
\z Теоремы операционного исчисления.
 \vfill

\vfill

\newpage


\shapk
\bilet{Экзаменационный билет}
\setcounter{zad}{0}

\vfill
\z Частные производные сложных и неявных функций.
 \vfill
\z Изображения некоторых функций.
 \vfill

\vfill

\newpage


\shapk
\bilet{Экзаменационный билет}
\setcounter{zad}{0}

\vfill
\z Производная по направлению и градиент.
 \vfill
\z Оригиналы и изображения.
 \vfill

\vfill

\newpage


\shapk
\bilet{Экзаменационный билет}
\setcounter{zad}{0}

\vfill
\z Экстремумы функции двух переменных, необходимое и достаточное условия.
 \vfill
\z Формулы Эйлера.
 \vfill

\vfill

\newpage


\shapk
\bilet{Экзаменационный билет}
\setcounter{zad}{0}

\vfill
\z Наибольшее и наименьшее значение функции, заданной в области.
 \vfill
\z Показательная форма комплексного числа.
 \vfill

\vfill

\newpage


\shapk
\bilet{Экзаменационный билет}
\setcounter{zad}{0}

\vfill
\z Метод наименьших квадратов.
 \vfill
\z Формула Муавра.
 \vfill

\vfill

\newpage


\shapk
\bilet{Экзаменационный билет}
\setcounter{zad}{0}

\vfill
\z Двойной интеграл как предел интегральных сумм, основные свойства, геометрический смысл.
 \vfill
\z Основные действия над комплексными числами.
 \vfill

\vfill

\newpage


\shapk
\bilet{Экзаменационный билет}
\setcounter{zad}{0}

\vfill
\z Изменение порядка интегрирования.
 \vfill
\z Алгебраическая и тригонометрическая формы записи.
 \vfill

\vfill

\newpage


\shapk
\bilet{Экзаменационный билет}
\setcounter{zad}{0}

\vfill
\z Приложения двойных интегралов.
 \vfill
\z Комплексные числа, изображение на плоскости.
 \vfill

\vfill

\newpage


\shapk
\bilet{Экзаменационный билет}
\setcounter{zad}{0}

\vfill
\z Тройной интеграл, его приложения.
 \vfill
\z Функции комплексного переменного.
 \vfill

\vfill

\newpage


\shapk
\bilet{Экзаменационный билет}
\setcounter{zad}{0}

\vfill
\z Замена переменных в кратных интегралах.
 \vfill
\z Ряды Фурье для четных и нечетных функций.
 \vfill

\vfill

\newpage


\shapk
\bilet{Экзаменационный билет}
\setcounter{zad}{0}

\vfill
\z Криволинейные интегралы I-го и II-го рода, свойства и вычисление.
 \vfill
\z Тригонометрический ряд Фурье, достаточное условие разложимости функций в ряд Фурье.
 \vfill

\vfill

\newpage


\shapk
\bilet{Экзаменационный билет}
\setcounter{zad}{0}

\vfill
\z Формула Грина.
 \vfill
\z Приложение рядов к приближенным вычислениям (значений функций, пределов, определенных интегралов, дифференциальных уравнений).
 \vfill

\vfill

\newpage


\shapk
\bilet{Экзаменационный билет}
\setcounter{zad}{0}

\vfill
\z Условие независимости криволинейного интеграла от пути интегрирования.
 \vfill
\z Разложение функций в ряд Тейлора.
 \vfill

\vfill

\newpage


\shapk
\bilet{Экзаменационный билет}
\setcounter{zad}{0}

\vfill
\z Векторная функция скалярного аргумента.
 \vfill
\z Ряды Тейлора и Маклорена.
 \vfill

\vfill

\newpage


\shapk
\bilet{Экзаменационный билет}
\setcounter{zad}{0}

\vfill
\z Производная, ее геометрический смысл.
 \vfill
\z Теорема Абеля, радиус сходимости.
 \vfill

\vfill

\newpage


\shapk
\bilet{Экзаменационный билет}
\setcounter{zad}{0}

\vfill
\z Уравнение касательной и нормальной плоскостей.
 \vfill
\z Степенные ряды.
 \vfill

\vfill

\newpage


\shapk
\bilet{Экзаменационный билет}
\setcounter{zad}{0}

\vfill
\z Поверхностный интеграл.
 \vfill
\z Функциональный ряд, область сходимости.
 \vfill

\vfill

\newpage


\shapk
\bilet{Экзаменационный билет}
\setcounter{zad}{0}

\vfill
\z Векторное поле
 \vfill
\z Абсолютная и условная сходимости ряда.
 \vfill

\vfill

\newpage


\shapk
\bilet{Экзаменационный билет}
\setcounter{zad}{0}

\vfill
\z Поток через поверхность.
 \vfill
\z Знакочередующиеся ряды, теорема Лейбница.
 \vfill

\vfill

\newpage


\shapk
\bilet{Экзаменационный билет}
\setcounter{zad}{0}

\vfill
\z Дивергенция и формула Остроградского.
 \vfill
\z Признаки сходимости Даламбера и Коши (радикальный и интегральный).
 \vfill

\vfill

\newpage


\shapk
\bilet{Экзаменационный билет}
\setcounter{zad}{0}

\vfill
\z Циркуляция и ротор векторного поля.
 \vfill
\z Признаки сравнения.
 \vfill

\vfill

\newpage


\shapk
\bilet{Экзаменационный билет}
\setcounter{zad}{0}

\vfill
\z Формула Стокса.
 \vfill
\z Ряды с положительными членами.
 \vfill

\vfill

\newpage


\shapk
\bilet{Экзаменационный билет}
\setcounter{zad}{0}

\vfill
\z Потенциальное векторное поле.
 \vfill
\z Необходимое условие сходимости.
 \vfill

\vfill

\newpage


\shapk
\bilet{Экзаменационный билет}
\setcounter{zad}{0}

\vfill
\z Дифференциальные уравнения I-го порядка, геометрический смысл. Общее частное и особые решения.
 \vfill
\z Числовые ряды. Свойства рядов. Сумма ряда.
 \vfill

\vfill

\newpage


\shapk
\bilet{Экзаменационный билет}
\setcounter{zad}{0}

\vfill
\z Задача Коши, теорема о существовании и единственности решения.
 \vfill
\z Системы линейных дифференциальных уравнений I-го порядка с постоянными коэффициентами.
 \vfill

\vfill

\newpage


\shapk
\bilet{Экзаменационный билет}
\setcounter{zad}{0}

\vfill
\z Дифференциальные уравнения с разделенными и разделяющимися переменными.
 \vfill
\z Линейные дифференциальные уравнения II-го порядка с постоянными коэффициентами и специальной правой частью.
 \vfill

\vfill

\newpage


\shapk
\bilet{Экзаменационный билет}
\setcounter{zad}{0}

\vfill
\z Линейные дифференциальные уравнения I-го порядка.
 \vfill
\z Линейные однородные дифференциальные уравнения II-го порядка с постоянными коэффициентами.
 \vfill

\vfill

\newpage


\shapk
\bilet{Экзаменационный билет}
\setcounter{zad}{0}

\vfill
\z Однородные дифференциальные уравнения в полных дифференциалах.
 \vfill
\z Определитель Вронского, структура общих решений.
 \vfill

\vfill

\newpage


\shapk
\bilet{Экзаменационный билет}
\setcounter{zad}{0}

\vfill
\z Дифференциальные уравнения высших порядков, общее решение, задача Коши.
 \vfill
\z Линейные однородные дифференциальные уравнения II-го порядка.
 \vfill

\vfill

\newpage



\end{document}