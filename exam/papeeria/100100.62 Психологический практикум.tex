\documentclass[
	14pt,
	a4paper,
	]
	{scrartcl}

\usepackage{mysty}

\newcommand{\spec}{100100.62}
\newcommand{\disc}{Психологический\\ \multicolumn{2}{r}{ практикум }}
\newcommand{\kafedra}{ГиСД}

\usepackage[
	papersize={210mm,99mm},
	top=.5cm,
	left=1.5cm,
	right=1.5cm,
	bottom=.5cm
	]{geometry}

\pagestyle{empty}

\begin{document}


\shapk
\bilet{Экзаменационный билет}
\setcounter{zad}{0}

\vfill
\z Характеристика понятия «поведения».
 \vfill
\z Групповой процесс (групповая динамика). \vfill

\vfill

\newpage


\shapk
\bilet{Экзаменационный билет}
\setcounter{zad}{0}

\vfill
\z Эволюционная теория поведения. Уровни поведения.
 \vfill
\z Модель Джозефа Лафта и Гарри Инграма.
 \vfill

\vfill

\newpage


\shapk
\bilet{Экзаменационный билет}
\setcounter{zad}{0}

\vfill
\z Деятельность и ее связь с поведением.
 \vfill
\z Групповая этика.
 \vfill

\vfill

\newpage


\shapk
\bilet{Экзаменационный билет}
\setcounter{zad}{0}

\vfill
\z Внешнее и внутреннее выражение поведения.
 \vfill
\z Типы поведения в группе.
 \vfill

\vfill

\newpage


\shapk
\bilet{Экзаменационный билет}
\setcounter{zad}{0}

\vfill
\z Имидж, его создание и влияние на эффективность профессиональной деятельности работника сервиса.
 \vfill
\z Функции руководителя группы.
 \vfill

\vfill

\newpage


\shapk
\bilet{Экзаменационный билет}
\setcounter{zad}{0}

\vfill
\z Практическая психология. Отрасли психологической науки.
 \vfill
\z Понятие группы.
 \vfill

\vfill

\newpage


\shapk
\bilet{Экзаменационный билет}
\setcounter{zad}{0}

\vfill
\z Понятие о психологической информации.
 \vfill
\z Цели и принципы психогимнастики.
 \vfill

\vfill

\newpage


\shapk
\bilet{Экзаменационный билет}
\setcounter{zad}{0}

\vfill
\z Понятие о психологической задаче и психологической помощи.
 \vfill
\z Цели и принципы ролевой игры.
 \vfill

\vfill

\newpage


\shapk
\bilet{Экзаменационный билет}
\setcounter{zad}{0}

\vfill
\z Методические основы решения психологических задач.
 \vfill
\z Цели и принципы групповой дискуссии.
 \vfill

\vfill

\newpage


\shapk
\bilet{Экзаменационный билет}
\setcounter{zad}{0}

\vfill
\z Психологическая информация, ее получение в работе с клиентом.
 \vfill
\z Средства решения задач СПТ: групповые дискуссии, ролевые игры, психогимнастика.
 \vfill

\vfill

\newpage


\shapk
\bilet{Экзаменационный билет}
\setcounter{zad}{0}

\vfill
\z Проблемы применения данных психодиагностики в педагогической и социальной практики.
 \vfill
\z История развития социально – психологического тренинга (К. Левин).
 \vfill

\vfill

\newpage


\shapk
\bilet{Экзаменационный билет}
\setcounter{zad}{0}

\vfill
\z Социальная диагностика.
 \vfill
\z Психическая саморегуляция специалиста сервиса.
 \vfill

\vfill

\newpage


\shapk
\bilet{Экзаменационный билет}
\setcounter{zad}{0}

\vfill
\z Психология сервиса.
 \vfill
\z Принципы социально-психологического тренинга.
 \vfill

\vfill

\newpage


\shapk
\bilet{Экзаменационный билет}
\setcounter{zad}{0}

\vfill
\z Практическая психология как отрасль психологической науки
 \vfill
\z Цели и задачи социально-психологического тренинга.
 \vfill

\vfill

\newpage


\shapk
\bilet{Экзаменационный билет}
\setcounter{zad}{0}

\vfill
\z Основные технологии практической психологии.
 \vfill
\z Понятие социально - психологического тренинга.
 \vfill

\vfill

\newpage


\shapk
\bilet{Экзаменационный билет}
\setcounter{zad}{0}

\vfill
\z Психодиагностика: структура, развитие психодиагностики в России и за рубежом. 
 \vfill
\z Разрешение и предупреждение конфликтов.
 \vfill

\vfill

\newpage


\shapk
\bilet{Экзаменационный билет}
\setcounter{zad}{0}

\vfill
\z Психотерапия и психологическая реабилитация (медицинская, профессиональная, социальная).
 \vfill
\z Типы и виды конфликтов.
 \vfill

\vfill

\newpage


\shapk
\bilet{Экзаменационный билет}
\setcounter{zad}{0}

\vfill
\z Социотерапия и ее характеристика.
 \vfill
\z Концепции социальных конфликтов.
 \vfill

\vfill

\newpage


\shapk
\bilet{Экзаменационный билет}
\setcounter{zad}{0}

\vfill
\z Поведенческая психотерапия (И.П. Павлов, В.М. Бехтерев, Е.Л. Торндайк). Психокоррекция.
 \vfill
\z Определение и характеристика понятия «конфликт».
 \vfill

\vfill

\newpage


\shapk
\bilet{Экзаменационный билет}
\setcounter{zad}{0}

\vfill
\z Психологическое консультирование.
 \vfill
\z Понятие «Я» - концепция.
 \vfill

\vfill

\newpage


\shapk
\bilet{Экзаменационный билет}
\setcounter{zad}{0}

\vfill
\z Понятие социальной ситуации, и ее виды.
 \vfill
\z Понятие интрапсихического конфликта.
 \vfill

\vfill

\newpage


\shapk
\bilet{Экзаменационный билет}
\setcounter{zad}{0}

\vfill
\z Понятие личности. Теории личности.
 \vfill
\z Классификация потребностей человека.
 \vfill

\vfill

\newpage


\shapk
\bilet{Экзаменационный билет}
\setcounter{zad}{0}

\vfill
\z Типология человеческой психики.
 \vfill
\z Понятие эмоционального состояния.
 \vfill

\vfill

\newpage


\shapk
\bilet{Экзаменационный билет}
\setcounter{zad}{0}

\vfill
\z Невербальные средства общения.
 \vfill
\z Теория когнитивного диссонанса.
 \vfill

\vfill

\newpage


\shapk
\bilet{Экзаменационный билет}
\setcounter{zad}{0}

\vfill
\z Социальная роль и социальный контроль.
 \vfill
\z Влияние установки и поведение человека.
 \vfill

\vfill

\newpage


\shapk
\bilet{Экзаменационный билет}
\setcounter{zad}{0}

\vfill
\z Социальная роль и черты личности.
 \vfill
\z Исследования установки.
 \vfill

\vfill

\newpage


\shapk
\bilet{Экзаменационный билет}
\setcounter{zad}{0}

\vfill
\z Социальное восприятие.
 \vfill
\z Установка.
 \vfill

\vfill

\newpage



\end{document}