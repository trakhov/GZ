\documentclass[
	14pt,
	a4paper,
	]
	{scrartcl}

\usepackage{mysty}

\newcommand{\spec}{100103.65}
\newcommand{\disc}{Концепции\\ \multicolumn{2}{r}{ современного естествознания }}
\newcommand{\kafedra}{ЕМиТД}

\usepackage[
	papersize={210mm,99mm},
	top=.5cm,
	left=1.5cm,
	right=1.5cm,
	bottom=.5cm
	]{geometry}

\pagestyle{empty}

\begin{document}


\shapk
\bilet{Экзаменационный билет}
\setcounter{zad}{0}

\vfill
\z Две культуры --- естественно-научная и гуманитарная --- как отражение двух типов мышления. Рациональное и образное мышление.
 \vfill
\z Современное естествознание и проблема социума. Техногенное общество. Роль современного естествознания в преодолении энергетического, экологического и информационного кризисов. \vfill

\vfill

\newpage


\shapk
\bilet{Экзаменационный билет}
\setcounter{zad}{0}

\vfill
\z Общенаучные методы эмпирического познания.
 \vfill
\z Синергетика и основные принципы самоорганизации систем.
 \vfill

\vfill

\newpage


\shapk
\bilet{Экзаменационный билет}
\setcounter{zad}{0}

\vfill
\z Общенаучные методы теоретического познания.
 \vfill
\z Ноосфера Вернадского и экология окружающей природной среды.
 \vfill

\vfill

\newpage


\shapk
\bilet{Экзаменационный билет}
\setcounter{zad}{0}

\vfill
\z Взаимосвязь теории и эксперимента. Наблюдение, измерение и лабораторный эксперимент в естествознании. Реальные и мысленные эксперименты.
 \vfill
\z Структурные уровни биосферы, взаимосвязь ее компонентов.
 \vfill

\vfill

\newpage


\shapk
\bilet{Экзаменационный билет}
\setcounter{zad}{0}

\vfill
\z История естествознания. Атомистика древних греков.
 \vfill
\z Биосфера, ее эволюция, ресурсы, пределы устойчивости.
 \vfill

\vfill

\newpage


\shapk
\bilet{Экзаменационный билет}
\setcounter{zad}{0}

\vfill
\z Особенности античного научного знания, концепция геоцентризма.
 \vfill
\z Биоэтика. Ранговая иерархия высших животных. Иерархия потребностей человека. Проблема жизни и смерти.
 \vfill

\vfill

\newpage


\shapk
\bilet{Экзаменационный билет}
\setcounter{zad}{0}

\vfill
\z Естествознание в эпоху Возрождения. Борьба за гелиоцентрическую систему мира.
 \vfill
\z Основные принципы и запреты биоэтики.
 \vfill

\vfill

\newpage


\shapk
\bilet{Экзаменационный билет}
\setcounter{zad}{0}

\vfill
\z Физика Средневековья. Достижения науки средневекового Востока. Европейская средневековая наука.
 \vfill
\z Научные и этические проблемы клонирования.
 \vfill

\vfill

\newpage


\shapk
\bilet{Экзаменационный билет}
\setcounter{zad}{0}

\vfill
\z Развитие науки в России в 18-19 веках.
 \vfill
\z Мутации и генная инженерия. Проблемы.
 \vfill

\vfill

\newpage


\shapk
\bilet{Экзаменационный билет}
\setcounter{zad}{0}

\vfill
\z Механическая картина мира и ее ограниченность.
 \vfill
\z Человек: эмоции, творчество, работоспособность.
 \vfill

\vfill

\newpage


\shapk
\bilet{Экзаменационный билет}
\setcounter{zad}{0}

\vfill
\z Электромагнитная картина мира и ее ограниченность.
 \vfill
\z Человек: поведение и высшая нервная деятельность.
 \vfill

\vfill

\newpage


\shapk
\bilet{Экзаменационный билет}
\setcounter{zad}{0}

\vfill
\z Роль диалектического и метафизического методов в создании естественнонаучной картины мира. Процесс диалектизации науки.
 \vfill
\z Происхождение и эволюция человека.
 \vfill

\vfill

\newpage


\shapk
\bilet{Экзаменационный билет}
\setcounter{zad}{0}

\vfill
\z Учение Дарвина как генеральная линия эволюционного естествознания.
 \vfill
\z Теории эволюции живых организмов. Возникновение и эволюция основных видов живых организмов по Дарвину.
 \vfill

\vfill

\newpage


\shapk
\bilet{Экзаменационный билет}
\setcounter{zad}{0}

\vfill
\z Успехи механической картины природы в описании тепловых явлений. Молекулярно-кинетическая теория вещества.
 \vfill
\z Эволюция форм жизни на Земле от анаэробных к аэробным.
 \vfill

\vfill

\newpage


\shapk
\bilet{Экзаменационный билет}
\setcounter{zad}{0}

\vfill
\z Начала термодинамики и понятие энтропии.
 \vfill
\z Нуклеиновые кислоты. ДНК --- основа генетического материала. Структура ДНК.
 \vfill

\vfill

\newpage


\shapk
\bilet{Экзаменационный билет}
\setcounter{zad}{0}

\vfill
\z Пространство и время. Свойства пространства и времени. Представления в древности и сейчас.
 \vfill
\z Необратимость времени для живых систем. Жизненный цикл организма: от зарождения до гибели. Проблемы старения и смерти организма.
 \vfill

\vfill

\newpage


\shapk
\bilet{Экзаменационный билет}
\setcounter{zad}{0}

\vfill
\z А. Эйнштейн и относительность пространства-времени.
 \vfill
\z Обмен веществ и энергии в клетке как модель классической динамики живых объектов.
 \vfill

\vfill

\newpage


\shapk
\bilet{Экзаменационный билет}
\setcounter{zad}{0}

\vfill
\z Основы специальной теории относительности. Релятивистское выражение для импульса и энергии. Взаимосвязь массы и энергии.
 \vfill
\z Клетка как фундаментальная модель живой материи на микроуровне. Жизненный цикл клетки. Единство и многообразие клеточных типов.
 \vfill

\vfill

\newpage


\shapk
\bilet{Экзаменационный билет}
\setcounter{zad}{0}

\vfill
\z Второй этап в развитии электромагнитной картины мира. Представление об общей теории относительности.
 \vfill
\z Концепция Опарина возникновения жизни на Земле и опыт Миллера.
 \vfill

\vfill

\newpage


\shapk
\bilet{Экзаменационный билет}
\setcounter{zad}{0}

\vfill
\z Электромагнитная природа света. Волновые свойства света: интерференция, дифракция, дисперсия и поляризация.
 \vfill
\z Концепции возникновения жизни на Земле. Биохимическая эволюция.
 \vfill

\vfill

\newpage


\shapk
\bilet{Экзаменационный билет}
\setcounter{zad}{0}

\vfill
\z Корпускулярная концепция описания природы. Основные законы классической механики Ньютона. Концепция дальнодействия.
 \vfill
\z Уравнения химических реакций как классические модели химических процессов. Типы химических связей и химических реакций.
 \vfill

\vfill

\newpage


\shapk
\bilet{Экзаменационный билет}
\setcounter{zad}{0}

\vfill
\z Импульс, момент импульса и энергия как меры движения. Законы сохранения.
 \vfill
\z Химические элементы и соединения как классические модели вещества. Периодическая система химических элементов.
 \vfill

\vfill

\newpage


\shapk
\bilet{Экзаменационный билет}
\setcounter{zad}{0}

\vfill
\z Становление квантово-полевой картины мира. Тепловое излучение и гипотеза Планка.
 \vfill
\z Климат на земле. Формирование и эволюция.
 \vfill

\vfill

\newpage


\shapk
\bilet{Экзаменационный билет}
\setcounter{zad}{0}

\vfill
\z Квантовые свойства света.
 \vfill
\z Формирование планеты Земля, ее строение и эволюция.
 \vfill

\vfill

\newpage


\shapk
\bilet{Экзаменационный билет}
\setcounter{zad}{0}

\vfill
\z Планетарная модель атома Резерфорда и ее особенности.
 \vfill
\z Планеты-гиганты Солнечной системы. Их особенности.
 \vfill

\vfill

\newpage


\shapk
\bilet{Экзаменационный билет}
\setcounter{zad}{0}

\vfill
\z Модели атома и теория Н.Бора.
 \vfill
\z Земля и планеты земной группы.
 \vfill

\vfill

\newpage


\shapk
\bilet{Экзаменационный билет}
\setcounter{zad}{0}

\vfill
\z Гипотеза де Бройля и формирование квантовой механики Шредингера-Гейзенберга-Дирака.
 \vfill
\z Происхождение и строение Солнечной системы. Солнце.
 \vfill

\vfill

\newpage


\shapk
\bilet{Экзаменационный билет}
\setcounter{zad}{0}

\vfill
\z Особенности свойств микромира. Принцип неопределенности Гейзенберга.
 \vfill
\z Образование звезд в галактиках. Классификация звезд и их эволюция. Источники энергии звезд.
 \vfill

\vfill

\newpage


\shapk
\bilet{Экзаменационный билет}
\setcounter{zad}{0}

\vfill
\z Корпускулярно-волновой дуализм и принцип дополнительности.
 \vfill
\z Закон Хаббла, «красное смещение» и нестационарность Вселенной.
 \vfill

\vfill

\newpage


\shapk
\bilet{Экзаменационный билет}
\setcounter{zad}{0}

\vfill
\z Иерархия структур природы. Микромир: ядра атомов, элементарные частицы, кварки. Фундаментальные взаимодействия.
 \vfill
\z Мегамир. «Горячее» рождение Вселенной. Модели развития Вселенной, неоднозначность сценария.
 \vfill

\vfill

\newpage



\end{document}