\documentclass[
	14pt,
	a4paper,
	]
	{scrartcl}

\usepackage{mysty}

\newcommand{\spec}{100101.65}
\newcommand{\disc}{Философия \\}
\newcommand{\kafedra}{ГиСКД}

\usepackage[
	papersize={210mm,99mm},
	top=.5cm,
	left=1.5cm,
	right=1.5cm,
	bottom=.5cm
	]{geometry}

\pagestyle{empty}

\begin{document}


\shapk
\bilet{Экзаменационный билет}
\setcounter{zad}{0}

\vfill
\z 	Мировоззрение, его структура и исторические типы.
 \vfill
\z 	Социальное предвидение и его научные критерии. \vfill

\vfill

\newpage


\shapk
\bilet{Экзаменационный билет}
\setcounter{zad}{0}

\vfill
\z 	Предмет философии, ее место и роль в обществе.
 \vfill
\z 	Идеи И.В. Вернадского о ноосфере и их актуальность.
 \vfill

\vfill

\newpage


\shapk
\bilet{Экзаменационный билет}
\setcounter{zad}{0}

\vfill
\z 	Структура философского знания. Функции философии.
 \vfill
\z 	Глобальные проблемы современности: причины обострения и пути решения.
 \vfill

\vfill

\newpage


\shapk
\bilet{Экзаменационный билет}
\setcounter{zad}{0}

\vfill
\z 	Становление философии и ее основной вопрос.
 \vfill
\z 	Будущее человечества. Взаимодействие цивилизаций и сценарии будущего.
 \vfill

\vfill

\newpage


\shapk
\bilet{Экзаменационный билет}
\setcounter{zad}{0}

\vfill
\z 	Основные направления и школы философии. 
 \vfill
\z 	Наука и техника.
 \vfill

\vfill

\newpage


\shapk
\bilet{Экзаменационный билет}
\setcounter{zad}{0}

\vfill
\z 	Этапы исторического развития философии.
 \vfill
\z 	Научные революции и смена типов рациональности.
 \vfill

\vfill

\newpage


\shapk
\bilet{Экзаменационный билет}
\setcounter{zad}{0}

\vfill
\z 	Античная философия.
 \vfill
\z 	Человек и исторический процесс: личность и массы, свобода и необходимость.
 \vfill

\vfill

\newpage


\shapk
\bilet{Экзаменационный билет}
\setcounter{zad}{0}

\vfill
\z 	Философия Средневековья.
 \vfill
\z 	Человек в системе социальных связей.
 \vfill

\vfill

\newpage


\shapk
\bilet{Экзаменационный билет}
\setcounter{zad}{0}

\vfill
\z 	Философия Возрождения.
 \vfill
\z 	Общественный прогресс и рубежи всемирной истории.
 \vfill

\vfill

\newpage


\shapk
\bilet{Экзаменационный билет}
\setcounter{zad}{0}

\vfill
\z 	Философия Нового времени.
 \vfill
\z 	Сущность общественного прогресса и его критерии.
 \vfill

\vfill

\newpage


\shapk
\bilet{Экзаменационный билет}
\setcounter{zad}{0}

\vfill
\z 	Проблема человека в философии Просвещения.
 \vfill
\z 	Роль материального производства в жизни общества.
 \vfill

\vfill

\newpage


\shapk
\bilet{Экзаменационный билет}
\setcounter{zad}{0}

\vfill
\z 	Классическая немецкая философия.
 \vfill
\z 	Гражданское общество и государство.
 \vfill

\vfill

\newpage


\shapk
\bilet{Экзаменационный билет}
\setcounter{zad}{0}

\vfill
\z 	Возникновение и развитие марксисткой философии.
 \vfill
\z 	Общество и его структура.
 \vfill

\vfill

\newpage


\shapk
\bilet{Экзаменационный билет}
\setcounter{zad}{0}

\vfill
\z 	Русская философия 19 начала 20 века.
 \vfill
\z 	Формационная и цивилизационная концепция общественного развития.
 \vfill

\vfill

\newpage


\shapk
\bilet{Экзаменационный билет}
\setcounter{zad}{0}

\vfill
\z 	Отношение к разуму и науке в философии 20 века.
 \vfill
\z 	Религиозные ценности и свобода совести.
 \vfill

\vfill

\newpage


\shapk
\bilet{Экзаменационный билет}
\setcounter{zad}{0}

\vfill
\z 	Человек в мире и мир человека (экзистенциализм).
 \vfill
\z 	Эстетические ценности и их роль в человеческой жизни.
 \vfill

\vfill

\newpage


\shapk
\bilet{Экзаменационный билет}
\setcounter{zad}{0}

\vfill
\z 	Герменевтика как метод философского познания.
 \vfill
\z 	Нравственные ценности. Представления о совершенном человеке в различных культурах.
 \vfill

\vfill

\newpage


\shapk
\bilet{Экзаменационный билет}
\setcounter{zad}{0}

\vfill
\z 	Исторические формы позитивизма. 
 \vfill
\z 	Насилие и ненасилие.
 \vfill

\vfill

\newpage


\shapk
\bilet{Экзаменационный билет}
\setcounter{zad}{0}

\vfill
\z 	 Учение о бытие. Монистические и плюралистические концепции.
 \vfill
\z 	Человек и природа.
 \vfill

\vfill

\newpage


\shapk
\bilet{Экзаменационный билет}
\setcounter{zad}{0}

\vfill
\z 	Материя и ее атрибуты.
 \vfill
\z 	Человек, общество, культура.
 \vfill

\vfill

\newpage


\shapk
\bilet{Экзаменационный билет}
\setcounter{zad}{0}

\vfill
\z 	Самоорганизация бытия. Понятие материального и идеального.
 \vfill
\z 	Сознание, самосознание и личность.
 \vfill

\vfill

\newpage


\shapk
\bilet{Экзаменационный билет}
\setcounter{zad}{0}

\vfill
\z 	Движение и развитие, диалектика.
 \vfill
\z 	Смысл человеческого бытия свобода и ответственность.
 \vfill

\vfill

\newpage


\shapk
\bilet{Экзаменационный билет}
\setcounter{zad}{0}

\vfill
\z 	Детерминизм и индетерминизм. Динамические и статистические закономерности.
 \vfill
\z 	Научное и  вненаучное знание. Критерий научности.
 \vfill

\vfill

\newpage


\shapk
\bilet{Экзаменационный билет}
\setcounter{zad}{0}

\vfill
\z 	 Пространство и время, их свойства.
 \vfill
\z 	Искусство спора. Основы логики.
 \vfill

\vfill

\newpage


\shapk
\bilet{Экзаменационный билет}
\setcounter{zad}{0}

\vfill
\z 	Материальные и духовные ценности в жизни общества.
 \vfill
\z 	Действительность, мышление, логика и язык.
 \vfill

\vfill

\newpage


\shapk
\bilet{Экзаменационный билет}
\setcounter{zad}{0}

\vfill
\z 	Диалектика, ее сущность и исторические типы.
 \vfill
\z 	Рациональное и иррациональное в познавательной деятельности.
 \vfill

\vfill

\newpage


\shapk
\bilet{Экзаменационный билет}
\setcounter{zad}{0}

\vfill
\z 	Теория диалектики.
 \vfill
\z 	Вера и знание.
 \vfill

\vfill

\newpage


\shapk
\bilet{Экзаменационный билет}
\setcounter{zad}{0}

\vfill
\z 	Сознание и познание.
 \vfill
\z 	Теоретическое познание и его стадии.
 \vfill

\vfill

\newpage


\shapk
\bilet{Экзаменационный билет}
\setcounter{zad}{0}

\vfill
\z 	Чувственное и рациональное познание.
 \vfill
\z 	Эмпирическое познание и его стадии.
 \vfill

\vfill

\newpage


\shapk
\bilet{Экзаменационный билет}
\setcounter{zad}{0}

\vfill
\z 	Познание, творчество, практика.
 \vfill
\z 	Структура научного познания его методы и формы.
 \vfill

\vfill

\newpage



\end{document}