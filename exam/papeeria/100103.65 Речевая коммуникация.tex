\documentclass[
	14pt,
	a4paper,
	]
	{scrartcl}

\usepackage{mysty}

\newcommand{\spec}{100103.65}
\newcommand{\disc}{Речевая\\ \multicolumn{2}{r}{ коммуникация }}
\newcommand{\kafedra}{сервиса}

\usepackage[
	papersize={210mm,99mm},
	top=.5cm,
	left=1.5cm,
	right=1.5cm,
	bottom=.5cm
	]{geometry}

\pagestyle{empty}

\begin{document}


\shapk
\bilet{Экзаменационный билет}
\setcounter{zad}{0}

\vfill
\z Понятие коммуникации.
 \vfill
\z Идеи Дейла Карнеги (или другого автора) о тактике спора. \vfill

\vfill

\newpage


\shapk
\bilet{Экзаменационный билет}
\setcounter{zad}{0}

\vfill
\z Успешность речевого взаимодействия.
 \vfill
\z Использование афоризмов, пословиц и поговорок как средств психологической защиты.
 \vfill

\vfill

\newpage


\shapk
\bilet{Экзаменационный билет}
\setcounter{zad}{0}

\vfill
\z Коммуникация и общение.
 \vfill
\z Нормы литературного языка (речевая норма).
 \vfill

\vfill

\newpage


\shapk
\bilet{Экзаменационный билет}
\setcounter{zad}{0}

\vfill
\z Модель речевой коммуникации.
 \vfill
\z Особенности составления официально-деловых текстов.
 \vfill

\vfill

\newpage


\shapk
\bilet{Экзаменационный билет}
\setcounter{zad}{0}

\vfill
\z Коммуникативный акт и его составляющие.
 \vfill
\z Рекомендации по составлению договора.
 \vfill

\vfill

\newpage


\shapk
\bilet{Экзаменационный билет}
\setcounter{zad}{0}

\vfill
\z Язык и речь.
 \vfill
\z Интонирование звучащей речи.
 \vfill

\vfill

\newpage


\shapk
\bilet{Экзаменационный билет}
\setcounter{zad}{0}

\vfill
\z Речевой акт.
 \vfill
\z Варианты составления деловых писем.
 \vfill

\vfill

\newpage


\shapk
\bilet{Экзаменационный билет}
\setcounter{zad}{0}

\vfill
\z Речь и мышление.
 \vfill
\z Рекомендации по сохранению и развитию голоса.
 \vfill

\vfill

\newpage


\shapk
\bilet{Экзаменационный билет}
\setcounter{zad}{0}

\vfill
\z Разговорная речь.
 \vfill
\z Взаимодействие интересов, состояния и отношений людей в процессе коммуникации.
 \vfill

\vfill

\newpage


\shapk
\bilet{Экзаменационный билет}
\setcounter{zad}{0}

\vfill
\z Функция языка и речи.
 \vfill
\z Психофизиология речевой коммуникации.
 \vfill

\vfill

\newpage


\shapk
\bilet{Экзаменационный билет}
\setcounter{zad}{0}

\vfill
\z Виды речевой деятельности.
 \vfill
\z Этапы передачи сообщения.
 \vfill

\vfill

\newpage


\shapk
\bilet{Экзаменационный билет}
\setcounter{zad}{0}

\vfill
\z Способы чтения.
 \vfill
\z Аксиомы коммуникации.
 \vfill

\vfill

\newpage


\shapk
\bilet{Экзаменационный билет}
\setcounter{zad}{0}

\vfill
\z Язык как код.
 \vfill
\z Логические ошибки в процессе речевой коммуникации.
 \vfill

\vfill

\newpage


\shapk
\bilet{Экзаменационный билет}
\setcounter{zad}{0}

\vfill
\z Инициация коммуникативного акта.
 \vfill
\z Фаза аргументации или принятия решения.
 \vfill

\vfill

\newpage


\shapk
\bilet{Экзаменационный билет}
\setcounter{zad}{0}

\vfill
\z Подготовительные условия коммуникативного акта.
 \vfill
\z Классификация речевых актов.
 \vfill

\vfill

\newpage


\shapk
\bilet{Экзаменационный билет}
\setcounter{zad}{0}

\vfill
\z Коммуникативная целесообразность речи.
 \vfill
\z Коммуникативный акт и референция.
 \vfill

\vfill

\newpage


\shapk
\bilet{Экзаменационный билет}
\setcounter{zad}{0}

\vfill
\z Этика речевой коммуникации.
 \vfill
\z Принцы вежливости в коммуникативном кодексе.
 \vfill

\vfill

\newpage


\shapk
\bilet{Экзаменационный билет}
\setcounter{zad}{0}

\vfill
\z Психология речевой коммуникации.
 \vfill
\z Максима такта.
 \vfill

\vfill

\newpage


\shapk
\bilet{Экзаменационный билет}
\setcounter{zad}{0}

\vfill
\z Модели коммуникативных промахов.
 \vfill
\z Максима манеры.
 \vfill

\vfill

\newpage


\shapk
\bilet{Экзаменационный билет}
\setcounter{zad}{0}

\vfill
\z Основные типы коммуникабельности людей.
 \vfill
\z Максима релевантности.
 \vfill

\vfill

\newpage


\shapk
\bilet{Экзаменационный билет}
\setcounter{zad}{0}

\vfill
\z Особенности поведения адресанта.
 \vfill
\z Максима качества информации.
 \vfill

\vfill

\newpage


\shapk
\bilet{Экзаменационный билет}
\setcounter{zad}{0}

\vfill
\z Коммуникативная роль адресанта.
 \vfill
\z Максима полноты информации.
 \vfill

\vfill

\newpage


\shapk
\bilet{Экзаменационный билет}
\setcounter{zad}{0}

\vfill
\z Неожидаемые коммуникативные акты.
 \vfill
\z Типы вербального поведения, способствующие пониманию партнера.
 \vfill

\vfill

\newpage


\shapk
\bilet{Экзаменационный билет}
\setcounter{zad}{0}

\vfill
\z Ожидаемые коммуникативные акты.
 \vfill
\z Уровни обмена информацией в процессе общения.
 \vfill

\vfill

\newpage


\shapk
\bilet{Экзаменационный билет}
\setcounter{zad}{0}

\vfill
\z Речевые ситуации (фреймы).
 \vfill
\z Воздействие в процессе коммуникации.
 \vfill

\vfill

\newpage


\shapk
\bilet{Экзаменационный билет}
\setcounter{zad}{0}

\vfill
\z Логичность изложения (логические законы).
 \vfill
\z Коммуникативные барьеры.
 \vfill

\vfill

\newpage


\shapk
\bilet{Экзаменационный билет}
\setcounter{zad}{0}

\vfill
\z Эго-состояние и речевое общение.
 \vfill
\z Коммуникативный кодекс.
 \vfill

\vfill

\newpage


\shapk
\bilet{Экзаменационный билет}
\setcounter{zad}{0}

\vfill
\z Роль контакта в теории и практике речевой коммуникации.
 \vfill
\z Путь к согласию в диалоге.
 \vfill

\vfill

\newpage


\shapk
\bilet{Экзаменационный билет}
\setcounter{zad}{0}

\vfill
\z Основы искусства речи.
 \vfill
\z Дискуссия как форма коммуникации.
 \vfill

\vfill

\newpage



\end{document}