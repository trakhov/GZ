\documentclass[
	14pt,
	a4paper,
	]
	{scrartcl}

\usepackage{mysty}

\newcommand{\spec}{100103.65}
\newcommand{\disc}{Стандартизация\\ \multicolumn{2}{r}{ и сертификация }}
\newcommand{\kafedra}{МиЭ}

\usepackage[
	papersize={210mm,99mm},
	top=.5cm,
	left=1.5cm,
	right=1.5cm,
	bottom=.5cm
	]{geometry}

\pagestyle{empty}

\begin{document}


\shapk
\bilet{Экзаменационный билет}
\setcounter{zad}{0}

\vfill
\z Сущность, задачи и цели стандартизации социально-культурных и туристских услуг.
 \vfill
\z Аккредитация органов по сертификации и испытательных (измерительных) лабораторий. \vfill

\vfill

\newpage


\shapk
\bilet{Экзаменационный билет}
\setcounter{zad}{0}

\vfill
\z Комплексная и опережающая стандартизация социально-культурных и туристских услуг, ее особенности и характеристики.
 \vfill
\z Органы по сертификации и испытательные лаборатории. 
 \vfill

\vfill

\newpage


\shapk
\bilet{Экзаменационный билет}
\setcounter{zad}{0}

\vfill
\z Требования, предъявляемые к стандартам в различных отраслях промышленности. Поясните на примере.
 \vfill
\z Правила и порядок проведения сертификации социально-культурных и туристских услуг.
 \vfill

\vfill

\newpage


\shapk
\bilet{Экзаменационный билет}
\setcounter{zad}{0}

\vfill
\z Охарактеризуйте сущность и назначение комплексной стандартизации социально-культурных и туристских услуг.
 \vfill
\z Качество социально-культурных и туристских услуг и защита потребителя.
 \vfill

\vfill

\newpage


\shapk
\bilet{Экзаменационный билет}
\setcounter{zad}{0}

\vfill
\z Дайте организационно–штатную структуру Госстандарта РФ и его служб.
 \vfill
\z Термины и определения в области сертификации социально-культурных и туристских услуг.
 \vfill

\vfill

\newpage


\shapk
\bilet{Экзаменационный билет}
\setcounter{zad}{0}

\vfill
\z Охарактеризуйте Государственную систему стандартизации России (ГССРФ). 
 \vfill
\z Объекты сертификации, их характеристики.
 \vfill

\vfill

\newpage


\shapk
\bilet{Экзаменационный билет}
\setcounter{zad}{0}

\vfill
\z Дайте общую характеристику системы стандартизации в России.
 \vfill
\z Роль государственного контроля и надзора за соблюдением требований государственных стандартов.
 \vfill

\vfill

\newpage


\shapk
\bilet{Экзаменационный билет}
\setcounter{zad}{0}

\vfill
\z Опишите органы стандартизации РФ.
 \vfill
\z Определение оптимального уровня унификации и стандартизации.
 \vfill

\vfill

\newpage


\shapk
\bilet{Экзаменационный билет}
\setcounter{zad}{0}

\vfill
\z Охарактеризуйте назначение общероссийских классификаторов продукции и услуг. Приведите пример.
 \vfill
\z Научная база стандартизации, ее особенности и характеристики.
 \vfill

\vfill

\newpage


\shapk
\bilet{Экзаменационный билет}
\setcounter{zad}{0}

\vfill
\z Что представляют собой стандарты отраслей, их назначение? Приведите пример.
 \vfill
\z Основные положения государственной системы стандартизации (ГСС).
 \vfill

\vfill

\newpage


\shapk
\bilet{Экзаменационный билет}
\setcounter{zad}{0}

\vfill
\z Назначение и принципы разработки стандарта предприятия (СТП). Приведите пример.
 \vfill
\z Роль сертификации в повышении качества социально-культурных и туристских услуг и ее развитие на национальном уровне.
 \vfill

\vfill

\newpage


\shapk
\bilet{Экзаменационный билет}
\setcounter{zad}{0}

\vfill
\z Дайте характеристику стандарта инженерного общества. Приведите пример.
 \vfill
\z Роль сертификации в повышении качества социально-культурных и туристских услуг и ее развитие на региональном уровне.
 \vfill

\vfill

\newpage


\shapk
\bilet{Экзаменационный билет}
\setcounter{zad}{0}

\vfill
\z Как осуществляются государственный контроль и надзор за соблюдением требований государственных стандартов социально-культурных и туристских услуг?
 \vfill
\z Роль сертификации в повышении качества социально-культурных и туристских услуг и ее развитие на международном уровне.
 \vfill

\vfill

\newpage


\shapk
\bilet{Экзаменационный билет}
\setcounter{zad}{0}

\vfill
\z Опишите перечень некоторых систем межгосударственных и государственных стандартов межотраслевого значения?
 \vfill
\z Исторические основы развития стандартизации.
 \vfill

\vfill

\newpage


\shapk
\bilet{Экзаменационный билет}
\setcounter{zad}{0}

\vfill
\z Приведите схему взаимосвязей государственных систем по стандартизации и сертификации. Определение оптимального уровня унификации и стандартизации.
 \vfill
\z Сертификация систем качества, ее особенности и характеристики.
 \vfill

\vfill

\newpage


\shapk
\bilet{Экзаменационный билет}
\setcounter{zad}{0}

\vfill
\z Как осуществляются классификация и кодирование технико–экономической и социальной информации в стандартизации России?
 \vfill
\z Нормативно–законодательная база сертификации в РФ.
 \vfill

\vfill

\newpage


\shapk
\bilet{Экзаменационный билет}
\setcounter{zad}{0}

\vfill
\z Охарактеризуйте международную систему стандартизации.
 \vfill
\z Опишите форму акта отбора образцов при проведении сертификации социально-культурных и туристских услуг.
 \vfill

\vfill

\newpage


\shapk
\bilet{Экзаменационный билет}
\setcounter{zad}{0}

\vfill
\z Какое значение международного сотрудничества в области стандартизации?
 \vfill
\z Как документально оформляется решение по заявке на проведение сертификации социально-культурных и туристских услуг.
 \vfill

\vfill

\newpage


\shapk
\bilet{Экзаменационный билет}
\setcounter{zad}{0}

\vfill
\z Какие международные и региональные организации по стандартизации социально-культурных и туристских услуг вы знаете?
 \vfill
\z Порядок оформления заявки на проведение сертификации продукции.
 \vfill

\vfill

\newpage


\shapk
\bilet{Экзаменационный билет}
\setcounter{zad}{0}

\vfill
\z Дайте краткую характеристику структуры ИСО (показать на схеме).
 \vfill
\z Правила и порядок проведения сертификации социально-культурных и туристских услуг.
 \vfill

\vfill

\newpage


\shapk
\bilet{Экзаменационный билет}
\setcounter{zad}{0}

\vfill
\z Какие задачи выполняют международные и региональные организации по разработке стандартов ИСО.
 \vfill
\z Как проводится инспекционный контроль за сертификацией социально-культурных и туристских услуг?
 \vfill

\vfill

\newpage


\shapk
\bilet{Экзаменационный билет}
\setcounter{zad}{0}

\vfill
\z Сущность, цели и задачи сертификации. Схемы и системы сертификации. 
 \vfill
\z Дайте характеристику и порядок аттестации производства продукции.
 \vfill

\vfill

\newpage


\shapk
\bilet{Экзаменационный билет}
\setcounter{zad}{0}

\vfill
\z Сертификация социально-культурных и туристских услуг. 
 \vfill
\z Дайте описание схемы сертификации социально-культурных и туристских услуг.
 \vfill

\vfill

\newpage


\shapk
\bilet{Экзаменационный билет}
\setcounter{zad}{0}

\vfill
\z Исторические основы развития сертификации. 
 \vfill
\z Охарактеризуйте 1–5 схемы сертификации.
 \vfill

\vfill

\newpage


\shapk
\bilet{Экзаменационный билет}
\setcounter{zad}{0}

\vfill
\z Основные понятия и термины, приятые в сертификации. 
 \vfill
\z Как распределена ответственность между участниками сертификации социально-культурных и туристских услуг?
 \vfill

\vfill

\newpage


\shapk
\bilet{Экзаменационный билет}
\setcounter{zad}{0}

\vfill
\z Обязательная и добровольная сертификация социально-культурных и туристских услуг.
 \vfill
\z Какие особенности выполняют: заявитель, третья сторона и орган по сертификации?
 \vfill

\vfill

\newpage


\shapk
\bilet{Экзаменационный билет}
\setcounter{zad}{0}

\vfill
\z Дайте определения понятиям: сертификат соответствия, знак соответствия, идентификация продукции. 
 \vfill
\z Назначение и задачи испытательной лаборатории (центра) по сертификации продукции и услуг.
 \vfill

\vfill

\newpage


\shapk
\bilet{Экзаменационный билет}
\setcounter{zad}{0}

\vfill
\z Условия осуществления сертификации социально-культурных и туристских услуг.
 \vfill
\z Дайте характеристику формы декларации о соответствии как способа доказательства соответствия в отдельных схемах сертификации.
 \vfill

\vfill

\newpage



\end{document}