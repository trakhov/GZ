\documentclass[
	14pt,
	a4paper,
	]
	{scrartcl}

\usepackage{mysty}

\newcommand{\spec}{100103.65}
\newcommand{\disc}{Математика \\}
\newcommand{\kafedra}{ЕМиТД}

\usepackage[
	papersize={210mm,99mm},
	top=.5cm,
	left=1.5cm,
	right=1.5cm,
	bottom=.5cm
	]{geometry}

\pagestyle{empty}

\begin{document}


\shapk
\bilet{Экзаменационный билет}
\setcounter{zad}{0}

\vfill
\z 	Элементы линейной алгебры
 \vfill
\z 	Элементы теории корреляции \vfill

\vfill

\newpage


\shapk
\bilet{Экзаменационный билет}
\setcounter{zad}{0}

\vfill
\z 	Матрицы и действия над ними.
 \vfill
\z 	Доверительный интервал.
 \vfill

\vfill

\newpage


\shapk
\bilet{Экзаменационный билет}
\setcounter{zad}{0}

\vfill
\z 	Определители, их свойства и вычисление.
 \vfill
\z 	Статистические оценки параметров распределения.
 \vfill

\vfill

\newpage


\shapk
\bilet{Экзаменационный билет}
\setcounter{zad}{0}

\vfill
\z 	Алгебраическое дополнение.
 \vfill
\z 	Полигон и гистограмма.
 \vfill

\vfill

\newpage


\shapk
\bilet{Экзаменационный билет}
\setcounter{zad}{0}

\vfill
\z 	Теорема о вычислении определителя через разложение по элементам строки.
 \vfill
\z 	Выборочная и генеральная совокупности. Методы отбора. 
 \vfill

\vfill

\newpage


\shapk
\bilet{Экзаменационный билет}
\setcounter{zad}{0}

\vfill
\z 	Обратная матрица.
 \vfill
\z 	Задачи математической статистики. Основные понятия.
 \vfill

\vfill

\newpage


\shapk
\bilet{Экзаменационный билет}
\setcounter{zad}{0}

\vfill
\z 	Элементарные преобразования.
 \vfill
\z 	Математическая статистика
 \vfill

\vfill

\newpage


\shapk
\bilet{Экзаменационный билет}
\setcounter{zad}{0}

\vfill
\z 	Ранг матрицы.
 \vfill
\z 	Показательное распределение
 \vfill

\vfill

\newpage


\shapk
\bilet{Экзаменационный билет}
\setcounter{zad}{0}

\vfill
\z 	Системы линейных уравнений, условие их совместности. Теорема Кронекера-Капелли.
 \vfill
\z 	Равномерное распределение.
 \vfill

\vfill

\newpage


\shapk
\bilet{Экзаменационный билет}
\setcounter{zad}{0}

\vfill
\z 	Матричный способ решения систем.
 \vfill
\z 	Правило 3-х сигм.
 \vfill

\vfill

\newpage


\shapk
\bilet{Экзаменационный билет}
\setcounter{zad}{0}

\vfill
\z 	Формулы Крамера. 
 \vfill
\z 	Вероятность попадания в заданной интервал для нормальной случайной величины.
 \vfill

\vfill

\newpage


\shapk
\bilet{Экзаменационный билет}
\setcounter{zad}{0}

\vfill
\z 	Метод Гаусса.
 \vfill
\z 	Нормальное распределение.
 \vfill

\vfill

\newpage


\shapk
\bilet{Экзаменационный билет}
\setcounter{zad}{0}

\vfill
\z 	Элементы векторной алгебры и аналитической геометрии
 \vfill
\z 	Числовые характеристики непрерывных случайных величин.
 \vfill

\vfill

\newpage


\shapk
\bilet{Экзаменационный билет}
\setcounter{zad}{0}

\vfill
\z 	Декартова система координат.
 \vfill
\z 	Вероятность попадания непрерывной случайной величины в заданный интервал.
 \vfill

\vfill

\newpage


\shapk
\bilet{Экзаменационный билет}
\setcounter{zad}{0}

\vfill
\z 	Простейшие задачи аналитической геометрии (расстояние между точками, деление отрезка в заданном отношении).
 \vfill
\z 	Понятие о дифференциальной функции распределения.
 \vfill

\vfill

\newpage


\shapk
\bilet{Экзаменационный билет}
\setcounter{zad}{0}

\vfill
\z 	Прямая на плоскости, различные виды ее уравнений.
 \vfill
\z 	Понятие об интегральной  функции распределения. 
 \vfill

\vfill

\newpage


\shapk
\bilet{Экзаменационный билет}
\setcounter{zad}{0}

\vfill
\z 	Угол между прямыми на плоскости.
 \vfill
\z 	Числовые характеристики дискретных случайных величин.
 \vfill

\vfill

\newpage


\shapk
\bilet{Экзаменационный билет}
\setcounter{zad}{0}

\vfill
\z 	Расстояние от точки до прямой.
 \vfill
\z 	Распределение Пуассона. Простейший поток событий.
 \vfill

\vfill

\newpage


\shapk
\bilet{Экзаменационный билет}
\setcounter{zad}{0}

\vfill
\z 	Кривые второго порядка их канонические уравнения: эллипс, парабола, гипербола.
 \vfill
\z 	Биномиальное распределение. Геометрическое распределение.
 \vfill

\vfill

\newpage


\shapk
\bilet{Экзаменационный билет}
\setcounter{zad}{0}

\vfill
\z 	Векторы, линейные операции над ними.
 \vfill
\z 	Дискретные случайные величины. Закон распределения. 
 \vfill

\vfill

\newpage


\shapk
\bilet{Экзаменационный билет}
\setcounter{zad}{0}

\vfill
\z 	Координаты вектора, его длина, направляющие косинусы.
 \vfill
\z 	Функции Лапласа.
 \vfill

\vfill

\newpage


\shapk
\bilet{Экзаменационный билет}
\setcounter{zad}{0}

\vfill
\z 	Разложение вектора по координатным ортам.
 \vfill
\z 	Повторение испытаний. Схема Бернулли. 
 \vfill

\vfill

\newpage


\shapk
\bilet{Экзаменационный билет}
\setcounter{zad}{0}

\vfill
\z 	Скалярное произведение векторов.
 \vfill
\z 	Вероятность гипотез. Формулы Бейеса.
 \vfill

\vfill

\newpage


\shapk
\bilet{Экзаменационный билет}
\setcounter{zad}{0}

\vfill
\z 	Векторное произведение векторов.
 \vfill
\z 	Формула полной вероятности. 
 \vfill

\vfill

\newpage


\shapk
\bilet{Экзаменационный билет}
\setcounter{zad}{0}

\vfill
\z 	Смешанное произведение векторов.
 \vfill
\z 	Условная вероятность, умножение вероятностей.
 \vfill

\vfill

\newpage


\shapk
\bilet{Экзаменационный билет}
\setcounter{zad}{0}

\vfill
\z 	Условие перпендикулярности, 
 \vfill
\z 	Сложение событий.
 \vfill

\vfill

\newpage


\shapk
\bilet{Экзаменационный билет}
\setcounter{zad}{0}

\vfill
\z 	Условие коллинеарности векторов.
 \vfill
\z 	Вероятность событий. 
 \vfill

\vfill

\newpage


\shapk
\bilet{Экзаменационный билет}
\setcounter{zad}{0}

\vfill
\z 	Дифференциальное и интегральное исчисление
 \vfill
\z 	Случайные события и их классификация.
 \vfill

\vfill

\newpage


\shapk
\bilet{Экзаменационный билет}
\setcounter{zad}{0}

\vfill
\z 	Введение в анализ
 \vfill
\z 	Предмет теории вероятностей.
 \vfill

\vfill

\newpage


\shapk
\bilet{Экзаменационный билет}
\setcounter{zad}{0}

\vfill
\z 	Функция, область определения, способы задания. Сложная и обратная функции.
 \vfill
\z 	Элементы комбинаторики: перестановки, размещения, сочетания.
 \vfill

\vfill

\newpage


\shapk
\bilet{Экзаменационный билет}
\setcounter{zad}{0}

\vfill
\z 	Предел функции.
 \vfill
\z 	Теория вероятностей
 \vfill

\vfill

\newpage


\shapk
\bilet{Экзаменационный билет}
\setcounter{zad}{0}

\vfill
\z 	Бесконечно малые функции, их свойства.
 \vfill
\z 	Геометрическое приложение определенного интеграла: вычисление площадей плоских фигур.
 \vfill

\vfill

\newpage


\shapk
\bilet{Экзаменационный билет}
\setcounter{zad}{0}

\vfill
\z 	Лемма о пределах.
 \vfill
\z 	Замена переменной и интегрирование по частям.
 \vfill

\vfill

\newpage


\shapk
\bilet{Экзаменационный билет}
\setcounter{zad}{0}

\vfill
\z 	Основные теоремы о пределах.
 \vfill
\z 	Формула Ньютона-Лейбница.
 \vfill

\vfill

\newpage


\shapk
\bilet{Экзаменационный билет}
\setcounter{zad}{0}

\vfill
\z 	Алгебраические методы вычисления пределов.
 \vfill
\z 	Определенный интеграл, геометрический смысл, свойства.
 \vfill

\vfill

\newpage


\shapk
\bilet{Экзаменационный билет}
\setcounter{zad}{0}

\vfill
\z 	Замечательные пределы.
 \vfill
\z 	Определенный интеграл.
 \vfill

\vfill

\newpage


\shapk
\bilet{Экзаменационный билет}
\setcounter{zad}{0}

\vfill
\z 	Непрерывность функции. Свойства непрерывных функций
 \vfill
\z 	Основные методы интегрирования: замена переменных, интегрирование по частям
 \vfill

\vfill

\newpage


\shapk
\bilet{Экзаменационный билет}
\setcounter{zad}{0}

\vfill
\z 	Точки разрыва.
 \vfill
\z 	Таблица интегралов.
 \vfill

\vfill

\newpage


\shapk
\bilet{Экзаменационный билет}
\setcounter{zad}{0}

\vfill
\z 	Производная и дифференциал функции одной переменной
 \vfill
\z 	Неопределенный интеграл, свойства.
 \vfill

\vfill

\newpage


\shapk
\bilet{Экзаменационный билет}
\setcounter{zad}{0}

\vfill
\z 	Задачи, приводящие к понятию производной.
 \vfill
\z 	Первообразная.
 \vfill

\vfill

\newpage


\shapk
\bilet{Экзаменационный билет}
\setcounter{zad}{0}

\vfill
\z 	Производная, ее геометрический смысл.
 \vfill
\z 	Неопределенный интеграл
 \vfill

\vfill

\newpage


\shapk
\bilet{Экзаменационный билет}
\setcounter{zad}{0}

\vfill
\z 	Правила дифференцирования.
 \vfill
\z 	Наибольшее и наименьшее значения функции на отрезке
 \vfill

\vfill

\newpage


\shapk
\bilet{Экзаменационный билет}
\setcounter{zad}{0}

\vfill
\z 	Производные основных элементарных функций.
 \vfill
\z 	Исследование функций и построение графика.
 \vfill

\vfill

\newpage


\shapk
\bilet{Экзаменационный билет}
\setcounter{zad}{0}

\vfill
\z 	Производная сложной  функции
 \vfill
\z 	Асимптоты кривой.
 \vfill

\vfill

\newpage


\shapk
\bilet{Экзаменационный билет}
\setcounter{zad}{0}

\vfill
\z 	Дифференциал функции, его геометрический смысл, правило вычисления.
 \vfill
\z 	Выпуклость и вогнутость, точки перегиба. 
 \vfill

\vfill

\newpage


\shapk
\bilet{Экзаменационный билет}
\setcounter{zad}{0}

\vfill
\z 	Производные и дифференциалы высших порядков. 
 \vfill
\z 	Понятие экстремума, основные теоремы.
 \vfill

\vfill

\newpage


\shapk
\bilet{Экзаменационный билет}
\setcounter{zad}{0}

\vfill
\z 	Теоремы Ролля, Лагранжа, Коши.
 \vfill
\z 	Монотонность.
 \vfill

\vfill

\newpage


\shapk
\bilet{Экзаменационный билет}
\setcounter{zad}{0}

\vfill
\z 	Приложения производной и дифференциала функции одной переменной
 \vfill
\z 	Правило Лопиталя.
 \vfill

\vfill

\newpage



\end{document}