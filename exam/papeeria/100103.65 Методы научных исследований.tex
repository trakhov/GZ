\documentclass[
	14pt,
	a4paper,
	]
	{scrartcl}

\usepackage{mysty}

\newcommand{\spec}{100103.65}
\newcommand{\disc}{Методы\\ \multicolumn{2}{r}{ научных исследований }}
\newcommand{\kafedra}{сервиса}

\usepackage[
	papersize={210mm,99mm},
	top=.5cm,
	left=1.5cm,
	right=1.5cm,
	bottom=.5cm
	]{geometry}

\pagestyle{empty}

\begin{document}


\shapk
\bilet{Экзаменационный билет}
\setcounter{zad}{0}

\vfill
\z Роль научных исследований в совершенствовании социально-культурного обслуживания.
 \vfill
\z Методы организации и проведение маркетинговых исследований. \vfill

\vfill

\newpage


\shapk
\bilet{Экзаменационный билет}
\setcounter{zad}{0}

\vfill
\z Методология научного исследования. Логика научного исследования. Научные парадигмы, исследовательские программы и логика конкретного исследования.
 \vfill
\z Основные социологические шкалы установок (Терстоун, Богардус, Лайкерт, Гутман и др.).
 \vfill

\vfill

\newpage


\shapk
\bilet{Экзаменационный билет}
\setcounter{zad}{0}

\vfill
\z Методика проведения научного исследования в СКСиТ.
 \vfill
\z Психологическая традиция в развитии метода опроса.
 \vfill

\vfill

\newpage


\shapk
\bilet{Экзаменационный билет}
\setcounter{zad}{0}

\vfill
\z Направления развития научных исследований в сфере социально-культурного сервиса.
 \vfill
\z Классификация исследовательских ролей по Р. Гоулду.
 \vfill

\vfill

\newpage


\shapk
\bilet{Экзаменационный билет}
\setcounter{zad}{0}

\vfill
\z Метод опроса в общественных науках.
 \vfill
\z ИТ их влияние на получение и обработку данных.
 \vfill

\vfill

\newpage


\shapk
\bilet{Экзаменационный билет}
\setcounter{zad}{0}

\vfill
\z Виды социологического опроса.
 \vfill
\z Классификация, принципы формирования и использования информации.
 \vfill

\vfill

\newpage


\shapk
\bilet{Экзаменационный билет}
\setcounter{zad}{0}

\vfill
\z Принципы разработки вопросников и вопросов.
 \vfill
\z  Методы отбора источников информации
 \vfill

\vfill

\newpage


\shapk
\bilet{Экзаменационный билет}
\setcounter{zad}{0}

\vfill
\z Метод письменного опроса.
 \vfill
\z Психографические методы изучения потребителя.
 \vfill

\vfill

\newpage


\shapk
\bilet{Экзаменационный билет}
\setcounter{zad}{0}

\vfill
\z Метод устного опроса.
 \vfill
\z Модели анализа качества услуг.
 \vfill

\vfill

\newpage


\shapk
\bilet{Экзаменационный билет}
\setcounter{zad}{0}

\vfill
\z Метод фокус-групп.
 \vfill
\z Теория причастности: общие положения.
 \vfill

\vfill

\newpage


\shapk
\bilet{Экзаменационный билет}
\setcounter{zad}{0}

\vfill
\z Метод опроса с помощью технических средств.
 \vfill
\z Методы исследования качества услуг.
 \vfill

\vfill

\newpage


\shapk
\bilet{Экзаменационный билет}
\setcounter{zad}{0}

\vfill
\z Метод массового опроса.
 \vfill
\z Методы оценки сегментов рынка.
 \vfill

\vfill

\newpage


\shapk
\bilet{Экзаменационный билет}
\setcounter{zad}{0}

\vfill
\z Включенное наблюдение и этнографический метод: определение и исторические истоки.
 \vfill
\z Построение графиков, диаграмм, гистограмм, схем.
 \vfill

\vfill

\newpage


\shapk
\bilet{Экзаменационный билет}
\setcounter{zad}{0}

\vfill
\z Планирование исследования методом наблюдения: определение проблемы, тактики наблюдателя.
 \vfill
\z Качественные методы: понятие и виды.
 \vfill

\vfill

\newpage


\shapk
\bilet{Экзаменационный билет}
\setcounter{zad}{0}

\vfill
\z Планирование исследования методом эксперимента: определение проблемы, построение экспериментальной модели.
 \vfill
\z Количественные методы: понятие и виды.
 \vfill

\vfill

\newpage


\shapk
\bilet{Экзаменационный билет}
\setcounter{zad}{0}

\vfill
\z Планирование исследования методом опроса.
 \vfill
\z Систематико-логические методы: понятие и виды.
 \vfill

\vfill

\newpage


\shapk
\bilet{Экзаменационный билет}
\setcounter{zad}{0}

\vfill
\z Ситуация наблюдения, роли наблюдателя, взаимоотношения «в поле».
 \vfill
\z Интуитивно-творческие методы: понятие и виды.
 \vfill

\vfill

\newpage


\shapk
\bilet{Экзаменационный билет}
\setcounter{zad}{0}

\vfill
\z Валидность включенного наблюдения: подход К. Гирца, подход Н. Дензина, подход М. Хэммерсли.
 \vfill
\z Творческие методы: сфера применения.
 \vfill

\vfill

\newpage


\shapk
\bilet{Экзаменационный билет}
\setcounter{zad}{0}

\vfill
\z Метод экспериментального исследования в социально-культурном сервисе и туризме: определение, общие принципы.
 \vfill
\z Постановка цели маркетингового исследования.
 \vfill

\vfill

\newpage


\shapk
\bilet{Экзаменационный билет}
\setcounter{zad}{0}

\vfill
\z Типы экспериментов в социально-культурном сервисе и туризме.
 \vfill
\z Основные этапы выполнения научно-исследовательской темы.
 \vfill

\vfill

\newpage


\shapk
\bilet{Экзаменационный билет}
\setcounter{zad}{0}

\vfill
\z Индексы и шкалирование в исследовании: понятие и виды индексов.
 \vfill
\z Общая линейная модель и эксплораторные подходы к анализу структуры эмпирических данных.
 \vfill

\vfill

\newpage


\shapk
\bilet{Экзаменационный билет}
\setcounter{zad}{0}

\vfill
\z Многомерный-многоматричный подход Д. Кэмпбелла.
 \vfill
\z Правила оформления таблиц, графиков, списка литературы.
 \vfill

\vfill

\newpage


\shapk
\bilet{Экзаменационный билет}
\setcounter{zad}{0}

\vfill
\z Конструирование индексов и шкал.
 \vfill
\z Метод уточнения в анализе связи между переменными.
 \vfill

\vfill

\newpage


\shapk
\bilet{Экзаменационный билет}
\setcounter{zad}{0}

\vfill
\z Многомерное шкалирование: общие принципы и социологические приложения.
 \vfill
\z Первичные источники информации.
 \vfill

\vfill

\newpage


\shapk
\bilet{Экзаменационный билет}
\setcounter{zad}{0}

\vfill
\z Генеральная совокупность, выборка, распределение, выборочная оценка и ошибка.
 \vfill
\z Типы источников внутренней информации.
 \vfill

\vfill

\newpage


\shapk
\bilet{Экзаменационный билет}
\setcounter{zad}{0}

\vfill
\z Типы вероятностных выборок и процедуры их построения.
 \vfill
\z Типы внешних источников информации.
 \vfill

\vfill

\newpage


\shapk
\bilet{Экзаменационный билет}
\setcounter{zad}{0}

\vfill
\z Основа выборки: виды, подходы к составлению.
 \vfill
\z Анализ поведения потребителей.
 \vfill

\vfill

\newpage


\shapk
\bilet{Экзаменационный билет}
\setcounter{zad}{0}

\vfill
\z Одномерный анализ: табулирование и представление данных.
 \vfill
\z Анализ внутренней информации.
 \vfill

\vfill

\newpage


\shapk
\bilet{Экзаменационный билет}
\setcounter{zad}{0}

\vfill
\z Типы графического представления данных.
 \vfill
\z SWOT-анализ.
 \vfill

\vfill

\newpage


\shapk
\bilet{Экзаменационный билет}
\setcounter{zad}{0}

\vfill
\z Использование основных статистик: меры центральной тенденции и меры разброса.
 \vfill
\z SPSS-анализ обработки данных.
 \vfill

\vfill

\newpage



\end{document}