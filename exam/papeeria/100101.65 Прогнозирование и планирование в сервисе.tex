\documentclass[
	14pt,
	a4paper,
	]
	{scrartcl}

\usepackage{mysty}

\newcommand{\spec}{100101.65}
\newcommand{\disc}{Прогнозирование\\ \multicolumn{2}{r}{ и планирование в сервисе }}
\newcommand{\kafedra}{МиЭ}

\usepackage[
	papersize={210mm,99mm},
	top=.5cm,
	left=1.5cm,
	right=1.5cm,
	bottom=.5cm
	]{geometry}

\pagestyle{empty}

\begin{document}


\shapk
\bilet{Экзаменационный билет}
\setcounter{zad}{0}

\vfill
\z Понятие планирования: директивное планирование, индикативное планирование, регулирование.
 \vfill
\z Показатели плана натуральные и стоимостные. \vfill

\vfill

\newpage


\shapk
\bilet{Экзаменационный билет}
\setcounter{zad}{0}

\vfill
\z Понятие экономического прогнозирования.
 \vfill
\z Финансовый план – раздел бизнес-плана предприятия.
 \vfill

\vfill

\newpage


\shapk
\bilet{Экзаменационный билет}
\setcounter{zad}{0}

\vfill
\z Классификация методов прогнозирования.
 \vfill
\z План доходов и расходов.
 \vfill

\vfill

\newpage


\shapk
\bilet{Экзаменационный билет}
\setcounter{zad}{0}

\vfill
\z Модели экономического прогнозирования.
 \vfill
\z Финансовое планирование – задачи, содержание.
 \vfill

\vfill

\newpage


\shapk
\bilet{Экзаменационный билет}
\setcounter{zad}{0}

\vfill
\z Организационные аспекты прогнозирования.
 \vfill
\z Планирование прибыли и рентабельности.
 \vfill

\vfill

\newpage


\shapk
\bilet{Экзаменационный билет}
\setcounter{zad}{0}

\vfill
\z Метод экстраполяции тенденции.
 \vfill
\z Задачи и содержание планирования издержек производства.
 \vfill

\vfill

\newpage


\shapk
\bilet{Экзаменационный билет}
\setcounter{zad}{0}

\vfill
\z Метод «мозговой атаки».
 \vfill
\z Расчет безубыточности предприятия.
 \vfill

\vfill

\newpage


\shapk
\bilet{Экзаменационный билет}
\setcounter{zad}{0}

\vfill
\z Метод «Дельфи».
 \vfill
\z Расчеты потребности в трудовых ресурсах.
 \vfill

\vfill

\newpage


\shapk
\bilet{Экзаменационный билет}
\setcounter{zad}{0}

\vfill
\z Метод написания сценариев.
 \vfill
\z Планирование персонала.
 \vfill

\vfill

\newpage


\shapk
\bilet{Экзаменационный билет}
\setcounter{zad}{0}

\vfill
\z Метод «Дерева целей».
 \vfill
\z Расчет точки безубыточности.
 \vfill

\vfill

\newpage


\shapk
\bilet{Экзаменационный билет}
\setcounter{zad}{0}

\vfill
\z Методы моделирования.
 \vfill
\z Формирование ассортимента оказываемых услуг на предприятии.
 \vfill

\vfill

\newpage


\shapk
\bilet{Экзаменационный билет}
\setcounter{zad}{0}

\vfill
\z Прогнозирование развития производства (на примере автомобильной промышленности). Прогнозирование показателей эффективности производств
 \vfill
\z Содержание и показатели производственной программы предприятия.
 \vfill

\vfill

\newpage


\shapk
\bilet{Экзаменационный билет}
\setcounter{zad}{0}

\vfill
\z Прогнозирование спроса и детерминант, его составляющих.
 \vfill
\z Планирование капитальных вложений.
 \vfill

\vfill

\newpage


\shapk
\bilet{Экзаменационный билет}
\setcounter{zad}{0}

\vfill
\z Прогнозирование предложения на рынке авто-услуг.
 \vfill
\z Планирование управления и организации труда.
 \vfill

\vfill

\newpage


\shapk
\bilet{Экзаменационный билет}
\setcounter{zad}{0}

\vfill
\z Прогноз предложения и его детерминант.
 \vfill
\z Основные требования, предъявляемые к бизнес-плану предприятия в рыночных условиях.
 \vfill

\vfill

\newpage


\shapk
\bilet{Экзаменационный билет}
\setcounter{zad}{0}

\vfill
\z Основные принципы планирования.
 \vfill
\z Планирование производственной мощности предприятия.
 \vfill

\vfill

\newpage


\shapk
\bilet{Экзаменационный билет}
\setcounter{zad}{0}

\vfill
\z Типы планов.
 \vfill
\z Планирование ассортимента оказываемых услуг.
 \vfill

\vfill

\newpage


\shapk
\bilet{Экзаменационный билет}
\setcounter{zad}{0}

\vfill
\z Характеристика и задачи планов.
 \vfill
\z Порядок формирования производственной программы предприятия.
 \vfill

\vfill

\newpage


\shapk
\bilet{Экзаменационный билет}
\setcounter{zad}{0}

\vfill
\z Экономико-математический метод планирования.
 \vfill
\z Организация планирования на предприятии.
 \vfill

\vfill

\newpage


\shapk
\bilet{Экзаменационный билет}
\setcounter{zad}{0}

\vfill
\z Балансовый метод планирования.
 \vfill
\z Процесс (этапы) планирования.
 \vfill

\vfill

\newpage


\shapk
\bilet{Экзаменационный билет}
\setcounter{zad}{0}

\vfill
\z Факторный метод планирования.
 \vfill
\z Функциональная стратегия предприятия.
 \vfill

\vfill

\newpage


\shapk
\bilet{Экзаменационный билет}
\setcounter{zad}{0}

\vfill
\z Индексный метод планирования.
 \vfill
\z Базовая стратегия предприятия.
 \vfill

\vfill

\newpage


\shapk
\bilet{Экзаменационный билет}
\setcounter{zad}{0}

\vfill
\z Нормативный метод планирования.
 \vfill
\z Определение политики предприятия (целеполагание).
 \vfill

\vfill

\newpage


\shapk
\bilet{Экзаменационный билет}
\setcounter{zad}{0}

\vfill
\z Оценка риска.
 \vfill
\z Стратегическое планирование развития предприятия.
 \vfill

\vfill

\newpage


\shapk
\bilet{Экзаменационный билет}
\setcounter{zad}{0}

\vfill
\z Плановые расчеты и показатели как инструмент планирования.
 \vfill
\z Взаимосвязь различных видов планов.
 \vfill

\vfill

\newpage


\shapk
\bilet{Экзаменационный билет}
\setcounter{zad}{0}

\vfill
\z Нормативная база планирования.
 \vfill
\z Структура планов, принципы их разработки.
 \vfill

\vfill

\newpage


\shapk
\bilet{Экзаменационный билет}
\setcounter{zad}{0}

\vfill
\z Основные разделы бизнес-плана.
 \vfill
\z Назначение бизнес-плана – цели, задачи.
 \vfill

\vfill

\newpage


\shapk
\bilet{Экзаменационный билет}
\setcounter{zad}{0}

\vfill
\z Долгосрочный план предприятия.
 \vfill
\z Оперативно-производственное планирование на предприятии.
 \vfill

\vfill

\newpage



\end{document}