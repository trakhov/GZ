\documentclass[
	14pt,
	a4paper,
	]
	{scrartcl}

\usepackage{mysty}

\newcommand{\spec}{100103.65}
\newcommand{\disc}{Региональная\\ \multicolumn{2}{r}{ экономика РФ }}
\newcommand{\kafedra}{экономики}

\usepackage[
	papersize={210mm,99mm},
	top=.5cm,
	left=1.5cm,
	right=1.5cm,
	bottom=.5cm
	]{geometry}

\pagestyle{empty}

\begin{document}


\shapk
\bilet{Экзаменационный билет}
\setcounter{zad}{0}

\vfill
\z Метод и задачи регионалистики.
 \vfill
\z Оценка влияния различных районов округа на его благосостояние. \vfill

\vfill

\newpage


\shapk
\bilet{Экзаменационный билет}
\setcounter{zad}{0}

\vfill
\z Регионализация, понятие о регионе, виды регионов.
 \vfill
\z Влияние соседних округов на развитие СЗФО.
 \vfill

\vfill

\newpage


\shapk
\bilet{Экзаменационный билет}
\setcounter{zad}{0}

\vfill
\z Государственная региональная политика и ее аспекты.
 \vfill
\z Проблемы развития СЗФО.
 \vfill

\vfill

\newpage


\shapk
\bilet{Экзаменационный билет}
\setcounter{zad}{0}

\vfill
\z Единый народохозяйственный комплекс Северо-Западного федерального округа.
 \vfill
\z Перспективы развития СЗФО.
 \vfill

\vfill

\newpage


\shapk
\bilet{Экзаменационный билет}
\setcounter{zad}{0}

\vfill
\z Классификация отраслей хозяйства по назначению продукции и по характеру предметов труда.
 \vfill
\z Особенности в развитии СЗФО от других округов.
 \vfill

\vfill

\newpage


\shapk
\bilet{Экзаменационный билет}
\setcounter{zad}{0}

\vfill
\z Особенности отраслевой и территориальной структуры народного хозяйства Северо-Западного федерального округа.
 \vfill
\z Роль химико-лесного комплекса в экономике России.
 \vfill

\vfill

\newpage


\shapk
\bilet{Экзаменационный билет}
\setcounter{zad}{0}

\vfill
\z Современное состояние управления ЕНХК.
 \vfill
\z Роль машиностроительного комплекса в экономике России. 
 \vfill

\vfill

\newpage


\shapk
\bilet{Экзаменационный билет}
\setcounter{zad}{0}

\vfill
\z Понятие об эффективности размещения промышленных предприятий.
 \vfill
\z Роль строительного комплекса в экономике России.
 \vfill

\vfill

\newpage


\shapk
\bilet{Экзаменационный билет}
\setcounter{zad}{0}

\vfill
\z Формы территориальной организации промышленности.
 \vfill
\z Административно-территориальное деление СЗФО.
 \vfill

\vfill

\newpage


\shapk
\bilet{Экзаменационный билет}
\setcounter{zad}{0}

\vfill
\z Топливно-энергетический комплекс Северо-Западного федерального округа (ТЭК).
 \vfill
\z Отрасли рыночной специализации Северо-Запада.
 \vfill

\vfill

\newpage


\shapk
\bilet{Экзаменационный билет}
\setcounter{zad}{0}

\vfill
\z Значение ТЭК в народном хозяйстве и его связь с другими комплексами.
 \vfill
\z Оценка влияния различных отраслей на благосостояние населения СЗФО.
 \vfill

\vfill

\newpage


\shapk
\bilet{Экзаменационный билет}
\setcounter{zad}{0}

\vfill
\z Значение металлургического комплекса и его роль в народном хозяйстве.
 \vfill
\z Региональные особенности размещения металлургического производства.
 \vfill

\vfill

\newpage


\shapk
\bilet{Экзаменационный билет}
\setcounter{zad}{0}

\vfill
\z Основные фонды, региональные особенности развития металлургического производства.
 \vfill
\z Регион как социально-экономическая система.
 \vfill

\vfill

\newpage


\shapk
\bilet{Экзаменационный билет}
\setcounter{zad}{0}

\vfill
\z Машиностроительный комплекс и его особенности.
 \vfill
\z Роль Северо-Западного федерального округа в социально-экономическом развитии России.
 \vfill

\vfill

\newpage


\shapk
\bilet{Экзаменационный билет}
\setcounter{zad}{0}

\vfill
\z Специфика химико-лесного комплекса и его значение в народном хозяйстве.
 \vfill
\z Основные отечественные и зарубежные инвестиционные потоки в основные отрасли и отдельные крупные предприятия округа.
 \vfill

\vfill

\newpage


\shapk
\bilet{Экзаменационный билет}
\setcounter{zad}{0}

\vfill
\z Строительный комплекс, его состав и структура.
 \vfill
\z Северо-Западный федеральный округ и его роль во внешнеэкономических связях России.
 \vfill

\vfill

\newpage


\shapk
\bilet{Экзаменационный билет}
\setcounter{zad}{0}

\vfill
\z Понятие стройиндустрии и ее особенности.
 \vfill
\z Порты Северо-Западного федерального округа --- проблемы, перспективы и их значение для страны
 \vfill

\vfill

\newpage


\shapk
\bilet{Экзаменационный билет}
\setcounter{zad}{0}

\vfill
\z Легкая промышленность и ее сущность.
 \vfill
\z История развития и освоения данной территории.
 \vfill

\vfill

\newpage


\shapk
\bilet{Экзаменационный билет}
\setcounter{zad}{0}

\vfill
\z Состав агропромышленного комплекса.
 \vfill
\z Характеристика Ненецкого автономного округа (STEP-анализ).
 \vfill

\vfill

\newpage


\shapk
\bilet{Экзаменационный билет}
\setcounter{zad}{0}

\vfill
\z Особенности сельского хозяйства на Северо-Западе.
 \vfill
\z Характеристика Республики Коми (STEP-анализ).
 \vfill

\vfill

\newpage


\shapk
\bilet{Экзаменационный билет}
\setcounter{zad}{0}

\vfill
\z Основные факторы размещения АПК.
 \vfill
\z Характеристика Республики Карелии (STEP-анализ).
 \vfill

\vfill

\newpage


\shapk
\bilet{Экзаменационный билет}
\setcounter{zad}{0}

\vfill
\z Роль транспорта в народном хозяйстве страны.
 \vfill
\z Характеристика Архангельской области (STEP-анализ).
 \vfill

\vfill

\newpage


\shapk
\bilet{Экзаменационный билет}
\setcounter{zad}{0}

\vfill
\z Технико-экономические особенности видов транспорта.
 \vfill
\z Характеристика Мурманской области (STEP-анализ).
 \vfill

\vfill

\newpage


\shapk
\bilet{Экзаменационный билет}
\setcounter{zad}{0}

\vfill
\z Показатели, характеризующие работу транспорта.
 \vfill
\z Характеристика Вологодской области (STEP-анализ).
 \vfill

\vfill

\newpage


\shapk
\bilet{Экзаменационный билет}
\setcounter{zad}{0}

\vfill
\z Уровень развития и показатели обеспеченности транспортом территории.
 \vfill
\z Характеристика Калининградской области (STEP-анализ).
 \vfill

\vfill

\newpage


\shapk
\bilet{Экзаменационный билет}
\setcounter{zad}{0}

\vfill
\z Природно-ресурсный потенциал Северо-Западного федерального округа России.
 \vfill
\z Характеристика Новгородской области (STEP-анализ).
 \vfill

\vfill

\newpage


\shapk
\bilet{Экзаменационный билет}
\setcounter{zad}{0}

\vfill
\z Классификация природных ресурсов.
 \vfill
\z Характеристика Псковской области (STEP-анализ).
 \vfill

\vfill

\newpage


\shapk
\bilet{Экзаменационный билет}
\setcounter{zad}{0}

\vfill
\z Закономерности и принципы размещения производительных сил.
 \vfill
\z Характеристика Ленинградской области (STEP-анализ).
 \vfill

\vfill

\newpage


\shapk
\bilet{Экзаменационный билет}
\setcounter{zad}{0}

\vfill
\z Агломерирование производства и населения.
 \vfill
\z Характеристика города Санкт-Петербург (сравнительный анализ).
 \vfill

\vfill

\newpage


\shapk
\bilet{Экзаменационный билет}
\setcounter{zad}{0}

\vfill
\z Население как составляющая региона (город, поселок).
 \vfill
\z Характеристика Северного региона (сравнительный анализ).
 \vfill

\vfill

\newpage


\shapk
\bilet{Экзаменационный билет}
\setcounter{zad}{0}

\vfill
\z Демографические проблемы.
 \vfill
\z Характеристика Северо-Западного региона (сравнительный анализ).
 \vfill

\vfill

\newpage


\shapk
\bilet{Экзаменационный билет}
\setcounter{zad}{0}

\vfill
\z Характеристика национального состава региона.
 \vfill
\z Региональные особенности размещения населения.
 \vfill

\vfill

\newpage


\shapk
\bilet{Экзаменационный билет}
\setcounter{zad}{0}

\vfill
\z Характеристика профессионального состава региона.
 \vfill
\z Свободные экономические зоны и их характеристика.
 \vfill

\vfill

\newpage


\shapk
\bilet{Экзаменационный билет}
\setcounter{zad}{0}

\vfill
\z Трудовые ресурсы и их размещение.
 \vfill
\z Приграничные территории, проблемы и формы территориального управления.
 \vfill

\vfill

\newpage



\end{document}