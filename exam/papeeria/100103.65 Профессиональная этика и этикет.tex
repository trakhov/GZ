\documentclass[
	14pt,
	a4paper,
	]
	{scrartcl}

\usepackage{mysty}

\newcommand{\spec}{100103.65}
\newcommand{\disc}{\\ \multicolumn{2}{r}{ Профессиональная этика и этикет }}
\newcommand{\kafedra}{ГиСД}

\usepackage[
	papersize={210mm,99mm},
	top=.5cm,
	left=1.5cm,
	right=1.5cm,
	bottom=.5cm
	]{geometry}

\pagestyle{empty}

\begin{document}


\shapk
\bilet{Экзаменационный билет}
\setcounter{zad}{0}

\vfill
\z Этика как наука. Предмет этики.
 \vfill
\z Историческое своеобразие этики \vfill

\vfill

\newpage


\shapk
\bilet{Экзаменационный билет}
\setcounter{zad}{0}

\vfill
\z Основные направления и школы в развитии этики.
 \vfill
\z Этика И. Канта. Сущность морали. Категорический императив
 \vfill

\vfill

\newpage


\shapk
\bilet{Экзаменационный билет}
\setcounter{zad}{0}

\vfill
\z Мораль, её сущность и структура.
 \vfill
\z Практическая мораль Т. Гоббса
 \vfill

\vfill

\newpage


\shapk
\bilet{Экзаменационный билет}
\setcounter{zad}{0}

\vfill
\z Основные функции морали
 \vfill
\z Практическая мораль Ф. Бэкона
 \vfill

\vfill

\newpage


\shapk
\bilet{Экзаменационный билет}
\setcounter{zad}{0}

\vfill
\z Исторические типы нравственности (особенности становления и развития) 
 \vfill
\z Этические воззрения М. Монтеня
 \vfill

\vfill

\newpage


\shapk
\bilet{Экзаменационный билет}
\setcounter{zad}{0}

\vfill
\z Особенности и общие черты античной этики
 \vfill
\z Особенности и своеобразие западноевропейской этики 16-17 вв.
 \vfill

\vfill

\newpage


\shapk
\bilet{Экзаменационный билет}
\setcounter{zad}{0}

\vfill
\z Историческое своеобразие этики Демокрита
 \vfill
\z Этическое учение А. Августина
 \vfill

\vfill

\newpage


\shapk
\bilet{Экзаменационный билет}
\setcounter{zad}{0}

\vfill
\z Этический реметивизм софистов
 \vfill
\z Христианская мораль - как важнейший элемент идеологии и обыденного сознания европейского средневековья
 \vfill

\vfill

\newpage


\shapk
\bilet{Экзаменационный билет}
\setcounter{zad}{0}

\vfill
\z Этическое учение Сократа (моральный абсолютизм)
 \vfill
\z Этика Эпикура. Удовольствие как критерий добродетельной жизни
 \vfill

\vfill

\newpage


\shapk
\bilet{Экзаменационный билет}
\setcounter{zad}{0}

\vfill
\z Связь понятий блага и добродетели с понятиями пользы и счастья. Тезис "добродетель есть знание". Его смысл и значение
 \vfill
\z Этика Аристотеля. Учение о нравственных добродетелях
 \vfill

\vfill

\newpage


\shapk
\bilet{Экзаменационный билет}
\setcounter{zad}{0}

\vfill
\z Философско-этическая концепция Платона
 \vfill
\z Философско-этическая концепция Платона
 \vfill

\vfill

\newpage


\shapk
\bilet{Экзаменационный билет}
\setcounter{zad}{0}

\vfill
\z Этика Аристотеля. Предмет и задачи этики. Проблема высшего блага
 \vfill
\z Этическое учение Сократа (моральный абсолютизм)
 \vfill

\vfill

\newpage


\shapk
\bilet{Экзаменационный билет}
\setcounter{zad}{0}

\vfill
\z Учение о нравственных добродетелях (Аристотель)
 \vfill
\z Этический релятивизм софистов.
 \vfill

\vfill

\newpage


\shapk
\bilet{Экзаменационный билет}
\setcounter{zad}{0}

\vfill
\z Этика Эпикура. Удовольствие как критерий добродетельной жизни
 \vfill
\z Историческое своеобразие этики Демокрита
 \vfill

\vfill

\newpage


\shapk
\bilet{Экзаменационный билет}
\setcounter{zad}{0}

\vfill
\z Римский стоицизм - этическое учение, предшествующее раннему Христианству (Сенека, Эликтет, Марк Аврелий)
 \vfill
\z Особенности и общие черты античной этики
 \vfill

\vfill

\newpage


\shapk
\bilet{Экзаменационный билет}
\setcounter{zad}{0}

\vfill
\z Этические воззрения раннего Христианства
 \vfill
\z Основные функции морали
 \vfill

\vfill

\newpage


\shapk
\bilet{Экзаменационный билет}
\setcounter{zad}{0}

\vfill
\z Христианская мораль - как важнейший элемент идеологии и обыденного сознания европейского средневековья
 \vfill
\z Король, её сущность структура,
 \vfill

\vfill

\newpage


\shapk
\bilet{Экзаменационный билет}
\setcounter{zad}{0}

\vfill
\z Этическое учение Аврелия Августина
 \vfill
\z Этика как наука. Предмет этики
 \vfill

\vfill

\newpage


\shapk
\bilet{Экзаменационный билет}
\setcounter{zad}{0}

\vfill
\z Этическое учение Фомы Аквинского
 \vfill
\z Правила делового этикета.
 \vfill

\vfill

\newpage


\shapk
\bilet{Экзаменационный билет}
\setcounter{zad}{0}

\vfill
\z Особенности и своеобразие западноевропейской этики ХV1-ХVШ вв.
 \vfill
\z Виды этикета. Специфика делового этикета
 \vfill

\vfill

\newpage


\shapk
\bilet{Экзаменационный билет}
\setcounter{zad}{0}

\vfill
\z Этические воззрения М.Монтеля
 \vfill
\z Этикет как составная часть культуры человека и общества
 \vfill

\vfill

\newpage


\shapk
\bilet{Экзаменационный билет}
\setcounter{zad}{0}

\vfill
\z Практическая мораль Ф.Бэкона и Т.Гоббса
 \vfill
\z Соотношение теоретической и прикладной этики. Профессиональная этика
 \vfill

\vfill

\newpage


\shapk
\bilet{Экзаменационный билет}
\setcounter{zad}{0}

\vfill
\z Этика И.Канта. Сущность морали. Категорический императив
 \vfill
\z Морально-значимые качества делового человека
 \vfill

\vfill

\newpage


\shapk
\bilet{Экзаменационный билет}
\setcounter{zad}{0}

\vfill
\z Этические воззрения Г.Гегеля. Мораль и нравственность
 \vfill
\z Сфера делового общения. Основные принципы делового общения
 \vfill

\vfill

\newpage


\shapk
\bilet{Экзаменационный билет}
\setcounter{zad}{0}

\vfill
\z Этические воззрения русских революционных демократов
 \vfill
\z Мораль мотивация, оценка, самооценка
 \vfill

\vfill

\newpage


\shapk
\bilet{Экзаменационный билет}
\setcounter{zad}{0}

\vfill
\z Этика "Всеединства" В.С.Соловьева
 \vfill
\z Нравственная ценность поступка
 \vfill

\vfill

\newpage


\shapk
\bilet{Экзаменационный билет}
\setcounter{zad}{0}

\vfill
\z Этические воззрения Н.А.Бердяева
 \vfill
\z Ненасилие как форма разрешения конфликтов. Методы ненасильственной борьбы
 \vfill

\vfill

\newpage


\shapk
\bilet{Экзаменационный билет}
\setcounter{zad}{0}

\vfill
\z Общая характеристика этических категорий
 \vfill
\z Нравственные конфликты и способы их предотвращения и разрешения
 \vfill

\vfill

\newpage


\shapk
\bilet{Экзаменационный билет}
\setcounter{zad}{0}

\vfill
\z Добро и зло как исходные представления морального сознания
 \vfill
\z Нравственный конфликт. Условия возникновения и развития
 \vfill

\vfill

\newpage


\shapk
\bilet{Экзаменационный билет}
\setcounter{zad}{0}

\vfill
\z Смысл жизни - как высшая нравственная ценность
 \vfill
\z Проблема морального выбора
 \vfill

\vfill

\newpage



\end{document}