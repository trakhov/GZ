\documentclass[
	14pt,
	a4paper,
	]
	{scrartcl}

\usepackage{mysty}

\newcommand{\spec}{100101.65}
\newcommand{\disc}{Химия \\}
\newcommand{\kafedra}{МиЕНД}

\usepackage[
	papersize={210mm,99mm},
	top=.5cm,
	left=1.5cm,
	right=1.5cm,
	bottom=.5cm
	]{geometry}

\pagestyle{empty}

\begin{document}


\shapk
\bilet{Экзаменационный билет}
\setcounter{zad}{0}

\vfill
\z Квантово - механическая модель атома. Квантовые числа.
 \vfill
\z Химический, физико-химический анализ индивидуальных веществ и смесей. \vfill

\vfill

\newpage


\shapk
\bilet{Экзаменационный билет}
\setcounter{zad}{0}

\vfill
\z Квантово - механическая модель атома.
 \vfill
\z Поликонденсационные полимеры – синтетические волокна нейлон и капрон.
 \vfill

\vfill

\newpage


\shapk
\bilet{Экзаменационный билет}
\setcounter{zad}{0}

\vfill
\z Принцип запрета Паули. Правила Гунда.
 \vfill
\z Полимеры и пластмассы.
 \vfill

\vfill

\newpage


\shapk
\bilet{Экзаменационный билет}
\setcounter{zad}{0}

\vfill
\z Правила Клечковского.
 \vfill
\z Основные свойства олигомеров и полимеров. Полимеризация.
 \vfill

\vfill

\newpage


\shapk
\bilet{Экзаменационный билет}
\setcounter{zad}{0}

\vfill
\z Строение атома и химические свойства элементов.
 \vfill
\z Общая характеристика металлов подгруппы железа периодической системы.
 \vfill

\vfill

\newpage


\shapk
\bilet{Экзаменационный билет}
\setcounter{zad}{0}

\vfill
\z Периодический закон и периодические свойства элементов.
 \vfill
\z Общая характеристика металлов: физические и химические свойства.
 \vfill

\vfill

\newpage


\shapk
\bilet{Экзаменационный билет}
\setcounter{zad}{0}

\vfill
\z Типы химической связи. Ковалентная связь. Ионная связь.
 \vfill
\z Методы защиты металлов от коррозии.
 \vfill

\vfill

\newpage


\shapk
\bilet{Экзаменационный билет}
\setcounter{zad}{0}

\vfill
\z Типы химической связи. Донорно-акцепторная связь.
 \vfill
\z Коррозия металлов.
 \vfill

\vfill

\newpage


\shapk
\bilet{Экзаменационный билет}
\setcounter{zad}{0}

\vfill
\z Типы химической связи. Водородная связь.
 \vfill
\z Законы электролиза.
 \vfill

\vfill

\newpage


\shapk
\bilet{Экзаменационный билет}
\setcounter{zad}{0}

\vfill
\z Типы химической связи. Металлическая связь.
 \vfill
\z Электролиз в расплавах.
 \vfill

\vfill

\newpage


\shapk
\bilet{Экзаменационный билет}
\setcounter{zad}{0}

\vfill
\z Типы химической связи.   и ? - связь.
 \vfill
\z Электролиз в растворах.
 \vfill

\vfill

\newpage


\shapk
\bilet{Экзаменационный билет}
\setcounter{zad}{0}

\vfill
\z Гибридизация атомных орбиталей.
 \vfill
\z Химические источники тока. Топливный элемент.
 \vfill

\vfill

\newpage


\shapk
\bilet{Экзаменационный билет}
\setcounter{zad}{0}

\vfill
\z Метод валентных связей.
 \vfill
\z Химические источники тока. Аккумуляторы.
 \vfill

\vfill

\newpage


\shapk
\bilet{Экзаменационный билет}
\setcounter{zad}{0}

\vfill
\z Метод молекулярных орбиталей.
 \vfill
\z Концентрированные химические источники тока.
 \vfill

\vfill

\newpage


\shapk
\bilet{Экзаменационный билет}
\setcounter{zad}{0}

\vfill
\z Полярные и неполярные молекулы.
 \vfill
\z Гальванические элементы - химические источники тока.
 \vfill

\vfill

\newpage


\shapk
\bilet{Экзаменационный билет}
\setcounter{zad}{0}

\vfill
\z Межмолекулярное взаимодействие и физические свойства.
 \vfill
\z Стандартный водородный электрод и водородная шкала потенциалов уравнения неравенства.
 \vfill

\vfill

\newpage


\shapk
\bilet{Экзаменационный билет}
\setcounter{zad}{0}

\vfill
\z Типы кристаллических решеток и физико-химические свойства твердых тел.
 \vfill
\z Электродные потенциалы металлов.
 \vfill

\vfill

\newpage


\shapk
\bilet{Экзаменационный билет}
\setcounter{zad}{0}

\vfill
\z Химическая связь в твердых телах.
 \vfill
\z Окислительно-восстановительные свойства веществ.
 \vfill

\vfill

\newpage


\shapk
\bilet{Экзаменационный билет}
\setcounter{zad}{0}

\vfill
\z Растворы. Растворимость.
 \vfill
\z Валентность. Степень  окисления.
 \vfill

\vfill

\newpage


\shapk
\bilet{Экзаменационный билет}
\setcounter{zad}{0}

\vfill
\z Диссоциация слабых электролитов. Константы диссоциации.
 \vfill
\z Энергия Гиббса. Условия самопроизвольного протекания химических реакций.
 \vfill

\vfill

\newpage


\shapk
\bilet{Экзаменационный билет}
\setcounter{zad}{0}

\vfill
\z Растворы. Общие свойства растворов.
 \vfill
\z Энтальпия. Энтропия.
 \vfill

\vfill

\newpage


\shapk
\bilet{Экзаменационный билет}
\setcounter{zad}{0}

\vfill
\z Растворы. Способы выражения концентрации растворов.
 \vfill
\z Тепловые эффекты химических реакций. Термохимические законы.
 \vfill

\vfill

\newpage


\shapk
\bilet{Экзаменационный билет}
\setcounter{zad}{0}

\vfill
\z Растворы неэлектролитов. Давление пара раствора.
 \vfill
\z Гетерогенные дисперсные системы.
 \vfill

\vfill

\newpage


\shapk
\bilet{Экзаменационный билет}
\setcounter{zad}{0}

\vfill
\z Растворы неэлектролитов. Повышение и понижение температуры кипения и замерзания растворов.
 \vfill
\z Химическое равновесие. Влияние давления.
 \vfill

\vfill

\newpage


\shapk
\bilet{Экзаменационный билет}
\setcounter{zad}{0}

\vfill
\z Растворы электролитов, их свойства. Ионные реакции.
 \vfill
\z Химическое равновесие. Влияние температуры.
 \vfill

\vfill

\newpage


\shapk
\bilet{Экзаменационный билет}
\setcounter{zad}{0}

\vfill
\z Произведение растворимости.
 \vfill
\z Химическое равновесие. Влияние концентрации.
 \vfill

\vfill

\newpage


\shapk
\bilet{Экзаменационный билет}
\setcounter{zad}{0}

\vfill
\z Электролитическая диссоциация воды. Водородный показатель.
 \vfill
\z Химическое равновесие. Смещение равновесия. Принцип Ле-Шателье.
 \vfill

\vfill

\newpage


\shapk
\bilet{Экзаменационный билет}
\setcounter{zad}{0}

\vfill
\z Гидролиз солей.
 \vfill
\z Скорость химической реакции. Влияние катализатора.
 \vfill

\vfill

\newpage


\shapk
\bilet{Экзаменационный билет}
\setcounter{zad}{0}

\vfill
\z Скорость химической реакции и факторы, влияющие на нее.
 \vfill
\z Скорость химической реакции. Влияние температуры.
 \vfill

\vfill

\newpage



\end{document}