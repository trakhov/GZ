\documentclass[
	14pt,
	a4paper,
	]
	{scrartcl}

\usepackage{mysty}

\newcommand{\spec}{100103.65}
\newcommand{\disc}{Инновации\\ \multicolumn{2}{r}{ в СКСиТ }}
\newcommand{\kafedra}{сервиса}

\usepackage[
	papersize={210mm,99mm},
	top=.5cm,
	left=1.5cm,
	right=1.5cm,
	bottom=.5cm
	]{geometry}

\pagestyle{empty}

\begin{document}


\shapk
\bilet{Экзаменационный билет}
\setcounter{zad}{0}

\vfill
\z Содержание понятий «инновация», «инновационный менеджмент», «инновационный маркетинг».
 \vfill
\z Организационные формы трансфера технологий. \vfill

\vfill

\newpage


\shapk
\bilet{Экзаменационный билет}
\setcounter{zad}{0}

\vfill
\z Становление концепции инновационного менеджмента.
 \vfill
\z Понятие «инновативной» внутрифирменной культуры.
 \vfill

\vfill

\newpage


\shapk
\bilet{Экзаменационный билет}
\setcounter{zad}{0}

\vfill
\z Приоритеты в управлении и тенденции развития.
 \vfill
\z Нейтрализация внешних факторов сопротивления инновационным процессам на фирме.
 \vfill

\vfill

\newpage


\shapk
\bilet{Экзаменационный билет}
\setcounter{zad}{0}

\vfill
\z Инновационное развитие фирмы – основа повышения эффективности ее деятельности.
 \vfill
\z Внутренние факторы сопротивления инновациям.
 \vfill

\vfill

\newpage


\shapk
\bilet{Экзаменационный билет}
\setcounter{zad}{0}

\vfill
\z Основные направления инновационного развития организации.
 \vfill
\z Способы стимулирования инновативности руководства высшего и среднего звеньев управления.
 \vfill

\vfill

\newpage


\shapk
\bilet{Экзаменационный билет}
\setcounter{zad}{0}

\vfill
\z Механизм управления инновациями.
 \vfill
\z Современное понятие эффективного руководителя – новатора. Инновативность компании и высшее звено руководства.
 \vfill

\vfill

\newpage


\shapk
\bilet{Экзаменационный билет}
\setcounter{zad}{0}

\vfill
\z Классификация инноваций по степени рыночной новизны.
 \vfill
\z Основные формы научно-технической кооперации как метод саморегулирования инновационной деятельности.
 \vfill

\vfill

\newpage


\shapk
\bilet{Экзаменационный билет}
\setcounter{zad}{0}

\vfill
\z Классификация инноваций по объекту проведения.
 \vfill
\z Саморегулирование межфирменных отношений в области инноваций.
 \vfill

\vfill

\newpage


\shapk
\bilet{Экзаменационный билет}
\setcounter{zad}{0}

\vfill
\z Классификация инноваций по причинам проведения.
 \vfill
\z Законодательное регулирование инновационной деятельности фирм.
 \vfill

\vfill

\newpage


\shapk
\bilet{Экзаменационный билет}
\setcounter{zad}{0}

\vfill
\z Внешние и внутренние предпосылки инновационной деятельности туристических компаний и предприятий социально-культурного сервиса.
 \vfill
\z Финансовые методы государственного регулирования инновационной деятельности фирм. Виды финансовых льгот.
 \vfill

\vfill

\newpage


\shapk
\bilet{Экзаменационный билет}
\setcounter{zad}{0}

\vfill
\z Особенности инновационной деятельности фирмы (туризм, социально-культурный сервис).
 \vfill
\z Необходимость государственного регулирования инновационной деятельности фирм.
 \vfill

\vfill

\newpage


\shapk
\bilet{Экзаменационный билет}
\setcounter{zad}{0}

\vfill
\z Инновации как фактор повышения конкурентоспособности фирм на рынке туризма и социально-культурного сервиса.
 \vfill
\z Количественные методы оценки инновационных проектов.
 \vfill

\vfill

\newpage


\shapk
\bilet{Экзаменационный билет}
\setcounter{zad}{0}

\vfill
\z Понятие нового товара в инновационном маркетинге.
 \vfill
\z Качественный и количественный подход к оценке эффективности инновационной деятельности.
 \vfill

\vfill

\newpage


\shapk
\bilet{Экзаменационный билет}
\setcounter{zad}{0}

\vfill
\z Новизна как важнейшее конкурентное преимущество.
 \vfill
\z Основные пути снижения риска в инновационной деятельности.
 \vfill

\vfill

\newpage


\shapk
\bilet{Экзаменационный билет}
\setcounter{zad}{0}

\vfill
\z Технологии управления жизненным циклом товара.
 \vfill
\z Понятие эффективности инноваций.
 \vfill

\vfill

\newpage


\shapk
\bilet{Экзаменационный билет}
\setcounter{zad}{0}

\vfill
\z Понятие и методы репозиционирования в инновационном маркетинге.
 \vfill
\z Причины неудач нового продукта (услуги).
 \vfill

\vfill

\newpage


\shapk
\bilet{Экзаменационный билет}
\setcounter{zad}{0}

\vfill
\z Характеристика понятий «ребрендинг», «кастомизация».
 \vfill
\z Технологические инновации в социально-культурном сервисе и туризме.
 \vfill

\vfill

\newpage


\shapk
\bilet{Экзаменационный билет}
\setcounter{zad}{0}

\vfill
\z Маркетинг нового товара. Этапы создания нового продукта.
 \vfill
\z Сопротивление инновациям и методы его нейтрализации.
 \vfill

\vfill

\newpage


\shapk
\bilet{Экзаменационный билет}
\setcounter{zad}{0}

\vfill
\z Разработка стратегии нового товара.
 \vfill
\z Роль руководителя в процессе инновационной деятельности.
 \vfill

\vfill

\newpage


\shapk
\bilet{Экзаменационный билет}
\setcounter{zad}{0}

\vfill
\z Виды и методы тестирования нового товара.
 \vfill
\z Адаптация внутрифирменной культуры к требованиям инновационного менеджмента.
 \vfill

\vfill

\newpage


\shapk
\bilet{Экзаменационный билет}
\setcounter{zad}{0}

\vfill
\z Определение места нового товара в ассортиментной матрице.
 \vfill
\z Понятие инновативной внутрифирменной культуры.
 \vfill

\vfill

\newpage


\shapk
\bilet{Экзаменационный билет}
\setcounter{zad}{0}

\vfill
\z Планирование инновационной деятельности компании.
 \vfill
\z Инновационные методы управления персоналом. Методы стимулирования инновационной активности персонала.
 \vfill

\vfill

\newpage


\shapk
\bilet{Экзаменационный билет}
\setcounter{zad}{0}

\vfill
\z Роль стратегического планирования в инновационной деятельности фирм.
 \vfill
\z Понятие эффективности инноваций. Методы оценки инновационной деятельности.
 \vfill

\vfill

\newpage


\shapk
\bilet{Экзаменационный билет}
\setcounter{zad}{0}

\vfill
\z Этапы принятия стратегических решений.
 \vfill
\z Венчурные подразделения в организационной структуре фирмы.
 \vfill

\vfill

\newpage


\shapk
\bilet{Экзаменационный билет}
\setcounter{zad}{0}

\vfill
\z Классификация инновационных стратегий.
 \vfill
\z Матричная структура инновационной деятельности.
 \vfill

\vfill

\newpage


\shapk
\bilet{Экзаменационный билет}
\setcounter{zad}{0}

\vfill
\z Реализация стратегии «технологического лидерства».
 \vfill
\z Внутренние источники финансирования инновационных проектов.
 \vfill

\vfill

\newpage


\shapk
\bilet{Экзаменационный билет}
\setcounter{zad}{0}

\vfill
\z Имитационные инновационные стратегии.
 \vfill
\z Внешние источники финансирования инновационных проектов.
 \vfill

\vfill

\newpage


\shapk
\bilet{Экзаменационный билет}
\setcounter{zad}{0}

\vfill
\z Процесс взаимодействия инноваций с внешней средой.
 \vfill
\z Финансирование инновационной деятельности: проблемы и методы.
 \vfill

\vfill

\newpage


\shapk
\bilet{Экзаменационный билет}
\setcounter{zad}{0}

\vfill
\z Бизнес-планирование инновационных проектов: понятие, функции, методология.
 \vfill
\z Формы организации инновационной деятельности.
 \vfill

\vfill

\newpage


\shapk
\bilet{Экзаменационный билет}
\setcounter{zad}{0}

\vfill
\z Краткая характеристика этапов бизнес-плана инновационного проекта.
 \vfill
\z Современные особенности организации инновационной деятельности туристических фирм и предприятий социально-культурного сервиса.
 \vfill

\vfill

\newpage



\end{document}