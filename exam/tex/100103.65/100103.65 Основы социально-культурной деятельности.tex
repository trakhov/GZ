\documentclass[
	14pt,
	a4paper,
	]
	{scrartcl}

\usepackage{mysty}

\newcommand{\spec}{100103.65}
\newcommand{\disc}{Основы\\ \multicolumn{2}{r}{ социально-культурной деятельности }}
\newcommand{\kafedra}{сервиса}

\usepackage[
	papersize={210mm,99mm},
	top=.5cm,
	left=1.5cm,
	right=1.5cm,
	bottom=.5cm
	]{geometry}

\pagestyle{empty}

\begin{document}


\shapk
\bilet{Экзаменационный билет}
\setcounter{zad}{0}

\vfill
\z Понятие социально-культурной деятельности.
 \vfill
\z Материально-технический ресурс. \vfill

\vfill

\newpage


\shapk
\bilet{Экзаменационный билет}
\setcounter{zad}{0}

\vfill
\z Стихийная и специально организованная социально-культурная деятельность.
 \vfill
\z Самофинансирование и фандрайзинг.
 \vfill

\vfill

\newpage


\shapk
\bilet{Экзаменационный билет}
\setcounter{zad}{0}

\vfill
\z Методология и методы исследования социально-культурной деятельности.
 \vfill
\z Кадровый и финансовый ресурс.
 \vfill

\vfill

\newpage


\shapk
\bilet{Экзаменационный билет}
\setcounter{zad}{0}

\vfill
\z Понятие христианства как государственной религии.
 \vfill
\z Понятие ресурсной базы.
 \vfill

\vfill

\newpage


\shapk
\bilet{Экзаменационный билет}
\setcounter{zad}{0}

\vfill
\z Новая историческая общность – Святая Русь.
 \vfill
\z Сущность и классификация социально-культурных технологий.
 \vfill

\vfill

\newpage


\shapk
\bilet{Экзаменационный билет}
\setcounter{zad}{0}

\vfill
\z Наука и религия. Православие и искусство.
 \vfill
\z Технологические основы социально-культурной деятельности.
 \vfill

\vfill

\newpage


\shapk
\bilet{Экзаменационный билет}
\setcounter{zad}{0}

\vfill
\z Социально-экономические преобразования петровского времени.
 \vfill
\z Культура и рынок.
 \vfill

\vfill

\newpage


\shapk
\bilet{Экзаменационный билет}
\setcounter{zad}{0}

\vfill
\z Новая историческая общность – европеизированные россияне.
 \vfill
\z Государство как ведущий субъект культурной политики.
 \vfill

\vfill

\newpage


\shapk
\bilet{Экзаменационный билет}
\setcounter{zad}{0}

\vfill
\z Досуг как государственная политика.
 \vfill
\z Уровни культурной политики.
 \vfill

\vfill

\newpage


\shapk
\bilet{Экзаменационный билет}
\setcounter{zad}{0}

\vfill
\z Петровский мир смеховой культуры.
 \vfill
\z Понятие культурной политики.
 \vfill

\vfill

\newpage


\shapk
\bilet{Экзаменационный билет}
\setcounter{zad}{0}

\vfill
\z Идеология просвещения – теоретическая база просветительной деятельности в России.
 \vfill
\z Социально-культурные институты гражданского общества.
 \vfill

\vfill

\newpage


\shapk
\bilet{Экзаменационный билет}
\setcounter{zad}{0}

\vfill
\z Новый социальный слой – первые граждане России. Кодекс нравственности дворянина.
 \vfill
\z Гражданское общество как совокупность независимых от государства само организованных институтов.
 \vfill

\vfill

\newpage


\shapk
\bilet{Экзаменационный билет}
\setcounter{zad}{0}

\vfill
\z Социально-культурные институты гражданского общества
 \vfill
\z Понятие праздника. Празднично-обрядовый календарь.
 \vfill

\vfill

\newpage


\shapk
\bilet{Экзаменационный билет}
\setcounter{zad}{0}

\vfill
\z Роль масонства в становлении гражданского общества.
 \vfill
\z Стихийно сложившиеся и специально организованные формы социально-культурной деятельности.
 \vfill

\vfill

\newpage


\shapk
\bilet{Экзаменационный билет}
\setcounter{zad}{0}

\vfill
\z Видные просветители екатерининской эпохи.
 \vfill
\z Средства социально-культурной деятельности.
 \vfill

\vfill

\newpage


\shapk
\bilet{Экзаменационный билет}
\setcounter{zad}{0}

\vfill
\z Тайные общества в России в 1812-1825 годах.
 \vfill
\z Методы социально-культурной деятельности.
 \vfill

\vfill

\newpage


\shapk
\bilet{Экзаменационный билет}
\setcounter{zad}{0}

\vfill
\z Программа нравственного самовоспитания декабристов.
 \vfill
\z Социальные институты.
 \vfill

\vfill

\newpage


\shapk
\bilet{Экзаменационный билет}
\setcounter{zad}{0}

\vfill
\z Просветительская деятельность декабристов в Сибири.
 \vfill
\z Музеи, парки культуры и отдыха.
 \vfill

\vfill

\newpage


\shapk
\bilet{Экзаменационный билет}
\setcounter{zad}{0}

\vfill
\z Нравственный подвиг жен декабристов.
 \vfill
\z Учреждения культуры клубного типа.
 \vfill

\vfill

\newpage


\shapk
\bilet{Экзаменационный билет}
\setcounter{zad}{0}

\vfill
\z Социально-культурная деятельность западников и славянофилов.
 \vfill
\z Библиотеки и модернизация библиотечного дела.
 \vfill

\vfill

\newpage


\shapk
\bilet{Экзаменационный билет}
\setcounter{zad}{0}

\vfill
\z Социально-культурный утопизм Н.Г. Чернышевского
 \vfill
\z Объекты национального наследия.
 \vfill

\vfill

\newpage


\shapk
\bilet{Экзаменационный билет}
\setcounter{zad}{0}

\vfill
\z Интеллигенция и народ.
 \vfill
\z Понятие культурной жизни.
 \vfill

\vfill

\newpage


\shapk
\bilet{Экзаменационный билет}
\setcounter{zad}{0}

\vfill
\z "Серебряный век" как социально-культурный феномен.
 \vfill
\z Понятие "субъект" и "объект" социально-культурной деятельности.
 \vfill

\vfill

\newpage


\shapk
\bilet{Экзаменационный билет}
\setcounter{zad}{0}

\vfill
\z Новое религиозное сознание. Религиозно-философские собрания.
 \vfill
\z Понятие свободного времени. Досуг как качество жизни.
 \vfill

\vfill

\newpage


\shapk
\bilet{Экзаменационный билет}
\setcounter{zad}{0}

\vfill
\z Творческие лаборатории петербургской богемы.
 \vfill
\z Перестройка. Радикальные реформы в экономике, открытое общество.
 \vfill

\vfill

\newpage


\shapk
\bilet{Экзаменационный билет}
\setcounter{zad}{0}

\vfill
\z Художественное объединение "Мир искусства".
 \vfill
\z Движение интеллигенции по перестройке мышления народа. Клубы "Перестройка", "Демократическая перестройка".
 \vfill

\vfill

\newpage


\shapk
\bilet{Экзаменационный билет}
\setcounter{zad}{0}

\vfill
\z Возникновение российской социал-демократии. Подпольные кружки. Г.В. Плеханов.
 \vfill
\z Диссидентское движение. Создание независимой периодики – самиздата.
 \vfill

\vfill

\newpage


\shapk
\bilet{Экзаменационный билет}
\setcounter{zad}{0}

\vfill
\z Русский вариант марксизма в теоретических произведениях В.И. Ленина.
 \vfill
\z Формирование новой исторической общности советского народа.
 \vfill

\vfill

\newpage


\shapk
\bilet{Экзаменационный билет}
\setcounter{zad}{0}

\vfill
\z Социально-культурная деятельность большевиков.
 \vfill
\z "Основы политпросвет работы" Н.К. Крупской.
 \vfill

\vfill

\newpage


\shapk
\bilet{Экзаменационный билет}
\setcounter{zad}{0}

\vfill
\z Понятие и сущность тоталитарного режима.
 \vfill
\z Теоретик социально-культурной деятельности А.В. Луначарский.
 \vfill

\vfill

\newpage



\end{document}