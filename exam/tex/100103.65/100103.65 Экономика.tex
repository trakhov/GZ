\documentclass[
	14pt,
	a4paper,
	]
	{scrartcl}

\usepackage{mysty}

\newcommand{\spec}{100103.65}
\newcommand{\disc}{Экономика \\}
\newcommand{\kafedra}{ЭиМ}

\usepackage[
	papersize={210mm,99mm},
	top=.5cm,
	left=1.5cm,
	right=1.5cm,
	bottom=.5cm
	]{geometry}

\pagestyle{empty}

\begin{document}


\shapk
\bilet{Экзаменационный билет}
\setcounter{zad}{0}

\vfill
\z 	Предмет и метод экономической теории.
 \vfill
\z 	Особенности современной валютной системы. \vfill

\vfill

\newpage


\shapk
\bilet{Экзаменационный билет}
\setcounter{zad}{0}

\vfill
\z 	Основные направления экономической мысли. Основные экономические школы.
 \vfill
\z 	Платежный баланс.
 \vfill

\vfill

\newpage


\shapk
\bilet{Экзаменационный билет}
\setcounter{zad}{0}

\vfill
\z 	Современное содержание парадигмы экономического и социального развития. Экономическая теория и экономическая политика.
 \vfill
\z 	Формирование курса валют.
 \vfill

\vfill

\newpage


\shapk
\bilet{Экзаменационный билет}
\setcounter{zad}{0}

\vfill
\z 	Понятие экономической системы.  
 \vfill
\z 	Валютная система мирового хозяйства.
 \vfill

\vfill

\newpage


\shapk
\bilet{Экзаменационный билет}
\setcounter{zad}{0}

\vfill
\z 	Экономическая система в условиях полной занятости ресурсов.
 \vfill
\z 	Глобализация экономики.
 \vfill

\vfill

\newpage


\shapk
\bilet{Экзаменационный билет}
\setcounter{zad}{0}

\vfill
\z 	Эффективное использование ресурсов.  Граница производственных возможностей.
 \vfill
\z 	Транснационализация экономики.
 \vfill

\vfill

\newpage


\shapk
\bilet{Экзаменационный билет}
\setcounter{zad}{0}

\vfill
\z 	Закон возвышающихся вмененных (альтернативных) издержек.
 \vfill
\z 	Международное разделение труда и теория сравнительных преимуществ.
 \vfill

\vfill

\newpage


\shapk
\bilet{Экзаменационный билет}
\setcounter{zad}{0}

\vfill
\z 	Основные типы экономических систем. 
 \vfill
\z 	Структурная перестройка экономики.
 \vfill

\vfill

\newpage


\shapk
\bilet{Экзаменационный билет}
\setcounter{zad}{0}

\vfill
\z 	Натуральное и товарное производство. Роль частной собственности.
 \vfill
\z 	Приватизация и разгосударствление общественной собственности.
 \vfill

\vfill

\newpage


\shapk
\bilet{Экзаменационный билет}
\setcounter{zad}{0}

\vfill
\z 	Конкуренция совершенная и несовершенная.
 \vfill
\z 	Либерализация цен и реформа ценообразования.
 \vfill

\vfill

\newpage


\shapk
\bilet{Экзаменационный билет}
\setcounter{zad}{0}

\vfill
\z 	Товар и его свойства
 \vfill
\z 	Переходная экономика как объект экономической теории.
 \vfill

\vfill

\newpage


\shapk
\bilet{Экзаменационный билет}
\setcounter{zad}{0}

\vfill
\z 	Деньги: происхождение, сущность и функции.
 \vfill
\z 	Аргументы в пользу равенства и неравенства доходов.
 \vfill

\vfill

\newpage


\shapk
\bilet{Экзаменационный билет}
\setcounter{zad}{0}

\vfill
\z 	Рынок : его функции, условия существования.
 \vfill
\z 	Дифференциация и нивелировка доходов.  Кривая Лоренца.
 \vfill

\vfill

\newpage


\shapk
\bilet{Экзаменационный билет}
\setcounter{zad}{0}

\vfill
\z 	Субъекты рыночной экономики.  Кругооборот благ и доходов.
 \vfill
\z 	Доходы населения и источники их  формирования.
 \vfill

\vfill

\newpage


\shapk
\bilet{Экзаменационный билет}
\setcounter{zad}{0}

\vfill
\z 	Рыночный механизм: решение основных экономических проблем.
 \vfill
\z 	Социальная политика в условиях рынка.
 \vfill

\vfill

\newpage


\shapk
\bilet{Экзаменационный билет}
\setcounter{zad}{0}

\vfill
\z 	Оценка роли рынка: альтернативные взгляды
 \vfill
\z 	Проблемы фискальной политики в России.
 \vfill

\vfill

\newpage


\shapk
\bilet{Экзаменационный билет}
\setcounter{zad}{0}

\vfill
\z 	Рыночный спрос и его факторы.  Закон спроса. Детерминанты спроса.
 \vfill
\z 	Кривая Лоренца.
 \vfill

\vfill

\newpage


\shapk
\bilet{Экзаменационный билет}
\setcounter{zad}{0}

\vfill
\z 	Рыночное предложение и его факторы.  Закон предложения. Детерминанты предложения.
 \vfill
\z 	Оценка эффективности фискальной политики.
 \vfill

\vfill

\newpage


\shapk
\bilet{Экзаменационный билет}
\setcounter{zad}{0}

\vfill
\z 	Рыночное равновесие и равновесная цена.
 \vfill
\z 	Пропорциональный налог, прямые и косвенные налоги, чистые налоги.
 \vfill

\vfill

\newpage


\shapk
\bilet{Экзаменационный билет}
\setcounter{zad}{0}

\vfill
\z 	Эластичность спроса и предложения.
 \vfill
\z 	Дискреционная и недискреционная фискальная политика.
 \vfill

\vfill

\newpage


\shapk
\bilet{Экзаменационный билет}
\setcounter{zad}{0}

\vfill
\z 	Кривые безразличия. Свойства кривых безразличия. Карта кривых безразличия..
 \vfill
\z 	Государственный долг. Причины возникновения, последствия.
 \vfill

\vfill

\newpage


\shapk
\bilet{Экзаменационный билет}
\setcounter{zad}{0}

\vfill
\z 	Выбор потребителя. Бюджетное ограничение потребителя.
 \vfill
\z 	Влияние кредитно-денежной политики на равновесный объем производства.	Финансы. Государственный бюджет.
 \vfill

\vfill

\newpage


\shapk
\bilet{Экзаменационный билет}
\setcounter{zad}{0}

\vfill
\z 	Производственная функция. Изокванта.
 \vfill
\z 	Политика дешевых и дорогих денег.
 \vfill

\vfill

\newpage


\shapk
\bilet{Экзаменационный билет}
\setcounter{zad}{0}

\vfill
\z 	Выбор производителя
 \vfill
\z 	Цели и инструменты кредитно-денежной политики.
 \vfill

\vfill

\newpage


\shapk
\bilet{Экзаменационный билет}
\setcounter{zad}{0}

\vfill
\z 	Виды издержек производства.
 \vfill
\z 	Механизм функционирования коммерческого банка.
 \vfill

\vfill

\newpage


\shapk
\bilet{Экзаменационный билет}
\setcounter{zad}{0}

\vfill
\z 	Закон возрастающих издержек.  Издержки в краткосрочном и долгосрочном периодах.
 \vfill
\z 	Центральный банк и его задачи.
 \vfill

\vfill

\newpage


\shapk
\bilet{Экзаменационный билет}
\setcounter{zad}{0}

\vfill
\z 	Виды прибыли.
 \vfill
\z 	Банковская система как организатор денежного рынка
 \vfill

\vfill

\newpage


\shapk
\bilet{Экзаменационный билет}
\setcounter{zad}{0}

\vfill
\z 	Равновесие конкурентного производителя в краткосрочном периоде.
 \vfill
\z 	Денежный рынок и его равновесие.
 \vfill

\vfill

\newpage


\shapk
\bilet{Экзаменационный билет}
\setcounter{zad}{0}

\vfill
\z 	Кривая предложения в условиях совершенной конкуренции.
 \vfill
\z 	Антициклическое регулирование. Стабилизационная политика.
 \vfill

\vfill

\newpage


\shapk
\bilet{Экзаменационный билет}
\setcounter{zad}{0}

\vfill
\z 	Равновесие конкурентного производителя в долгосрочном периоде.
 \vfill
\z 	Прогнозирование, программирование, индикативное планирование.
 \vfill

\vfill

\newpage


\shapk
\bilet{Экзаменационный билет}
\setcounter{zad}{0}

\vfill
\z 	Монополия в условиях равновесия.
 \vfill
\z 	Экономическая роль государства. Экономические функции государства.
 \vfill

\vfill

\newpage


\shapk
\bilet{Экзаменационный билет}
\setcounter{zad}{0}

\vfill
\z 	Принципы монополистического ценообразования.  Диверсификация  (дискриминация) цен.
 \vfill
\z 	Антиинфляционные меры.
 \vfill

\vfill

\newpage


\shapk
\bilet{Экзаменационный билет}
\setcounter{zad}{0}

\vfill
\z 	Монополия и ее экономические последствия.  Регулируемая монополия.
 \vfill
\z 	Инфляция и безработица.  Кривая Филипса. Проблема стагфляции.
 \vfill

\vfill

\newpage


\shapk
\bilet{Экзаменационный билет}
\setcounter{zad}{0}

\vfill
\z 	Монополистическая конкуренция.  Неценовая конкуренция.
 \vfill
\z 	Инфляция спроса и инфляция издержек.
 \vfill

\vfill

\newpage


\shapk
\bilet{Экзаменационный билет}
\setcounter{zad}{0}

\vfill
\z 	Виды олигополии.
 \vfill
\z 	Понятие инфляции. Виды инфляции.
 \vfill

\vfill

\newpage


\shapk
\bilet{Экзаменационный билет}
\setcounter{zad}{0}

\vfill
\z 	Особенности рынка рабочей силы.  Теория предельной производительности.
 \vfill
\z 	Теория занятости в классической и кейнсианской трактовке. 
 \vfill

\vfill

\newpage


\shapk
\bilet{Экзаменационный билет}
\setcounter{zad}{0}

\vfill
\z 	Спрос на рабочую силу в условиях совершенной и несовершенной конкуренции. ценовая эластичность спроса на рабочую силу.
 \vfill
\z 	Безработица, ее типы. Фактический и естественный уровни. Полная занятость.
 \vfill

\vfill

\newpage


\shapk
\bilet{Экзаменационный билет}
\setcounter{zad}{0}

\vfill
\z 	Заработная плата. Компромисс между трудом и досугом.
 \vfill
\z 	Антициклическое регулирование.
 \vfill

\vfill

\newpage


\shapk
\bilet{Экзаменационный билет}
\setcounter{zad}{0}

\vfill
\z 	Монопсония на рынке труда.
 \vfill
\z 	Экономический кризис в России в 90-х годах ХХ века.
 \vfill

\vfill

\newpage


\shapk
\bilet{Экзаменационный билет}
\setcounter{zad}{0}

\vfill
\z 	Роль профсоюзов. Минимальная заработная плата.  Двусторонняя монополия.
 \vfill
\z 	Понятие структурного кризиса.
 \vfill

\vfill

\newpage


\shapk
\bilet{Экзаменационный билет}
\setcounter{zad}{0}

\vfill
\z 	Проблемы инвестиций в человеческий капитал.
 \vfill
\z 	Модификация экономического цикла в послевоенный период.
 \vfill

\vfill

\newpage


\shapk
\bilet{Экзаменационный билет}
\setcounter{zad}{0}

\vfill
\z 	 Рынок капитала.
 \vfill
\z 	Теория экономического цикла. Производственный цикл в условиях индустриального общества. Фазы цикла.
 \vfill

\vfill

\newpage


\shapk
\bilet{Экзаменационный билет}
\setcounter{zad}{0}

\vfill
\z 	Кругооборот и оборот предпринимательского капитала.
 \vfill
\z 	Теории адаптивных и рациональных ожиданий. 
 \vfill

\vfill

\newpage


\shapk
\bilet{Экзаменационный билет}
\setcounter{zad}{0}

\vfill
\z 	Специфические особенности рынка земли.
 \vfill
\z 	Рецессионный и инфляционный разрывы
 \vfill

\vfill

\newpage


\shapk
\bilet{Экзаменационный билет}
\setcounter{zad}{0}

\vfill
\z 	Роль частной собственности на землю: негативный и позитивный взгляды.
 \vfill
\z 	Потребление и сбережение. Эффект мультипликатора. Мультипликатор и акселератор.
 \vfill

\vfill

\newpage


\shapk
\bilet{Экзаменационный билет}
\setcounter{zad}{0}

\vfill
\z 	Понятие ренты и арендной платы. Цена земли.
 \vfill
\z 	Изменения в макроэкономическом равновесии.  Эффект храповика.
 \vfill

\vfill

\newpage


\shapk
\bilet{Экзаменационный билет}
\setcounter{zad}{0}

\vfill
\z 	Понятие предприятия. Основные формы предпринимательской деятельности.
 \vfill
\z 	Равновесный уровень цен и равновесный реальный объем национального производства.
 \vfill

\vfill

\newpage


\shapk
\bilet{Экзаменационный билет}
\setcounter{zad}{0}

\vfill
\z 	Фирма – основная структурная единица бизнеса.
 \vfill
\z 	Совокупное предложение. Особенности кривой совокупного предложения.
 \vfill

\vfill

\newpage


\shapk
\bilet{Экзаменационный билет}
\setcounter{zad}{0}

\vfill
\z 	Венчурный капитал. Экономические риски и неопределенность.
 \vfill
\z 	Совокупный спрос и его особенности.
 \vfill

\vfill

\newpage


\shapk
\bilet{Экзаменационный билет}
\setcounter{zad}{0}

\vfill
\z 	Концентрация и централизация производства.
 \vfill
\z 	Оценка последствий экономического роста.
 \vfill

\vfill

\newpage


\shapk
\bilet{Экзаменационный билет}
\setcounter{zad}{0}

\vfill
\z 	Теория предельной полезности.  Кардинализм и ординализм.
 \vfill
\z 	Государственное регулирование экономического роста.
 \vfill

\vfill

\newpage


\shapk
\bilet{Экзаменационный билет}
\setcounter{zad}{0}

\vfill
\z 	Понятие экономических интересов.
 \vfill
\z 	Неоклассические и кейнсианские модели экономического роста.
 \vfill

\vfill

\newpage


\shapk
\bilet{Экзаменационный билет}
\setcounter{zad}{0}

\vfill
\z 	Понятие альтернативных издержек (издержки отвергнутых возможностей).
 \vfill
\z 	Экономический рост, его типы.  Факторы экономического роста.
 \vfill

\vfill

\newpage


\shapk
\bilet{Экзаменационный билет}
\setcounter{zad}{0}

\vfill
\z 	Неопределенность: технологическая, внутренней и внешней среды.
 \vfill
\z 	Чистое экономическое благосостояние и национальное богатство.
 \vfill

\vfill

\newpage


\shapk
\bilet{Экзаменационный билет}
\setcounter{zad}{0}

\vfill
\z 	Современное содержание парадигмы экономического и социального развития.
 \vfill
\z 	Номинальный и реальный ВНП.
 \vfill

\vfill

\newpage


\shapk
\bilet{Экзаменационный билет}
\setcounter{zad}{0}

\vfill
\z 	Понятие экономической стратегии.
 \vfill
\z 	Производные показатели системы национальных счетов: чистый национальный продукт, национальный доход, личный, располагаемый доход.
 \vfill

\vfill

\newpage


\shapk
\bilet{Экзаменационный билет}
\setcounter{zad}{0}

\vfill
\z 	Понятие экономической политики.
 \vfill
\z 	Структура валового национального продукта по доходам
 \vfill

\vfill

\newpage


\shapk
\bilet{Экзаменационный билет}
\setcounter{zad}{0}

\vfill
\z 	Отдача от масштаба производства.
 \vfill
\z 	Структура валового национального продукта по расходам.
 \vfill

\vfill

\newpage


\shapk
\bilet{Экзаменационный билет}
\setcounter{zad}{0}

\vfill
\z 	Риски и необходимость их страхования.
 \vfill
\z 	Проблема повторного счета. Добавленная стоимость.
 \vfill

\vfill

\newpage


\shapk
\bilet{Экзаменационный билет}
\setcounter{zad}{0}

\vfill
\z 	Максимизация прибыли в краткосрочном и долгосрочном периодах.
 \vfill
\z 	Валовой национальный продукт.  Валовой внутренний продукт.
 \vfill

\vfill

\newpage



\end{document}