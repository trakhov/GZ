Вася и Света условились встретиться в определенном месте между 06:00 и 08:00. Каждый из них может прийти в любое время в течение указанного промежутка и ждет второго некоторое время. Вася ждет 30 минут, после чего уходит; Света ждет 50 минут, после чего уходит. В 08:00 любой из них уходит, сколько бы до этого он ни ждал. Чему равна вероятность того, что Вася и Света не встретятся?
Петя и Надя условились встретиться в определенном месте между 15:00 и 16:00. Каждый из них может прийти в любое время в течение указанного промежутка и ждет второго некоторое время. Петя ждет 20 минут, после чего уходит; Надя ждет 40 минут, после чего уходит. В 16:00 любой из них уходит, сколько бы до этого он ни ждал. Чему равна вероятность того, что встреча состоится в первые полчаса?
Алексей и Света условились встретиться в определенном месте между 13:00 и 14:00. Каждый из них может прийти в любое время в течение указанного промежутка и ждет второго некоторое время. Алексей ждет 40 минут, после чего уходит; Света ждет 20 минут, после чего уходит. В 14:00 любой из них уходит, сколько бы до этого он ни ждал. Чему равна вероятность того, что Алексей и Света встретятся?
Аня и Света условились встретиться в определенном месте между 05:00 и 06:00. Каждый из них может прийти в любое время в течение указанного промежутка и ждет второго некоторое время. Аня ждет 30 минут, после чего уходит; Света ждет 40 минут, после чего уходит. В 06:00 любой из них уходит, сколько бы до этого он ни ждал. Чему равна вероятность того, что Аня и Света встретятся?
Валя и Иван Гермогенович Енотов условились встретиться в определенном месте между 11:00 и 14:00. Каждый из них может прийти в любое время в течение указанного промежутка и ждет второго некоторое время. Валя ждет 40 минут, после чего уходит; Иван Гермогенович Енотов ждет 60 минут, после чего уходит. В 14:00 любой из них уходит, сколько бы до этого он ни ждал. Чему равна вероятность того, что Валя придет раньше, чем Иван Гермогенович Енотов?
Джон Фаулз и Габриэль Гарсия Маркес условились встретиться в определенном месте между 08:00 и 11:00. Каждый из них может прийти в любое время в течение указанного промежутка и ждет второго некоторое время. Джон Фаулз ждет 60 минут, после чего уходит; Габриэль Гарсия Маркес ждет 50 минут, после чего уходит. В 11:00 любой из них уходит, сколько бы до этого он ни ждал. Чему равна вероятность того, что встреча состоится в первые полчаса?
Барак Обама и Владимир Путин условились встретиться в определенном месте между 18:00 и 19:00. Каждый из них может прийти в любое время в течение указанного промежутка и ждет второго некоторое время. Барак Обама ждет 40 минут, после чего уходит; Владимир Путин ждет 30 минут, после чего уходит. В 19:00 любой из них уходит, сколько бы до этого он ни ждал. Чему равна вероятность того, что Барак Обама и Владимир Путин не встретятся?
Владимир Путин и Барак Обама условились встретиться в определенном месте между 04:00 и 07:00. Каждый из них может прийти в любое время в течение указанного промежутка и ждет второго некоторое время. Владимир Путин ждет 60 минут, после чего уходит; Барак Обама ждет 40 минут, после чего уходит. В 07:00 любой из них уходит, сколько бы до этого он ни ждал. Чему равна вероятность того, что Владимир Путин и Барак Обама не встретятся?
Бэтмен и Робин условились встретиться в определенном месте между 05:00 и 08:00. Каждый из них может прийти в любое время в течение указанного промежутка и ждет второго некоторое время. Бэтмен ждет 50 минут, после чего уходит; Робин ждет 40 минут, после чего уходит. В 08:00 любой из них уходит, сколько бы до этого он ни ждал. Чему равна вероятность того, что встреча состоится в первые полчаса?
Аня и Алексей условились встретиться в определенном месте между 12:00 и 15:00. Каждый из них может прийти в любое время в течение указанного промежутка и ждет второго некоторое время. Аня ждет 50 минут, после чего уходит; Алексей ждет 40 минут, после чего уходит. В 15:00 любой из них уходит, сколько бы до этого он ни ждал. Чему равна вероятность того, что Аня опоздает более чем на полчаса?
Карик и Иван Гермогенович Енотов условились встретиться в определенном месте между 16:00 и 19:00. Каждый из них может прийти в любое время в течение указанного промежутка и ждет второго некоторое время. Карик ждет 50 минут, после чего уходит; Иван Гермогенович Енотов ждет 40 минут, после чего уходит. В 19:00 любой из них уходит, сколько бы до этого он ни ждал. Чему равна вероятность того, что встреча состоится в первые двадцать минут?
Валя и Карик условились встретиться в определенном месте между 17:00 и 18:00. Каждый из них может прийти в любое время в течение указанного промежутка и ждет второго некоторое время. Валя ждет 30 минут, после чего уходит; Карик ждет 40 минут, после чего уходит. В 18:00 любой из них уходит, сколько бы до этого он ни ждал. Чему равна вероятность того, что встреча состоится в последние полчаса?
д'Артаньян и Арамис условились встретиться в определенном месте между 08:00 и 10:00. Каждый из них может прийти в любое время в течение указанного промежутка и ждет второго некоторое время. д'Артаньян ждет 50 минут, после чего уходит; Арамис ждет 30 минут, после чего уходит. В 10:00 любой из них уходит, сколько бы до этого он ни ждал. Чему равна вероятность того, что встреча состоится в последние полчаса?
Бэтмен и Робин условились встретиться в определенном месте между 07:00 и 09:00. Каждый из них может прийти в любое время в течение указанного промежутка и ждет второго некоторое время. Бэтмен ждет 50 минут, после чего уходит; Робин ждет 40 минут, после чего уходит. В 09:00 любой из них уходит, сколько бы до этого он ни ждал. Чему равна вероятность того, что встреча произойдет не ранее чем без четверти 09:00?
Андрей и Алексей условились встретиться в определенном месте между 16:00 и 18:00. Каждый из них может прийти в любое время в течение указанного промежутка и ждет второго некоторое время. Андрей ждет 40 минут, после чего уходит; Алексей ждет 30 минут, после чего уходит. В 18:00 любой из них уходит, сколько бы до этого он ни ждал. Чему равна вероятность того, что встреча произойдет не ранее чем без четверти 18:00?
Аня и Петя условились встретиться в определенном месте между 07:00 и 08:00. Каждый из них может прийти в любое время в течение указанного промежутка и ждет второго некоторое время. Аня ждет 40 минут, после чего уходит; Петя ждет 20 минут, после чего уходит. В 08:00 любой из них уходит, сколько бы до этого он ни ждал. Чему равна вероятность того, что встреча состоится в первые двадцать минут?
Дэйв Гилмор и Ян Пэйс условились встретиться в определенном месте между 05:00 и 06:00. Каждый из них может прийти в любое время в течение указанного промежутка и ждет второго некоторое время. Дэйв Гилмор ждет 30 минут, после чего уходит; Ян Пэйс ждет 20 минут, после чего уходит. В 06:00 любой из них уходит, сколько бы до этого он ни ждал. Чему равна вероятность того, что встреча состоится в первые полчаса?
Портос и Атос условились встретиться в определенном месте между 15:00 и 18:00. Каждый из них может прийти в любое время в течение указанного промежутка и ждет второго некоторое время. Портос ждет 40 минут, после чего уходит; Атос ждет 60 минут, после чего уходит. В 18:00 любой из них уходит, сколько бы до этого он ни ждал. Чему равна вероятность того, что встреча состоится в первые двадцать минут?
Леонтий и Пафнутий условились встретиться в определенном месте между 17:00 и 20:00. Каждый из них может прийти в любое время в течение указанного промежутка и ждет второго некоторое время. Леонтий ждет 60 минут, после чего уходит; Пафнутий ждет 40 минут, после чего уходит. В 20:00 любой из них уходит, сколько бы до этого он ни ждал. Чему равна вероятность того, что встреча состоится в первые полчаса?
Света и Надя условились встретиться в определенном месте между 09:00 и 12:00. Каждый из них может прийти в любое время в течение указанного промежутка и ждет второго некоторое время. Света ждет 50 минут, после чего уходит; Надя ждет 60 минут, после чего уходит. В 12:00 любой из них уходит, сколько бы до этого он ни ждал. Чему равна вероятность того, что встреча произойдет не ранее чем без четверти 12:00?
Женя Богорад и Степан Степаныч Пивораки условились встретиться в определенном месте между 05:00 и 07:00. Каждый из них может прийти в любое время в течение указанного промежутка и ждет второго некоторое время. Женя Богорад ждет 30 минут, после чего уходит; Степан Степаныч Пивораки ждет 50 минут, после чего уходит. В 07:00 любой из них уходит, сколько бы до этого он ни ждал. Чему равна вероятность того, что Женя Богорад опоздает более чем на полчаса?
Джим Моррисон и Кит Ричардс условились встретиться в определенном месте между 02:00 и 04:00. Каждый из них может прийти в любое время в течение указанного промежутка и ждет второго некоторое время. Джим Моррисон ждет 50 минут, после чего уходит; Кит Ричардс ждет 40 минут, после чего уходит. В 04:00 любой из них уходит, сколько бы до этого он ни ждал. Чему равна вероятность того, что Джим Моррисон опоздает более чем на полчаса?
Владимир Путин и Барак Обама условились встретиться в определенном месте между 16:00 и 17:00. Каждый из них может прийти в любое время в течение указанного промежутка и ждет второго некоторое время. Владимир Путин ждет 20 минут, после чего уходит; Барак Обама ждет 40 минут, после чего уходит. В 17:00 любой из них уходит, сколько бы до этого он ни ждал. Чему равна вероятность того, что Владимир Путин опоздает более чем на полчаса?
Вася и Надя условились встретиться в определенном месте между 05:00 и 07:00. Каждый из них может прийти в любое время в течение указанного промежутка и ждет второго некоторое время. Вася ждет 30 минут, после чего уходит; Надя ждет 40 минут, после чего уходит. В 07:00 любой из них уходит, сколько бы до этого он ни ждал. Чему равна вероятность того, что Вася опоздает менее чем на полчаса?
Степан Степаныч Пивораки и Женя Богорад условились встретиться в определенном месте между 16:00 и 19:00. Каждый из них может прийти в любое время в течение указанного промежутка и ждет второго некоторое время. Степан Степаныч Пивораки ждет 50 минут, после чего уходит; Женя Богорад ждет 60 минут, после чего уходит. В 19:00 любой из них уходит, сколько бы до этого он ни ждал. Чему равна вероятность того, что Степан Степаныч Пивораки и Женя Богорад встретятся?
Эрнест Хэмингуэй и Марк Твен условились встретиться в определенном месте между 03:00 и 06:00. Каждый из них может прийти в любое время в течение указанного промежутка и ждет второго некоторое время. Эрнест Хэмингуэй ждет 60 минут, после чего уходит; Марк Твен ждет 40 минут, после чего уходит. В 06:00 любой из них уходит, сколько бы до этого он ни ждал. Чему равна вероятность того, что Эрнест Хэмингуэй опоздает менее чем на полчаса?
Алексей и Катя условились встретиться в определенном месте между 02:00 и 04:00. Каждый из них может прийти в любое время в течение указанного промежутка и ждет второго некоторое время. Алексей ждет 50 минут, после чего уходит; Катя ждет 40 минут, после чего уходит. В 04:00 любой из них уходит, сколько бы до этого он ни ждал. Чему равна вероятность того, что встреча состоится в последние полчаса?
Алексей и Света условились встретиться в определенном месте между 16:00 и 19:00. Каждый из них может прийти в любое время в течение указанного промежутка и ждет второго некоторое время. Алексей ждет 50 минут, после чего уходит; Света ждет 40 минут, после чего уходит. В 19:00 любой из них уходит, сколько бы до этого он ни ждал. Чему равна вероятность того, что Алексей придет раньше, чем Света?
Света и Надя условились встретиться в определенном месте между 13:00 и 14:00. Каждый из них может прийти в любое время в течение указанного промежутка и ждет второго некоторое время. Света ждет 40 минут, после чего уходит; Надя ждет 20 минут, после чего уходит. В 14:00 любой из них уходит, сколько бы до этого он ни ждал. Чему равна вероятность того, что встреча состоится в последние полчаса?
Леонардо и Микеланджело условились встретиться в определенном месте между 03:00 и 06:00. Каждый из них может прийти в любое время в течение указанного промежутка и ждет второго некоторое время. Леонардо ждет 60 минут, после чего уходит; Микеланджело ждет 50 минут, после чего уходит. В 06:00 любой из них уходит, сколько бы до этого он ни ждал. Чему равна вероятность того, что Леонардо и Микеланджело встретятся?
Джордж Оруэлл и Марк Твен условились встретиться в определенном месте между 07:00 и 08:00. Каждый из них может прийти в любое время в течение указанного промежутка и ждет второго некоторое время. Джордж Оруэлл ждет 20 минут, после чего уходит; Марк Твен ждет 30 минут, после чего уходит. В 08:00 любой из них уходит, сколько бы до этого он ни ждал. Чему равна вероятность того, что встреча произойдет не ранее чем без четверти 08:00?
Женя Богорад и Хоттабыч условились встретиться в определенном месте между 15:00 и 16:00. Каждый из них может прийти в любое время в течение указанного промежутка и ждет второго некоторое время. Женя Богорад ждет 40 минут, после чего уходит; Хоттабыч ждет 30 минут, после чего уходит. В 16:00 любой из них уходит, сколько бы до этого он ни ждал. Чему равна вероятность того, что Женя Богорад и Хоттабыч встретятся?
Робин и Бэтмен условились встретиться в определенном месте между 19:00 и 22:00. Каждый из них может прийти в любое время в течение указанного промежутка и ждет второго некоторое время. Робин ждет 40 минут, после чего уходит; Бэтмен ждет 50 минут, после чего уходит. В 22:00 любой из них уходит, сколько бы до этого он ни ждал. Чему равна вероятность того, что Робин и Бэтмен не встретятся?
Надя и Андрей условились встретиться в определенном месте между 16:00 и 18:00. Каждый из них может прийти в любое время в течение указанного промежутка и ждет второго некоторое время. Надя ждет 50 минут, после чего уходит; Андрей ждет 40 минут, после чего уходит. В 18:00 любой из них уходит, сколько бы до этого он ни ждал. Чему равна вероятность того, что встреча состоится в первые двадцать минут?
Валя и Карик условились встретиться в определенном месте между 03:00 и 04:00. Каждый из них может прийти в любое время в течение указанного промежутка и ждет второго некоторое время. Валя ждет 20 минут, после чего уходит; Карик ждет 30 минут, после чего уходит. В 04:00 любой из них уходит, сколько бы до этого он ни ждал. Чему равна вероятность того, что встреча состоится в первые полчаса?
Джон Фаулз и Эрнест Хэмингуэй условились встретиться в определенном месте между 03:00 и 05:00. Каждый из них может прийти в любое время в течение указанного промежутка и ждет второго некоторое время. Джон Фаулз ждет 30 минут, после чего уходит; Эрнест Хэмингуэй ждет 40 минут, после чего уходит. В 05:00 любой из них уходит, сколько бы до этого он ни ждал. Чему равна вероятность того, что встреча состоится в первые полчаса?
Пелагия и Дуня условились встретиться в определенном месте между 07:00 и 09:00. Каждый из них может прийти в любое время в течение указанного промежутка и ждет второго некоторое время. Пелагия ждет 50 минут, после чего уходит; Дуня ждет 40 минут, после чего уходит. В 09:00 любой из них уходит, сколько бы до этого он ни ждал. Чему равна вероятность того, что встреча состоится в первые двадцать минут?
Андрей и Аня условились встретиться в определенном месте между 13:00 и 16:00. Каждый из них может прийти в любое время в течение указанного промежутка и ждет второго некоторое время. Андрей ждет 50 минут, после чего уходит; Аня ждет 60 минут, после чего уходит. В 16:00 любой из них уходит, сколько бы до этого он ни ждал. Чему равна вероятность того, что встреча состоится в последние полчаса?
Вася и Андрей условились встретиться в определенном месте между 13:00 и 14:00. Каждый из них может прийти в любое время в течение указанного промежутка и ждет второго некоторое время. Вася ждет 30 минут, после чего уходит; Андрей ждет 20 минут, после чего уходит. В 14:00 любой из них уходит, сколько бы до этого он ни ждал. Чему равна вероятность того, что Вася придет раньше, чем Андрей?
Робин и Бэтмен условились встретиться в определенном месте между 10:00 и 13:00. Каждый из них может прийти в любое время в течение указанного промежутка и ждет второго некоторое время. Робин ждет 50 минут, после чего уходит; Бэтмен ждет 60 минут, после чего уходит. В 13:00 любой из них уходит, сколько бы до этого он ни ждал. Чему равна вероятность того, что встреча состоится в первые двадцать минут?
Пелагия и Архип условились встретиться в определенном месте между 03:00 и 05:00. Каждый из них может прийти в любое время в течение указанного промежутка и ждет второго некоторое время. Пелагия ждет 40 минут, после чего уходит; Архип ждет 50 минут, после чего уходит. В 05:00 любой из них уходит, сколько бы до этого он ни ждал. Чему равна вероятность того, что Пелагия опоздает более чем на полчаса?
Робин и Бэтмен условились встретиться в определенном месте между 17:00 и 18:00. Каждый из них может прийти в любое время в течение указанного промежутка и ждет второго некоторое время. Робин ждет 30 минут, после чего уходит; Бэтмен ждет 40 минут, после чего уходит. В 18:00 любой из них уходит, сколько бы до этого он ни ждал. Чему равна вероятность того, что Робин придет раньше, чем Бэтмен?
Женя Богорад и Степан Степаныч Пивораки условились встретиться в определенном месте между 07:00 и 10:00. Каждый из них может прийти в любое время в течение указанного промежутка и ждет второго некоторое время. Женя Богорад ждет 60 минут, после чего уходит; Степан Степаныч Пивораки ждет 40 минут, после чего уходит. В 10:00 любой из них уходит, сколько бы до этого он ни ждал. Чему равна вероятность того, что Женя Богорад и Степан Степаныч Пивораки встретятся?
Саша и Света условились встретиться в определенном месте между 01:00 и 03:00. Каждый из них может прийти в любое время в течение указанного промежутка и ждет второго некоторое время. Саша ждет 50 минут, после чего уходит; Света ждет 30 минут, после чего уходит. В 03:00 любой из них уходит, сколько бы до этого он ни ждал. Чему равна вероятность того, что встреча произойдет не ранее чем без четверти 03:00?
Аня и Алексей условились встретиться в определенном месте между 05:00 и 06:00. Каждый из них может прийти в любое время в течение указанного промежутка и ждет второго некоторое время. Аня ждет 20 минут, после чего уходит; Алексей ждет 30 минут, после чего уходит. В 06:00 любой из них уходит, сколько бы до этого он ни ждал. Чему равна вероятность того, что встреча состоится в первые двадцать минут?
Микеланджело и Рафаэль условились встретиться в определенном месте между 09:00 и 10:00. Каждый из них может прийти в любое время в течение указанного промежутка и ждет второго некоторое время. Микеланджело ждет 20 минут, после чего уходит; Рафаэль ждет 40 минут, после чего уходит. В 10:00 любой из них уходит, сколько бы до этого он ни ждал. Чему равна вероятность того, что Микеланджело опоздает менее чем на полчаса?
Надя и Петя условились встретиться в определенном месте между 01:00 и 02:00. Каждый из них может прийти в любое время в течение указанного промежутка и ждет второго некоторое время. Надя ждет 30 минут, после чего уходит; Петя ждет 40 минут, после чего уходит. В 02:00 любой из них уходит, сколько бы до этого он ни ждал. Чему равна вероятность того, что встреча состоится в последние полчаса?
Алексей и Саша условились встретиться в определенном месте между 05:00 и 06:00. Каждый из них может прийти в любое время в течение указанного промежутка и ждет второго некоторое время. Алексей ждет 40 минут, после чего уходит; Саша ждет 20 минут, после чего уходит. В 06:00 любой из них уходит, сколько бы до этого он ни ждал. Чему равна вероятность того, что Алексей и Саша не встретятся?
Барак Обама и Владимир Путин условились встретиться в определенном месте между 13:00 и 15:00. Каждый из них может прийти в любое время в течение указанного промежутка и ждет второго некоторое время. Барак Обама ждет 30 минут, после чего уходит; Владимир Путин ждет 40 минут, после чего уходит. В 15:00 любой из них уходит, сколько бы до этого он ни ждал. Чему равна вероятность того, что встреча состоится в первые полчаса?
Хоттабыч и Волька Костыльков условились встретиться в определенном месте между 08:00 и 10:00. Каждый из них может прийти в любое время в течение указанного промежутка и ждет второго некоторое время. Хоттабыч ждет 30 минут, после чего уходит; Волька Костыльков ждет 50 минут, после чего уходит. В 10:00 любой из них уходит, сколько бы до этого он ни ждал. Чему равна вероятность того, что Хоттабыч опоздает более чем на полчаса?