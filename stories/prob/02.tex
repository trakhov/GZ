Пелагия и Пафнутий условились встретиться в определенном месте между 11:00 и 13:00. Каждый из них может прийти в любое время в течение указанного промежутка и ждет второго некоторое время. Пелагия ждет 40 минут, после чего уходит; Пафнутий ждет 50 минут, после чего уходит. В 13:00 любой из них уходит, сколько бы до этого он ни ждал. Чему равна вероятность того, что Пелагия опоздает менее чем на полчаса?
Кит Ричардс и Мик Джаггер условились встретиться в определенном месте между 19:00 и 21:00. Каждый из них может прийти в любое время в течение указанного промежутка и ждет второго некоторое время. Кит Ричардс ждет 40 минут, после чего уходит; Мик Джаггер ждет 30 минут, после чего уходит. В 21:00 любой из них уходит, сколько бы до этого он ни ждал. Чему равна вероятность того, что Кит Ричардс опоздает менее чем на полчаса?
Катя и Алексей условились встретиться в определенном месте между 14:00 и 16:00. Каждый из них может прийти в любое время в течение указанного промежутка и ждет второго некоторое время. Катя ждет 40 минут, после чего уходит; Алексей ждет 30 минут, после чего уходит. В 16:00 любой из них уходит, сколько бы до этого он ни ждал. Чему равна вероятность того, что встреча произойдет не ранее чем без четверти 16:00?
Алексей и Надя условились встретиться в определенном месте между 12:00 и 13:00. Каждый из них может прийти в любое время в течение указанного промежутка и ждет второго некоторое время. Алексей ждет 20 минут, после чего уходит; Надя ждет 30 минут, после чего уходит. В 13:00 любой из них уходит, сколько бы до этого он ни ждал. Чему равна вероятность того, что Алексей опоздает более чем на полчаса?
Полина и Вася условились встретиться в определенном месте между 13:00 и 14:00. Каждый из них может прийти в любое время в течение указанного промежутка и ждет второго некоторое время. Полина ждет 30 минут, после чего уходит; Вася ждет 20 минут, после чего уходит. В 14:00 любой из них уходит, сколько бы до этого он ни ждал. Чему равна вероятность того, что Полина придет раньше, чем Вася?
Владимир Путин и Барак Обама условились встретиться в определенном месте между 15:00 и 17:00. Каждый из них может прийти в любое время в течение указанного промежутка и ждет второго некоторое время. Владимир Путин ждет 40 минут, после чего уходит; Барак Обама ждет 50 минут, после чего уходит. В 17:00 любой из них уходит, сколько бы до этого он ни ждал. Чему равна вероятность того, что встреча состоится в первые полчаса?
Бэтмен и Робин условились встретиться в определенном месте между 09:00 и 11:00. Каждый из них может прийти в любое время в течение указанного промежутка и ждет второго некоторое время. Бэтмен ждет 40 минут, после чего уходит; Робин ждет 50 минут, после чего уходит. В 11:00 любой из них уходит, сколько бы до этого он ни ждал. Чему равна вероятность того, что Бэтмен придет раньше, чем Робин?
Архип и Дуня условились встретиться в определенном месте между 11:00 и 12:00. Каждый из них может прийти в любое время в течение указанного промежутка и ждет второго некоторое время. Архип ждет 40 минут, после чего уходит; Дуня ждет 20 минут, после чего уходит. В 12:00 любой из них уходит, сколько бы до этого он ни ждал. Чему равна вероятность того, что встреча произойдет не ранее чем без четверти 12:00?
Валя и Карик условились встретиться в определенном месте между 07:00 и 10:00. Каждый из них может прийти в любое время в течение указанного промежутка и ждет второго некоторое время. Валя ждет 50 минут, после чего уходит; Карик ждет 40 минут, после чего уходит. В 10:00 любой из них уходит, сколько бы до этого он ни ждал. Чему равна вероятность того, что встреча состоится в последние полчаса?
Барак Обама и Владимир Путин условились встретиться в определенном месте между 13:00 и 15:00. Каждый из них может прийти в любое время в течение указанного промежутка и ждет второго некоторое время. Барак Обама ждет 40 минут, после чего уходит; Владимир Путин ждет 30 минут, после чего уходит. В 15:00 любой из них уходит, сколько бы до этого он ни ждал. Чему равна вероятность того, что встреча состоится в первые двадцать минут?
Марк Твен и Джон Фаулз условились встретиться в определенном месте между 03:00 и 06:00. Каждый из них может прийти в любое время в течение указанного промежутка и ждет второго некоторое время. Марк Твен ждет 60 минут, после чего уходит; Джон Фаулз ждет 40 минут, после чего уходит. В 06:00 любой из них уходит, сколько бы до этого он ни ждал. Чему равна вероятность того, что Марк Твен придет раньше, чем Джон Фаулз?
Пафнутий и Архип условились встретиться в определенном месте между 13:00 и 16:00. Каждый из них может прийти в любое время в течение указанного промежутка и ждет второго некоторое время. Пафнутий ждет 60 минут, после чего уходит; Архип ждет 50 минут, после чего уходит. В 16:00 любой из них уходит, сколько бы до этого он ни ждал. Чему равна вероятность того, что встреча состоится в первые двадцать минут?
Вася и Света условились встретиться в определенном месте между 09:00 и 11:00. Каждый из них может прийти в любое время в течение указанного промежутка и ждет второго некоторое время. Вася ждет 50 минут, после чего уходит; Света ждет 30 минут, после чего уходит. В 11:00 любой из них уходит, сколько бы до этого он ни ждал. Чему равна вероятность того, что Вася опоздает более чем на полчаса?
Вася и Петя условились встретиться в определенном месте между 03:00 и 06:00. Каждый из них может прийти в любое время в течение указанного промежутка и ждет второго некоторое время. Вася ждет 50 минут, после чего уходит; Петя ждет 40 минут, после чего уходит. В 06:00 любой из них уходит, сколько бы до этого он ни ждал. Чему равна вероятность того, что Вася и Петя не встретятся?
Валя и Карик условились встретиться в определенном месте между 06:00 и 09:00. Каждый из них может прийти в любое время в течение указанного промежутка и ждет второго некоторое время. Валя ждет 40 минут, после чего уходит; Карик ждет 60 минут, после чего уходит. В 09:00 любой из них уходит, сколько бы до этого он ни ждал. Чему равна вероятность того, что Валя и Карик встретятся?
Вася и Надя условились встретиться в определенном месте между 08:00 и 09:00. Каждый из них может прийти в любое время в течение указанного промежутка и ждет второго некоторое время. Вася ждет 30 минут, после чего уходит; Надя ждет 20 минут, после чего уходит. В 09:00 любой из них уходит, сколько бы до этого он ни ждал. Чему равна вероятность того, что встреча состоится в первые двадцать минут?
Архип и Пафнутий условились встретиться в определенном месте между 07:00 и 09:00. Каждый из них может прийти в любое время в течение указанного промежутка и ждет второго некоторое время. Архип ждет 40 минут, после чего уходит; Пафнутий ждет 30 минут, после чего уходит. В 09:00 любой из них уходит, сколько бы до этого он ни ждал. Чему равна вероятность того, что Архип и Пафнутий не встретятся?
Валя и Карик условились встретиться в определенном месте между 12:00 и 14:00. Каждый из них может прийти в любое время в течение указанного промежутка и ждет второго некоторое время. Валя ждет 50 минут, после чего уходит; Карик ждет 30 минут, после чего уходит. В 14:00 любой из них уходит, сколько бы до этого он ни ждал. Чему равна вероятность того, что Валя и Карик встретятся?
Габриэль Гарсия Маркес и Джордж Оруэлл условились встретиться в определенном месте между 10:00 и 11:00. Каждый из них может прийти в любое время в течение указанного промежутка и ждет второго некоторое время. Габриэль Гарсия Маркес ждет 40 минут, после чего уходит; Джордж Оруэлл ждет 30 минут, после чего уходит. В 11:00 любой из них уходит, сколько бы до этого он ни ждал. Чему равна вероятность того, что Габриэль Гарсия Маркес придет раньше, чем Джордж Оруэлл?
Катя и Вася условились встретиться в определенном месте между 08:00 и 11:00. Каждый из них может прийти в любое время в течение указанного промежутка и ждет второго некоторое время. Катя ждет 60 минут, после чего уходит; Вася ждет 40 минут, после чего уходит. В 11:00 любой из них уходит, сколько бы до этого он ни ждал. Чему равна вероятность того, что Катя опоздает более чем на полчаса?
Барак Обама и Владимир Путин условились встретиться в определенном месте между 03:00 и 05:00. Каждый из них может прийти в любое время в течение указанного промежутка и ждет второго некоторое время. Барак Обама ждет 30 минут, после чего уходит; Владимир Путин ждет 40 минут, после чего уходит. В 05:00 любой из них уходит, сколько бы до этого он ни ждал. Чему равна вероятность того, что Барак Обама опоздает менее чем на полчаса?
Архип и Леонтий условились встретиться в определенном месте между 18:00 и 21:00. Каждый из них может прийти в любое время в течение указанного промежутка и ждет второго некоторое время. Архип ждет 40 минут, после чего уходит; Леонтий ждет 60 минут, после чего уходит. В 21:00 любой из них уходит, сколько бы до этого он ни ждал. Чему равна вероятность того, что встреча состоится в первые двадцать минут?
Владимир Путин и Барак Обама условились встретиться в определенном месте между 15:00 и 17:00. Каждый из них может прийти в любое время в течение указанного промежутка и ждет второго некоторое время. Владимир Путин ждет 50 минут, после чего уходит; Барак Обама ждет 40 минут, после чего уходит. В 17:00 любой из них уходит, сколько бы до этого он ни ждал. Чему равна вероятность того, что Владимир Путин и Барак Обама встретятся?
Владимир Путин и Барак Обама условились встретиться в определенном месте между 09:00 и 10:00. Каждый из них может прийти в любое время в течение указанного промежутка и ждет второго некоторое время. Владимир Путин ждет 20 минут, после чего уходит; Барак Обама ждет 30 минут, после чего уходит. В 10:00 любой из них уходит, сколько бы до этого он ни ждал. Чему равна вероятность того, что встреча произойдет не ранее чем без четверти 10:00?
Атос и Арамис условились встретиться в определенном месте между 06:00 и 09:00. Каждый из них может прийти в любое время в течение указанного промежутка и ждет второго некоторое время. Атос ждет 60 минут, после чего уходит; Арамис ждет 40 минут, после чего уходит. В 09:00 любой из них уходит, сколько бы до этого он ни ждал. Чему равна вероятность того, что Атос опоздает менее чем на полчаса?
Атос и Портос условились встретиться в определенном месте между 07:00 и 10:00. Каждый из них может прийти в любое время в течение указанного промежутка и ждет второго некоторое время. Атос ждет 60 минут, после чего уходит; Портос ждет 40 минут, после чего уходит. В 10:00 любой из них уходит, сколько бы до этого он ни ждал. Чему равна вероятность того, что Атос придет раньше, чем Портос?
Алексей и Аня условились встретиться в определенном месте между 10:00 и 12:00. Каждый из них может прийти в любое время в течение указанного промежутка и ждет второго некоторое время. Алексей ждет 30 минут, после чего уходит; Аня ждет 50 минут, после чего уходит. В 12:00 любой из них уходит, сколько бы до этого он ни ждал. Чему равна вероятность того, что Алексей придет раньше, чем Аня?
Портос и Атос условились встретиться в определенном месте между 16:00 и 17:00. Каждый из них может прийти в любое время в течение указанного промежутка и ждет второго некоторое время. Портос ждет 30 минут, после чего уходит; Атос ждет 20 минут, после чего уходит. В 17:00 любой из них уходит, сколько бы до этого он ни ждал. Чему равна вероятность того, что встреча произойдет не ранее чем без четверти 17:00?
Катя и Саша условились встретиться в определенном месте между 05:00 и 07:00. Каждый из них может прийти в любое время в течение указанного промежутка и ждет второго некоторое время. Катя ждет 30 минут, после чего уходит; Саша ждет 40 минут, после чего уходит. В 07:00 любой из них уходит, сколько бы до этого он ни ждал. Чему равна вероятность того, что встреча произойдет не ранее чем без четверти 07:00?
Саша и Надя условились встретиться в определенном месте между 13:00 и 15:00. Каждый из них может прийти в любое время в течение указанного промежутка и ждет второго некоторое время. Саша ждет 40 минут, после чего уходит; Надя ждет 50 минут, после чего уходит. В 15:00 любой из них уходит, сколько бы до этого он ни ждал. Чему равна вероятность того, что Саша придет раньше, чем Надя?
Аня и Саша условились встретиться в определенном месте между 16:00 и 19:00. Каждый из них может прийти в любое время в течение указанного промежутка и ждет второго некоторое время. Аня ждет 60 минут, после чего уходит; Саша ждет 40 минут, после чего уходит. В 19:00 любой из них уходит, сколько бы до этого он ни ждал. Чему равна вероятность того, что Аня и Саша не встретятся?
Карик и Валя условились встретиться в определенном месте между 17:00 и 18:00. Каждый из них может прийти в любое время в течение указанного промежутка и ждет второго некоторое время. Карик ждет 40 минут, после чего уходит; Валя ждет 30 минут, после чего уходит. В 18:00 любой из них уходит, сколько бы до этого он ни ждал. Чему равна вероятность того, что встреча состоится в первые двадцать минут?
Валя и Карик условились встретиться в определенном месте между 08:00 и 10:00. Каждый из них может прийти в любое время в течение указанного промежутка и ждет второго некоторое время. Валя ждет 40 минут, после чего уходит; Карик ждет 30 минут, после чего уходит. В 10:00 любой из них уходит, сколько бы до этого он ни ждал. Чему равна вероятность того, что Валя и Карик встретятся?
д'Артаньян и Арамис условились встретиться в определенном месте между 16:00 и 18:00. Каждый из них может прийти в любое время в течение указанного промежутка и ждет второго некоторое время. д'Артаньян ждет 30 минут, после чего уходит; Арамис ждет 40 минут, после чего уходит. В 18:00 любой из них уходит, сколько бы до этого он ни ждал. Чему равна вероятность того, что д'Артаньян опоздает менее чем на полчаса?
Андрей и Аня условились встретиться в определенном месте между 16:00 и 17:00. Каждый из них может прийти в любое время в течение указанного промежутка и ждет второго некоторое время. Андрей ждет 30 минут, после чего уходит; Аня ждет 40 минут, после чего уходит. В 17:00 любой из них уходит, сколько бы до этого он ни ждал. Чему равна вероятность того, что Андрей и Аня встретятся?
Вася и Катя условились встретиться в определенном месте между 02:00 и 03:00. Каждый из них может прийти в любое время в течение указанного промежутка и ждет второго некоторое время. Вася ждет 20 минут, после чего уходит; Катя ждет 40 минут, после чего уходит. В 03:00 любой из них уходит, сколько бы до этого он ни ждал. Чему равна вероятность того, что Вася и Катя встретятся?
Хоттабыч и Волька Костыльков условились встретиться в определенном месте между 16:00 и 19:00. Каждый из них может прийти в любое время в течение указанного промежутка и ждет второго некоторое время. Хоттабыч ждет 60 минут, после чего уходит; Волька Костыльков ждет 40 минут, после чего уходит. В 19:00 любой из них уходит, сколько бы до этого он ни ждал. Чему равна вероятность того, что Хоттабыч придет раньше, чем Волька Костыльков?
Ян Пэйс и Дэйв Гилмор условились встретиться в определенном месте между 06:00 и 09:00. Каждый из них может прийти в любое время в течение указанного промежутка и ждет второго некоторое время. Ян Пэйс ждет 60 минут, после чего уходит; Дэйв Гилмор ждет 40 минут, после чего уходит. В 09:00 любой из них уходит, сколько бы до этого он ни ждал. Чему равна вероятность того, что Ян Пэйс и Дэйв Гилмор встретятся?
Надя и Аня условились встретиться в определенном месте между 18:00 и 20:00. Каждый из них может прийти в любое время в течение указанного промежутка и ждет второго некоторое время. Надя ждет 50 минут, после чего уходит; Аня ждет 30 минут, после чего уходит. В 20:00 любой из них уходит, сколько бы до этого он ни ждал. Чему равна вероятность того, что Надя опоздает менее чем на полчаса?
Надя и Света условились встретиться в определенном месте между 17:00 и 18:00. Каждый из них может прийти в любое время в течение указанного промежутка и ждет второго некоторое время. Надя ждет 30 минут, после чего уходит; Света ждет 20 минут, после чего уходит. В 18:00 любой из них уходит, сколько бы до этого он ни ждал. Чему равна вероятность того, что Надя опоздает более чем на полчаса?
Петя и Надя условились встретиться в определенном месте между 02:00 и 05:00. Каждый из них может прийти в любое время в течение указанного промежутка и ждет второго некоторое время. Петя ждет 40 минут, после чего уходит; Надя ждет 60 минут, после чего уходит. В 05:00 любой из них уходит, сколько бы до этого он ни ждал. Чему равна вероятность того, что встреча состоится в последние полчаса?
Катя и Андрей условились встретиться в определенном месте между 19:00 и 22:00. Каждый из них может прийти в любое время в течение указанного промежутка и ждет второго некоторое время. Катя ждет 60 минут, после чего уходит; Андрей ждет 50 минут, после чего уходит. В 22:00 любой из них уходит, сколько бы до этого он ни ждал. Чему равна вероятность того, что Катя опоздает менее чем на полчаса?
Полина и Петя условились встретиться в определенном месте между 12:00 и 13:00. Каждый из них может прийти в любое время в течение указанного промежутка и ждет второго некоторое время. Полина ждет 40 минут, после чего уходит; Петя ждет 30 минут, после чего уходит. В 13:00 любой из них уходит, сколько бы до этого он ни ждал. Чему равна вероятность того, что встреча состоится в первые двадцать минут?
Петя и Света условились встретиться в определенном месте между 02:00 и 03:00. Каждый из них может прийти в любое время в течение указанного промежутка и ждет второго некоторое время. Петя ждет 30 минут, после чего уходит; Света ждет 40 минут, после чего уходит. В 03:00 любой из них уходит, сколько бы до этого он ни ждал. Чему равна вероятность того, что встреча произойдет не ранее чем без четверти 03:00?
Катя и Света условились встретиться в определенном месте между 13:00 и 14:00. Каждый из них может прийти в любое время в течение указанного промежутка и ждет второго некоторое время. Катя ждет 20 минут, после чего уходит; Света ждет 40 минут, после чего уходит. В 14:00 любой из них уходит, сколько бы до этого он ни ждал. Чему равна вероятность того, что Катя и Света не встретятся?
Эрнест Хэмингуэй и Марк Твен условились встретиться в определенном месте между 11:00 и 14:00. Каждый из них может прийти в любое время в течение указанного промежутка и ждет второго некоторое время. Эрнест Хэмингуэй ждет 60 минут, после чего уходит; Марк Твен ждет 50 минут, после чего уходит. В 14:00 любой из них уходит, сколько бы до этого он ни ждал. Чему равна вероятность того, что встреча произойдет не ранее чем без четверти 14:00?
Пелагия и Архип условились встретиться в определенном месте между 14:00 и 16:00. Каждый из них может прийти в любое время в течение указанного промежутка и ждет второго некоторое время. Пелагия ждет 40 минут, после чего уходит; Архип ждет 30 минут, после чего уходит. В 16:00 любой из них уходит, сколько бы до этого он ни ждал. Чему равна вероятность того, что Пелагия придет раньше, чем Архип?
Саша и Надя условились встретиться в определенном месте между 06:00 и 07:00. Каждый из них может прийти в любое время в течение указанного промежутка и ждет второго некоторое время. Саша ждет 40 минут, после чего уходит; Надя ждет 20 минут, после чего уходит. В 07:00 любой из них уходит, сколько бы до этого он ни ждал. Чему равна вероятность того, что Саша придет раньше, чем Надя?
Ян Пэйс и Мик Джаггер условились встретиться в определенном месте между 09:00 и 10:00. Каждый из них может прийти в любое время в течение указанного промежутка и ждет второго некоторое время. Ян Пэйс ждет 30 минут, после чего уходит; Мик Джаггер ждет 20 минут, после чего уходит. В 10:00 любой из них уходит, сколько бы до этого он ни ждал. Чему равна вероятность того, что встреча состоится в первые полчаса?
Джим Моррисон и Ян Пэйс условились встретиться в определенном месте между 03:00 и 06:00. Каждый из них может прийти в любое время в течение указанного промежутка и ждет второго некоторое время. Джим Моррисон ждет 40 минут, после чего уходит; Ян Пэйс ждет 60 минут, после чего уходит. В 06:00 любой из них уходит, сколько бы до этого он ни ждал. Чему равна вероятность того, что встреча состоится в последние полчаса?