

\z Случайная величина $X$ задана плотностью распределения: $$f(x) = \begin{cases} -\frac{1}{18}(x-3)^2, & x \in (0; 3), \\ 0, & x \not\in (0; 3). \end{cases}$$ Найдите моду, медиану и математическое ожидание этой случайной величины.


\vfill

\z Дискретная случайная величина $X$ задана законом распределения: $$ X = \left(\begin{array}{rrr}0 & 1 & 2\\0{,}1 & 0{,}8 & 0{,}1\end{array}\right).$$ Найдите начальные и центральные моменты первого, второго и третьего порядков.
 

\vfill

\newpage\setcounter{zad}{0}

\z Случайная величина $X$ задана плотностью распределения: $$f(x) = \begin{cases} -\frac{3}{16}(x-2)^2, & x \in (0; 2), \\ 0, & x \not\in (0; 2). \end{cases}$$ Найдите моду, медиану и математическое ожидание этой случайной величины.


\vfill

\z Случайная величина $X$ задана плотностью распределения $$f(x) = \begin{cases}0, & x\not\in(0; 4), \\ \frac{1}{8}x, & x\in(0; 4).\end{cases}$$ Найдите начальные и центральные моменты первого, второго и третьего порядков.
 

\vfill

\newpage\setcounter{zad}{0}

\z Случайная величина $X$ задана плотностью распределения: $$f(x) = \begin{cases} -\frac{1}{18}(x-3)^2, & x \in (0; 3), \\ 0, & x \not\in (0; 3). \end{cases}$$ Найдите моду, медиану и математическое ожидание этой случайной величины.


\vfill

\z Дискретная случайная величина $X$ задана законом распределения: $$ X = \left(\begin{array}{rrr}-2 & 1 & 3\\0{,}8 & 0{,}1 & 0{,}1\end{array}\right).$$ Найдите начальные и центральные моменты первого, второго и третьего порядков.
 

\vfill

\newpage\setcounter{zad}{0}

\z Случайная величина $X$ задана плотностью распределения: $$f(x) = \begin{cases} \frac{3}{8}(x-2)^2, & x \in (0; 2), \\ 0, & x \not\in (0; 2). \end{cases}$$ Найдите моду, медиану и математическое ожидание этой случайной величины.


\vfill

\z Дискретная случайная величина $X$ задана законом распределения: $$ X = \left(\begin{array}{rrr}-3 & 0 & 3\\0{,}2 & 0{,}1 & 0{,}7\end{array}\right).$$ Найдите начальные и центральные моменты первого, второго и третьего порядков.
 

\vfill

\newpage\setcounter{zad}{0}

\z Случайная величина $X$ задана плотностью распределения: $$f(x) = \begin{cases} -\frac{1}{18}(x+3)^2, & x \in (-3; 0), \\ 0, & x \not\in (-3; 0). \end{cases}$$ Найдите моду, медиану и математическое ожидание этой случайной величины.


\vfill

\z Дискретная случайная величина $X$ задана законом распределения: $$ X = \left(\begin{array}{rrr}1 & 3 & 4\\0{,}2 & 0{,}4 & 0{,}4\end{array}\right).$$ Найдите начальные и центральные моменты первого, второго и третьего порядков.
 

\vfill

\newpage\setcounter{zad}{0}

\z Случайная величина $X$ задана плотностью распределения: $$f(x) = \begin{cases} -\frac{1}{18}(x-3)^2, & x \in (0; 3), \\ 0, & x \not\in (0; 3). \end{cases}$$ Найдите моду, медиану и математическое ожидание этой случайной величины.


\vfill

\z Дискретная случайная величина $X$ задана законом распределения: $$ X = \left(\begin{array}{rrr}-4 & -3 & 0\\0{,}2 & 0{,}1 & 0{,}7\end{array}\right).$$ Найдите начальные и центральные моменты первого, второго и третьего порядков.
 

\vfill

\newpage\setcounter{zad}{0}

\z Случайная величина $X$ задана плотностью распределения: $$f(x) = \begin{cases} -\frac{3}{16}(x+2)^2, & x \in (-2; 0), \\ 0, & x \not\in (-2; 0). \end{cases}$$ Найдите моду, медиану и математическое ожидание этой случайной величины.


\vfill

\z Дискретная случайная величина $X$ задана законом распределения: $$ X = \left(\begin{array}{rrr}2 & 4 & 5\\0{,}2 & 0{,}7 & 0{,}1\end{array}\right).$$ Найдите начальные и центральные моменты первого, второго и третьего порядков.
 

\vfill

\newpage\setcounter{zad}{0}

\z Случайная величина $X$ задана плотностью распределения: $$f(x) = \begin{cases} -\frac{1}{18}(x-3)^2, & x \in (0; 3), \\ 0, & x \not\in (0; 3). \end{cases}$$ Найдите моду, медиану и математическое ожидание этой случайной величины.


\vfill

\z Случайная величина $X$ задана плотностью распределения $$f(x) = \begin{cases}0, & x\not\in(3; 5), \\ \frac{1}{8}x, & x\in(3; 5).\end{cases}$$ Найдите начальные и центральные моменты первого, второго и третьего порядков.
 

\vfill

\newpage\setcounter{zad}{0}

\z Случайная величина $X$ задана плотностью распределения: $$f(x) = \begin{cases} 3(x+1)^2, & x \in (-1; 0), \\ 0, & x \not\in (-1; 0). \end{cases}$$ Найдите моду, медиану и математическое ожидание этой случайной величины.


\vfill

\z Дискретная случайная величина $X$ задана законом распределения: $$ X = \left(\begin{array}{rrr}3 & 4 & 6\\0{,}3 & 0{,}3 & 0{,}4\end{array}\right).$$ Найдите начальные и центральные моменты первого, второго и третьего порядков.
 

\vfill

\newpage\setcounter{zad}{0}

\z Случайная величина $X$ задана плотностью распределения: $$f(x) = \begin{cases} -\frac{3}{16}(x+2)^2, & x \in (-2; 0), \\ 0, & x \not\in (-2; 0). \end{cases}$$ Найдите моду, медиану и математическое ожидание этой случайной величины.


\vfill

\z Случайная величина $X$ задана плотностью распределения $$f(x) = \begin{cases}0, & x\not\in(0; 2), \\ \frac{1}{2}x, & x\in(0; 2).\end{cases}$$ Найдите начальные и центральные моменты первого, второго и третьего порядков.
 

\vfill

\newpage\setcounter{zad}{0}

\z Случайная величина $X$ задана плотностью распределения: $$f(x) = \begin{cases} -\frac{3}{16}(x-2)^2, & x \in (0; 2), \\ 0, & x \not\in (0; 2). \end{cases}$$ Найдите моду, медиану и математическое ожидание этой случайной величины.


\vfill

\z Дискретная случайная величина $X$ задана законом распределения: $$ X = \left(\begin{array}{rrr}0 & 2 & 4\\0{,}2 & 0{,}7 & 0{,}1\end{array}\right).$$ Найдите начальные и центральные моменты первого, второго и третьего порядков.
 

\vfill

\newpage\setcounter{zad}{0}

\z Случайная величина $X$ задана плотностью распределения: $$f(x) = \begin{cases} -\frac{1}{18}(x+3)^2, & x \in (-3; 0), \\ 0, & x \not\in (-3; 0). \end{cases}$$ Найдите моду, медиану и математическое ожидание этой случайной величины.


\vfill

\z Дискретная случайная величина $X$ задана законом распределения: $$ X = \left(\begin{array}{rrr}-4 & -2 & 1\\0{,}2 & 0{,}7 & 0{,}1\end{array}\right).$$ Найдите начальные и центральные моменты первого, второго и третьего порядков.
 

\vfill

\newpage\setcounter{zad}{0}

\z Случайная величина $X$ задана плотностью распределения: $$f(x) = \begin{cases} 3(x+1)^2, & x \in (-1; 0), \\ 0, & x \not\in (-1; 0). \end{cases}$$ Найдите моду, медиану и математическое ожидание этой случайной величины.


\vfill

\z Дискретная случайная величина $X$ задана законом распределения: $$ X = \left(\begin{array}{rrr}2 & 4 & 6\\0{,}1 & 0{,}2 & 0{,}7\end{array}\right).$$ Найдите начальные и центральные моменты первого, второго и третьего порядков.
 

\vfill

\newpage\setcounter{zad}{0}

\z Случайная величина $X$ задана плотностью распределения: $$f(x) = \begin{cases} \frac{3}{8}(x+2)^2, & x \in (-2; 0), \\ 0, & x \not\in (-2; 0). \end{cases}$$ Найдите моду, медиану и математическое ожидание этой случайной величины.


\vfill

\z Дискретная случайная величина $X$ задана законом распределения: $$ X = \left(\begin{array}{rrr}3 & 4 & 7\\0{,}2 & 0{,}2 & 0{,}6\end{array}\right).$$ Найдите начальные и центральные моменты первого, второго и третьего порядков.
 

\vfill

\newpage\setcounter{zad}{0}

\z Случайная величина $X$ задана плотностью распределения: $$f(x) = \begin{cases} -\frac{3}{16}(x+2)^2, & x \in (-2; 0), \\ 0, & x \not\in (-2; 0). \end{cases}$$ Найдите моду, медиану и математическое ожидание этой случайной величины.


\vfill

\z Дискретная случайная величина $X$ задана законом распределения: $$ X = \left(\begin{array}{rrr}-4 & -3 & -2\\0{,}8 & 0{,}1 & 0{,}1\end{array}\right).$$ Найдите начальные и центральные моменты первого, второго и третьего порядков.
 

\vfill

\newpage\setcounter{zad}{0}

\z Случайная величина $X$ задана плотностью распределения: $$f(x) = \begin{cases} 3(x-1)^2, & x \in (0; 1), \\ 0, & x \not\in (0; 1). \end{cases}$$ Найдите моду, медиану и математическое ожидание этой случайной величины.


\vfill

\z Дискретная случайная величина $X$ задана законом распределения: $$ X = \left(\begin{array}{rrr}0 & 2 & 3\\0{,}6 & 0{,}3 & 0{,}1\end{array}\right).$$ Найдите начальные и центральные моменты первого, второго и третьего порядков.
 

\vfill

\newpage\setcounter{zad}{0}

\z Случайная величина $X$ задана плотностью распределения: $$f(x) = \begin{cases} -\frac{3}{2}(x-1)^2, & x \in (0; 1), \\ 0, & x \not\in (0; 1). \end{cases}$$ Найдите моду, медиану и математическое ожидание этой случайной величины.


\vfill

\z Случайная величина $X$ задана плотностью распределения $$f(x) = \begin{cases}0, & x\not\in(0; 4), \\ \frac{1}{8}x, & x\in(0; 4).\end{cases}$$ Найдите начальные и центральные моменты первого, второго и третьего порядков.
 

\vfill

\newpage\setcounter{zad}{0}

\z Случайная величина $X$ задана плотностью распределения: $$f(x) = \begin{cases} -\frac{3}{2}(x-1)^2, & x \in (0; 1), \\ 0, & x \not\in (0; 1). \end{cases}$$ Найдите моду, медиану и математическое ожидание этой случайной величины.


\vfill

\z Дискретная случайная величина $X$ задана законом распределения: $$ X = \left(\begin{array}{rrr}0 & 1 & 3\\0{,}7 & 0{,}2 & 0{,}1\end{array}\right).$$ Найдите начальные и центральные моменты первого, второго и третьего порядков.
 

\vfill

\newpage\setcounter{zad}{0}

\z Случайная величина $X$ задана плотностью распределения: $$f(x) = \begin{cases} 3(x+1)^2, & x \in (-1; 0), \\ 0, & x \not\in (-1; 0). \end{cases}$$ Найдите моду, медиану и математическое ожидание этой случайной величины.


\vfill

\z Дискретная случайная величина $X$ задана законом распределения: $$ X = \left(\begin{array}{rrr}1 & 3 & 4\\0{,}6 & 0{,}1 & 0{,}3\end{array}\right).$$ Найдите начальные и центральные моменты первого, второго и третьего порядков.
 

\vfill

\newpage\setcounter{zad}{0}

\z Случайная величина $X$ задана плотностью распределения: $$f(x) = \begin{cases} \frac{1}{9}(x-3)^2, & x \in (0; 3), \\ 0, & x \not\in (0; 3). \end{cases}$$ Найдите моду, медиану и математическое ожидание этой случайной величины.


\vfill

\z Дискретная случайная величина $X$ задана законом распределения: $$ X = \left(\begin{array}{rrr}0 & 1 & 2\\0{,}2 & 0{,}7 & 0{,}1\end{array}\right).$$ Найдите начальные и центральные моменты первого, второго и третьего порядков.
 

\vfill

\newpage\setcounter{zad}{0}

\z Случайная величина $X$ задана плотностью распределения: $$f(x) = \begin{cases} -\frac{3}{2}(x-1)^2, & x \in (0; 1), \\ 0, & x \not\in (0; 1). \end{cases}$$ Найдите моду, медиану и математическое ожидание этой случайной величины.


\vfill

\z Случайная величина $X$ задана плотностью распределения $$f(x) = \begin{cases}0, & x\not\in(4; 5), \\ \frac{2}{9}x, & x\in(4; 5).\end{cases}$$ Найдите начальные и центральные моменты первого, второго и третьего порядков.
 

\vfill

\newpage\setcounter{zad}{0}

\z Случайная величина $X$ задана плотностью распределения: $$f(x) = \begin{cases} \frac{1}{9}(x+3)^2, & x \in (-3; 0), \\ 0, & x \not\in (-3; 0). \end{cases}$$ Найдите моду, медиану и математическое ожидание этой случайной величины.


\vfill

\z Случайная величина $X$ задана плотностью распределения $$f(x) = \begin{cases}0, & x\not\in(3; 4), \\ \frac{2}{7}x, & x\in(3; 4).\end{cases}$$ Найдите начальные и центральные моменты первого, второго и третьего порядков.
 

\vfill

\newpage\setcounter{zad}{0}

\z Случайная величина $X$ задана плотностью распределения: $$f(x) = \begin{cases} 3(x-1)^2, & x \in (0; 1), \\ 0, & x \not\in (0; 1). \end{cases}$$ Найдите моду, медиану и математическое ожидание этой случайной величины.


\vfill

\z Случайная величина $X$ задана плотностью распределения $$f(x) = \begin{cases}0, & x\not\in(1; 5), \\ \frac{1}{12}x, & x\in(1; 5).\end{cases}$$ Найдите начальные и центральные моменты первого, второго и третьего порядков.
 

\vfill

\newpage\setcounter{zad}{0}

\z Случайная величина $X$ задана плотностью распределения: $$f(x) = \begin{cases} -\frac{3}{16}(x-2)^2, & x \in (0; 2), \\ 0, & x \not\in (0; 2). \end{cases}$$ Найдите моду, медиану и математическое ожидание этой случайной величины.


\vfill

\z Дискретная случайная величина $X$ задана законом распределения: $$ X = \left(\begin{array}{rrr}-2 & 1 & 4\\0{,}2 & 0{,}1 & 0{,}7\end{array}\right).$$ Найдите начальные и центральные моменты первого, второго и третьего порядков.
 

\vfill

\newpage\setcounter{zad}{0}

\z Случайная величина $X$ задана плотностью распределения: $$f(x) = \begin{cases} -\frac{3}{16}(x+2)^2, & x \in (-2; 0), \\ 0, & x \not\in (-2; 0). \end{cases}$$ Найдите моду, медиану и математическое ожидание этой случайной величины.


\vfill

\z Дискретная случайная величина $X$ задана законом распределения: $$ X = \left(\begin{array}{rrr}1 & 4 & 5\\0{,}1 & 0{,}3 & 0{,}6\end{array}\right).$$ Найдите начальные и центральные моменты первого, второго и третьего порядков.
 

\vfill

\newpage\setcounter{zad}{0}

\z Случайная величина $X$ задана плотностью распределения: $$f(x) = \begin{cases} 3(x+1)^2, & x \in (-1; 0), \\ 0, & x \not\in (-1; 0). \end{cases}$$ Найдите моду, медиану и математическое ожидание этой случайной величины.


\vfill

\z Дискретная случайная величина $X$ задана законом распределения: $$ X = \left(\begin{array}{rrr}3 & 4 & 6\\0{,}6 & 0{,}2 & 0{,}2\end{array}\right).$$ Найдите начальные и центральные моменты первого, второго и третьего порядков.
 

\vfill

\newpage\setcounter{zad}{0}

\z Случайная величина $X$ задана плотностью распределения: $$f(x) = \begin{cases} -\frac{3}{2}(x-1)^2, & x \in (0; 1), \\ 0, & x \not\in (0; 1). \end{cases}$$ Найдите моду, медиану и математическое ожидание этой случайной величины.


\vfill

\z Дискретная случайная величина $X$ задана законом распределения: $$ X = \left(\begin{array}{rrr}4 & 5 & 6\\0{,}7 & 0{,}1 & 0{,}2\end{array}\right).$$ Найдите начальные и центральные моменты первого, второго и третьего порядков.
 

\vfill

\newpage\setcounter{zad}{0}

\z Случайная величина $X$ задана плотностью распределения: $$f(x) = \begin{cases} -\frac{3}{16}(x-2)^2, & x \in (0; 2), \\ 0, & x \not\in (0; 2). \end{cases}$$ Найдите моду, медиану и математическое ожидание этой случайной величины.


\vfill

\z Дискретная случайная величина $X$ задана законом распределения: $$ X = \left(\begin{array}{rrr}0 & 2 & 4\\0{,}5 & 0{,}2 & 0{,}3\end{array}\right).$$ Найдите начальные и центральные моменты первого, второго и третьего порядков.
 

\vfill

\newpage\setcounter{zad}{0}

\z Случайная величина $X$ задана плотностью распределения: $$f(x) = \begin{cases} -\frac{1}{18}(x+3)^2, & x \in (-3; 0), \\ 0, & x \not\in (-3; 0). \end{cases}$$ Найдите моду, медиану и математическое ожидание этой случайной величины.


\vfill

\z Дискретная случайная величина $X$ задана законом распределения: $$ X = \left(\begin{array}{rrr}1 & 4 & 7\\0{,}8 & 0{,}1 & 0{,}1\end{array}\right).$$ Найдите начальные и центральные моменты первого, второго и третьего порядков.
 

\vfill

\newpage\setcounter{zad}{0}

\z Случайная величина $X$ задана плотностью распределения: $$f(x) = \begin{cases} \frac{3}{8}(x+2)^2, & x \in (-2; 0), \\ 0, & x \not\in (-2; 0). \end{cases}$$ Найдите моду, медиану и математическое ожидание этой случайной величины.


\vfill

\z Дискретная случайная величина $X$ задана законом распределения: $$ X = \left(\begin{array}{rrr}-3 & -1 & 1\\0{,}3 & 0{,}3 & 0{,}4\end{array}\right).$$ Найдите начальные и центральные моменты первого, второго и третьего порядков.
 

\vfill

\newpage\setcounter{zad}{0}

\z Случайная величина $X$ задана плотностью распределения: $$f(x) = \begin{cases} 3(x+1)^2, & x \in (-1; 0), \\ 0, & x \not\in (-1; 0). \end{cases}$$ Найдите моду, медиану и математическое ожидание этой случайной величины.


\vfill

\z Случайная величина $X$ задана плотностью распределения $$f(x) = \begin{cases}0, & x\not\in(2; 4), \\ \frac{1}{6}x, & x\in(2; 4).\end{cases}$$ Найдите начальные и центральные моменты первого, второго и третьего порядков.
 

\vfill

\newpage\setcounter{zad}{0}

\z Случайная величина $X$ задана плотностью распределения: $$f(x) = \begin{cases} 3(x+1)^2, & x \in (-1; 0), \\ 0, & x \not\in (-1; 0). \end{cases}$$ Найдите моду, медиану и математическое ожидание этой случайной величины.


\vfill

\z Случайная величина $X$ задана плотностью распределения $$f(x) = \begin{cases}0, & x\not\in(3; 6), \\ \frac{2}{27}x, & x\in(3; 6).\end{cases}$$ Найдите начальные и центральные моменты первого, второго и третьего порядков.
 

\vfill

\newpage\setcounter{zad}{0}

\z Случайная величина $X$ задана плотностью распределения: $$f(x) = \begin{cases} \frac{3}{8}(x-2)^2, & x \in (0; 2), \\ 0, & x \not\in (0; 2). \end{cases}$$ Найдите моду, медиану и математическое ожидание этой случайной величины.


\vfill

\z Дискретная случайная величина $X$ задана законом распределения: $$ X = \left(\begin{array}{rrr}2 & 3 & 6\\0{,}1 & 0{,}5 & 0{,}4\end{array}\right).$$ Найдите начальные и центральные моменты первого, второго и третьего порядков.
 

\vfill

\newpage\setcounter{zad}{0}

\z Случайная величина $X$ задана плотностью распределения: $$f(x) = \begin{cases} -\frac{3}{16}(x-2)^2, & x \in (0; 2), \\ 0, & x \not\in (0; 2). \end{cases}$$ Найдите моду, медиану и математическое ожидание этой случайной величины.


\vfill

\z Случайная величина $X$ задана плотностью распределения $$f(x) = \begin{cases}0, & x\not\in(4; 8), \\ \frac{1}{24}x, & x\in(4; 8).\end{cases}$$ Найдите начальные и центральные моменты первого, второго и третьего порядков.
 

\vfill

\newpage\setcounter{zad}{0}

\z Случайная величина $X$ задана плотностью распределения: $$f(x) = \begin{cases} 3(x-1)^2, & x \in (0; 1), \\ 0, & x \not\in (0; 1). \end{cases}$$ Найдите моду, медиану и математическое ожидание этой случайной величины.


\vfill

\z Дискретная случайная величина $X$ задана законом распределения: $$ X = \left(\begin{array}{rrr}-1 & 1 & 2\\0{,}8 & 0{,}1 & 0{,}1\end{array}\right).$$ Найдите начальные и центральные моменты первого, второго и третьего порядков.
 

\vfill

\newpage\setcounter{zad}{0}

\z Случайная величина $X$ задана плотностью распределения: $$f(x) = \begin{cases} \frac{3}{8}(x-2)^2, & x \in (0; 2), \\ 0, & x \not\in (0; 2). \end{cases}$$ Найдите моду, медиану и математическое ожидание этой случайной величины.


\vfill

\z Случайная величина $X$ задана плотностью распределения $$f(x) = \begin{cases}0, & x\not\in(0; 3), \\ \frac{2}{9}x, & x\in(0; 3).\end{cases}$$ Найдите начальные и центральные моменты первого, второго и третьего порядков.
 

\vfill

\newpage\setcounter{zad}{0}

\z Случайная величина $X$ задана плотностью распределения: $$f(x) = \begin{cases} 3(x+1)^2, & x \in (-1; 0), \\ 0, & x \not\in (-1; 0). \end{cases}$$ Найдите моду, медиану и математическое ожидание этой случайной величины.


\vfill

\z Дискретная случайная величина $X$ задана законом распределения: $$ X = \left(\begin{array}{rrr}0 & 3 & 6\\0{,}3 & 0{,}5 & 0{,}2\end{array}\right).$$ Найдите начальные и центральные моменты первого, второго и третьего порядков.
 

\vfill

\newpage\setcounter{zad}{0}

\z Случайная величина $X$ задана плотностью распределения: $$f(x) = \begin{cases} \frac{3}{8}(x+2)^2, & x \in (-2; 0), \\ 0, & x \not\in (-2; 0). \end{cases}$$ Найдите моду, медиану и математическое ожидание этой случайной величины.


\vfill

\z Дискретная случайная величина $X$ задана законом распределения: $$ X = \left(\begin{array}{rrr}0 & 3 & 4\\0{,}1 & 0{,}2 & 0{,}7\end{array}\right).$$ Найдите начальные и центральные моменты первого, второго и третьего порядков.
 

\vfill

\newpage\setcounter{zad}{0}

\z Случайная величина $X$ задана плотностью распределения: $$f(x) = \begin{cases} -\frac{1}{18}(x+3)^2, & x \in (-3; 0), \\ 0, & x \not\in (-3; 0). \end{cases}$$ Найдите моду, медиану и математическое ожидание этой случайной величины.


\vfill

\z Случайная величина $X$ задана плотностью распределения $$f(x) = \begin{cases}0, & x\not\in(2; 4), \\ \frac{1}{6}x, & x\in(2; 4).\end{cases}$$ Найдите начальные и центральные моменты первого, второго и третьего порядков.
 

\vfill

\newpage\setcounter{zad}{0}

\z Случайная величина $X$ задана плотностью распределения: $$f(x) = \begin{cases} -\frac{3}{2}(x+1)^2, & x \in (-1; 0), \\ 0, & x \not\in (-1; 0). \end{cases}$$ Найдите моду, медиану и математическое ожидание этой случайной величины.


\vfill

\z Дискретная случайная величина $X$ задана законом распределения: $$ X = \left(\begin{array}{rrr}-4 & -3 & -2\\0{,}7 & 0{,}2 & 0{,}1\end{array}\right).$$ Найдите начальные и центральные моменты первого, второго и третьего порядков.
 

\vfill

\newpage\setcounter{zad}{0}

\z Случайная величина $X$ задана плотностью распределения: $$f(x) = \begin{cases} \frac{1}{9}(x-3)^2, & x \in (0; 3), \\ 0, & x \not\in (0; 3). \end{cases}$$ Найдите моду, медиану и математическое ожидание этой случайной величины.


\vfill

\z Случайная величина $X$ задана плотностью распределения $$f(x) = \begin{cases}0, & x\not\in(4; 6), \\ \frac{1}{10}x, & x\in(4; 6).\end{cases}$$ Найдите начальные и центральные моменты первого, второго и третьего порядков.
 

\vfill

\newpage\setcounter{zad}{0}

\z Случайная величина $X$ задана плотностью распределения: $$f(x) = \begin{cases} \frac{3}{8}(x+2)^2, & x \in (-2; 0), \\ 0, & x \not\in (-2; 0). \end{cases}$$ Найдите моду, медиану и математическое ожидание этой случайной величины.


\vfill

\z Дискретная случайная величина $X$ задана законом распределения: $$ X = \left(\begin{array}{rrr}-2 & -1 & 2\\0{,}3 & 0{,}2 & 0{,}5\end{array}\right).$$ Найдите начальные и центральные моменты первого, второго и третьего порядков.
 

\vfill

\newpage\setcounter{zad}{0}

\z Случайная величина $X$ задана плотностью распределения: $$f(x) = \begin{cases} 3(x+1)^2, & x \in (-1; 0), \\ 0, & x \not\in (-1; 0). \end{cases}$$ Найдите моду, медиану и математическое ожидание этой случайной величины.


\vfill

\z Случайная величина $X$ задана плотностью распределения $$f(x) = \begin{cases}0, & x\not\in(2; 3), \\ \frac{2}{5}x, & x\in(2; 3).\end{cases}$$ Найдите начальные и центральные моменты первого, второго и третьего порядков.
 

\vfill

\newpage\setcounter{zad}{0}

\z Случайная величина $X$ задана плотностью распределения: $$f(x) = \begin{cases} \frac{3}{8}(x-2)^2, & x \in (0; 2), \\ 0, & x \not\in (0; 2). \end{cases}$$ Найдите моду, медиану и математическое ожидание этой случайной величины.


\vfill

\z Дискретная случайная величина $X$ задана законом распределения: $$ X = \left(\begin{array}{rrr}0 & 1 & 2\\0{,}1 & 0{,}1 & 0{,}8\end{array}\right).$$ Найдите начальные и центральные моменты первого, второго и третьего порядков.
 

\vfill

\newpage\setcounter{zad}{0}

\z Случайная величина $X$ задана плотностью распределения: $$f(x) = \begin{cases} \frac{3}{8}(x-2)^2, & x \in (0; 2), \\ 0, & x \not\in (0; 2). \end{cases}$$ Найдите моду, медиану и математическое ожидание этой случайной величины.


\vfill

\z Случайная величина $X$ задана плотностью распределения $$f(x) = \begin{cases}0, & x\not\in(3; 4), \\ \frac{2}{7}x, & x\in(3; 4).\end{cases}$$ Найдите начальные и центральные моменты первого, второго и третьего порядков.
 

\vfill

\newpage\setcounter{zad}{0}

\z Случайная величина $X$ задана плотностью распределения: $$f(x) = \begin{cases} -\frac{3}{2}(x-1)^2, & x \in (0; 1), \\ 0, & x \not\in (0; 1). \end{cases}$$ Найдите моду, медиану и математическое ожидание этой случайной величины.


\vfill

\z Дискретная случайная величина $X$ задана законом распределения: $$ X = \left(\begin{array}{rrr}4 & 7 & 8\\0{,}8 & 0{,}1 & 0{,}1\end{array}\right).$$ Найдите начальные и центральные моменты первого, второго и третьего порядков.
 

\vfill

\newpage\setcounter{zad}{0}

\z Случайная величина $X$ задана плотностью распределения: $$f(x) = \begin{cases} -\frac{1}{18}(x-3)^2, & x \in (0; 3), \\ 0, & x \not\in (0; 3). \end{cases}$$ Найдите моду, медиану и математическое ожидание этой случайной величины.


\vfill

\z Дискретная случайная величина $X$ задана законом распределения: $$ X = \left(\begin{array}{rrr}0 & 3 & 6\\0{,}8 & 0{,}1 & 0{,}1\end{array}\right).$$ Найдите начальные и центральные моменты первого, второго и третьего порядков.
 

\vfill

\newpage\setcounter{zad}{0}

\z Случайная величина $X$ задана плотностью распределения: $$f(x) = \begin{cases} \frac{1}{9}(x-3)^2, & x \in (0; 3), \\ 0, & x \not\in (0; 3). \end{cases}$$ Найдите моду, медиану и математическое ожидание этой случайной величины.


\vfill

\z Случайная величина $X$ задана плотностью распределения $$f(x) = \begin{cases}0, & x\not\in(0; 1), \\ \frac{2}{1}x, & x\in(0; 1).\end{cases}$$ Найдите начальные и центральные моменты первого, второго и третьего порядков.
 

\vfill

\newpage\setcounter{zad}{0}

\z Случайная величина $X$ задана плотностью распределения: $$f(x) = \begin{cases} -\frac{3}{16}(x-2)^2, & x \in (0; 2), \\ 0, & x \not\in (0; 2). \end{cases}$$ Найдите моду, медиану и математическое ожидание этой случайной величины.


\vfill

\z Случайная величина $X$ задана плотностью распределения $$f(x) = \begin{cases}0, & x\not\in(0; 2), \\ \frac{1}{2}x, & x\in(0; 2).\end{cases}$$ Найдите начальные и центральные моменты первого, второго и третьего порядков.
 

\vfill

\newpage\setcounter{zad}{0}

\z Случайная величина $X$ задана плотностью распределения: $$f(x) = \begin{cases} \frac{3}{8}(x-2)^2, & x \in (0; 2), \\ 0, & x \not\in (0; 2). \end{cases}$$ Найдите моду, медиану и математическое ожидание этой случайной величины.


\vfill

\z Случайная величина $X$ задана плотностью распределения $$f(x) = \begin{cases}0, & x\not\in(4; 6), \\ \frac{1}{10}x, & x\in(4; 6).\end{cases}$$ Найдите начальные и центральные моменты первого, второго и третьего порядков.
 

\vfill

\newpage\setcounter{zad}{0}

\z Случайная величина $X$ задана плотностью распределения: $$f(x) = \begin{cases} -\frac{3}{2}(x+1)^2, & x \in (-1; 0), \\ 0, & x \not\in (-1; 0). \end{cases}$$ Найдите моду, медиану и математическое ожидание этой случайной величины.


\vfill

\z Случайная величина $X$ задана плотностью распределения $$f(x) = \begin{cases}0, & x\not\in(3; 7), \\ \frac{1}{20}x, & x\in(3; 7).\end{cases}$$ Найдите начальные и центральные моменты первого, второго и третьего порядков.
 

\vfill

\newpage\setcounter{zad}{0}

\z Случайная величина $X$ задана плотностью распределения: $$f(x) = \begin{cases} \frac{3}{8}(x-2)^2, & x \in (0; 2), \\ 0, & x \not\in (0; 2). \end{cases}$$ Найдите моду, медиану и математическое ожидание этой случайной величины.


\vfill

\z Случайная величина $X$ задана плотностью распределения $$f(x) = \begin{cases}0, & x\not\in(3; 7), \\ \frac{1}{20}x, & x\in(3; 7).\end{cases}$$ Найдите начальные и центральные моменты первого, второго и третьего порядков.
 

\vfill

\newpage\setcounter{zad}{0}

\z Случайная величина $X$ задана плотностью распределения: $$f(x) = \begin{cases} 3(x-1)^2, & x \in (0; 1), \\ 0, & x \not\in (0; 1). \end{cases}$$ Найдите моду, медиану и математическое ожидание этой случайной величины.


\vfill

\z Дискретная случайная величина $X$ задана законом распределения: $$ X = \left(\begin{array}{rrr}3 & 5 & 7\\0{,}4 & 0{,}5 & 0{,}1\end{array}\right).$$ Найдите начальные и центральные моменты первого, второго и третьего порядков.
 

\vfill

\newpage\setcounter{zad}{0}

\z Случайная величина $X$ задана плотностью распределения: $$f(x) = \begin{cases} 3(x+1)^2, & x \in (-1; 0), \\ 0, & x \not\in (-1; 0). \end{cases}$$ Найдите моду, медиану и математическое ожидание этой случайной величины.


\vfill

\z Дискретная случайная величина $X$ задана законом распределения: $$ X = \left(\begin{array}{rrr}4 & 6 & 9\\0{,}6 & 0{,}1 & 0{,}3\end{array}\right).$$ Найдите начальные и центральные моменты первого, второго и третьего порядков.
 

\vfill

\newpage\setcounter{zad}{0}

\z Случайная величина $X$ задана плотностью распределения: $$f(x) = \begin{cases} -\frac{3}{2}(x+1)^2, & x \in (-1; 0), \\ 0, & x \not\in (-1; 0). \end{cases}$$ Найдите моду, медиану и математическое ожидание этой случайной величины.


\vfill

\z Дискретная случайная величина $X$ задана законом распределения: $$ X = \left(\begin{array}{rrr}3 & 6 & 9\\0{,}4 & 0{,}2 & 0{,}4\end{array}\right).$$ Найдите начальные и центральные моменты первого, второго и третьего порядков.
 

\vfill

\newpage\setcounter{zad}{0}

\z Случайная величина $X$ задана плотностью распределения: $$f(x) = \begin{cases} \frac{3}{8}(x+2)^2, & x \in (-2; 0), \\ 0, & x \not\in (-2; 0). \end{cases}$$ Найдите моду, медиану и математическое ожидание этой случайной величины.


\vfill

\z Дискретная случайная величина $X$ задана законом распределения: $$ X = \left(\begin{array}{rrr}4 & 6 & 9\\0{,}7 & 0{,}2 & 0{,}1\end{array}\right).$$ Найдите начальные и центральные моменты первого, второго и третьего порядков.
 

\vfill

\newpage\setcounter{zad}{0}

\z Случайная величина $X$ задана плотностью распределения: $$f(x) = \begin{cases} 3(x-1)^2, & x \in (0; 1), \\ 0, & x \not\in (0; 1). \end{cases}$$ Найдите моду, медиану и математическое ожидание этой случайной величины.


\vfill

\z Случайная величина $X$ задана плотностью распределения $$f(x) = \begin{cases}0, & x\not\in(4; 6), \\ \frac{1}{10}x, & x\in(4; 6).\end{cases}$$ Найдите начальные и центральные моменты первого, второго и третьего порядков.
 

\vfill

\newpage\setcounter{zad}{0}

\z Случайная величина $X$ задана плотностью распределения: $$f(x) = \begin{cases} 3(x-1)^2, & x \in (0; 1), \\ 0, & x \not\in (0; 1). \end{cases}$$ Найдите моду, медиану и математическое ожидание этой случайной величины.


\vfill

\z Случайная величина $X$ задана плотностью распределения $$f(x) = \begin{cases}0, & x\not\in(2; 5), \\ \frac{2}{21}x, & x\in(2; 5).\end{cases}$$ Найдите начальные и центральные моменты первого, второго и третьего порядков.
 

\vfill

\newpage\setcounter{zad}{0}

\z Случайная величина $X$ задана плотностью распределения: $$f(x) = \begin{cases} -\frac{1}{18}(x+3)^2, & x \in (-3; 0), \\ 0, & x \not\in (-3; 0). \end{cases}$$ Найдите моду, медиану и математическое ожидание этой случайной величины.


\vfill

\z Дискретная случайная величина $X$ задана законом распределения: $$ X = \left(\begin{array}{rrr}4 & 6 & 8\\0{,}8 & 0{,}1 & 0{,}1\end{array}\right).$$ Найдите начальные и центральные моменты первого, второго и третьего порядков.
 

\vfill

\newpage\setcounter{zad}{0}

\z Случайная величина $X$ задана плотностью распределения: $$f(x) = \begin{cases} 3(x+1)^2, & x \in (-1; 0), \\ 0, & x \not\in (-1; 0). \end{cases}$$ Найдите моду, медиану и математическое ожидание этой случайной величины.


\vfill

\z Случайная величина $X$ задана плотностью распределения $$f(x) = \begin{cases}0, & x\not\in(0; 2), \\ \frac{1}{2}x, & x\in(0; 2).\end{cases}$$ Найдите начальные и центральные моменты первого, второго и третьего порядков.
 

\vfill

\newpage\setcounter{zad}{0}