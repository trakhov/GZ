

\z Непрерывная случайная величина $X$ на всей числовой оси задана своей функцией распределения\footnote{$\Phi(x)$ --- функция Лапласа.}: $$ F(x) = \frac{1}{2} + \frac{1}{2}\Phi\left( \frac{x - 50}{30} \right). $$ Найдите: а) вероятность $P(-92 < X < -9)$; б) $30\%$-ную точку; в) медиану $Me(X)$.


\vfill

\z Непрерывная случайная величина $X$ задана плотностью вероятности: $$ f(x) = \begin{cases}0, & x < 0 \\ C\mathrm{e}^{-15x}, & x \geqslant 0.\end{cases} $$ Найдите а) параметр $C$; б) вероятность $P(X < 0{,}57)$; в) математическое ожидание $M(X)$.
 

\vfill

\newpage\setcounter{zad}{0}

\z Непрерывная случайная величина $X$ на всей числовой оси задана своей функцией распределения\footnote{$\Phi(x)$ --- функция Лапласа.}: $$ F(x) = \frac{1}{2} + \frac{1}{2}\Phi\left( \frac{x + 40}{30} \right). $$ Найдите: а) вероятность $P(-1 < X < 89)$; б) квантиль уровня $0{,}7$; в) медиану $Me(X)$.


\vfill

\z Непрерывная случайная величина $X$ задана плотностью вероятности: $$ f(x) = \begin{cases}0, & x < 0 \\ C\mathrm{e}^{-8x}, & x \geqslant 0.\end{cases} $$ Найдите а) параметр $C$; б) вероятность $P(X < 0{,}9)$; в) математическое ожидание $M(X)$.
 

\vfill

\newpage\setcounter{zad}{0}

\z Непрерывная случайная величина $X$ на всей числовой оси задана своей функцией распределения\footnote{$\Phi(x)$ --- функция Лапласа.}: $$ F(x) = \frac{1}{2} + \frac{1}{2}\Phi\left( \frac{x + 80}{20} \right). $$ Найдите: а) вероятность $P(42 < X < 118)$; б) квантиль уровня $0{,}7$; в) медиану $Me(X)$.


\vfill

\z Непрерывная случайная величина $X$ задана плотностью вероятности: $$ f(x) = \begin{cases}0, & x < 0 \\ C\mathrm{e}^{-7x}, & x \geqslant 0.\end{cases} $$ Найдите а) параметр $C$; б) вероятность $P(X < 0{,}76)$; в) математическое ожидание $M(X)$.
 

\vfill

\newpage\setcounter{zad}{0}

\z Непрерывная случайная величина $X$ на всей числовой оси задана своей функцией распределения\footnote{$\Phi(x)$ --- функция Лапласа.}: $$ F(x) = \frac{1}{2} + \frac{1}{2}\Phi\left( \frac{x - 20}{50} \right). $$ Найдите: а) вероятность $P(-101 < X < 71)$; б) квантиль уровня $0{,}2$; в) медиану $Me(X)$.


\vfill

\z Непрерывная случайная величина $X$ задана плотностью вероятности: $$ f(x) = \begin{cases}0, & x < 0 \\ C\mathrm{e}^{-2x}, & x \geqslant 0.\end{cases} $$ Найдите а) параметр $C$; б) вероятность $P(X < 1{,}05)$; в) математическое ожидание $M(X)$.
 

\vfill

\newpage\setcounter{zad}{0}

\z Непрерывная случайная величина $X$ на всей числовой оси задана своей функцией распределения\footnote{$\Phi(x)$ --- функция Лапласа.}: $$ F(x) = \frac{1}{2} + \frac{1}{2}\Phi\left( \frac{x + 10}{2} \right). $$ Найдите: а) вероятность $P(7 < X < 14)$; б) квантиль уровня $0{,}7$; в) медиану $Me(X)$.


\vfill

\z Непрерывная случайная величина $X$ задана плотностью вероятности: $$ f(x) = \begin{cases}0, & x < 0 \\ C\mathrm{e}^{-2x}, & x \geqslant 0.\end{cases} $$ Найдите а) параметр $C$; б) вероятность $P(X < 1{,}05)$; в) математическое ожидание $M(X)$.
 

\vfill

\newpage\setcounter{zad}{0}

\z Непрерывная случайная величина $X$ на всей числовой оси задана своей функцией распределения\footnote{$\Phi(x)$ --- функция Лапласа.}: $$ F(x) = \frac{1}{2} + \frac{1}{2}\Phi\left( \frac{x + 30}{40} \right). $$ Найдите: а) вероятность $P(-25 < X < 102)$; б) квантиль уровня $0{,}7$; в) медиану $Me(X)$.


\vfill

\z Непрерывная случайная величина $X$ задана плотностью вероятности: $$ f(x) = \begin{cases}0, & x < 0 \\ C\mathrm{e}^{-7x}, & x \geqslant 0.\end{cases} $$ Найдите а) параметр $C$; б) вероятность $P(X < 1{,}01)$; в) математическое ожидание $M(X)$.
 

\vfill

\newpage\setcounter{zad}{0}

\z Непрерывная случайная величина $X$ на всей числовой оси задана своей функцией распределения\footnote{$\Phi(x)$ --- функция Лапласа.}: $$ F(x) = \frac{1}{2} + \frac{1}{2}\Phi\left( \frac{x - 130}{40} \right). $$ Найдите: а) вероятность $P(-187 < X < -62)$; б) $90\%$-ную точку; в) медиану $Me(X)$.


\vfill

\z Непрерывная случайная величина $X$ задана плотностью вероятности: $$ f(x) = \begin{cases}0, & x < 0 \\ C\mathrm{e}^{-2x}, & x \geqslant 0.\end{cases} $$ Найдите а) параметр $C$; б) вероятность $P(X < 0{,}05)$; в) математическое ожидание $M(X)$.
 

\vfill

\newpage\setcounter{zad}{0}

\z Непрерывная случайная величина $X$ на всей числовой оси задана своей функцией распределения\footnote{$\Phi(x)$ --- функция Лапласа.}: $$ F(x) = \frac{1}{2} + \frac{1}{2}\Phi\left( \frac{x + 30}{40} \right). $$ Найдите: а) вероятность $P(-27 < X < 93)$; б) $70\%$-ную точку; в) медиану $Me(X)$.


\vfill

\z Непрерывная случайная величина $X$ задана плотностью вероятности: $$ f(x) = \begin{cases}0, & x < 0 \\ C\mathrm{e}^{-4x}, & x \geqslant 0.\end{cases} $$ Найдите а) параметр $C$; б) вероятность $P(X < 0{,}05)$; в) математическое ожидание $M(X)$.
 

\vfill

\newpage\setcounter{zad}{0}

\z Непрерывная случайная величина $X$ на всей числовой оси задана своей функцией распределения\footnote{$\Phi(x)$ --- функция Лапласа.}: $$ F(x) = \frac{1}{2} + \frac{1}{2}\Phi\left( \frac{x - 90}{20} \right). $$ Найдите: а) вероятность $P(-129 < X < -58)$; б) квантиль уровня $0{,}8$; в) медиану $Me(X)$.


\vfill

\z Непрерывная случайная величина $X$ задана плотностью вероятности: $$ f(x) = \begin{cases}0, & x < 0 \\ C\mathrm{e}^{-3x}, & x \geqslant 0.\end{cases} $$ Найдите а) параметр $C$; б) вероятность $P(X < 0{,}05)$; в) математическое ожидание $M(X)$.
 

\vfill

\newpage\setcounter{zad}{0}

\z Непрерывная случайная величина $X$ на всей числовой оси задана своей функцией распределения\footnote{$\Phi(x)$ --- функция Лапласа.}: $$ F(x) = \frac{1}{2} + \frac{1}{2}\Phi\left( \frac{x + 10}{4} \right). $$ Найдите: а) вероятность $P(5 < X < 15)$; б) квантиль уровня $0{,}7$; в) медиану $Me(X)$.


\vfill

\z Непрерывная случайная величина $X$ задана плотностью вероятности: $$ f(x) = \begin{cases}0, & x < 0 \\ C\mathrm{e}^{-13x}, & x \geqslant 0.\end{cases} $$ Найдите а) параметр $C$; б) вероятность $P(X < 0{,}18)$; в) математическое ожидание $M(X)$.
 

\vfill

\newpage\setcounter{zad}{0}

\z Непрерывная случайная величина $X$ на всей числовой оси задана своей функцией распределения\footnote{$\Phi(x)$ --- функция Лапласа.}: $$ F(x) = \frac{1}{2} + \frac{1}{2}\Phi\left( \frac{x - 40}{30} \right). $$ Найдите: а) вероятность $P(-89 < X < 15)$; б) квантиль уровня $0{,}9$; в) медиану $Me(X)$.


\vfill

\z Непрерывная случайная величина $X$ задана плотностью вероятности: $$ f(x) = \begin{cases}0, & x < 0 \\ C\mathrm{e}^{-8x}, & x \geqslant 0.\end{cases} $$ Найдите а) параметр $C$; б) вероятность $P(X < 0{,}49)$; в) математическое ожидание $M(X)$.
 

\vfill

\newpage\setcounter{zad}{0}

\z Непрерывная случайная величина $X$ на всей числовой оси задана своей функцией распределения\footnote{$\Phi(x)$ --- функция Лапласа.}: $$ F(x) = \frac{1}{2} + \frac{1}{2}\Phi\left( \frac{x + 14}{5} \right). $$ Найдите: а) вероятность $P(5 < X < 21)$; б) квантиль уровня $0{,}9$; в) медиану $Me(X)$.


\vfill

\z Непрерывная случайная величина $X$ задана плотностью вероятности: $$ f(x) = \begin{cases}0, & x < 0 \\ C\mathrm{e}^{-8x}, & x \geqslant 0.\end{cases} $$ Найдите а) параметр $C$; б) вероятность $P(X < 0{,}07)$; в) математическое ожидание $M(X)$.
 

\vfill

\newpage\setcounter{zad}{0}

\z Непрерывная случайная величина $X$ на всей числовой оси задана своей функцией распределения\footnote{$\Phi(x)$ --- функция Лапласа.}: $$ F(x) = \frac{1}{2} + \frac{1}{2}\Phi\left( \frac{x - 9}{5} \right). $$ Найдите: а) вероятность $P(-18 < X < 1)$; б) квантиль уровня $0{,}6$; в) медиану $Me(X)$.


\vfill

\z Непрерывная случайная величина $X$ задана плотностью вероятности: $$ f(x) = \begin{cases}0, & x < 0 \\ C\mathrm{e}^{-11x}, & x \geqslant 0.\end{cases} $$ Найдите а) параметр $C$; б) вероятность $P(X < 0{,}74)$; в) математическое ожидание $M(X)$.
 

\vfill

\newpage\setcounter{zad}{0}

\z Непрерывная случайная величина $X$ на всей числовой оси задана своей функцией распределения\footnote{$\Phi(x)$ --- функция Лапласа.}: $$ F(x) = \frac{1}{2} + \frac{1}{2}\Phi\left( \frac{x - 6}{5} \right). $$ Найдите: а) вероятность $P(-14 < X < 2)$; б) квантиль уровня $0{,}5$; в) медиану $Me(X)$.


\vfill

\z Непрерывная случайная величина $X$ задана плотностью вероятности: $$ f(x) = \begin{cases}0, & x < 0 \\ C\mathrm{e}^{-8x}, & x \geqslant 0.\end{cases} $$ Найдите а) параметр $C$; б) вероятность $P(X < 0{,}63)$; в) математическое ожидание $M(X)$.
 

\vfill

\newpage\setcounter{zad}{0}

\z Непрерывная случайная величина $X$ на всей числовой оси задана своей функцией распределения\footnote{$\Phi(x)$ --- функция Лапласа.}: $$ F(x) = \frac{1}{2} + \frac{1}{2}\Phi\left( \frac{x + 60}{40} \right). $$ Найдите: а) вероятность $P(-16 < X < 120)$; б) квантиль уровня $0{,}1$; в) медиану $Me(X)$.


\vfill

\z Непрерывная случайная величина $X$ задана плотностью вероятности: $$ f(x) = \begin{cases}0, & x < 0 \\ C\mathrm{e}^{-13x}, & x \geqslant 0.\end{cases} $$ Найдите а) параметр $C$; б) вероятность $P(X < 0{,}22)$; в) математическое ожидание $M(X)$.
 

\vfill

\newpage\setcounter{zad}{0}

\z Непрерывная случайная величина $X$ на всей числовой оси задана своей функцией распределения\footnote{$\Phi(x)$ --- функция Лапласа.}: $$ F(x) = \frac{1}{2} + \frac{1}{2}\Phi\left( \frac{x + 70}{20} \right). $$ Найдите: а) вероятность $P(34 < X < 107)$; б) квантиль уровня $0{,}2$; в) медиану $Me(X)$.


\vfill

\z Непрерывная случайная величина $X$ задана плотностью вероятности: $$ f(x) = \begin{cases}0, & x < 0 \\ C\mathrm{e}^{-11x}, & x \geqslant 0.\end{cases} $$ Найдите а) параметр $C$; б) вероятность $P(X < 0{,}25)$; в) математическое ожидание $M(X)$.
 

\vfill

\newpage\setcounter{zad}{0}

\z Непрерывная случайная величина $X$ на всей числовой оси задана своей функцией распределения\footnote{$\Phi(x)$ --- функция Лапласа.}: $$ F(x) = \frac{1}{2} + \frac{1}{2}\Phi\left( \frac{x - 120}{50} \right). $$ Найдите: а) вероятность $P(-181 < X < -50)$; б) квантиль уровня $0{,}2$; в) медиану $Me(X)$.


\vfill

\z Непрерывная случайная величина $X$ задана плотностью вероятности: $$ f(x) = \begin{cases}0, & x < 0 \\ C\mathrm{e}^{-3x}, & x \geqslant 0.\end{cases} $$ Найдите а) параметр $C$; б) вероятность $P(X < 0{,}05)$; в) математическое ожидание $M(X)$.
 

\vfill

\newpage\setcounter{zad}{0}

\z Непрерывная случайная величина $X$ на всей числовой оси задана своей функцией распределения\footnote{$\Phi(x)$ --- функция Лапласа.}: $$ F(x) = \frac{1}{2} + \frac{1}{2}\Phi\left( \frac{x + 80}{40} \right). $$ Найдите: а) вероятность $P(21 < X < 160)$; б) квантиль уровня $0{,}6$; в) медиану $Me(X)$.


\vfill

\z Непрерывная случайная величина $X$ задана плотностью вероятности: $$ f(x) = \begin{cases}0, & x < 0 \\ C\mathrm{e}^{-2x}, & x \geqslant 0.\end{cases} $$ Найдите а) параметр $C$; б) вероятность $P(X < 0{,}05)$; в) математическое ожидание $M(X)$.
 

\vfill

\newpage\setcounter{zad}{0}

\z Непрерывная случайная величина $X$ на всей числовой оси задана своей функцией распределения\footnote{$\Phi(x)$ --- функция Лапласа.}: $$ F(x) = \frac{1}{2} + \frac{1}{2}\Phi\left( \frac{x - 1}{2} \right). $$ Найдите: а) вероятность $P(-5 < X < 3)$; б) квантиль уровня $0{,}2$; в) медиану $Me(X)$.


\vfill

\z Непрерывная случайная величина $X$ задана плотностью вероятности: $$ f(x) = \begin{cases}0, & x < 0 \\ C\mathrm{e}^{-5x}, & x \geqslant 0.\end{cases} $$ Найдите а) параметр $C$; б) вероятность $P(X < 0{,}6)$; в) математическое ожидание $M(X)$.
 

\vfill

\newpage\setcounter{zad}{0}

\z Непрерывная случайная величина $X$ на всей числовой оси задана своей функцией распределения\footnote{$\Phi(x)$ --- функция Лапласа.}: $$ F(x) = \frac{1}{2} + \frac{1}{2}\Phi\left( \frac{x + 3}{5} \right). $$ Найдите: а) вероятность $P(-4 < X < 9)$; б) $60\%$-ную точку; в) медиану $Me(X)$.


\vfill

\z Непрерывная случайная величина $X$ задана плотностью вероятности: $$ f(x) = \begin{cases}0, & x < 0 \\ C\mathrm{e}^{-8x}, & x \geqslant 0.\end{cases} $$ Найдите а) параметр $C$; б) вероятность $P(X < 0{,}66)$; в) математическое ожидание $M(X)$.
 

\vfill

\newpage\setcounter{zad}{0}

\z Непрерывная случайная величина $X$ на всей числовой оси задана своей функцией распределения\footnote{$\Phi(x)$ --- функция Лапласа.}: $$ F(x) = \frac{1}{2} + \frac{1}{2}\Phi\left( \frac{x + 11}{5} \right). $$ Найдите: а) вероятность $P(2 < X < 20)$; б) квантиль уровня $0{,}8$; в) медиану $Me(X)$.


\vfill

\z Непрерывная случайная величина $X$ задана плотностью вероятности: $$ f(x) = \begin{cases}0, & x < 0 \\ C\mathrm{e}^{-2x}, & x \geqslant 0.\end{cases} $$ Найдите а) параметр $C$; б) вероятность $P(X < 0{,}05)$; в) математическое ожидание $M(X)$.
 

\vfill

\newpage\setcounter{zad}{0}

\z Непрерывная случайная величина $X$ на всей числовой оси задана своей функцией распределения\footnote{$\Phi(x)$ --- функция Лапласа.}: $$ F(x) = \frac{1}{2} + \frac{1}{2}\Phi\left( \frac{x + 130}{20} \right). $$ Найдите: а) вероятность $P(99 < X < 163)$; б) $50\%$-ную точку; в) медиану $Me(X)$.


\vfill

\z Непрерывная случайная величина $X$ задана плотностью вероятности: $$ f(x) = \begin{cases}0, & x < 0 \\ C\mathrm{e}^{-3x}, & x \geqslant 0.\end{cases} $$ Найдите а) параметр $C$; б) вероятность $P(X < 0{,}05)$; в) математическое ожидание $M(X)$.
 

\vfill

\newpage\setcounter{zad}{0}

\z Непрерывная случайная величина $X$ на всей числовой оси задана своей функцией распределения\footnote{$\Phi(x)$ --- функция Лапласа.}: $$ F(x) = \frac{1}{2} + \frac{1}{2}\Phi\left( \frac{x + 90}{50} \right). $$ Найдите: а) вероятность $P(21 < X < 188)$; б) $90\%$-ную точку; в) медиану $Me(X)$.


\vfill

\z Непрерывная случайная величина $X$ задана плотностью вероятности: $$ f(x) = \begin{cases}0, & x < 0 \\ C\mathrm{e}^{-13x}, & x \geqslant 0.\end{cases} $$ Найдите а) параметр $C$; б) вероятность $P(X < 0{,}7)$; в) математическое ожидание $M(X)$.
 

\vfill

\newpage\setcounter{zad}{0}

\z Непрерывная случайная величина $X$ на всей числовой оси задана своей функцией распределения\footnote{$\Phi(x)$ --- функция Лапласа.}: $$ F(x) = \frac{1}{2} + \frac{1}{2}\Phi\left( \frac{x + 1}{3} \right). $$ Найдите: а) вероятность $P(-4 < X < 7)$; б) $30\%$-ную точку; в) медиану $Me(X)$.


\vfill

\z Непрерывная случайная величина $X$ задана плотностью вероятности: $$ f(x) = \begin{cases}0, & x < 0 \\ C\mathrm{e}^{-5x}, & x \geqslant 0.\end{cases} $$ Найдите а) параметр $C$; б) вероятность $P(X < 0{,}64)$; в) математическое ожидание $M(X)$.
 

\vfill

\newpage\setcounter{zad}{0}

\z Непрерывная случайная величина $X$ на всей числовой оси задана своей функцией распределения\footnote{$\Phi(x)$ --- функция Лапласа.}: $$ F(x) = \frac{1}{2} + \frac{1}{2}\Phi\left( \frac{x - 3}{5} \right). $$ Найдите: а) вероятность $P(-9 < X < 4)$; б) $80\%$-ную точку; в) медиану $Me(X)$.


\vfill

\z Непрерывная случайная величина $X$ задана плотностью вероятности: $$ f(x) = \begin{cases}0, & x < 0 \\ C\mathrm{e}^{-6x}, & x \geqslant 0.\end{cases} $$ Найдите а) параметр $C$; б) вероятность $P(X < 0{,}62)$; в) математическое ожидание $M(X)$.
 

\vfill

\newpage\setcounter{zad}{0}

\z Непрерывная случайная величина $X$ на всей числовой оси задана своей функцией распределения\footnote{$\Phi(x)$ --- функция Лапласа.}: $$ F(x) = \frac{1}{2} + \frac{1}{2}\Phi\left( \frac{x + 90}{20} \right). $$ Найдите: а) вероятность $P(59 < X < 121)$; б) квантиль уровня $0{,}7$; в) медиану $Me(X)$.


\vfill

\z Непрерывная случайная величина $X$ задана плотностью вероятности: $$ f(x) = \begin{cases}0, & x < 0 \\ C\mathrm{e}^{-12x}, & x \geqslant 0.\end{cases} $$ Найдите а) параметр $C$; б) вероятность $P(X < 0{,}14)$; в) математическое ожидание $M(X)$.
 

\vfill

\newpage\setcounter{zad}{0}

\z Непрерывная случайная величина $X$ на всей числовой оси задана своей функцией распределения\footnote{$\Phi(x)$ --- функция Лапласа.}: $$ F(x) = \frac{1}{2} + \frac{1}{2}\Phi\left( \frac{x - 10}{50} \right). $$ Найдите: а) вероятность $P(-102 < X < 69)$; б) квантиль уровня $0{,}9$; в) медиану $Me(X)$.


\vfill

\z Непрерывная случайная величина $X$ задана плотностью вероятности: $$ f(x) = \begin{cases}0, & x < 0 \\ C\mathrm{e}^{-3x}, & x \geqslant 0.\end{cases} $$ Найдите а) параметр $C$; б) вероятность $P(X < 0{,}05)$; в) математическое ожидание $M(X)$.
 

\vfill

\newpage\setcounter{zad}{0}

\z Непрерывная случайная величина $X$ на всей числовой оси задана своей функцией распределения\footnote{$\Phi(x)$ --- функция Лапласа.}: $$ F(x) = \frac{1}{2} + \frac{1}{2}\Phi\left( \frac{x + 20}{30} \right). $$ Найдите: а) вероятность $P(-39 < X < 72)$; б) $20\%$-ную точку; в) медиану $Me(X)$.


\vfill

\z Непрерывная случайная величина $X$ задана плотностью вероятности: $$ f(x) = \begin{cases}0, & x < 0 \\ C\mathrm{e}^{-4x}, & x \geqslant 0.\end{cases} $$ Найдите а) параметр $C$; б) вероятность $P(X < 0{,}05)$; в) математическое ожидание $M(X)$.
 

\vfill

\newpage\setcounter{zad}{0}

\z Непрерывная случайная величина $X$ на всей числовой оси задана своей функцией распределения\footnote{$\Phi(x)$ --- функция Лапласа.}: $$ F(x) = \frac{1}{2} + \frac{1}{2}\Phi\left( \frac{x - 60}{50} \right). $$ Найдите: а) вероятность $P(-159 < X < 35)$; б) $80\%$-ную точку; в) медиану $Me(X)$.


\vfill

\z Непрерывная случайная величина $X$ задана плотностью вероятности: $$ f(x) = \begin{cases}0, & x < 0 \\ C\mathrm{e}^{-10x}, & x \geqslant 0.\end{cases} $$ Найдите а) параметр $C$; б) вероятность $P(X < 0{,}74)$; в) математическое ожидание $M(X)$.
 

\vfill

\newpage\setcounter{zad}{0}

\z Непрерывная случайная величина $X$ на всей числовой оси задана своей функцией распределения\footnote{$\Phi(x)$ --- функция Лапласа.}: $$ F(x) = \frac{1}{2} + \frac{1}{2}\Phi\left( \frac{x - 100}{20} \right). $$ Найдите: а) вероятность $P(-132 < X < -67)$; б) квантиль уровня $0{,}7$; в) медиану $Me(X)$.


\vfill

\z Непрерывная случайная величина $X$ задана плотностью вероятности: $$ f(x) = \begin{cases}0, & x < 0 \\ C\mathrm{e}^{-4x}, & x \geqslant 0.\end{cases} $$ Найдите а) параметр $C$; б) вероятность $P(X < 0{,}05)$; в) математическое ожидание $M(X)$.
 

\vfill

\newpage\setcounter{zad}{0}

\z Непрерывная случайная величина $X$ на всей числовой оси задана своей функцией распределения\footnote{$\Phi(x)$ --- функция Лапласа.}: $$ F(x) = \frac{1}{2} + \frac{1}{2}\Phi\left( \frac{x + 80}{30} \right). $$ Найдите: а) вероятность $P(40 < X < 126)$; б) квантиль уровня $0{,}6$; в) медиану $Me(X)$.


\vfill

\z Непрерывная случайная величина $X$ задана плотностью вероятности: $$ f(x) = \begin{cases}0, & x < 0 \\ C\mathrm{e}^{-7x}, & x \geqslant 0.\end{cases} $$ Найдите а) параметр $C$; б) вероятность $P(X < 0{,}96)$; в) математическое ожидание $M(X)$.
 

\vfill

\newpage\setcounter{zad}{0}

\z Непрерывная случайная величина $X$ на всей числовой оси задана своей функцией распределения\footnote{$\Phi(x)$ --- функция Лапласа.}: $$ F(x) = \frac{1}{2} + \frac{1}{2}\Phi\left( \frac{x + 40}{40} \right). $$ Найдите: а) вероятность $P(-39 < X < 93)$; б) квантиль уровня $0{,}7$; в) медиану $Me(X)$.


\vfill

\z Непрерывная случайная величина $X$ задана плотностью вероятности: $$ f(x) = \begin{cases}0, & x < 0 \\ C\mathrm{e}^{-10x}, & x \geqslant 0.\end{cases} $$ Найдите а) параметр $C$; б) вероятность $P(X < 0{,}27)$; в) математическое ожидание $M(X)$.
 

\vfill

\newpage\setcounter{zad}{0}

\z Непрерывная случайная величина $X$ на всей числовой оси задана своей функцией распределения\footnote{$\Phi(x)$ --- функция Лапласа.}: $$ F(x) = \frac{1}{2} + \frac{1}{2}\Phi\left( \frac{x}{20} \right). $$ Найдите: а) вероятность $P(-31 < X < 37)$; б) $90\%$-ную точку; в) медиану $Me(X)$.


\vfill

\z Непрерывная случайная величина $X$ задана плотностью вероятности: $$ f(x) = \begin{cases}0, & x < 0 \\ C\mathrm{e}^{-12x}, & x \geqslant 0.\end{cases} $$ Найдите а) параметр $C$; б) вероятность $P(X < 0{,}48)$; в) математическое ожидание $M(X)$.
 

\vfill

\newpage\setcounter{zad}{0}

\z Непрерывная случайная величина $X$ на всей числовой оси задана своей функцией распределения\footnote{$\Phi(x)$ --- функция Лапласа.}: $$ F(x) = \frac{1}{2} + \frac{1}{2}\Phi\left( \frac{x - 50}{20} \right). $$ Найдите: а) вероятность $P(-85 < X < -12)$; б) квантиль уровня $0{,}3$; в) медиану $Me(X)$.


\vfill

\z Непрерывная случайная величина $X$ задана плотностью вероятности: $$ f(x) = \begin{cases}0, & x < 0 \\ C\mathrm{e}^{-12x}, & x \geqslant 0.\end{cases} $$ Найдите а) параметр $C$; б) вероятность $P(X < 0{,}49)$; в) математическое ожидание $M(X)$.
 

\vfill

\newpage\setcounter{zad}{0}

\z Непрерывная случайная величина $X$ на всей числовой оси задана своей функцией распределения\footnote{$\Phi(x)$ --- функция Лапласа.}: $$ F(x) = \frac{1}{2} + \frac{1}{2}\Phi\left( \frac{x - 6}{2} \right). $$ Найдите: а) вероятность $P(-9 < X < -3)$; б) квантиль уровня $0{,}6$; в) медиану $Me(X)$.


\vfill

\z Непрерывная случайная величина $X$ задана плотностью вероятности: $$ f(x) = \begin{cases}0, & x < 0 \\ C\mathrm{e}^{-10x}, & x \geqslant 0.\end{cases} $$ Найдите а) параметр $C$; б) вероятность $P(X < 0{,}72)$; в) математическое ожидание $M(X)$.
 

\vfill

\newpage\setcounter{zad}{0}

\z Непрерывная случайная величина $X$ на всей числовой оси задана своей функцией распределения\footnote{$\Phi(x)$ --- функция Лапласа.}: $$ F(x) = \frac{1}{2} + \frac{1}{2}\Phi\left( \frac{x - 8}{3} \right). $$ Найдите: а) вероятность $P(-13 < X < -3)$; б) $60\%$-ную точку; в) медиану $Me(X)$.


\vfill

\z Непрерывная случайная величина $X$ задана плотностью вероятности: $$ f(x) = \begin{cases}0, & x < 0 \\ C\mathrm{e}^{-9x}, & x \geqslant 0.\end{cases} $$ Найдите а) параметр $C$; б) вероятность $P(X < 0{,}66)$; в) математическое ожидание $M(X)$.
 

\vfill

\newpage\setcounter{zad}{0}

\z Непрерывная случайная величина $X$ на всей числовой оси задана своей функцией распределения\footnote{$\Phi(x)$ --- функция Лапласа.}: $$ F(x) = \frac{1}{2} + \frac{1}{2}\Phi\left( \frac{x + 40}{40} \right). $$ Найдите: а) вероятность $P(-32 < X < 101)$; б) $60\%$-ную точку; в) медиану $Me(X)$.


\vfill

\z Непрерывная случайная величина $X$ задана плотностью вероятности: $$ f(x) = \begin{cases}0, & x < 0 \\ C\mathrm{e}^{-4x}, & x \geqslant 0.\end{cases} $$ Найдите а) параметр $C$; б) вероятность $P(X < 0{,}05)$; в) математическое ожидание $M(X)$.
 

\vfill

\newpage\setcounter{zad}{0}

\z Непрерывная случайная величина $X$ на всей числовой оси задана своей функцией распределения\footnote{$\Phi(x)$ --- функция Лапласа.}: $$ F(x) = \frac{1}{2} + \frac{1}{2}\Phi\left( \frac{x - 30}{20} \right). $$ Найдите: а) вероятность $P(-70 < X < 3)$; б) квантиль уровня $0{,}3$; в) медиану $Me(X)$.


\vfill

\z Непрерывная случайная величина $X$ задана плотностью вероятности: $$ f(x) = \begin{cases}0, & x < 0 \\ C\mathrm{e}^{-3x}, & x \geqslant 0.\end{cases} $$ Найдите а) параметр $C$; б) вероятность $P(X < 0{,}05)$; в) математическое ожидание $M(X)$.
 

\vfill

\newpage\setcounter{zad}{0}

\z Непрерывная случайная величина $X$ на всей числовой оси задана своей функцией распределения\footnote{$\Phi(x)$ --- функция Лапласа.}: $$ F(x) = \frac{1}{2} + \frac{1}{2}\Phi\left( \frac{x - 10}{3} \right). $$ Найдите: а) вероятность $P(-14 < X < -6)$; б) $20\%$-ную точку; в) медиану $Me(X)$.


\vfill

\z Непрерывная случайная величина $X$ задана плотностью вероятности: $$ f(x) = \begin{cases}0, & x < 0 \\ C\mathrm{e}^{-8x}, & x \geqslant 0.\end{cases} $$ Найдите а) параметр $C$; б) вероятность $P(X < 0{,}16)$; в) математическое ожидание $M(X)$.
 

\vfill

\newpage\setcounter{zad}{0}

\z Непрерывная случайная величина $X$ на всей числовой оси задана своей функцией распределения\footnote{$\Phi(x)$ --- функция Лапласа.}: $$ F(x) = \frac{1}{2} + \frac{1}{2}\Phi\left( \frac{x + 120}{30} \right). $$ Найдите: а) вероятность $P(64 < X < 173)$; б) квантиль уровня $0{,}7$; в) медиану $Me(X)$.


\vfill

\z Непрерывная случайная величина $X$ задана плотностью вероятности: $$ f(x) = \begin{cases}0, & x < 0 \\ C\mathrm{e}^{-12x}, & x \geqslant 0.\end{cases} $$ Найдите а) параметр $C$; б) вероятность $P(X < 0{,}52)$; в) математическое ожидание $M(X)$.
 

\vfill

\newpage\setcounter{zad}{0}

\z Непрерывная случайная величина $X$ на всей числовой оси задана своей функцией распределения\footnote{$\Phi(x)$ --- функция Лапласа.}: $$ F(x) = \frac{1}{2} + \frac{1}{2}\Phi\left( \frac{x + 140}{40} \right). $$ Найдите: а) вероятность $P(72 < X < 220)$; б) квантиль уровня $0{,}5$; в) медиану $Me(X)$.


\vfill

\z Непрерывная случайная величина $X$ задана плотностью вероятности: $$ f(x) = \begin{cases}0, & x < 0 \\ C\mathrm{e}^{-12x}, & x \geqslant 0.\end{cases} $$ Найдите а) параметр $C$; б) вероятность $P(X < 0{,}89)$; в) математическое ожидание $M(X)$.
 

\vfill

\newpage\setcounter{zad}{0}

\z Непрерывная случайная величина $X$ на всей числовой оси задана своей функцией распределения\footnote{$\Phi(x)$ --- функция Лапласа.}: $$ F(x) = \frac{1}{2} + \frac{1}{2}\Phi\left( \frac{x - 14}{5} \right). $$ Найдите: а) вероятность $P(-22 < X < -6)$; б) $30\%$-ную точку; в) медиану $Me(X)$.


\vfill

\z Непрерывная случайная величина $X$ задана плотностью вероятности: $$ f(x) = \begin{cases}0, & x < 0 \\ C\mathrm{e}^{-12x}, & x \geqslant 0.\end{cases} $$ Найдите а) параметр $C$; б) вероятность $P(X < 0{,}9)$; в) математическое ожидание $M(X)$.
 

\vfill

\newpage\setcounter{zad}{0}

\z Непрерывная случайная величина $X$ на всей числовой оси задана своей функцией распределения\footnote{$\Phi(x)$ --- функция Лапласа.}: $$ F(x) = \frac{1}{2} + \frac{1}{2}\Phi\left( \frac{x + 100}{40} \right). $$ Найдите: а) вероятность $P(40 < X < 155)$; б) квантиль уровня $0{,}7$; в) медиану $Me(X)$.


\vfill

\z Непрерывная случайная величина $X$ задана плотностью вероятности: $$ f(x) = \begin{cases}0, & x < 0 \\ C\mathrm{e}^{-14x}, & x \geqslant 0.\end{cases} $$ Найдите а) параметр $C$; б) вероятность $P(X < 0{,}5)$; в) математическое ожидание $M(X)$.
 

\vfill

\newpage\setcounter{zad}{0}

\z Непрерывная случайная величина $X$ на всей числовой оси задана своей функцией распределения\footnote{$\Phi(x)$ --- функция Лапласа.}: $$ F(x) = \frac{1}{2} + \frac{1}{2}\Phi\left( \frac{x}{5} \right). $$ Найдите: а) вероятность $P(-9 < X < 8)$; б) $70\%$-ную точку; в) медиану $Me(X)$.


\vfill

\z Непрерывная случайная величина $X$ задана плотностью вероятности: $$ f(x) = \begin{cases}0, & x < 0 \\ C\mathrm{e}^{-6x}, & x \geqslant 0.\end{cases} $$ Найдите а) параметр $C$; б) вероятность $P(X < 0{,}6)$; в) математическое ожидание $M(X)$.
 

\vfill

\newpage\setcounter{zad}{0}

\z Непрерывная случайная величина $X$ на всей числовой оси задана своей функцией распределения\footnote{$\Phi(x)$ --- функция Лапласа.}: $$ F(x) = \frac{1}{2} + \frac{1}{2}\Phi\left( \frac{x - 9}{3} \right). $$ Найдите: а) вероятность $P(-14 < X < -4)$; б) квантиль уровня $0{,}6$; в) медиану $Me(X)$.


\vfill

\z Непрерывная случайная величина $X$ задана плотностью вероятности: $$ f(x) = \begin{cases}0, & x < 0 \\ C\mathrm{e}^{-14x}, & x \geqslant 0.\end{cases} $$ Найдите а) параметр $C$; б) вероятность $P(X < 0{,}65)$; в) математическое ожидание $M(X)$.
 

\vfill

\newpage\setcounter{zad}{0}

\z Непрерывная случайная величина $X$ на всей числовой оси задана своей функцией распределения\footnote{$\Phi(x)$ --- функция Лапласа.}: $$ F(x) = \frac{1}{2} + \frac{1}{2}\Phi\left( \frac{x}{3} \right). $$ Найдите: а) вероятность $P(-5 < X < 4)$; б) $40\%$-ную точку; в) медиану $Me(X)$.


\vfill

\z Непрерывная случайная величина $X$ задана плотностью вероятности: $$ f(x) = \begin{cases}0, & x < 0 \\ C\mathrm{e}^{-12x}, & x \geqslant 0.\end{cases} $$ Найдите а) параметр $C$; б) вероятность $P(X < 0{,}81)$; в) математическое ожидание $M(X)$.
 

\vfill

\newpage\setcounter{zad}{0}

\z Непрерывная случайная величина $X$ на всей числовой оси задана своей функцией распределения\footnote{$\Phi(x)$ --- функция Лапласа.}: $$ F(x) = \frac{1}{2} + \frac{1}{2}\Phi\left( \frac{x + 70}{30} \right). $$ Найдите: а) вероятность $P(16 < X < 113)$; б) квантиль уровня $0{,}5$; в) медиану $Me(X)$.


\vfill

\z Непрерывная случайная величина $X$ задана плотностью вероятности: $$ f(x) = \begin{cases}0, & x < 0 \\ C\mathrm{e}^{-8x}, & x \geqslant 0.\end{cases} $$ Найдите а) параметр $C$; б) вероятность $P(X < 0{,}12)$; в) математическое ожидание $M(X)$.
 

\vfill

\newpage\setcounter{zad}{0}

\z Непрерывная случайная величина $X$ на всей числовой оси задана своей функцией распределения\footnote{$\Phi(x)$ --- функция Лапласа.}: $$ F(x) = \frac{1}{2} + \frac{1}{2}\Phi\left( \frac{x - 40}{30} \right). $$ Найдите: а) вероятность $P(-80 < X < 13)$; б) $90\%$-ную точку; в) медиану $Me(X)$.


\vfill

\z Непрерывная случайная величина $X$ задана плотностью вероятности: $$ f(x) = \begin{cases}0, & x < 0 \\ C\mathrm{e}^{-8x}, & x \geqslant 0.\end{cases} $$ Найдите а) параметр $C$; б) вероятность $P(X < 0{,}98)$; в) математическое ожидание $M(X)$.
 

\vfill

\newpage\setcounter{zad}{0}

\z Непрерывная случайная величина $X$ на всей числовой оси задана своей функцией распределения\footnote{$\Phi(x)$ --- функция Лапласа.}: $$ F(x) = \frac{1}{2} + \frac{1}{2}\Phi\left( \frac{x + 4}{4} \right). $$ Найдите: а) вероятность $P(-1 < X < 9)$; б) $60\%$-ную точку; в) медиану $Me(X)$.


\vfill

\z Непрерывная случайная величина $X$ задана плотностью вероятности: $$ f(x) = \begin{cases}0, & x < 0 \\ C\mathrm{e}^{-5x}, & x \geqslant 0.\end{cases} $$ Найдите а) параметр $C$; б) вероятность $P(X < 0{,}77)$; в) математическое ожидание $M(X)$.
 

\vfill

\newpage\setcounter{zad}{0}

\z Непрерывная случайная величина $X$ на всей числовой оси задана своей функцией распределения\footnote{$\Phi(x)$ --- функция Лапласа.}: $$ F(x) = \frac{1}{2} + \frac{1}{2}\Phi\left( \frac{x + 2}{5} \right). $$ Найдите: а) вероятность $P(-8 < X < 10)$; б) квантиль уровня $0{,}7$; в) медиану $Me(X)$.


\vfill

\z Непрерывная случайная величина $X$ задана плотностью вероятности: $$ f(x) = \begin{cases}0, & x < 0 \\ C\mathrm{e}^{-4x}, & x \geqslant 0.\end{cases} $$ Найдите а) параметр $C$; б) вероятность $P(X < 0{,}05)$; в) математическое ожидание $M(X)$.
 

\vfill

\newpage\setcounter{zad}{0}

\z Непрерывная случайная величина $X$ на всей числовой оси задана своей функцией распределения\footnote{$\Phi(x)$ --- функция Лапласа.}: $$ F(x) = \frac{1}{2} + \frac{1}{2}\Phi\left( \frac{x + 50}{40} \right). $$ Найдите: а) вероятность $P(-7 < X < 109)$; б) $90\%$-ную точку; в) медиану $Me(X)$.


\vfill

\z Непрерывная случайная величина $X$ задана плотностью вероятности: $$ f(x) = \begin{cases}0, & x < 0 \\ C\mathrm{e}^{-3x}, & x \geqslant 0.\end{cases} $$ Найдите а) параметр $C$; б) вероятность $P(X < 0{,}05)$; в) математическое ожидание $M(X)$.
 

\vfill

\newpage\setcounter{zad}{0}

\z Непрерывная случайная величина $X$ на всей числовой оси задана своей функцией распределения\footnote{$\Phi(x)$ --- функция Лапласа.}: $$ F(x) = \frac{1}{2} + \frac{1}{2}\Phi\left( \frac{x + 80}{30} \right). $$ Найдите: а) вероятность $P(24 < X < 120)$; б) $70\%$-ную точку; в) медиану $Me(X)$.


\vfill

\z Непрерывная случайная величина $X$ задана плотностью вероятности: $$ f(x) = \begin{cases}0, & x < 0 \\ C\mathrm{e}^{-8x}, & x \geqslant 0.\end{cases} $$ Найдите а) параметр $C$; б) вероятность $P(X < 0{,}83)$; в) математическое ожидание $M(X)$.
 

\vfill

\newpage\setcounter{zad}{0}

\z Непрерывная случайная величина $X$ на всей числовой оси задана своей функцией распределения\footnote{$\Phi(x)$ --- функция Лапласа.}: $$ F(x) = \frac{1}{2} + \frac{1}{2}\Phi\left( \frac{x + 80}{20} \right). $$ Найдите: а) вероятность $P(50 < X < 113)$; б) квантиль уровня $0{,}1$; в) медиану $Me(X)$.


\vfill

\z Непрерывная случайная величина $X$ задана плотностью вероятности: $$ f(x) = \begin{cases}0, & x < 0 \\ C\mathrm{e}^{-9x}, & x \geqslant 0.\end{cases} $$ Найдите а) параметр $C$; б) вероятность $P(X < 0{,}94)$; в) математическое ожидание $M(X)$.
 

\vfill

\newpage\setcounter{zad}{0}

\z Непрерывная случайная величина $X$ на всей числовой оси задана своей функцией распределения\footnote{$\Phi(x)$ --- функция Лапласа.}: $$ F(x) = \frac{1}{2} + \frac{1}{2}\Phi\left( \frac{x}{20} \right). $$ Найдите: а) вероятность $P(-37 < X < 34)$; б) квантиль уровня $0{,}8$; в) медиану $Me(X)$.


\vfill

\z Непрерывная случайная величина $X$ задана плотностью вероятности: $$ f(x) = \begin{cases}0, & x < 0 \\ C\mathrm{e}^{-4x}, & x \geqslant 0.\end{cases} $$ Найдите а) параметр $C$; б) вероятность $P(X < 0{,}05)$; в) математическое ожидание $M(X)$.
 

\vfill

\newpage\setcounter{zad}{0}

\z Непрерывная случайная величина $X$ на всей числовой оси задана своей функцией распределения\footnote{$\Phi(x)$ --- функция Лапласа.}: $$ F(x) = \frac{1}{2} + \frac{1}{2}\Phi\left( \frac{x + 14}{2} \right). $$ Найдите: а) вероятность $P(11 < X < 17)$; б) $10\%$-ную точку; в) медиану $Me(X)$.


\vfill

\z Непрерывная случайная величина $X$ задана плотностью вероятности: $$ f(x) = \begin{cases}0, & x < 0 \\ C\mathrm{e}^{-15x}, & x \geqslant 0.\end{cases} $$ Найдите а) параметр $C$; б) вероятность $P(X < 0{,}5)$; в) математическое ожидание $M(X)$.
 

\vfill

\newpage\setcounter{zad}{0}

\z Непрерывная случайная величина $X$ на всей числовой оси задана своей функцией распределения\footnote{$\Phi(x)$ --- функция Лапласа.}: $$ F(x) = \frac{1}{2} + \frac{1}{2}\Phi\left( \frac{x + 140}{30} \right). $$ Найдите: а) вероятность $P(90 < X < 192)$; б) $90\%$-ную точку; в) медиану $Me(X)$.


\vfill

\z Непрерывная случайная величина $X$ задана плотностью вероятности: $$ f(x) = \begin{cases}0, & x < 0 \\ C\mathrm{e}^{-12x}, & x \geqslant 0.\end{cases} $$ Найдите а) параметр $C$; б) вероятность $P(X < 0{,}27)$; в) математическое ожидание $M(X)$.
 

\vfill

\newpage\setcounter{zad}{0}

\z Непрерывная случайная величина $X$ на всей числовой оси задана своей функцией распределения\footnote{$\Phi(x)$ --- функция Лапласа.}: $$ F(x) = \frac{1}{2} + \frac{1}{2}\Phi\left( \frac{x + 140}{50} \right). $$ Найдите: а) вероятность $P(71 < X < 202)$; б) $80\%$-ную точку; в) медиану $Me(X)$.


\vfill

\z Непрерывная случайная величина $X$ задана плотностью вероятности: $$ f(x) = \begin{cases}0, & x < 0 \\ C\mathrm{e}^{-10x}, & x \geqslant 0.\end{cases} $$ Найдите а) параметр $C$; б) вероятность $P(X < 0{,}34)$; в) математическое ожидание $M(X)$.
 

\vfill

\newpage\setcounter{zad}{0}

\z Непрерывная случайная величина $X$ на всей числовой оси задана своей функцией распределения\footnote{$\Phi(x)$ --- функция Лапласа.}: $$ F(x) = \frac{1}{2} + \frac{1}{2}\Phi\left( \frac{x + 4}{2} \right). $$ Найдите: а) вероятность $P(0 < X < 7)$; б) $50\%$-ную точку; в) медиану $Me(X)$.


\vfill

\z Непрерывная случайная величина $X$ задана плотностью вероятности: $$ f(x) = \begin{cases}0, & x < 0 \\ C\mathrm{e}^{-2x}, & x \geqslant 0.\end{cases} $$ Найдите а) параметр $C$; б) вероятность $P(X < 0{,}05)$; в) математическое ожидание $M(X)$.
 

\vfill

\newpage\setcounter{zad}{0}

\z Непрерывная случайная величина $X$ на всей числовой оси задана своей функцией распределения\footnote{$\Phi(x)$ --- функция Лапласа.}: $$ F(x) = \frac{1}{2} + \frac{1}{2}\Phi\left( \frac{x - 10}{20} \right). $$ Найдите: а) вероятность $P(-43 < X < 24)$; б) $70\%$-ную точку; в) медиану $Me(X)$.


\vfill

\z Непрерывная случайная величина $X$ задана плотностью вероятности: $$ f(x) = \begin{cases}0, & x < 0 \\ C\mathrm{e}^{-6x}, & x \geqslant 0.\end{cases} $$ Найдите а) параметр $C$; б) вероятность $P(X < 0{,}32)$; в) математическое ожидание $M(X)$.
 

\vfill

\newpage\setcounter{zad}{0}

\z Непрерывная случайная величина $X$ на всей числовой оси задана своей функцией распределения\footnote{$\Phi(x)$ --- функция Лапласа.}: $$ F(x) = \frac{1}{2} + \frac{1}{2}\Phi\left( \frac{x - 10}{50} \right). $$ Найдите: а) вероятность $P(-103 < X < 70)$; б) $70\%$-ную точку; в) медиану $Me(X)$.


\vfill

\z Непрерывная случайная величина $X$ задана плотностью вероятности: $$ f(x) = \begin{cases}0, & x < 0 \\ C\mathrm{e}^{-11x}, & x \geqslant 0.\end{cases} $$ Найдите а) параметр $C$; б) вероятность $P(X < 0{,}56)$; в) математическое ожидание $M(X)$.
 

\vfill

\newpage\setcounter{zad}{0}