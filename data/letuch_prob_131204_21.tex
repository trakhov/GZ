

\z Непрерывная случайная величина $X$ распределена равномерно на отрезке $[-3; 1]$. Найдите: \\ \quad а) математическое ожидание $M(X)$; \\ \quad б) вероятность $P(X>-2{,}6)$; \\ \quad в) $90\%$-ную точку этой случайной величины.


\vfill

\z Время между появлениями двух последовательных покупателей в некотором магазине имеет показательное распределение с параметром $\lambda = 0.01$. Найдите: \\ \quad а) дисперсию времени между появлениями двух покупателей; \\ \quad б) вероятность того, что за время длительностью $t = 90$ не произойдет появление покупателя.
 

\vfill

\newpage\setcounter{zad}{0}

\z Непрерывная случайная величина $X$ распределена равномерно на отрезке $[2; 5]$. Найдите: \\ \quad а) математическое ожидание $M(X)$; \\ \quad б) вероятность $P(X>2{,}9)$; \\ \quad в) медиану этой случайной величины.


\vfill

\z Длительность времени безотказной работы прибора имеет показательное распределение с параметром $\lambda = 0.05$. Найдите: \\ \quad а) дисперсию времени безотказной работы прибора; \\ \quad б) вероятность того, что за время длительностью $t = 50$ прибор не откажет.
 

\vfill

\newpage\setcounter{zad}{0}

\z Непрерывная случайная величина $X$ распределена равномерно на отрезке $[1; 3]$. Найдите: \\ \quad а) среднее квадратическое отклонение $\sigma_X$; \\ \quad б) вероятность $P(X>1{,}6)$; \\ \quad в) $70\%$-ную точку этой случайной величины.


\vfill

\z Время между двумя последовательными попаданиями метеорита в космический корабль имеет показательное распределение с параметром $\lambda = 0.025$. Найдите: \\ \quad а) дисперсию времени между двумя попаданиями метеорита; \\ \quad б) вероятность того, что за время длительностью $t = 20$ корабль  столкнется с метеоритом.
 

\vfill

\newpage\setcounter{zad}{0}

\z Непрерывная случайная величина $X$ распределена равномерно на отрезке $[3; 5]$. Найдите: \\ \quad а) математическое ожидание $M(X)$; \\ \quad б) вероятность $P(X>4{,}8)$; \\ \quad в) $20\%$-ную точку этой случайной величины.


\vfill

\z Длительность времени безотказной работы прибора имеет показательное распределение с параметром $\lambda = 0.05$. Найдите: \\ \quad а) дисперсию времени безотказной работы прибора; \\ \quad б) вероятность того, что за время длительностью $t = 50$ прибор  откажет.
 

\vfill

\newpage\setcounter{zad}{0}

\z Непрерывная случайная величина $X$ распределена равномерно на отрезке $[4; 9]$. Найдите: \\ \quad а) дисперсию $D(X)$; \\ \quad б) вероятность $P(X>5{,}0)$; \\ \quad в) $50\%$-ную точку этой случайной величины.


\vfill

\z Время между двумя последовательными попаданиями метеорита в космический корабль имеет показательное распределение с параметром $\lambda = 0.04$. Найдите: \\ \quad а) среднее квадратическое отклонение времени между двумя попаданиями метеорита; \\ \quad б) вероятность того, что за время длительностью $t = 70$ корабль не столкнется с метеоритом.
 

\vfill

\newpage\setcounter{zad}{0}

\z Непрерывная случайная величина $X$ распределена равномерно на отрезке $[2; 7]$. Найдите: \\ \quad а) среднее квадратическое отклонение $\sigma_X$; \\ \quad б) вероятность $P(X>3{,}0)$; \\ \quad в) $60\%$-ную точку этой случайной величины.


\vfill

\z Длительность времени безотказной работы прибора имеет показательное распределение с параметром $\lambda = 0.02$. Найдите: \\ \quad а) среднее квадратическое отклонение времени безотказной работы прибора; \\ \quad б) вероятность того, что за время длительностью $t = 50$ прибор не откажет.
 

\vfill

\newpage\setcounter{zad}{0}

\z Непрерывная случайная величина $X$ распределена равномерно на отрезке $[3; 7]$. Найдите: \\ \quad а) среднее квадратическое отклонение $\sigma_X$; \\ \quad б) вероятность $P(X>4{,}2)$; \\ \quad в) $80\%$-ную точку этой случайной величины.


\vfill

\z Время между появлениями двух последовательных покупателей в некотором магазине имеет показательное распределение с параметром $\lambda = 0.025$. Найдите: \\ \quad а) математическое ожидание времени между появлениями двух покупателей; \\ \quad б) вероятность того, что за время длительностью $t = 90$ не произойдет появление покупателя.
 

\vfill

\newpage\setcounter{zad}{0}

\z Непрерывная случайная величина $X$ распределена равномерно на отрезке $[3; 6]$. Найдите: \\ \quad а) дисперсию $D(X)$; \\ \quad б) вероятность $P(X>5{,}1)$; \\ \quad в) медиану этой случайной величины.


\vfill

\z Длительность времени безотказной работы прибора имеет показательное распределение с параметром $\lambda = 0.025$. Найдите: \\ \quad а) среднее время безотказной работы прибора; \\ \quad б) вероятность того, что за время длительностью $t = 80$ прибор  откажет.
 

\vfill

\newpage\setcounter{zad}{0}

\z Непрерывная случайная величина $X$ распределена равномерно на отрезке $[1; 3]$. Найдите: \\ \quad а) математическое ожидание $M(X)$; \\ \quad б) вероятность $P(X>2{,}0)$; \\ \quad в) $60\%$-ную точку этой случайной величины.


\vfill

\z Время между появлениями двух последовательных покупателей в некотором магазине имеет показательное распределение с параметром $\lambda = 0.008$. Найдите: \\ \quad а) среднее время между появлениями двух покупателей; \\ \quad б) вероятность того, что за время длительностью $t = 90$  произойдет появление покупателя.
 

\vfill

\newpage\setcounter{zad}{0}

\z Непрерывная случайная величина $X$ распределена равномерно на отрезке $[3; 6]$. Найдите: \\ \quad а) среднее квадратическое отклонение $\sigma_X$; \\ \quad б) вероятность $P(X>3{,}3)$; \\ \quad в) медиану этой случайной величины.


\vfill

\z Время между появлениями двух последовательных покупателей в некотором магазине имеет показательное распределение с параметром $\lambda = 0.02$. Найдите: \\ \quad а) среднее время между появлениями двух покупателей; \\ \quad б) вероятность того, что за время длительностью $t = 70$ не произойдет появление покупателя.
 

\vfill

\newpage\setcounter{zad}{0}

\z Непрерывная случайная величина $X$ распределена равномерно на отрезке $[0; 4]$. Найдите: \\ \quad а) среднее квадратическое отклонение $\sigma_X$; \\ \quad б) вероятность $P(X>1{,}6)$; \\ \quad в) медиану этой случайной величины.


\vfill

\z Время между двумя последовательными попаданиями метеорита в космический корабль имеет показательное распределение с параметром $\lambda = 0.04$. Найдите: \\ \quad а) дисперсию времени между двумя попаданиями метеорита; \\ \quad б) вероятность того, что за время длительностью $t = 70$ корабль  столкнется с метеоритом.
 

\vfill

\newpage\setcounter{zad}{0}

\z Непрерывная случайная величина $X$ распределена равномерно на отрезке $[-1; 2]$. Найдите: \\ \quad а) дисперсию $D(X)$; \\ \quad б) вероятность $P(X>1{,}4)$; \\ \quad в) $70\%$-ную точку этой случайной величины.


\vfill

\z Время между двумя последовательными попаданиями метеорита в космический корабль имеет показательное распределение с параметром $\lambda = 0.0625$. Найдите: \\ \quad а) дисперсию времени между двумя попаданиями метеорита; \\ \quad б) вероятность того, что за время длительностью $t = 30$ корабль  столкнется с метеоритом.
 

\vfill

\newpage\setcounter{zad}{0}

\z Непрерывная случайная величина $X$ распределена равномерно на отрезке $[2; 4]$. Найдите: \\ \quad а) среднее квадратическое отклонение $\sigma_X$; \\ \quad б) вероятность $P(X>3{,}4)$; \\ \quad в) $50\%$-ную точку этой случайной величины.


\vfill

\z Длительность времени безотказной работы прибора имеет показательное распределение с параметром $\lambda = 0.05$. Найдите: \\ \quad а) дисперсию времени безотказной работы прибора; \\ \quad б) вероятность того, что за время длительностью $t = 60$ прибор не откажет.
 

\vfill

\newpage\setcounter{zad}{0}

\z Непрерывная случайная величина $X$ распределена равномерно на отрезке $[-2; 2]$. Найдите: \\ \quad а) дисперсию $D(X)$; \\ \quad б) вероятность $P(X>-0{,}8)$; \\ \quad в) $60\%$-ную точку этой случайной величины.


\vfill

\z Длительность времени безотказной работы прибора имеет показательное распределение с параметром $\lambda = 0.008$. Найдите: \\ \quad а) среднее квадратическое отклонение времени безотказной работы прибора; \\ \quad б) вероятность того, что за время длительностью $t = 40$ прибор  откажет.
 

\vfill

\newpage\setcounter{zad}{0}

\z Непрерывная случайная величина $X$ распределена равномерно на отрезке $[0; 5]$. Найдите: \\ \quad а) среднее квадратическое отклонение $\sigma_X$; \\ \quad б) вероятность $P(X>3{,}0)$; \\ \quad в) $50\%$-ную точку этой случайной величины.


\vfill

\z Время между двумя последовательными попаданиями метеорита в космический корабль имеет показательное распределение с параметром $\lambda = 0.02$. Найдите: \\ \quad а) математическое ожидание времени между двумя попаданиями метеорита; \\ \quad б) вероятность того, что за время длительностью $t = 50$ корабль  столкнется с метеоритом.
 

\vfill

\newpage\setcounter{zad}{0}

\z Непрерывная случайная величина $X$ распределена равномерно на отрезке $[3; 4]$. Найдите: \\ \quad а) дисперсию $D(X)$; \\ \quad б) вероятность $P(X>3{,}1)$; \\ \quad в) квантиль $x_{0{,}3}$ этой случайной величины.


\vfill

\z Длительность времени безотказной работы прибора имеет показательное распределение с параметром $\lambda = 0.04$. Найдите: \\ \quad а) математическое ожидание времени безотказной работы прибора; \\ \quad б) вероятность того, что за время длительностью $t = 20$ прибор  откажет.
 

\vfill

\newpage\setcounter{zad}{0}

\z Непрерывная случайная величина $X$ распределена равномерно на отрезке $[-3; 2]$. Найдите: \\ \quad а) дисперсию $D(X)$; \\ \quad б) вероятность $P(X>-2{,}5)$; \\ \quad в) квантиль $x_{0{,}1}$ этой случайной величины.


\vfill

\z Длительность времени безотказной работы прибора имеет показательное распределение с параметром $\lambda = 0.04$. Найдите: \\ \quad а) среднее квадратическое отклонение времени безотказной работы прибора; \\ \quad б) вероятность того, что за время длительностью $t = 60$ прибор не откажет.
 

\vfill

\newpage\setcounter{zad}{0}

\z Непрерывная случайная величина $X$ распределена равномерно на отрезке $[4; 5]$. Найдите: \\ \quad а) математическое ожидание $M(X)$; \\ \quad б) вероятность $P(X>4{,}3)$; \\ \quad в) квантиль $x_{0{,}7}$ этой случайной величины.


\vfill

\z Время между появлениями двух последовательных покупателей в некотором магазине имеет показательное распределение с параметром $\lambda = 0.05$. Найдите: \\ \quad а) среднее квадратическое отклонение времени между появлениями двух покупателей; \\ \quad б) вероятность того, что за время длительностью $t = 50$ не произойдет появление покупателя.
 

\vfill

\newpage\setcounter{zad}{0}

\z Непрерывная случайная величина $X$ распределена равномерно на отрезке $[-2; 2]$. Найдите: \\ \quad а) математическое ожидание $M(X)$; \\ \quad б) вероятность $P(X>0{,}4)$; \\ \quad в) квантиль $x_{0{,}5}$ этой случайной величины.


\vfill

\z Длительность времени безотказной работы прибора имеет показательное распределение с параметром $\lambda = 0.04$. Найдите: \\ \quad а) дисперсию времени безотказной работы прибора; \\ \quad б) вероятность того, что за время длительностью $t = 30$ прибор не откажет.
 

\vfill

\newpage\setcounter{zad}{0}

\z Непрерывная случайная величина $X$ распределена равномерно на отрезке $[-2; 1]$. Найдите: \\ \quad а) математическое ожидание $M(X)$; \\ \quad б) вероятность $P(X>0{,}1)$; \\ \quad в) медиану этой случайной величины.


\vfill

\z Время между двумя последовательными попаданиями метеорита в космический корабль имеет показательное распределение с параметром $\lambda = 0.02$. Найдите: \\ \quad а) дисперсию времени между двумя попаданиями метеорита; \\ \quad б) вероятность того, что за время длительностью $t = 70$ корабль  столкнется с метеоритом.
 

\vfill

\newpage\setcounter{zad}{0}

\z Непрерывная случайная величина $X$ распределена равномерно на отрезке $[1; 4]$. Найдите: \\ \quad а) дисперсию $D(X)$; \\ \quad б) вероятность $P(X>2{,}8)$; \\ \quad в) медиану этой случайной величины.


\vfill

\z Длительность времени безотказной работы прибора имеет показательное распределение с параметром $\lambda = 0.04$. Найдите: \\ \quad а) дисперсию времени безотказной работы прибора; \\ \quad б) вероятность того, что за время длительностью $t = 20$ прибор не откажет.
 

\vfill

\newpage\setcounter{zad}{0}

\z Непрерывная случайная величина $X$ распределена равномерно на отрезке $[-1; 1]$. Найдите: \\ \quad а) среднее квадратическое отклонение $\sigma_X$; \\ \quad б) вероятность $P(X>-0{,}6)$; \\ \quad в) медиану этой случайной величины.


\vfill

\z Время между двумя последовательными попаданиями метеорита в космический корабль имеет показательное распределение с параметром $\lambda = 0.04$. Найдите: \\ \quad а) дисперсию времени между двумя попаданиями метеорита; \\ \quad б) вероятность того, что за время длительностью $t = 70$ корабль  столкнется с метеоритом.
 

\vfill

\newpage\setcounter{zad}{0}

\z Непрерывная случайная величина $X$ распределена равномерно на отрезке $[1; 3]$. Найдите: \\ \quad а) математическое ожидание $M(X)$; \\ \quad б) вероятность $P(X>2{,}8)$; \\ \quad в) медиану этой случайной величины.


\vfill

\z Время между двумя последовательными попаданиями метеорита в космический корабль имеет показательное распределение с параметром $\lambda = 0.04$. Найдите: \\ \quad а) среднее квадратическое отклонение времени между двумя попаданиями метеорита; \\ \quad б) вероятность того, что за время длительностью $t = 30$ корабль не столкнется с метеоритом.
 

\vfill

\newpage\setcounter{zad}{0}

\z Непрерывная случайная величина $X$ распределена равномерно на отрезке $[-1; 1]$. Найдите: \\ \quad а) математическое ожидание $M(X)$; \\ \quad б) вероятность $P(X>0{,}2)$; \\ \quad в) $90\%$-ную точку этой случайной величины.


\vfill

\z Время между появлениями двух последовательных покупателей в некотором магазине имеет показательное распределение с параметром $\lambda = 0.0625$. Найдите: \\ \quad а) математическое ожидание времени между появлениями двух покупателей; \\ \quad б) вероятность того, что за время длительностью $t = 40$  произойдет появление покупателя.
 

\vfill

\newpage\setcounter{zad}{0}

\z Непрерывная случайная величина $X$ распределена равномерно на отрезке $[1; 4]$. Найдите: \\ \quad а) среднее квадратическое отклонение $\sigma_X$; \\ \quad б) вероятность $P(X>3{,}7)$; \\ \quad в) квантиль $x_{0{,}9}$ этой случайной величины.


\vfill

\z Время между двумя последовательными попаданиями метеорита в космический корабль имеет показательное распределение с параметром $\lambda = 0.025$. Найдите: \\ \quad а) дисперсию времени между двумя попаданиями метеорита; \\ \quad б) вероятность того, что за время длительностью $t = 50$ корабль не столкнется с метеоритом.
 

\vfill

\newpage\setcounter{zad}{0}

\z Непрерывная случайная величина $X$ распределена равномерно на отрезке $[4; 7]$. Найдите: \\ \quad а) среднее квадратическое отклонение $\sigma_X$; \\ \quad б) вероятность $P(X>4{,}9)$; \\ \quad в) $80\%$-ную точку этой случайной величины.


\vfill

\z Длительность времени безотказной работы прибора имеет показательное распределение с параметром $\lambda = 0.025$. Найдите: \\ \quad а) дисперсию времени безотказной работы прибора; \\ \quad б) вероятность того, что за время длительностью $t = 60$ прибор не откажет.
 

\vfill

\newpage\setcounter{zad}{0}

\z Непрерывная случайная величина $X$ распределена равномерно на отрезке $[-3; 1]$. Найдите: \\ \quad а) математическое ожидание $M(X)$; \\ \quad б) вероятность $P(X>0{,}6)$; \\ \quad в) квантиль $x_{0{,}5}$ этой случайной величины.


\vfill

\z Время между двумя последовательными попаданиями метеорита в космический корабль имеет показательное распределение с параметром $\lambda = 0.0625$. Найдите: \\ \quad а) среднее квадратическое отклонение времени между двумя попаданиями метеорита; \\ \quad б) вероятность того, что за время длительностью $t = 20$ корабль  столкнется с метеоритом.
 

\vfill

\newpage\setcounter{zad}{0}

\z Непрерывная случайная величина $X$ распределена равномерно на отрезке $[-1; 2]$. Найдите: \\ \quad а) дисперсию $D(X)$; \\ \quad б) вероятность $P(X>-0{,}1)$; \\ \quad в) квантиль $x_{0{,}1}$ этой случайной величины.


\vfill

\z Длительность времени безотказной работы прибора имеет показательное распределение с параметром $\lambda = 0.04$. Найдите: \\ \quad а) математическое ожидание времени безотказной работы прибора; \\ \quad б) вероятность того, что за время длительностью $t = 20$ прибор  откажет.
 

\vfill

\newpage\setcounter{zad}{0}

\z Непрерывная случайная величина $X$ распределена равномерно на отрезке $[-1; 3]$. Найдите: \\ \quad а) дисперсию $D(X)$; \\ \quad б) вероятность $P(X>-0{,}6)$; \\ \quad в) $90\%$-ную точку этой случайной величины.


\vfill

\z Время между появлениями двух последовательных покупателей в некотором магазине имеет показательное распределение с параметром $\lambda = 0.02$. Найдите: \\ \quad а) дисперсию времени между появлениями двух покупателей; \\ \quad б) вероятность того, что за время длительностью $t = 20$ не произойдет появление покупателя.
 

\vfill

\newpage\setcounter{zad}{0}

\z Непрерывная случайная величина $X$ распределена равномерно на отрезке $[0; 5]$. Найдите: \\ \quad а) дисперсию $D(X)$; \\ \quad б) вероятность $P(X>0{,}5)$; \\ \quad в) медиану этой случайной величины.


\vfill

\z Длительность времени безотказной работы прибора имеет показательное распределение с параметром $\lambda = 0.0125$. Найдите: \\ \quad а) математическое ожидание времени безотказной работы прибора; \\ \quad б) вероятность того, что за время длительностью $t = 20$ прибор не откажет.
 

\vfill

\newpage\setcounter{zad}{0}

\z Непрерывная случайная величина $X$ распределена равномерно на отрезке $[4; 6]$. Найдите: \\ \quad а) математическое ожидание $M(X)$; \\ \quad б) вероятность $P(X>5{,}6)$; \\ \quad в) $50\%$-ную точку этой случайной величины.


\vfill

\z Длительность времени безотказной работы прибора имеет показательное распределение с параметром $\lambda = 0.01$. Найдите: \\ \quad а) дисперсию времени безотказной работы прибора; \\ \quad б) вероятность того, что за время длительностью $t = 30$ прибор  откажет.
 

\vfill

\newpage\setcounter{zad}{0}

\z Непрерывная случайная величина $X$ распределена равномерно на отрезке $[-2; -1]$. Найдите: \\ \quad а) математическое ожидание $M(X)$; \\ \quad б) вероятность $P(X>-1{,}6)$; \\ \quad в) квантиль $x_{0{,}1}$ этой случайной величины.


\vfill

\z Время между двумя последовательными попаданиями метеорита в космический корабль имеет показательное распределение с параметром $\lambda = 0.04$. Найдите: \\ \quad а) среднее квадратическое отклонение времени между двумя попаданиями метеорита; \\ \quad б) вероятность того, что за время длительностью $t = 50$ корабль не столкнется с метеоритом.
 

\vfill

\newpage\setcounter{zad}{0}

\z Непрерывная случайная величина $X$ распределена равномерно на отрезке $[-1; 1]$. Найдите: \\ \quad а) математическое ожидание $M(X)$; \\ \quad б) вероятность $P(X>-0{,}4)$; \\ \quad в) медиану этой случайной величины.


\vfill

\z Длительность времени безотказной работы прибора имеет показательное распределение с параметром $\lambda = 0.025$. Найдите: \\ \quad а) среднее квадратическое отклонение времени безотказной работы прибора; \\ \quad б) вероятность того, что за время длительностью $t = 20$ прибор не откажет.
 

\vfill

\newpage\setcounter{zad}{0}

\z Непрерывная случайная величина $X$ распределена равномерно на отрезке $[0; 3]$. Найдите: \\ \quad а) дисперсию $D(X)$; \\ \quad б) вероятность $P(X>1{,}5)$; \\ \quad в) медиану этой случайной величины.


\vfill

\z Длительность времени безотказной работы прибора имеет показательное распределение с параметром $\lambda = 0.04$. Найдите: \\ \quad а) среднее время безотказной работы прибора; \\ \quad б) вероятность того, что за время длительностью $t = 20$ прибор  откажет.
 

\vfill

\newpage\setcounter{zad}{0}

\z Непрерывная случайная величина $X$ распределена равномерно на отрезке $[1; 4]$. Найдите: \\ \quad а) среднее квадратическое отклонение $\sigma_X$; \\ \quad б) вероятность $P(X>2{,}5)$; \\ \quad в) $90\%$-ную точку этой случайной величины.


\vfill

\z Время между двумя последовательными попаданиями метеорита в космический корабль имеет показательное распределение с параметром $\lambda = 0.0125$. Найдите: \\ \quad а) дисперсию времени между двумя попаданиями метеорита; \\ \quad б) вероятность того, что за время длительностью $t = 110$ корабль  столкнется с метеоритом.
 

\vfill

\newpage\setcounter{zad}{0}

\z Непрерывная случайная величина $X$ распределена равномерно на отрезке $[1; 6]$. Найдите: \\ \quad а) математическое ожидание $M(X)$; \\ \quad б) вероятность $P(X>2{,}0)$; \\ \quad в) квантиль $x_{0{,}5}$ этой случайной величины.


\vfill

\z Длительность времени безотказной работы прибора имеет показательное распределение с параметром $\lambda = 0.0625$. Найдите: \\ \quad а) математическое ожидание времени безотказной работы прибора; \\ \quad б) вероятность того, что за время длительностью $t = 20$ прибор  откажет.
 

\vfill

\newpage\setcounter{zad}{0}

\z Непрерывная случайная величина $X$ распределена равномерно на отрезке $[3; 8]$. Найдите: \\ \quad а) дисперсию $D(X)$; \\ \quad б) вероятность $P(X>5{,}0)$; \\ \quad в) $60\%$-ную точку этой случайной величины.


\vfill

\z Время между двумя последовательными попаданиями метеорита в космический корабль имеет показательное распределение с параметром $\lambda = 0.04$. Найдите: \\ \quad а) математическое ожидание времени между двумя попаданиями метеорита; \\ \quad б) вероятность того, что за время длительностью $t = 50$ корабль не столкнется с метеоритом.
 

\vfill

\newpage\setcounter{zad}{0}

\z Непрерывная случайная величина $X$ распределена равномерно на отрезке $[-2; 3]$. Найдите: \\ \quad а) математическое ожидание $M(X)$; \\ \quad б) вероятность $P(X>-1{,}0)$; \\ \quad в) $30\%$-ную точку этой случайной величины.


\vfill

\z Время между появлениями двух последовательных покупателей в некотором магазине имеет показательное распределение с параметром $\lambda = 0.04$. Найдите: \\ \quad а) математическое ожидание времени между появлениями двух покупателей; \\ \quad б) вероятность того, что за время длительностью $t = 40$  произойдет появление покупателя.
 

\vfill

\newpage\setcounter{zad}{0}

\z Непрерывная случайная величина $X$ распределена равномерно на отрезке $[4; 5]$. Найдите: \\ \quad а) среднее квадратическое отклонение $\sigma_X$; \\ \quad б) вероятность $P(X>4{,}6)$; \\ \quad в) медиану этой случайной величины.


\vfill

\z Время между появлениями двух последовательных покупателей в некотором магазине имеет показательное распределение с параметром $\lambda = 0.01$. Найдите: \\ \quad а) математическое ожидание времени между появлениями двух покупателей; \\ \quad б) вероятность того, что за время длительностью $t = 70$  произойдет появление покупателя.
 

\vfill

\newpage\setcounter{zad}{0}

\z Непрерывная случайная величина $X$ распределена равномерно на отрезке $[1; 2]$. Найдите: \\ \quad а) дисперсию $D(X)$; \\ \quad б) вероятность $P(X>1{,}4)$; \\ \quad в) медиану этой случайной величины.


\vfill

\z Время между двумя последовательными попаданиями метеорита в космический корабль имеет показательное распределение с параметром $\lambda = 0.0625$. Найдите: \\ \quad а) математическое ожидание времени между двумя попаданиями метеорита; \\ \quad б) вероятность того, что за время длительностью $t = 40$ корабль  столкнется с метеоритом.
 

\vfill

\newpage\setcounter{zad}{0}

\z Непрерывная случайная величина $X$ распределена равномерно на отрезке $[-2; 1]$. Найдите: \\ \quad а) математическое ожидание $M(X)$; \\ \quad б) вероятность $P(X>0{,}7)$; \\ \quad в) медиану этой случайной величины.


\vfill

\z Время между появлениями двух последовательных покупателей в некотором магазине имеет показательное распределение с параметром $\lambda = 0.04$. Найдите: \\ \quad а) математическое ожидание времени между появлениями двух покупателей; \\ \quad б) вероятность того, что за время длительностью $t = 40$ не произойдет появление покупателя.
 

\vfill

\newpage\setcounter{zad}{0}

\z Непрерывная случайная величина $X$ распределена равномерно на отрезке $[1; 2]$. Найдите: \\ \quad а) дисперсию $D(X)$; \\ \quad б) вероятность $P(X>1{,}7)$; \\ \quad в) $70\%$-ную точку этой случайной величины.


\vfill

\z Время между появлениями двух последовательных покупателей в некотором магазине имеет показательное распределение с параметром $\lambda = 0.025$. Найдите: \\ \quad а) математическое ожидание времени между появлениями двух покупателей; \\ \quad б) вероятность того, что за время длительностью $t = 40$  произойдет появление покупателя.
 

\vfill

\newpage\setcounter{zad}{0}

\z Непрерывная случайная величина $X$ распределена равномерно на отрезке $[-2; 0]$. Найдите: \\ \quad а) среднее квадратическое отклонение $\sigma_X$; \\ \quad б) вероятность $P(X>-1{,}4)$; \\ \quad в) квантиль $x_{0{,}8}$ этой случайной величины.


\vfill

\z Время между появлениями двух последовательных покупателей в некотором магазине имеет показательное распределение с параметром $\lambda = 0.0625$. Найдите: \\ \quad а) дисперсию времени между появлениями двух покупателей; \\ \quad б) вероятность того, что за время длительностью $t = 40$ не произойдет появление покупателя.
 

\vfill

\newpage\setcounter{zad}{0}

\z Непрерывная случайная величина $X$ распределена равномерно на отрезке $[4; 9]$. Найдите: \\ \quad а) среднее квадратическое отклонение $\sigma_X$; \\ \quad б) вероятность $P(X>7{,}5)$; \\ \quad в) квантиль $x_{0{,}4}$ этой случайной величины.


\vfill

\z Длительность времени безотказной работы прибора имеет показательное распределение с параметром $\lambda = 0.0625$. Найдите: \\ \quad а) дисперсию времени безотказной работы прибора; \\ \quad б) вероятность того, что за время длительностью $t = 20$ прибор  откажет.
 

\vfill

\newpage\setcounter{zad}{0}

\z Непрерывная случайная величина $X$ распределена равномерно на отрезке $[-3; -1]$. Найдите: \\ \quad а) дисперсию $D(X)$; \\ \quad б) вероятность $P(X>-2{,}2)$; \\ \quad в) $40\%$-ную точку этой случайной величины.


\vfill

\z Время между двумя последовательными попаданиями метеорита в космический корабль имеет показательное распределение с параметром $\lambda = 0.04$. Найдите: \\ \quad а) дисперсию времени между двумя попаданиями метеорита; \\ \quad б) вероятность того, что за время длительностью $t = 60$ корабль не столкнется с метеоритом.
 

\vfill

\newpage\setcounter{zad}{0}

\z Непрерывная случайная величина $X$ распределена равномерно на отрезке $[1; 6]$. Найдите: \\ \quad а) дисперсию $D(X)$; \\ \quad б) вероятность $P(X>3{,}5)$; \\ \quad в) $70\%$-ную точку этой случайной величины.


\vfill

\z Длительность времени безотказной работы прибора имеет показательное распределение с параметром $\lambda = 0.01$. Найдите: \\ \quad а) среднее квадратическое отклонение времени безотказной работы прибора; \\ \quad б) вероятность того, что за время длительностью $t = 120$ прибор  откажет.
 

\vfill

\newpage\setcounter{zad}{0}

\z Непрерывная случайная величина $X$ распределена равномерно на отрезке $[-3; -2]$. Найдите: \\ \quad а) среднее квадратическое отклонение $\sigma_X$; \\ \quad б) вероятность $P(X>-2{,}3)$; \\ \quad в) квантиль $x_{0{,}2}$ этой случайной величины.


\vfill

\z Время между появлениями двух последовательных покупателей в некотором магазине имеет показательное распределение с параметром $\lambda = 0.05$. Найдите: \\ \quad а) дисперсию времени между появлениями двух покупателей; \\ \quad б) вероятность того, что за время длительностью $t = 20$  произойдет появление покупателя.
 

\vfill

\newpage\setcounter{zad}{0}

\z Непрерывная случайная величина $X$ распределена равномерно на отрезке $[-3; -1]$. Найдите: \\ \quad а) дисперсию $D(X)$; \\ \quad б) вероятность $P(X>-2{,}6)$; \\ \quad в) $80\%$-ную точку этой случайной величины.


\vfill

\z Длительность времени безотказной работы прибора имеет показательное распределение с параметром $\lambda = 0.05$. Найдите: \\ \quad а) среднее квадратическое отклонение времени безотказной работы прибора; \\ \quad б) вероятность того, что за время длительностью $t = 20$ прибор  откажет.
 

\vfill

\newpage\setcounter{zad}{0}

\z Непрерывная случайная величина $X$ распределена равномерно на отрезке $[1; 4]$. Найдите: \\ \quad а) математическое ожидание $M(X)$; \\ \quad б) вероятность $P(X>3{,}7)$; \\ \quad в) квантиль $x_{0{,}5}$ этой случайной величины.


\vfill

\z Время между появлениями двух последовательных покупателей в некотором магазине имеет показательное распределение с параметром $\lambda = 0.02$. Найдите: \\ \quad а) математическое ожидание времени между появлениями двух покупателей; \\ \quad б) вероятность того, что за время длительностью $t = 40$  произойдет появление покупателя.
 

\vfill

\newpage\setcounter{zad}{0}

\z Непрерывная случайная величина $X$ распределена равномерно на отрезке $[-4; 0]$. Найдите: \\ \quad а) математическое ожидание $M(X)$; \\ \quad б) вероятность $P(X>-3{,}6)$; \\ \quad в) квантиль $x_{0{,}7}$ этой случайной величины.


\vfill

\z Время между появлениями двух последовательных покупателей в некотором магазине имеет показательное распределение с параметром $\lambda = 0.02$. Найдите: \\ \quad а) среднее квадратическое отклонение времени между появлениями двух покупателей; \\ \quad б) вероятность того, что за время длительностью $t = 80$ не произойдет появление покупателя.
 

\vfill

\newpage\setcounter{zad}{0}

\z Непрерывная случайная величина $X$ распределена равномерно на отрезке $[-1; 4]$. Найдите: \\ \quad а) математическое ожидание $M(X)$; \\ \quad б) вероятность $P(X>1{,}0)$; \\ \quad в) медиану этой случайной величины.


\vfill

\z Время между появлениями двух последовательных покупателей в некотором магазине имеет показательное распределение с параметром $\lambda = 0.05$. Найдите: \\ \quad а) среднее время между появлениями двух покупателей; \\ \quad б) вероятность того, что за время длительностью $t = 20$ не произойдет появление покупателя.
 

\vfill

\newpage\setcounter{zad}{0}

\z Непрерывная случайная величина $X$ распределена равномерно на отрезке $[1; 6]$. Найдите: \\ \quad а) дисперсию $D(X)$; \\ \quad б) вероятность $P(X>2{,}0)$; \\ \quad в) $10\%$-ную точку этой случайной величины.


\vfill

\z Время между двумя последовательными попаданиями метеорита в космический корабль имеет показательное распределение с параметром $\lambda = 0.02$. Найдите: \\ \quad а) среднее время между двумя попаданиями метеорита; \\ \quad б) вероятность того, что за время длительностью $t = 40$ корабль  столкнется с метеоритом.
 

\vfill

\newpage\setcounter{zad}{0}

\z Непрерывная случайная величина $X$ распределена равномерно на отрезке $[3; 7]$. Найдите: \\ \quad а) дисперсию $D(X)$; \\ \quad б) вероятность $P(X>5{,}4)$; \\ \quad в) $50\%$-ную точку этой случайной величины.


\vfill

\z Время между появлениями двух последовательных покупателей в некотором магазине имеет показательное распределение с параметром $\lambda = 0.0125$. Найдите: \\ \quad а) среднее квадратическое отклонение времени между появлениями двух покупателей; \\ \quad б) вероятность того, что за время длительностью $t = 40$ не произойдет появление покупателя.
 

\vfill

\newpage\setcounter{zad}{0}

\z Непрерывная случайная величина $X$ распределена равномерно на отрезке $[1; 5]$. Найдите: \\ \quad а) дисперсию $D(X)$; \\ \quad б) вероятность $P(X>1{,}4)$; \\ \quad в) квантиль $x_{0{,}3}$ этой случайной величины.


\vfill

\z Время между двумя последовательными попаданиями метеорита в космический корабль имеет показательное распределение с параметром $\lambda = 0.05$. Найдите: \\ \quad а) среднее время между двумя попаданиями метеорита; \\ \quad б) вероятность того, что за время длительностью $t = 40$ корабль  столкнется с метеоритом.
 

\vfill

\newpage\setcounter{zad}{0}

\z Непрерывная случайная величина $X$ распределена равномерно на отрезке $[4; 8]$. Найдите: \\ \quad а) математическое ожидание $M(X)$; \\ \quad б) вероятность $P(X>7{,}2)$; \\ \quad в) медиану этой случайной величины.


\vfill

\z Время между двумя последовательными попаданиями метеорита в космический корабль имеет показательное распределение с параметром $\lambda = 0.04$. Найдите: \\ \quad а) дисперсию времени между двумя попаданиями метеорита; \\ \quad б) вероятность того, что за время длительностью $t = 30$ корабль не столкнется с метеоритом.
 

\vfill

\newpage\setcounter{zad}{0}

\z Непрерывная случайная величина $X$ распределена равномерно на отрезке $[-3; 1]$. Найдите: \\ \quad а) математическое ожидание $M(X)$; \\ \quad б) вероятность $P(X>-0{,}6)$; \\ \quad в) $40\%$-ную точку этой случайной величины.


\vfill

\z Время между двумя последовательными попаданиями метеорита в космический корабль имеет показательное распределение с параметром $\lambda = 0.01$. Найдите: \\ \quad а) среднее время между двумя попаданиями метеорита; \\ \quad б) вероятность того, что за время длительностью $t = 30$ корабль не столкнется с метеоритом.
 

\vfill

\newpage\setcounter{zad}{0}

\z Непрерывная случайная величина $X$ распределена равномерно на отрезке $[3; 4]$. Найдите: \\ \quad а) математическое ожидание $M(X)$; \\ \quad б) вероятность $P(X>3{,}6)$; \\ \quad в) медиану этой случайной величины.


\vfill

\z Длительность времени безотказной работы прибора имеет показательное распределение с параметром $\lambda = 0.04$. Найдите: \\ \quad а) среднее квадратическое отклонение времени безотказной работы прибора; \\ \quad б) вероятность того, что за время длительностью $t = 70$ прибор не откажет.
 

\vfill

\newpage\setcounter{zad}{0}

\z Непрерывная случайная величина $X$ распределена равномерно на отрезке $[4; 8]$. Найдите: \\ \quad а) математическое ожидание $M(X)$; \\ \quad б) вероятность $P(X>5{,}2)$; \\ \quad в) квантиль $x_{0{,}5}$ этой случайной величины.


\vfill

\z Время между появлениями двух последовательных покупателей в некотором магазине имеет показательное распределение с параметром $\lambda = 0.025$. Найдите: \\ \quad а) математическое ожидание времени между появлениями двух покупателей; \\ \quad б) вероятность того, что за время длительностью $t = 80$ не произойдет появление покупателя.
 

\vfill

\newpage\setcounter{zad}{0}

\z Непрерывная случайная величина $X$ распределена равномерно на отрезке $[0; 4]$. Найдите: \\ \quad а) дисперсию $D(X)$; \\ \quad б) вероятность $P(X>3{,}2)$; \\ \quad в) квантиль $x_{0{,}8}$ этой случайной величины.


\vfill

\z Длительность времени безотказной работы прибора имеет показательное распределение с параметром $\lambda = 0.04$. Найдите: \\ \quad а) дисперсию времени безотказной работы прибора; \\ \quad б) вероятность того, что за время длительностью $t = 40$ прибор  откажет.
 

\vfill

\newpage\setcounter{zad}{0}

\z Непрерывная случайная величина $X$ распределена равномерно на отрезке $[-2; -1]$. Найдите: \\ \quad а) среднее квадратическое отклонение $\sigma_X$; \\ \quad б) вероятность $P(X>-1{,}1)$; \\ \quad в) медиану этой случайной величины.


\vfill

\z Длительность времени безотказной работы прибора имеет показательное распределение с параметром $\lambda = 0.0125$. Найдите: \\ \quad а) среднее время безотказной работы прибора; \\ \quad б) вероятность того, что за время длительностью $t = 120$ прибор не откажет.
 

\vfill

\newpage\setcounter{zad}{0}