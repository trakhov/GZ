

\z Случайная величина $X$ распределена равномерно на отрезке $[4; 9]$. Найдите: \\ \quad а) математическое ожидание $M(X)$; \\ \quad б) среднее квадратическое отклонение $\sigma_X$; \\ \quad в) вероятность $P(X>5{,}0)$.Случайная величина $X$ распределена равномерно на отрезке $[0; 1]$. Найдите: \\ \quad а) математическое ожидание $M(X)$; \\ \quad б) среднее квадратическое отклонение $\sigma_X$; \\ \quad в) вероятность $P(X>0{,}6)$.Случайная величина $X$ распределена равномерно на отрезке $[1; 4]$. Найдите: \\ \quad а) математическое ожидание $M(X)$; \\ \quad б) среднее квадратическое отклонение $\sigma_X$; \\ \quad в) вероятность $P(X>1{,}6)$.Случайная величина $X$ распределена равномерно на отрезке $[-1; 3]$. Найдите: \\ \quad а) математическое ожидание $M(X)$; \\ \quad б) среднее квадратическое отклонение $\sigma_X$; \\ \quad в) вероятность $P(X>-0{,}2)$.Случайная величина $X$ распределена равномерно на отрезке $[1; 2]$. Найдите: \\ \quad а) математическое ожидание $M(X)$; \\ \quad б) среднее квадратическое отклонение $\sigma_X$; \\ \quad в) вероятность $P(X>1{,}3)$.Случайная величина $X$ распределена равномерно на отрезке $[-4; 0]$. Найдите: \\ \quad а) математическое ожидание $M(X)$; \\ \quad б) среднее квадратическое отклонение $\sigma_X$; \\ \quad в) вероятность $P(X>-3{,}2)$.Случайная величина $X$ распределена равномерно на отрезке $[4; 7]$. Найдите: \\ \quad а) математическое ожидание $M(X)$; \\ \quad б) среднее квадратическое отклонение $\sigma_X$; \\ \quad в) вероятность $P(X>5{,}2)$.Случайная величина $X$ распределена равномерно на отрезке $[-3; -2]$. Найдите: \\ \quad а) математическое ожидание $M(X)$; \\ \quad б) среднее квадратическое отклонение $\sigma_X$; \\ \quad в) вероятность $P(X>-2{,}5)$.Случайная величина $X$ распределена равномерно на отрезке $[-1; 1]$. Найдите: \\ \quad а) математическое ожидание $M(X)$; \\ \quad б) среднее квадратическое отклонение $\sigma_X$; \\ \quad в) вероятность $P(X>-0{,}4)$.Случайная величина $X$ распределена равномерно на отрезке $[-4; -1]$. Найдите: \\ \quad а) математическое ожидание $M(X)$; \\ \quad б) среднее квадратическое отклонение $\sigma_X$; \\ \quad в) вероятность $P(X>-1{,}6)$.Случайная величина $X$ распределена равномерно на отрезке $[-1; 4]$. Найдите: \\ \quad а) математическое ожидание $M(X)$; \\ \quad б) среднее квадратическое отклонение $\sigma_X$; \\ \quad в) вероятность $P(X>-0{,}5)$.Случайная величина $X$ распределена равномерно на отрезке $[3; 4]$. Найдите: \\ \quad а) математическое ожидание $M(X)$; \\ \quad б) среднее квадратическое отклонение $\sigma_X$; \\ \quad в) вероятность $P(X>3{,}1)$.Случайная величина $X$ распределена равномерно на отрезке $[0; 3]$. Найдите: \\ \quad а) математическое ожидание $M(X)$; \\ \quad б) среднее квадратическое отклонение $\sigma_X$; \\ \quad в) вероятность $P(X>0{,}6)$.Случайная величина $X$ распределена равномерно на отрезке $[3; 8]$. Найдите: \\ \quad а) математическое ожидание $M(X)$; \\ \quad б) среднее квадратическое отклонение $\sigma_X$; \\ \quad в) вероятность $P(X>7{,}0)$.Случайная величина $X$ распределена равномерно на отрезке $[-3; -2]$. Найдите: \\ \quad а) математическое ожидание $M(X)$; \\ \quad б) среднее квадратическое отклонение $\sigma_X$; \\ \quad в) вероятность $P(X>-2{,}1)$.Случайная величина $X$ распределена равномерно на отрезке $[3; 7]$. Найдите: \\ \quad а) математическое ожидание $M(X)$; \\ \quad б) среднее квадратическое отклонение $\sigma_X$; \\ \quad в) вероятность $P(X>4{,}6)$.Случайная величина $X$ распределена равномерно на отрезке $[-1; 0]$. Найдите: \\ \quad а) математическое ожидание $M(X)$; \\ \quad б) среднее квадратическое отклонение $\sigma_X$; \\ \quad в) вероятность $P(X>-0{,}8)$.Случайная величина $X$ распределена равномерно на отрезке $[-1; 1]$. Найдите: \\ \quad а) математическое ожидание $M(X)$; \\ \quad б) среднее квадратическое отклонение $\sigma_X$; \\ \quad в) вероятность $P(X>0{,}0)$.Случайная величина $X$ распределена равномерно на отрезке $[2; 5]$. Найдите: \\ \quad а) математическое ожидание $M(X)$; \\ \quad б) среднее квадратическое отклонение $\sigma_X$; \\ \quad в) вероятность $P(X>4{,}1)$.Случайная величина $X$ распределена равномерно на отрезке $[0; 1]$. Найдите: \\ \quad а) математическое ожидание $M(X)$; \\ \quad б) среднее квадратическое отклонение $\sigma_X$; \\ \quad в) вероятность $P(X>0{,}2)$.Случайная величина $X$ распределена равномерно на отрезке $[0; 4]$. Найдите: \\ \quad а) математическое ожидание $M(X)$; \\ \quad б) среднее квадратическое отклонение $\sigma_X$; \\ \quad в) вероятность $P(X>1{,}6)$.Случайная величина $X$ распределена равномерно на отрезке $[-3; -2]$. Найдите: \\ \quad а) математическое ожидание $M(X)$; \\ \quad б) среднее квадратическое отклонение $\sigma_X$; \\ \quad в) вероятность $P(X>-2{,}6)$.Случайная величина $X$ распределена равномерно на отрезке $[0; 5]$. Найдите: \\ \quad а) математическое ожидание $M(X)$; \\ \quad б) среднее квадратическое отклонение $\sigma_X$; \\ \quad в) вероятность $P(X>4{,}5)$.Случайная величина $X$ распределена равномерно на отрезке $[-4; -3]$. Найдите: \\ \quad а) математическое ожидание $M(X)$; \\ \quad б) среднее квадратическое отклонение $\sigma_X$; \\ \quad в) вероятность $P(X>-3{,}3)$.Случайная величина $X$ распределена равномерно на отрезке $[0; 1]$. Найдите: \\ \quad а) математическое ожидание $M(X)$; \\ \quad б) среднее квадратическое отклонение $\sigma_X$; \\ \quad в) вероятность $P(X>0{,}1)$.Случайная величина $X$ распределена равномерно на отрезке $[-4; -3]$. Найдите: \\ \quad а) математическое ожидание $M(X)$; \\ \quad б) среднее квадратическое отклонение $\sigma_X$; \\ \quad в) вероятность $P(X>-3{,}9)$.Случайная величина $X$ распределена равномерно на отрезке $[3; 5]$. Найдите: \\ \quad а) математическое ожидание $M(X)$; \\ \quad б) среднее квадратическое отклонение $\sigma_X$; \\ \quad в) вероятность $P(X>4{,}6)$.Случайная величина $X$ распределена равномерно на отрезке $[4; 9]$. Найдите: \\ \quad а) математическое ожидание $M(X)$; \\ \quad б) среднее квадратическое отклонение $\sigma_X$; \\ \quad в) вероятность $P(X>6{,}5)$.Случайная величина $X$ распределена равномерно на отрезке $[2; 6]$. Найдите: \\ \quad а) математическое ожидание $M(X)$; \\ \quad б) среднее квадратическое отклонение $\sigma_X$; \\ \quad в) вероятность $P(X>5{,}6)$.Случайная величина $X$ распределена равномерно на отрезке $[-3; 0]$. Найдите: \\ \quad а) математическое ожидание $M(X)$; \\ \quad б) среднее квадратическое отклонение $\sigma_X$; \\ \quad в) вероятность $P(X>-1{,}2)$.Случайная величина $X$ распределена равномерно на отрезке $[-3; -2]$. Найдите: \\ \quad а) математическое ожидание $M(X)$; \\ \quad б) среднее квадратическое отклонение $\sigma_X$; \\ \quad в) вероятность $P(X>-2{,}2)$.Случайная величина $X$ распределена равномерно на отрезке $[1; 6]$. Найдите: \\ \quad а) математическое ожидание $M(X)$; \\ \quad б) среднее квадратическое отклонение $\sigma_X$; \\ \quad в) вероятность $P(X>2{,}0)$.Случайная величина $X$ распределена равномерно на отрезке $[2; 3]$. Найдите: \\ \quad а) математическое ожидание $M(X)$; \\ \quad б) среднее квадратическое отклонение $\sigma_X$; \\ \quad в) вероятность $P(X>2{,}4)$.Случайная величина $X$ распределена равномерно на отрезке $[4; 5]$. Найдите: \\ \quad а) математическое ожидание $M(X)$; \\ \quad б) среднее квадратическое отклонение $\sigma_X$; \\ \quad в) вероятность $P(X>4{,}3)$.Случайная величина $X$ распределена равномерно на отрезке $[-4; -3]$. Найдите: \\ \quad а) математическое ожидание $M(X)$; \\ \quad б) среднее квадратическое отклонение $\sigma_X$; \\ \quad в) вероятность $P(X>-3{,}7)$.Случайная величина $X$ распределена равномерно на отрезке $[1; 4]$. Найдите: \\ \quad а) математическое ожидание $M(X)$; \\ \quad б) среднее квадратическое отклонение $\sigma_X$; \\ \quad в) вероятность $P(X>2{,}2)$.Случайная величина $X$ распределена равномерно на отрезке $[-3; 0]$. Найдите: \\ \quad а) математическое ожидание $M(X)$; \\ \quad б) среднее квадратическое отклонение $\sigma_X$; \\ \quad в) вероятность $P(X>-1{,}8)$.Случайная величина $X$ распределена равномерно на отрезке $[-3; -1]$. Найдите: \\ \quad а) математическое ожидание $M(X)$; \\ \quad б) среднее квадратическое отклонение $\sigma_X$; \\ \quad в) вероятность $P(X>-1{,}2)$.Случайная величина $X$ распределена равномерно на отрезке $[1; 4]$. Найдите: \\ \quad а) математическое ожидание $M(X)$; \\ \quad б) среднее квадратическое отклонение $\sigma_X$; \\ \quad в) вероятность $P(X>2{,}5)$.Случайная величина $X$ распределена равномерно на отрезке $[3; 6]$. Найдите: \\ \quad а) математическое ожидание $M(X)$; \\ \quad б) среднее квадратическое отклонение $\sigma_X$; \\ \quad в) вероятность $P(X>4{,}5)$.Случайная величина $X$ распределена равномерно на отрезке $[-4; -1]$. Найдите: \\ \quad а) математическое ожидание $M(X)$; \\ \quad б) среднее квадратическое отклонение $\sigma_X$; \\ \quad в) вероятность $P(X>-2{,}2)$.Случайная величина $X$ распределена равномерно на отрезке $[-3; -2]$. Найдите: \\ \quad а) математическое ожидание $M(X)$; \\ \quad б) среднее квадратическое отклонение $\sigma_X$; \\ \quad в) вероятность $P(X>-2{,}6)$.Случайная величина $X$ распределена равномерно на отрезке $[0; 5]$. Найдите: \\ \quad а) математическое ожидание $M(X)$; \\ \quad б) среднее квадратическое отклонение $\sigma_X$; \\ \quad в) вероятность $P(X>3{,}0)$.Случайная величина $X$ распределена равномерно на отрезке $[3; 4]$. Найдите: \\ \quad а) математическое ожидание $M(X)$; \\ \quad б) среднее квадратическое отклонение $\sigma_X$; \\ \quad в) вероятность $P(X>3{,}8)$.Случайная величина $X$ распределена равномерно на отрезке $[0; 3]$. Найдите: \\ \quad а) математическое ожидание $M(X)$; \\ \quad б) среднее квадратическое отклонение $\sigma_X$; \\ \quad в) вероятность $P(X>2{,}1)$.Случайная величина $X$ распределена равномерно на отрезке $[-1; 0]$. Найдите: \\ \quad а) математическое ожидание $M(X)$; \\ \quad б) среднее квадратическое отклонение $\sigma_X$; \\ \quad в) вероятность $P(X>-0{,}6)$.Случайная величина $X$ распределена равномерно на отрезке $[3; 4]$. Найдите: \\ \quad а) математическое ожидание $M(X)$; \\ \quad б) среднее квадратическое отклонение $\sigma_X$; \\ \quad в) вероятность $P(X>3{,}9)$.Случайная величина $X$ распределена равномерно на отрезке $[3; 6]$. Найдите: \\ \quad а) математическое ожидание $M(X)$; \\ \quad б) среднее квадратическое отклонение $\sigma_X$; \\ \quad в) вероятность $P(X>5{,}1)$.Случайная величина $X$ распределена равномерно на отрезке $[3; 6]$. Найдите: \\ \quad а) математическое ожидание $M(X)$; \\ \quad б) среднее квадратическое отклонение $\sigma_X$; \\ \quad в) вероятность $P(X>3{,}9)$.Случайная величина $X$ распределена равномерно на отрезке $[0; 3]$. Найдите: \\ \quad а) математическое ожидание $M(X)$; \\ \quad б) среднее квадратическое отклонение $\sigma_X$; \\ \quad в) вероятность $P(X>0{,}9)$.Случайная величина $X$ распределена равномерно на отрезке $[0; 1]$. Найдите: \\ \quad а) математическое ожидание $M(X)$; \\ \quad б) среднее квадратическое отклонение $\sigma_X$; \\ \quad в) вероятность $P(X>0{,}1)$.Случайная величина $X$ распределена равномерно на отрезке $[-2; 2]$. Найдите: \\ \quad а) математическое ожидание $M(X)$; \\ \quad б) среднее квадратическое отклонение $\sigma_X$; \\ \quad в) вероятность $P(X>0{,}8)$.Случайная величина $X$ распределена равномерно на отрезке $[0; 1]$. Найдите: \\ \quad а) математическое ожидание $M(X)$; \\ \quad б) среднее квадратическое отклонение $\sigma_X$; \\ \quad в) вероятность $P(X>0{,}5)$.Случайная величина $X$ распределена равномерно на отрезке $[-4; 0]$. Найдите: \\ \quad а) математическое ожидание $M(X)$; \\ \quad б) среднее квадратическое отклонение $\sigma_X$; \\ \quad в) вероятность $P(X>-3{,}6)$.Случайная величина $X$ распределена равномерно на отрезке $[4; 7]$. Найдите: \\ \quad а) математическое ожидание $M(X)$; \\ \quad б) среднее квадратическое отклонение $\sigma_X$; \\ \quad в) вероятность $P(X>6{,}4)$.Случайная величина $X$ распределена равномерно на отрезке $[-2; 0]$. Найдите: \\ \quad а) математическое ожидание $M(X)$; \\ \quad б) среднее квадратическое отклонение $\sigma_X$; \\ \quad в) вероятность $P(X>-1{,}2)$.Случайная величина $X$ распределена равномерно на отрезке $[4; 8]$. Найдите: \\ \quad а) математическое ожидание $M(X)$; \\ \quad б) среднее квадратическое отклонение $\sigma_X$; \\ \quad в) вероятность $P(X>6{,}8)$.Случайная величина $X$ распределена равномерно на отрезке $[-3; 0]$. Найдите: \\ \quad а) математическое ожидание $M(X)$; \\ \quad б) среднее квадратическое отклонение $\sigma_X$; \\ \quad в) вероятность $P(X>-0{,}6)$.Случайная величина $X$ распределена равномерно на отрезке $[2; 4]$. Найдите: \\ \quad а) математическое ожидание $M(X)$; \\ \quad б) среднее квадратическое отклонение $\sigma_X$; \\ \quad в) вероятность $P(X>3{,}4)$.Случайная величина $X$ распределена равномерно на отрезке $[-4; -1]$. Найдите: \\ \quad а) математическое ожидание $M(X)$; \\ \quad б) среднее квадратическое отклонение $\sigma_X$; \\ \quad в) вероятность $P(X>-1{,}6)$.Случайная величина $X$ распределена равномерно на отрезке $[-1; 4]$. Найдите: \\ \quad а) математическое ожидание $M(X)$; \\ \quad б) среднее квадратическое отклонение $\sigma_X$; \\ \quad в) вероятность $P(X>3{,}0)$.Случайная величина $X$ распределена равномерно на отрезке $[0; 4]$. Найдите: \\ \quad а) математическое ожидание $M(X)$; \\ \quad б) среднее квадратическое отклонение $\sigma_X$; \\ \quad в) вероятность $P(X>2{,}8)$.Случайная величина $X$ распределена равномерно на отрезке $[-4; -3]$. Найдите: \\ \quad а) математическое ожидание $M(X)$; \\ \quad б) среднее квадратическое отклонение $\sigma_X$; \\ \quad в) вероятность $P(X>-3{,}1)$.Случайная величина $X$ распределена равномерно на отрезке $[4; 9]$. Найдите: \\ \quad а) математическое ожидание $M(X)$; \\ \quad б) среднее квадратическое отклонение $\sigma_X$; \\ \quad в) вероятность $P(X>4{,}5)$.Случайная величина $X$ распределена равномерно на отрезке $[2; 6]$. Найдите: \\ \quad а) математическое ожидание $M(X)$; \\ \quad б) среднее квадратическое отклонение $\sigma_X$; \\ \quad в) вероятность $P(X>4{,}0)$.Случайная величина $X$ распределена равномерно на отрезке $[-2; 2]$. Найдите: \\ \quad а) математическое ожидание $M(X)$; \\ \quad б) среднее квадратическое отклонение $\sigma_X$; \\ \quad в) вероятность $P(X>1{,}2)$.Случайная величина $X$ распределена равномерно на отрезке $[1; 5]$. Найдите: \\ \quad а) математическое ожидание $M(X)$; \\ \quad б) среднее квадратическое отклонение $\sigma_X$; \\ \quad в) вероятность $P(X>3{,}8)$.Случайная величина $X$ распределена равномерно на отрезке $[4; 5]$. Найдите: \\ \quad а) математическое ожидание $M(X)$; \\ \quad б) среднее квадратическое отклонение $\sigma_X$; \\ \quad в) вероятность $P(X>4{,}5)$.Случайная величина $X$ распределена равномерно на отрезке $[-3; 1]$. Найдите: \\ \quad а) математическое ожидание $M(X)$; \\ \quad б) среднее квадратическое отклонение $\sigma_X$; \\ \quad в) вероятность $P(X>0{,}6)$.Случайная величина $X$ распределена равномерно на отрезке $[-1; 2]$. Найдите: \\ \quad а) математическое ожидание $M(X)$; \\ \quad б) среднее квадратическое отклонение $\sigma_X$; \\ \quad в) вероятность $P(X>1{,}4)$.Случайная величина $X$ распределена равномерно на отрезке $[3; 7]$. Найдите: \\ \quad а) математическое ожидание $M(X)$; \\ \quad б) среднее квадратическое отклонение $\sigma_X$; \\ \quad в) вероятность $P(X>5{,}0)$.Случайная величина $X$ распределена равномерно на отрезке $[2; 4]$. Найдите: \\ \quad а) математическое ожидание $M(X)$; \\ \quad б) среднее квадратическое отклонение $\sigma_X$; \\ \quad в) вероятность $P(X>3{,}6)$.Случайная величина $X$ распределена равномерно на отрезке $[4; 8]$. Найдите: \\ \quad а) математическое ожидание $M(X)$; \\ \quad б) среднее квадратическое отклонение $\sigma_X$; \\ \quad в) вероятность $P(X>7{,}2)$.Случайная величина $X$ распределена равномерно на отрезке $[-1; 0]$. Найдите: \\ \quad а) математическое ожидание $M(X)$; \\ \quad б) среднее квадратическое отклонение $\sigma_X$; \\ \quad в) вероятность $P(X>-0{,}6)$.Случайная величина $X$ распределена равномерно на отрезке $[-3; 2]$. Найдите: \\ \quad а) математическое ожидание $M(X)$; \\ \quad б) среднее квадратическое отклонение $\sigma_X$; \\ \quad в) вероятность $P(X>-1{,}5)$.Случайная величина $X$ распределена равномерно на отрезке $[-3; 0]$. Найдите: \\ \quad а) математическое ожидание $M(X)$; \\ \quad б) среднее квадратическое отклонение $\sigma_X$; \\ \quad в) вероятность $P(X>-1{,}2)$.Случайная величина $X$ распределена равномерно на отрезке $[1; 5]$. Найдите: \\ \quad а) математическое ожидание $M(X)$; \\ \quad б) среднее квадратическое отклонение $\sigma_X$; \\ \quad в) вероятность $P(X>3{,}4)$.Случайная величина $X$ распределена равномерно на отрезке $[0; 2]$. Найдите: \\ \quad а) математическое ожидание $M(X)$; \\ \quad б) среднее квадратическое отклонение $\sigma_X$; \\ \quad в) вероятность $P(X>0{,}2)$.Случайная величина $X$ распределена равномерно на отрезке $[4; 8]$. Найдите: \\ \quad а) математическое ожидание $M(X)$; \\ \quad б) среднее квадратическое отклонение $\sigma_X$; \\ \quad в) вероятность $P(X>6{,}8)$.Случайная величина $X$ распределена равномерно на отрезке $[2; 6]$. Найдите: \\ \quad а) математическое ожидание $M(X)$; \\ \quad б) среднее квадратическое отклонение $\sigma_X$; \\ \quad в) вероятность $P(X>2{,}4)$.Случайная величина $X$ распределена равномерно на отрезке $[2; 5]$. Найдите: \\ \quad а) математическое ожидание $M(X)$; \\ \quad б) среднее квадратическое отклонение $\sigma_X$; \\ \quad в) вероятность $P(X>3{,}2)$.Случайная величина $X$ распределена равномерно на отрезке $[0; 2]$. Найдите: \\ \quad а) математическое ожидание $M(X)$; \\ \quad б) среднее квадратическое отклонение $\sigma_X$; \\ \quad в) вероятность $P(X>0{,}8)$.Случайная величина $X$ распределена равномерно на отрезке $[4; 8]$. Найдите: \\ \quad а) математическое ожидание $M(X)$; \\ \quad б) среднее квадратическое отклонение $\sigma_X$; \\ \quad в) вероятность $P(X>5{,}6)$.Случайная величина $X$ распределена равномерно на отрезке $[2; 7]$. Найдите: \\ \quad а) математическое ожидание $M(X)$; \\ \quad б) среднее квадратическое отклонение $\sigma_X$; \\ \quad в) вероятность $P(X>3{,}5)$.Случайная величина $X$ распределена равномерно на отрезке $[2; 5]$. Найдите: \\ \quad а) математическое ожидание $M(X)$; \\ \quad б) среднее квадратическое отклонение $\sigma_X$; \\ \quad в) вероятность $P(X>4{,}4)$.Случайная величина $X$ распределена равномерно на отрезке $[-2; 1]$. Найдите: \\ \quad а) математическое ожидание $M(X)$; \\ \quad б) среднее квадратическое отклонение $\sigma_X$; \\ \quad в) вероятность $P(X>-0{,}8)$.Случайная величина $X$ распределена равномерно на отрезке $[4; 9]$. Найдите: \\ \quad а) математическое ожидание $M(X)$; \\ \quad б) среднее квадратическое отклонение $\sigma_X$; \\ \quad в) вероятность $P(X>8{,}5)$.Случайная величина $X$ распределена равномерно на отрезке $[0; 2]$. Найдите: \\ \quad а) математическое ожидание $M(X)$; \\ \quad б) среднее квадратическое отклонение $\sigma_X$; \\ \quad в) вероятность $P(X>0{,}2)$.Случайная величина $X$ распределена равномерно на отрезке $[4; 6]$. Найдите: \\ \quad а) математическое ожидание $M(X)$; \\ \quad б) среднее квадратическое отклонение $\sigma_X$; \\ \quad в) вероятность $P(X>5{,}0)$.Случайная величина $X$ распределена равномерно на отрезке $[1; 4]$. Найдите: \\ \quad а) математическое ожидание $M(X)$; \\ \quad б) среднее квадратическое отклонение $\sigma_X$; \\ \quad в) вероятность $P(X>1{,}9)$.Случайная величина $X$ распределена равномерно на отрезке $[-2; -1]$. Найдите: \\ \quad а) математическое ожидание $M(X)$; \\ \quad б) среднее квадратическое отклонение $\sigma_X$; \\ \quad в) вероятность $P(X>-1{,}1)$.Случайная величина $X$ распределена равномерно на отрезке $[3; 5]$. Найдите: \\ \quad а) математическое ожидание $M(X)$; \\ \quad б) среднее квадратическое отклонение $\sigma_X$; \\ \quad в) вероятность $P(X>4{,}2)$.Случайная величина $X$ распределена равномерно на отрезке $[0; 3]$. Найдите: \\ \quad а) математическое ожидание $M(X)$; \\ \quad б) среднее квадратическое отклонение $\sigma_X$; \\ \quad в) вероятность $P(X>2{,}4)$.Случайная величина $X$ распределена равномерно на отрезке $[1; 4]$. Найдите: \\ \quad а) математическое ожидание $M(X)$; \\ \quad б) среднее квадратическое отклонение $\sigma_X$; \\ \quad в) вероятность $P(X>3{,}1)$.Случайная величина $X$ распределена равномерно на отрезке $[2; 5]$. Найдите: \\ \quad а) математическое ожидание $M(X)$; \\ \quad б) среднее квадратическое отклонение $\sigma_X$; \\ \quad в) вероятность $P(X>4{,}1)$.Случайная величина $X$ распределена равномерно на отрезке $[-3; -1]$. Найдите: \\ \quad а) математическое ожидание $M(X)$; \\ \quad б) среднее квадратическое отклонение $\sigma_X$; \\ \quad в) вероятность $P(X>-2{,}4)$.Случайная величина $X$ распределена равномерно на отрезке $[-3; -1]$. Найдите: \\ \quad а) математическое ожидание $M(X)$; \\ \quad б) среднее квадратическое отклонение $\sigma_X$; \\ \quad в) вероятность $P(X>-1{,}2)$.Случайная величина $X$ распределена равномерно на отрезке $[0; 2]$. Найдите: \\ \quad а) математическое ожидание $M(X)$; \\ \quad б) среднее квадратическое отклонение $\sigma_X$; \\ \quad в) вероятность $P(X>1{,}2)$.Случайная величина $X$ распределена равномерно на отрезке $[-1; 0]$. Найдите: \\ \quad а) математическое ожидание $M(X)$; \\ \quad б) среднее квадратическое отклонение $\sigma_X$; \\ \quad в) вероятность $P(X>-0{,}8)$.Случайная величина $X$ распределена равномерно на отрезке $[0; 4]$. Найдите: \\ \quad а) математическое ожидание $M(X)$; \\ \quad б) среднее квадратическое отклонение $\sigma_X$; \\ \quad в) вероятность $P(X>3{,}2)$.

\vfill

\z Длительность времени безотказной работы прибора имеет показательное распределение с параметром $\lambda = 0.01$. Найдите: \\ \quad а) математическое ожидание времени безотказной работы прибора; \\ \quad б) вероятность того, что за время длительностью $t = 90$ прибор откажет.Длительность времени безотказной работы прибора имеет показательное распределение с параметром $\lambda = 0.02$. Найдите: \\ \quad а) дисперсию времени безотказной работы прибора; \\ \quad б) вероятность того, что за время длительностью $t = 60$ прибор откажет.Длительность времени безотказной работы прибора имеет показательное распределение с параметром $\lambda = 0.02$. Найдите: \\ \quad а) математическое ожидание времени безотказной работы прибора; \\ \quad б) вероятность того, что за время длительностью $t = 80$ прибор откажет.Длительность времени безотказной работы прибора имеет показательное распределение с параметром $\lambda = 0.02$. Найдите: \\ \quad а) среднее квадратическое отклонение времени безотказной работы прибора; \\ \quad б) вероятность того, что за время длительностью $t = 80$ прибор откажет.Длительность времени безотказной работы прибора имеет показательное распределение с параметром $\lambda = 0.03$. Найдите: \\ \quad а) среднее время безотказной работы прибора; \\ \quad б) вероятность того, что за время длительностью $t = 70$ прибор откажет.Длительность времени безотказной работы прибора имеет показательное распределение с параметром $\lambda = 0.01$. Найдите: \\ \quad а) среднее время безотказной работы прибора; \\ \quad б) вероятность того, что за время длительностью $t = 80$ прибор не откажет.Длительность времени безотказной работы прибора имеет показательное распределение с параметром $\lambda = 0.01$. Найдите: \\ \quad а) математическое ожидание времени безотказной работы прибора; \\ \quad б) вероятность того, что за время длительностью $t = 120$ прибор откажет.Длительность времени безотказной работы прибора имеет показательное распределение с параметром $\lambda = 0.02$. Найдите: \\ \quad а) дисперсию времени безотказной работы прибора; \\ \quad б) вероятность того, что за время длительностью $t = 60$ прибор не откажет.Длительность времени безотказной работы прибора имеет показательное распределение с параметром $\lambda = 0.02$. Найдите: \\ \quad а) дисперсию времени безотказной работы прибора; \\ \quad б) вероятность того, что за время длительностью $t = 60$ прибор не откажет.Длительность времени безотказной работы прибора имеет показательное распределение с параметром $\lambda = 0.01$. Найдите: \\ \quad а) среднее квадратическое отклонение времени безотказной работы прибора; \\ \quad б) вероятность того, что за время длительностью $t = 80$ прибор откажет.Длительность времени безотказной работы прибора имеет показательное распределение с параметром $\lambda = 0.01$. Найдите: \\ \quad а) математическое ожидание времени безотказной работы прибора; \\ \quad б) вероятность того, что за время длительностью $t = 90$ прибор откажет.Длительность времени безотказной работы прибора имеет показательное распределение с параметром $\lambda = 0.02$. Найдите: \\ \quad а) дисперсию времени безотказной работы прибора; \\ \quad б) вероятность того, что за время длительностью $t = 70$ прибор откажет.Длительность времени безотказной работы прибора имеет показательное распределение с параметром $\lambda = 0.02$. Найдите: \\ \quad а) среднее квадратическое отклонение времени безотказной работы прибора; \\ \quad б) вероятность того, что за время длительностью $t = 100$ прибор откажет.Длительность времени безотказной работы прибора имеет показательное распределение с параметром $\lambda = 0.01$. Найдите: \\ \quad а) среднее квадратическое отклонение времени безотказной работы прибора; \\ \quad б) вероятность того, что за время длительностью $t = 130$ прибор откажет.Длительность времени безотказной работы прибора имеет показательное распределение с параметром $\lambda = 0.03$. Найдите: \\ \quad а) дисперсию времени безотказной работы прибора; \\ \quad б) вероятность того, что за время длительностью $t = 70$ прибор откажет.Длительность времени безотказной работы прибора имеет показательное распределение с параметром $\lambda = 0.02$. Найдите: \\ \quad а) математическое ожидание времени безотказной работы прибора; \\ \quad б) вероятность того, что за время длительностью $t = 50$ прибор не откажет.Длительность времени безотказной работы прибора имеет показательное распределение с параметром $\lambda = 0.04$. Найдите: \\ \quad а) среднее время безотказной работы прибора; \\ \quad б) вероятность того, что за время длительностью $t = 60$ прибор не откажет.Длительность времени безотказной работы прибора имеет показательное распределение с параметром $\lambda = 0.07$. Найдите: \\ \quad а) среднее время безотказной работы прибора; \\ \quad б) вероятность того, что за время длительностью $t = 30$ прибор откажет.Длительность времени безотказной работы прибора имеет показательное распределение с параметром $\lambda = 0.05$. Найдите: \\ \quad а) математическое ожидание времени безотказной работы прибора; \\ \quad б) вероятность того, что за время длительностью $t = 30$ прибор не откажет.Длительность времени безотказной работы прибора имеет показательное распределение с параметром $\lambda = 0.12$. Найдите: \\ \quad а) дисперсию времени безотказной работы прибора; \\ \quad б) вероятность того, что за время длительностью $t = 20$ прибор не откажет.Длительность времени безотказной работы прибора имеет показательное распределение с параметром $\lambda = 0.01$. Найдите: \\ \quad а) среднее время безотказной работы прибора; \\ \quad б) вероятность того, что за время длительностью $t = 130$ прибор не откажет.Длительность времени безотказной работы прибора имеет показательное распределение с параметром $\lambda = 0.01$. Найдите: \\ \quad а) математическое ожидание времени безотказной работы прибора; \\ \quad б) вероятность того, что за время длительностью $t = 40$ прибор не откажет.Длительность времени безотказной работы прибора имеет показательное распределение с параметром $\lambda = 0.0$. Найдите: \\ \quad а) среднее время безотказной работы прибора; \\ \quad б) вероятность того, что за время длительностью $t = 130$ прибор откажет.Длительность времени безотказной работы прибора имеет показательное распределение с параметром $\lambda = 0.01$. Найдите: \\ \quad а) среднее квадратическое отклонение времени безотказной работы прибора; \\ \quad б) вероятность того, что за время длительностью $t = 50$ прибор откажет.Длительность времени безотказной работы прибора имеет показательное распределение с параметром $\lambda = 0.01$. Найдите: \\ \quad а) среднее квадратическое отклонение времени безотказной работы прибора; \\ \quad б) вероятность того, что за время длительностью $t = 130$ прибор не откажет.Длительность времени безотказной работы прибора имеет показательное распределение с параметром $\lambda = 0.15$. Найдите: \\ \quad а) среднее время безотказной работы прибора; \\ \quad б) вероятность того, что за время длительностью $t = 20$ прибор не откажет.Длительность времени безотказной работы прибора имеет показательное распределение с параметром $\lambda = 0.02$. Найдите: \\ \quad а) среднее квадратическое отклонение времени безотказной работы прибора; \\ \quad б) вероятность того, что за время длительностью $t = 110$ прибор не откажет.Длительность времени безотказной работы прибора имеет показательное распределение с параметром $\lambda = 0.04$. Найдите: \\ \quad а) среднее квадратическое отклонение времени безотказной работы прибора; \\ \quad б) вероятность того, что за время длительностью $t = 40$ прибор не откажет.Длительность времени безотказной работы прибора имеет показательное распределение с параметром $\lambda = 0.02$. Найдите: \\ \quad а) дисперсию времени безотказной работы прибора; \\ \quad б) вероятность того, что за время длительностью $t = 150$ прибор не откажет.Длительность времени безотказной работы прибора имеет показательное распределение с параметром $\lambda = 0.0$. Найдите: \\ \quad а) среднее квадратическое отклонение времени безотказной работы прибора; \\ \quad б) вероятность того, что за время длительностью $t = 150$ прибор откажет.Длительность времени безотказной работы прибора имеет показательное распределение с параметром $\lambda = 0.04$. Найдите: \\ \quad а) математическое ожидание времени безотказной работы прибора; \\ \quad б) вероятность того, что за время длительностью $t = 60$ прибор не откажет.Длительность времени безотказной работы прибора имеет показательное распределение с параметром $\lambda = 0.07$. Найдите: \\ \quad а) дисперсию времени безотказной работы прибора; \\ \quad б) вероятность того, что за время длительностью $t = 20$ прибор не откажет.Длительность времени безотказной работы прибора имеет показательное распределение с параметром $\lambda = 0.0$. Найдите: \\ \quad а) дисперсию времени безотказной работы прибора; \\ \quad б) вероятность того, что за время длительностью $t = 150$ прибор не откажет.Длительность времени безотказной работы прибора имеет показательное распределение с параметром $\lambda = 0.02$. Найдите: \\ \quad а) дисперсию времени безотказной работы прибора; \\ \quad б) вероятность того, что за время длительностью $t = 110$ прибор не откажет.Длительность времени безотказной работы прибора имеет показательное распределение с параметром $\lambda = 0.04$. Найдите: \\ \quad а) математическое ожидание времени безотказной работы прибора; \\ \quad б) вероятность того, что за время длительностью $t = 80$ прибор откажет.Длительность времени безотказной работы прибора имеет показательное распределение с параметром $\lambda = 0.04$. Найдите: \\ \quad а) математическое ожидание времени безотказной работы прибора; \\ \quad б) вероятность того, что за время длительностью $t = 40$ прибор откажет.Длительность времени безотказной работы прибора имеет показательное распределение с параметром $\lambda = 0.01$. Найдите: \\ \quad а) среднее квадратическое отклонение времени безотказной работы прибора; \\ \quad б) вероятность того, что за время длительностью $t = 120$ прибор откажет.Длительность времени безотказной работы прибора имеет показательное распределение с параметром $\lambda = 0.01$. Найдите: \\ \quad а) среднее квадратическое отклонение времени безотказной работы прибора; \\ \quad б) вероятность того, что за время длительностью $t = 90$ прибор откажет.Длительность времени безотказной работы прибора имеет показательное распределение с параметром $\lambda = 0.02$. Найдите: \\ \quad а) среднее время безотказной работы прибора; \\ \quad б) вероятность того, что за время длительностью $t = 110$ прибор откажет.Длительность времени безотказной работы прибора имеет показательное распределение с параметром $\lambda = 0.03$. Найдите: \\ \quad а) среднее квадратическое отклонение времени безотказной работы прибора; \\ \quad б) вероятность того, что за время длительностью $t = 110$ прибор откажет.Длительность времени безотказной работы прибора имеет показательное распределение с параметром $\lambda = 0.01$. Найдите: \\ \quad а) дисперсию времени безотказной работы прибора; \\ \quad б) вероятность того, что за время длительностью $t = 100$ прибор не откажет.Длительность времени безотказной работы прибора имеет показательное распределение с параметром $\lambda = 0.05$. Найдите: \\ \quad а) дисперсию времени безотказной работы прибора; \\ \quad б) вероятность того, что за время длительностью $t = 40$ прибор не откажет.Длительность времени безотказной работы прибора имеет показательное распределение с параметром $\lambda = 0.01$. Найдите: \\ \quad а) математическое ожидание времени безотказной работы прибора; \\ \quad б) вероятность того, что за время длительностью $t = 130$ прибор не откажет.Длительность времени безотказной работы прибора имеет показательное распределение с параметром $\lambda = 0.02$. Найдите: \\ \quad а) среднее квадратическое отклонение времени безотказной работы прибора; \\ \quad б) вероятность того, что за время длительностью $t = 80$ прибор откажет.Длительность времени безотказной работы прибора имеет показательное распределение с параметром $\lambda = 0.09$. Найдите: \\ \quad а) среднее время безотказной работы прибора; \\ \quad б) вероятность того, что за время длительностью $t = 30$ прибор откажет.Длительность времени безотказной работы прибора имеет показательное распределение с параметром $\lambda = 0.08$. Найдите: \\ \quad а) дисперсию времени безотказной работы прибора; \\ \quad б) вероятность того, что за время длительностью $t = 30$ прибор откажет.Длительность времени безотказной работы прибора имеет показательное распределение с параметром $\lambda = 0.08$. Найдите: \\ \quad а) среднее время безотказной работы прибора; \\ \quad б) вероятность того, что за время длительностью $t = 30$ прибор не откажет.Длительность времени безотказной работы прибора имеет показательное распределение с параметром $\lambda = 0.01$. Найдите: \\ \quad а) среднее время безотказной работы прибора; \\ \quad б) вероятность того, что за время длительностью $t = 80$ прибор не откажет.Длительность времени безотказной работы прибора имеет показательное распределение с параметром $\lambda = 0.09$. Найдите: \\ \quad а) среднее квадратическое отклонение времени безотказной работы прибора; \\ \quad б) вероятность того, что за время длительностью $t = 20$ прибор не откажет.Длительность времени безотказной работы прибора имеет показательное распределение с параметром $\lambda = 0.01$. Найдите: \\ \quad а) среднее время безотказной работы прибора; \\ \quad б) вероятность того, что за время длительностью $t = 100$ прибор откажет.Длительность времени безотказной работы прибора имеет показательное распределение с параметром $\lambda = 0.06$. Найдите: \\ \quad а) математическое ожидание времени безотказной работы прибора; \\ \quad б) вероятность того, что за время длительностью $t = 30$ прибор не откажет.Длительность времени безотказной работы прибора имеет показательное распределение с параметром $\lambda = 0.01$. Найдите: \\ \quad а) среднее время безотказной работы прибора; \\ \quad б) вероятность того, что за время длительностью $t = 130$ прибор откажет.Длительность времени безотказной работы прибора имеет показательное распределение с параметром $\lambda = 0.04$. Найдите: \\ \quad а) дисперсию времени безотказной работы прибора; \\ \quad б) вероятность того, что за время длительностью $t = 60$ прибор не откажет.Длительность времени безотказной работы прибора имеет показательное распределение с параметром $\lambda = 0.07$. Найдите: \\ \quad а) среднее время безотказной работы прибора; \\ \quad б) вероятность того, что за время длительностью $t = 40$ прибор откажет.Длительность времени безотказной работы прибора имеет показательное распределение с параметром $\lambda = 0.02$. Найдите: \\ \quad а) среднее квадратическое отклонение времени безотказной работы прибора; \\ \quad б) вероятность того, что за время длительностью $t = 110$ прибор откажет.Длительность времени безотказной работы прибора имеет показательное распределение с параметром $\lambda = 0.03$. Найдите: \\ \quad а) среднее квадратическое отклонение времени безотказной работы прибора; \\ \quad б) вероятность того, что за время длительностью $t = 90$ прибор откажет.Длительность времени безотказной работы прибора имеет показательное распределение с параметром $\lambda = 0.07$. Найдите: \\ \quad а) дисперсию времени безотказной работы прибора; \\ \quad б) вероятность того, что за время длительностью $t = 30$ прибор откажет.Длительность времени безотказной работы прибора имеет показательное распределение с параметром $\lambda = 0.01$. Найдите: \\ \quad а) среднее квадратическое отклонение времени безотказной работы прибора; \\ \quad б) вероятность того, что за время длительностью $t = 90$ прибор откажет.Длительность времени безотказной работы прибора имеет показательное распределение с параметром $\lambda = 0.01$. Найдите: \\ \quad а) среднее квадратическое отклонение времени безотказной работы прибора; \\ \quad б) вероятность того, что за время длительностью $t = 120$ прибор не откажет.Длительность времени безотказной работы прибора имеет показательное распределение с параметром $\lambda = 0.02$. Найдите: \\ \quad а) дисперсию времени безотказной работы прибора; \\ \quad б) вероятность того, что за время длительностью $t = 100$ прибор не откажет.Длительность времени безотказной работы прибора имеет показательное распределение с параметром $\lambda = 0.01$. Найдите: \\ \quad а) дисперсию времени безотказной работы прибора; \\ \quad б) вероятность того, что за время длительностью $t = 130$ прибор не откажет.Длительность времени безотказной работы прибора имеет показательное распределение с параметром $\lambda = 0.02$. Найдите: \\ \quad а) дисперсию времени безотказной работы прибора; \\ \quad б) вероятность того, что за время длительностью $t = 80$ прибор не откажет.Длительность времени безотказной работы прибора имеет показательное распределение с параметром $\lambda = 0.0$. Найдите: \\ \quad а) дисперсию времени безотказной работы прибора; \\ \quad б) вероятность того, что за время длительностью $t = 70$ прибор откажет.Длительность времени безотказной работы прибора имеет показательное распределение с параметром $\lambda = 0.0$. Найдите: \\ \quad а) математическое ожидание времени безотказной работы прибора; \\ \quad б) вероятность того, что за время длительностью $t = 130$ прибор не откажет.Длительность времени безотказной работы прибора имеет показательное распределение с параметром $\lambda = 0.09$. Найдите: \\ \quad а) дисперсию времени безотказной работы прибора; \\ \quad б) вероятность того, что за время длительностью $t = 30$ прибор не откажет.Длительность времени безотказной работы прибора имеет показательное распределение с параметром $\lambda = 0.02$. Найдите: \\ \quad а) дисперсию времени безотказной работы прибора; \\ \quad б) вероятность того, что за время длительностью $t = 150$ прибор не откажет.Длительность времени безотказной работы прибора имеет показательное распределение с параметром $\lambda = 0.07$. Найдите: \\ \quad а) среднее время безотказной работы прибора; \\ \quad б) вероятность того, что за время длительностью $t = 30$ прибор откажет.Длительность времени безотказной работы прибора имеет показательное распределение с параметром $\lambda = 0.01$. Найдите: \\ \quad а) среднее время безотказной работы прибора; \\ \quad б) вероятность того, что за время длительностью $t = 40$ прибор не откажет.Длительность времени безотказной работы прибора имеет показательное распределение с параметром $\lambda = 0.02$. Найдите: \\ \quad а) среднее время безотказной работы прибора; \\ \quad б) вероятность того, что за время длительностью $t = 110$ прибор откажет.Длительность времени безотказной работы прибора имеет показательное распределение с параметром $\lambda = 0.02$. Найдите: \\ \quad а) среднее время безотказной работы прибора; \\ \quad б) вероятность того, что за время длительностью $t = 120$ прибор откажет.Длительность времени безотказной работы прибора имеет показательное распределение с параметром $\lambda = 0.02$. Найдите: \\ \quad а) среднее квадратическое отклонение времени безотказной работы прибора; \\ \quad б) вероятность того, что за время длительностью $t = 140$ прибор откажет.Длительность времени безотказной работы прибора имеет показательное распределение с параметром $\lambda = 0.03$. Найдите: \\ \quad а) среднее время безотказной работы прибора; \\ \quad б) вероятность того, что за время длительностью $t = 100$ прибор откажет.Длительность времени безотказной работы прибора имеет показательное распределение с параметром $\lambda = 0.0$. Найдите: \\ \quad а) среднее квадратическое отклонение времени безотказной работы прибора; \\ \quad б) вероятность того, что за время длительностью $t = 140$ прибор не откажет.Длительность времени безотказной работы прибора имеет показательное распределение с параметром $\lambda = 0.01$. Найдите: \\ \quad а) среднее время безотказной работы прибора; \\ \quad б) вероятность того, что за время длительностью $t = 90$ прибор не откажет.Длительность времени безотказной работы прибора имеет показательное распределение с параметром $\lambda = 0.02$. Найдите: \\ \quad а) дисперсию времени безотказной работы прибора; \\ \quad б) вероятность того, что за время длительностью $t = 130$ прибор откажет.Длительность времени безотказной работы прибора имеет показательное распределение с параметром $\lambda = 0.02$. Найдите: \\ \quad а) дисперсию времени безотказной работы прибора; \\ \quad б) вероятность того, что за время длительностью $t = 50$ прибор откажет.Длительность времени безотказной работы прибора имеет показательное распределение с параметром $\lambda = 0.15$. Найдите: \\ \quad а) среднее время безотказной работы прибора; \\ \quad б) вероятность того, что за время длительностью $t = 20$ прибор откажет.Длительность времени безотказной работы прибора имеет показательное распределение с параметром $\lambda = 0.03$. Найдите: \\ \quad а) дисперсию времени безотказной работы прибора; \\ \quad б) вероятность того, что за время длительностью $t = 70$ прибор откажет.Длительность времени безотказной работы прибора имеет показательное распределение с параметром $\lambda = 0.01$. Найдите: \\ \quad а) среднее время безотказной работы прибора; \\ \quad б) вероятность того, что за время длительностью $t = 100$ прибор не откажет.Длительность времени безотказной работы прибора имеет показательное распределение с параметром $\lambda = 0.0$. Найдите: \\ \quad а) математическое ожидание времени безотказной работы прибора; \\ \quad б) вероятность того, что за время длительностью $t = 100$ прибор не откажет.Длительность времени безотказной работы прибора имеет показательное распределение с параметром $\lambda = 0.01$. Найдите: \\ \quad а) математическое ожидание времени безотказной работы прибора; \\ \quad б) вероятность того, что за время длительностью $t = 40$ прибор откажет.Длительность времени безотказной работы прибора имеет показательное распределение с параметром $\lambda = 0.04$. Найдите: \\ \quad а) среднее квадратическое отклонение времени безотказной работы прибора; \\ \quad б) вероятность того, что за время длительностью $t = 60$ прибор откажет.Длительность времени безотказной работы прибора имеет показательное распределение с параметром $\lambda = 0.04$. Найдите: \\ \quad а) среднее квадратическое отклонение времени безотказной работы прибора; \\ \quad б) вероятность того, что за время длительностью $t = 70$ прибор не откажет.Длительность времени безотказной работы прибора имеет показательное распределение с параметром $\lambda = 0.02$. Найдите: \\ \quad а) математическое ожидание времени безотказной работы прибора; \\ \quad б) вероятность того, что за время длительностью $t = 90$ прибор не откажет.Длительность времени безотказной работы прибора имеет показательное распределение с параметром $\lambda = 0.0$. Найдите: \\ \quad а) дисперсию времени безотказной работы прибора; \\ \quad б) вероятность того, что за время длительностью $t = 60$ прибор не откажет.Длительность времени безотказной работы прибора имеет показательное распределение с параметром $\lambda = 0.03$. Найдите: \\ \quad а) математическое ожидание времени безотказной работы прибора; \\ \quad б) вероятность того, что за время длительностью $t = 60$ прибор не откажет.Длительность времени безотказной работы прибора имеет показательное распределение с параметром $\lambda = 0.02$. Найдите: \\ \quad а) среднее квадратическое отклонение времени безотказной работы прибора; \\ \quad б) вероятность того, что за время длительностью $t = 80$ прибор откажет.Длительность времени безотказной работы прибора имеет показательное распределение с параметром $\lambda = 0.01$. Найдите: \\ \quad а) среднее квадратическое отклонение времени безотказной работы прибора; \\ \quad б) вероятность того, что за время длительностью $t = 120$ прибор откажет.Длительность времени безотказной работы прибора имеет показательное распределение с параметром $\lambda = 0.02$. Найдите: \\ \quad а) математическое ожидание времени безотказной работы прибора; \\ \quad б) вероятность того, что за время длительностью $t = 90$ прибор не откажет.Длительность времени безотказной работы прибора имеет показательное распределение с параметром $\lambda = 0.02$. Найдите: \\ \quad а) дисперсию времени безотказной работы прибора; \\ \quad б) вероятность того, что за время длительностью $t = 110$ прибор откажет.Длительность времени безотказной работы прибора имеет показательное распределение с параметром $\lambda = 0.01$. Найдите: \\ \quad а) среднее квадратическое отклонение времени безотказной работы прибора; \\ \quad б) вероятность того, что за время длительностью $t = 90$ прибор не откажет.Длительность времени безотказной работы прибора имеет показательное распределение с параметром $\lambda = 0.01$. Найдите: \\ \quad а) математическое ожидание времени безотказной работы прибора; \\ \quad б) вероятность того, что за время длительностью $t = 140$ прибор откажет.Длительность времени безотказной работы прибора имеет показательное распределение с параметром $\lambda = 0.01$. Найдите: \\ \quad а) среднее квадратическое отклонение времени безотказной работы прибора; \\ \quad б) вероятность того, что за время длительностью $t = 100$ прибор не откажет.Длительность времени безотказной работы прибора имеет показательное распределение с параметром $\lambda = 0.02$. Найдите: \\ \quad а) математическое ожидание времени безотказной работы прибора; \\ \quad б) вероятность того, что за время длительностью $t = 150$ прибор не откажет.Длительность времени безотказной работы прибора имеет показательное распределение с параметром $\lambda = 0.02$. Найдите: \\ \quad а) среднее время безотказной работы прибора; \\ \quad б) вероятность того, что за время длительностью $t = 120$ прибор не откажет.Длительность времени безотказной работы прибора имеет показательное распределение с параметром $\lambda = 0.01$. Найдите: \\ \quad а) дисперсию времени безотказной работы прибора; \\ \quad б) вероятность того, что за время длительностью $t = 130$ прибор не откажет.Длительность времени безотказной работы прибора имеет показательное распределение с параметром $\lambda = 0.08$. Найдите: \\ \quad а) математическое ожидание времени безотказной работы прибора; \\ \quad б) вероятность того, что за время длительностью $t = 30$ прибор не откажет.Длительность времени безотказной работы прибора имеет показательное распределение с параметром $\lambda = 0.01$. Найдите: \\ \quad а) среднее время безотказной работы прибора; \\ \quad б) вероятность того, что за время длительностью $t = 150$ прибор не откажет.Длительность времени безотказной работы прибора имеет показательное распределение с параметром $\lambda = 0.01$. Найдите: \\ \quad а) дисперсию времени безотказной работы прибора; \\ \quad б) вероятность того, что за время длительностью $t = 110$ прибор не откажет.Длительность времени безотказной работы прибора имеет показательное распределение с параметром $\lambda = 0.0$. Найдите: \\ \quad а) среднее квадратическое отклонение времени безотказной работы прибора; \\ \quad б) вероятность того, что за время длительностью $t = 150$ прибор не откажет. 

\vfill

\newpage\setcounter{zad}{0}

\z 

\vfill

\z  

\vfill

\newpage\setcounter{zad}{0}

\z 

\vfill

\z  

\vfill

\newpage\setcounter{zad}{0}

\z 

\vfill

\z  

\vfill

\newpage\setcounter{zad}{0}

\z 

\vfill

\z  

\vfill

\newpage\setcounter{zad}{0}

\z 

\vfill

\z  

\vfill

\newpage\setcounter{zad}{0}

\z 

\vfill

\z  

\vfill

\newpage\setcounter{zad}{0}

\z 

\vfill

\z  

\vfill

\newpage\setcounter{zad}{0}

\z 

\vfill

\z  

\vfill

\newpage\setcounter{zad}{0}

\z 

\vfill

\z  

\vfill

\newpage\setcounter{zad}{0}

\z 

\vfill

\z  

\vfill

\newpage\setcounter{zad}{0}

\z 

\vfill

\z  

\vfill

\newpage\setcounter{zad}{0}

\z 

\vfill

\z  

\vfill

\newpage\setcounter{zad}{0}

\z 

\vfill

\z  

\vfill

\newpage\setcounter{zad}{0}

\z 

\vfill

\z  

\vfill

\newpage\setcounter{zad}{0}

\z 

\vfill

\z  

\vfill

\newpage\setcounter{zad}{0}

\z 

\vfill

\z  

\vfill

\newpage\setcounter{zad}{0}

\z 

\vfill

\z  

\vfill

\newpage\setcounter{zad}{0}

\z 

\vfill

\z  

\vfill

\newpage\setcounter{zad}{0}

\z 

\vfill

\z  

\vfill

\newpage\setcounter{zad}{0}

\z 

\vfill

\z  

\vfill

\newpage\setcounter{zad}{0}

\z 

\vfill

\z  

\vfill

\newpage\setcounter{zad}{0}

\z 

\vfill

\z  

\vfill

\newpage\setcounter{zad}{0}

\z 

\vfill

\z  

\vfill

\newpage\setcounter{zad}{0}

\z 

\vfill

\z  

\vfill

\newpage\setcounter{zad}{0}

\z 

\vfill

\z  

\vfill

\newpage\setcounter{zad}{0}

\z 

\vfill

\z  

\vfill

\newpage\setcounter{zad}{0}

\z 

\vfill

\z  

\vfill

\newpage\setcounter{zad}{0}

\z 

\vfill

\z  

\vfill

\newpage\setcounter{zad}{0}

\z 

\vfill

\z  

\vfill

\newpage\setcounter{zad}{0}

\z 

\vfill

\z  

\vfill

\newpage\setcounter{zad}{0}

\z 

\vfill

\z  

\vfill

\newpage\setcounter{zad}{0}

\z 

\vfill

\z  

\vfill

\newpage\setcounter{zad}{0}

\z 

\vfill

\z  

\vfill

\newpage\setcounter{zad}{0}

\z 

\vfill

\z  

\vfill

\newpage\setcounter{zad}{0}

\z 

\vfill

\z  

\vfill

\newpage\setcounter{zad}{0}

\z 

\vfill

\z  

\vfill

\newpage\setcounter{zad}{0}

\z 

\vfill

\z  

\vfill

\newpage\setcounter{zad}{0}

\z 

\vfill

\z  

\vfill

\newpage\setcounter{zad}{0}

\z 

\vfill

\z  

\vfill

\newpage\setcounter{zad}{0}

\z 

\vfill

\z  

\vfill

\newpage\setcounter{zad}{0}

\z 

\vfill

\z  

\vfill

\newpage\setcounter{zad}{0}

\z 

\vfill

\z  

\vfill

\newpage\setcounter{zad}{0}

\z 

\vfill

\z  

\vfill

\newpage\setcounter{zad}{0}

\z 

\vfill

\z  

\vfill

\newpage\setcounter{zad}{0}

\z 

\vfill

\z  

\vfill

\newpage\setcounter{zad}{0}

\z 

\vfill

\z  

\vfill

\newpage\setcounter{zad}{0}

\z 

\vfill

\z  

\vfill

\newpage\setcounter{zad}{0}

\z 

\vfill

\z  

\vfill

\newpage\setcounter{zad}{0}

\z 

\vfill

\z  

\vfill

\newpage\setcounter{zad}{0}

\z 

\vfill

\z  

\vfill

\newpage\setcounter{zad}{0}

\z 

\vfill

\z  

\vfill

\newpage\setcounter{zad}{0}

\z 

\vfill

\z  

\vfill

\newpage\setcounter{zad}{0}

\z 

\vfill

\z  

\vfill

\newpage\setcounter{zad}{0}

\z 

\vfill

\z  

\vfill

\newpage\setcounter{zad}{0}

\z 

\vfill

\z  

\vfill

\newpage\setcounter{zad}{0}

\z 

\vfill

\z  

\vfill

\newpage\setcounter{zad}{0}

\z 

\vfill

\z  

\vfill

\newpage\setcounter{zad}{0}

\z 

\vfill

\z  

\vfill

\newpage\setcounter{zad}{0}

\z 

\vfill

\z  

\vfill

\newpage\setcounter{zad}{0}