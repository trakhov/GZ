<<<<<<< HEAD
=======


\begin{zkrW}{20}\noindent 
	Три студента пришли сдавать экзамен. Вероятность того, что первый студент сдаст экзамен, равна $0{,}1$, второй --- $0{,}9$, третий --- $0{,}8$. Найдите вероятность того, что хотя бы двое сдадут экзамен.
 
\end{zkrW}

\begin{zkrW}{20}\noindent 
	Пелагия и Архип условились встретиться в определенном месте между 4:00 и 5:00. Каждый из них может прийти в любое время в течение указанного промежутка и ждет второго некоторое время. Пелагия ждет 40 минут, после чего уходит; Архип ждет 50 минут, после чего уходит. В 5:00 любой из них уходит, сколько бы до этого он ни ждал. Чему равна вероятность того, что встреча произойдет не ранее чем без четверти 5:00?
 
\end{zkrW}

\begin{zkrW}{20}\noindent 
	В каждом из трех ящиков 12 желтых и 11 голубых шаров. Из первого ящика в третий перекладывают два наудачу выбранных шара, а из второго ящика в третий перекладывают один наудачу взятый шар. Затем из третьего ящика извлекается один шар. \\ \indent а) Найти вероятность того, что он желтый. \\ \indent б) Известно, что этот шар желтый; найти вероятность того, что из первого ящика во второй переложили желтые шары.
 
\end{zkrW}

\begin{zkrW}{20}\noindent 
	В студии находятся 5 телевизионных камер. Для каждой камеры вероятность того, что она включена в данный момент, равна $5/6$. Найти вероятность того, что в данный момент оказались выключены не более чем 4 камеры.
 
\end{zkrW}

\begin{zkrW}{20}\noindent 
	Стрелок попадает в цель из пистолета с вероятностью $0{,}25$, а из снайперской винтовки --- с вероятностью $0{,}9925$. Найти вероятность того, что, сделав 1200 выстрелов по цели из каждого оружия, стрелок \\ \indent а) промахнется из пистолета от 309 до 315 раз; \\ \indent б) промахнется из пистолета ровно 324 раз; \\ \indent в) допустит не более чем 3 промаха из снайперской винтовки.
 
\end{zkrW}

\newpage\setcounter{zad}{0}



\begin{zkrW}{20}\noindent 
	Вероятность наступления некоторого случайного события в каждом опыте одинакова и равна $1/2$. Опыты проводятся последовательно до наступления этого события. Определить вероятность того, что придется проводить 4-й опыт.
 
\end{zkrW}

\begin{zkrW}{20}\noindent 
	Карик и Иван Гермогенович Енотов условились встретиться в определенном месте между 4:00 и 8:00. Каждый из них может прийти в любое время в течение указанного промежутка и ждет второго некоторое время. Карик ждет 80 минут, после чего уходит; Иван Гермогенович Енотов ждет 70 минут, после чего уходит. В 8:00 любой из них уходит, сколько бы до этого он ни ждал. Чему равна вероятность того, что Карик опоздает более чем на полчаса?
 
\end{zkrW}

\begin{zkrW}{20}\noindent 
	В альбоме 10 чистых и 9 гашеных марок. Из альбома изымаются 3 наудачу извлеченные марки. После этого из альбома вновь наудачу извлекаются 4 марки. \\ \indent а) Найти вероятность того, что эти марки гашеные. \\ \indent б) Известно, что эти 4 марки гашеные; найти вероятность того, что первоначально изъятые 3 марки --- чистые.
 
\end{zkrW}

\begin{zkrW}{20}\noindent 
	Рабочий обслуживает 6 однотипных станков. Вероятность того, что станок потребует внимания рабочего в течение дня, равна $1/8$. Найти вероятность того, что в течение дня этих требований будет не более чем 2.
 
\end{zkrW}

\begin{zkrW}{20}\noindent 
	Вероятность выиграть отдельному игроку 1000 рублей в игре <<Кто хочет стать миллионером>> равна $0{,}8$, а 32000 рублей --- $0{,}04$. За сезон в этой игре принимает участие 100 человек. Найти вероятность того, что за сезон \\ \indent а) 1000 рублей получат ровно 73 человек; \\ \indent б) 1000 рублей получат от 86 до 89 человек; \\ \indent в) хотя бы 2 человек получат крупный выигрыш в 32000 рублей.
 
\end{zkrW}

\newpage\setcounter{zad}{0}



\begin{zkrW}{20}\noindent 
	Контролер ОТК, проверив качество сшитых 21 пальто, установил, что 7 из них --- первого сорта, а остальные --- второго. Найдите вероятность того, что среди взятых наудачу из этой партии 6 пальто не менее чем 2 будут второго сорта.
 
\end{zkrW}

\begin{zkrW}{20}\noindent 
	Андрей и Алексей условились встретиться в определенном месте между 7:00 и 11:00. Каждый из них может прийти в любое время в течение указанного промежутка и ждет второго некоторое время. Андрей ждет 80 минут, после чего уходит; Алексей ждет 70 минут, после чего уходит. В 11:00 любой из них уходит, сколько бы до этого он ни ждал. Чему равна вероятность того, что Андрей и Алексей встретятся?
 
\end{zkrW}

\begin{zkrW}{20}\noindent 
	В каждом из трех ящиков 8 желтых и 6 голубых шаров. Из первого и второго ящиков наудачу извлекается по одному шару и кладется в третий ящик. Затем из третьего ящика извлекается один шар. \\ \indent а) Найти вероятность того, что он желтый. \\ \indent б) Известно, что этот шар желтый; найти вероятность того, что шары, извлеченные из первого и второго ящиков, --- голубые.
 
\end{zkrW}

\begin{zkrW}{20}\noindent 
	По новым правилам в волейбольном матче игра происходит до тех пор, пока одна из команд не выиграет 5 партии. Вероятность победы российской сборной в каждой партии равна $0{,}1$. Определить вероятность того, что в ближайшем матче сборная России победит со счетом 5:4.
 
\end{zkrW}

\begin{zkrW}{20}\noindent 
	Студент за все время обучения в вузе в среднем выполняет 3750 задач по математике. Вероятность неверно решить отдельную задачу при условии стопроцентного посещения и активной работы на всех занятиях равна $0{,}0024$, в противном случае --- $0{,}6$. Найти вероятность того, что за время обучения в вузе \\ \indent а) абсолютно прилежный студент решил неверно ровно 3 задачи; \\ \indent б) обычный студент решил правильно ровно 2565 задач; \\ \indent в) обычный студент неверно решил от 2182 до 2205 задач.
 
\end{zkrW}

\newpage\setcounter{zad}{0}



\begin{zkrW}{20}\noindent 
	Произведено три выстрела по цели из орудия. Вероятность попадания при первом выстреле равна $2/7$; при втором --- $1/8$; при третьем --- $2/3$. Определить вероятность того, что будет хотя бы одно попадание.
 
\end{zkrW}

\begin{zkrW}{20}\noindent 
	Надя и Андрей условились встретиться в определенном месте между 3:00 и 7:00. Каждый из них может прийти в любое время в течение указанного промежутка и ждет второго некоторое время. Надя ждет 50 минут, после чего уходит; Андрей ждет 70 минут, после чего уходит. В 7:00 любой из них уходит, сколько бы до этого он ни ждал. Чему равна вероятность того, что встреча состоится в первые двадцать минут?
 
\end{zkrW}

\begin{zkrW}{20}\noindent 
	В каждом из трех ящиков 7 черных и 8 голубых шаров. Из первого ящика в третий перекладывают два наудачу выбранных шара, а из второго ящика в третий перекладывают один наудачу взятый шар. Затем из третьего ящика извлекается один шар. \\ \indent а) Найти вероятность того, что он голубой. \\ \indent б) Известно, что этот шар голубой; найти вероятность того, что из первого ящика во второй переложили голубые шары.
 
\end{zkrW}

\begin{zkrW}{20}\noindent 
	При данном технологическом процессе $70\%$ всех сходящих с конвейера автозавода автомобилей имеют цвет «металлик». Найти вероятность того, что из 9 случайно отобранных автомобилей не более чем 3 будут иметь этот цвет.
 
\end{zkrW}

\begin{zkrW}{20}\noindent 
	Вероятность выиграть отдельному игроку 1000 рублей в игре <<Кто хочет стать миллионером>> равна $0{,}4$, а 32000 рублей --- $0{,}006$. За сезон в этой игре принимает участие 150 человек. Найти вероятность того, что за сезон \\ \indent а) 1000 рублей получат ровно 67 человек; \\ \indent б) 1000 рублей получат от 52 до 56 человек; \\ \indent в) более чем 5 человек получат крупный выигрыш в 32000 рублей.
 
\end{zkrW}

\newpage\setcounter{zad}{0}



\begin{zkrW}{20}\noindent 
	Игра проводится до выигрыша одним из двух игроков двух партий подряд (ничьи исключаются). Вероятность выигрыша партии каждым из игроков равна $0{,}5$ и не зависит от исходов предыдущих партий. Найдите вероятность того, что игра окончится до 4-й партии.
 
\end{zkrW}

\begin{zkrW}{20}\noindent 
	Саша и Андрей условились встретиться в определенном месте между 8:00 и 11:00. Каждый из них может прийти в любое время в течение указанного промежутка и ждет второго некоторое время. Саша ждет 60 минут, после чего уходит; Андрей ждет 70 минут, после чего уходит. В 11:00 любой из них уходит, сколько бы до этого он ни ждал. Чему равна вероятность того, что встреча состоится в первые двадцать минут?
 
\end{zkrW}

\begin{zkrW}{20}\noindent 
	В каждом из трех ящиков 11 белых и 12 синих шаров. Из первого ящика в третий перекладывают два наудачу выбранных шара, а из второго ящика в третий перекладывают один наудачу взятый шар. Затем из третьего ящика извлекается один шар. \\ \indent а) Найти вероятность того, что он белый. \\ \indent б) Известно, что этот шар белый; найти вероятность того, что из первого ящика во второй переложили белые шары.
 
\end{zkrW}

\begin{zkrW}{20}\noindent 
	Эксплуатируется устройство, состоящее из 9 независимо работающих элементов. Вероятность отказа каждого из них за время работы устройства равна $2/3$. Найти вероятность того, что за время работы устройства откажут по крайней мере 3 элемента.
 
\end{zkrW}

\begin{zkrW}{20}\noindent 
	Студент за все время обучения в вузе в среднем выполняет 900 задач по математике. Вероятность неверно решить отдельную задачу при условии стопроцентного посещения и активной работы на всех занятиях равна $0{,}001$, в противном случае --- $0{,}1$. Найти вероятность того, что за время обучения в вузе \\ \indent а) абсолютно прилежный студент решил неверно не более чем 4 задачи; \\ \indent б) обычный студент решил правильно ровно 101 задач; \\ \indent в) обычный студент неверно решил от 96 до 101 задач.
 
\end{zkrW}

\newpage\setcounter{zad}{0}



\begin{zkrW}{20}\noindent 
	В двух урнах находятся шары, отличающиеся только цветом. В первой урне 11 белых, 12 черных и 9 красных шаров. Во второй урне 9 белых, 13 черных и 15 красных. Из каждой урны наудачу извлекаются по одному шару. Какова вероятность того, что извлеченные шары будут одинакового цвета?
 
\end{zkrW}

\begin{zkrW}{20}\noindent 
	Робин и Бэтмен условились встретиться в определенном месте между 5:00 и 6:00. Каждый из них может прийти в любое время в течение указанного промежутка и ждет второго некоторое время. Робин ждет 40 минут, после чего уходит; Бэтмен ждет 30 минут, после чего уходит. В 6:00 любой из них уходит, сколько бы до этого он ни ждал. Чему равна вероятность того, что встреча состоится в последние полчаса?
 
\end{zkrW}

\begin{zkrW}{20}\noindent 
	В альбоме 7 чистых и 10 гашеных марок. Из альбома наудачу извлекаются 3 марки и подвергаются гашению, а затем возвращаются в альбом. После этого вновь наудачу извлекаются 5 марки. \\ \indent а) Найти вероятность того, что эти марки гашеные. \\ \indent б) Известно, что эти 5 марки гашеные; найти вероятность того, что первоначально извлеченные 3 марки --- чистые.
 
\end{zkrW}

\begin{zkrW}{20}\noindent 
	$90\%$ изделий данного предприятия — это продукция высшего сорта. Некто приобрел 5 изделий, изготовленных на этом предприятии. Чему равна вероятность того, что не более чем 4 из них — высшего сорта?
 
\end{zkrW}

\begin{zkrW}{20}\noindent 
	Предполагая рождение ребенка в любой день года равновозможным, найти вероятность того, что в группе из 100 человек \\ \indent а) по меньшей мере 3 родились 16 февраля; \\ \indent б) ровно 45 родились осенью; \\ \indent в) от 46 до 53 родились весной.
 
\end{zkrW}

\newpage\setcounter{zad}{0}



\begin{zkrW}{20}\noindent 
	Три студента пришли сдавать экзамен. Вероятность того, что первый студент сдаст экзамен, равна $7/9$, второй --- $7/9$, третий --- $3/4$. Найдите вероятность того, что хотя бы двое сдадут экзамен.
 
\end{zkrW}

\begin{zkrW}{20}\noindent 
	Хоттабыч и Женя Богорад условились встретиться в определенном месте между 2:00 и 4:00. Каждый из них может прийти в любое время в течение указанного промежутка и ждет второго некоторое время. Хоттабыч ждет 60 минут, после чего уходит; Женя Богорад ждет 50 минут, после чего уходит. В 4:00 любой из них уходит, сколько бы до этого он ни ждал. Чему равна вероятность того, что встреча произойдет не ранее чем без четверти 4:00?
 
\end{zkrW}

\begin{zkrW}{20}\noindent 
	В каждом из трех ящиков 10 зеленых и 8 белых шаров. Из первого и второго ящиков наудачу извлекается по одному шару и кладется в третий ящик. Затем из третьего ящика извлекается один шар. \\ \indent а) Найти вероятность того, что он зеленый. \\ \indent б) Известно, что этот шар зеленый; найти вероятность того, что шары, извлеченные из первого и второго ящиков, --- белые.
 
\end{zkrW}

\begin{zkrW}{20}\noindent 
	При данном технологическом процессе $30\%$ всех сходящих с конвейера автозавода автомобилей имеют цвет «металлик». Найти вероятность того, что из 8 случайно отобранных автомобилей ровно 4 будут иметь этот цвет.
 
\end{zkrW}

\begin{zkrW}{20}\noindent 
	В ралли принимает участие 2400 экипажей. Каждый экипаж может сойти с дистанции из-за технических неполадок с вероятностью $0{,}6$, а из-за болезни водителя --- с вероятностью $0{,}0025$. Найти вероятность того, что \\ \indent а) по крайней мере 4 экипажей сойдут с дистанции из-за болезни водителя; \\ \indent б) ровно 1613 экипажей не смогут продолжать ралли из-за технических неполадок; \\ \indent в) от 1382 до 1483 экипажей пострадают от технических проблем.
 
\end{zkrW}

\newpage\setcounter{zad}{0}



\begin{zkrW}{20}\noindent 
	Среди 15 лампочек 5 стандартные. Одновременно берут наудачу 5 лампочки. Найдите вероятность того, что хотя бы одна из них нестандартная.
 
\end{zkrW}

\begin{zkrW}{20}\noindent 
	Катя и Света условились встретиться в определенном месте между 6:00 и 7:00. Каждый из них может прийти в любое время в течение указанного промежутка и ждет второго некоторое время. Катя ждет 40 минут, после чего уходит; Света ждет 20 минут, после чего уходит. В 7:00 любой из них уходит, сколько бы до этого он ни ждал. Чему равна вероятность того, что Катя опоздает более чем на полчаса?
 
\end{zkrW}

\begin{zkrW}{20}\noindent 
	В каждом из трех ящиков 5 желтых и 6 черных шаров. Из первого и второго ящиков наудачу извлекается по одному шару и кладется в третий ящик. Затем из третьего ящика извлекается один шар. \\ \indent а) Найти вероятность того, что он черный. \\ \indent б) Известно, что этот шар черный; найти вероятность того, что шары, извлеченные из первого и второго ящиков, --- черные.
 
\end{zkrW}

\begin{zkrW}{20}\noindent 
	Вероятность того, что за рабочий день расход электроэнергии не превысит норму, равна $1/2$. Найти вероятность того, что за 7 дней работы норма будет превышена не менее чем 2 раза.
 
\end{zkrW}

\begin{zkrW}{20}\noindent 
	Вероятность появления опечатки на отдельной странице книги равна $0{,}005$, а погрешности верстки --- $0{,}5$. Найти вероятность того, что в книге из 1600 страниц \\ \indent а) не менее чем 2 страниц будут иметь опечатки; \\ \indent б) от 760 до 832 страниц будут иметь погрешности верстки; \\ \indent в) погрешности верстки будут присутствовать ровно на 752 страницах.
 
\end{zkrW}

\newpage\setcounter{zad}{0}



\begin{zkrW}{20}\noindent 
	Вероятность того, что при первом измерении некоторой физической величины будет допущена ошибка, превышающая заданную точность, равна $0{,}6$; при последующих измерениях --- $0{,}9$. Произведены три независимых измерения. Найдите вероятность того, что не более чем в одном измерении допущенная ошибка превысит заданную точность.
 
\end{zkrW}

\begin{zkrW}{20}\noindent 
	Рафаэль и Донателло условились встретиться в определенном месте между 11:00 и 14:00. Каждый из них может прийти в любое время в течение указанного промежутка и ждет второго некоторое время. Рафаэль ждет 60 минут, после чего уходит; Донателло ждет 70 минут, после чего уходит. В 14:00 любой из них уходит, сколько бы до этого он ни ждал. Чему равна вероятность того, что Рафаэль и Донателло не встретятся?
 
\end{zkrW}

\begin{zkrW}{20}\noindent 
	В альбоме 6 чистых и 12 гашеных марок. Из альбома наудачу извлекаются 3 марки и подвергаются гашению, а затем возвращаются в альбом. После этого вновь наудачу извлекаются 4 марки. \\ \indent а) Найти вероятность того, что эти марки чистые. \\ \indent б) Известно, что эти 4 марки чистые; найти вероятность того, что первоначально извлеченные 3 марки --- гашеные.
 
\end{zkrW}

\begin{zkrW}{20}\noindent 
	Известно, что $55\%$ семян огурцов не всходят при посеве. Какова вероятность того, что из 5 посеянных семян взойдут не менее чем 3?
 
\end{zkrW}

\begin{zkrW}{20}\noindent 
	Стрелок попадает в цель из пистолета с вероятностью $0{,}9$, а из снайперской винтовки --- с вероятностью $0{,}99$. Найти вероятность того, что, сделав 900 выстрелов по цели из каждого оружия, стрелок \\ \indent а) промахнется из пистолета от 786 до 826 раз; \\ \indent б) промахнется из пистолета ровно 859 раз; \\ \indent в) допустит ровно 3 промаха из снайперской винтовки.
 
\end{zkrW}

\newpage\setcounter{zad}{0}



\begin{zkrW}{20}\noindent 
	Мастер обслуживает четыре станка, работающих независимо друг от друга. Вероятность того, что первыйй станок в течение смены потребует внимания мастера, равна $5/8$, второй --- $1/3$, третий --- $1/4$ и четвертый --- $1/4$. Найдите вероятность того, что в течение смены хотя бы один станок не потребует внимания мастера.
 
\end{zkrW}

\begin{zkrW}{20}\noindent 
	Саша и Надя условились встретиться в определенном месте между 4:00 и 6:00. Каждый из них может прийти в любое время в течение указанного промежутка и ждет второго некоторое время. Саша ждет 60 минут, после чего уходит; Надя ждет 50 минут, после чего уходит. В 6:00 любой из них уходит, сколько бы до этого он ни ждал. Чему равна вероятность того, что Саша опоздает менее чем на полчаса?
 
\end{zkrW}

\begin{zkrW}{20}\noindent 
	В альбоме 9 чистых и 8 гашеных марок. Из альбома наудачу извлекаются 2 марки и заменяются на чистые. После этого вновь наудачу извлекаются 4 марки. \\ \indent а) Найти вероятность того, что эти марки гашеные. \\ \indent б) Известно, что эти 4 марки гашеные; найти вероятность того, что первоначально извлеченные 2 марки --- чистые.
 
\end{zkrW}

\begin{zkrW}{20}\noindent 
	В тестовом задании 7 вопросов, на каждый дано 6 варианта ответа, среди которых один правильный. Какова вероятность того, что, выбирая вариант ответа наугад, отвечающий правильно ответит не менее чем на 3 вопроса?
 
\end{zkrW}

\begin{zkrW}{20}\noindent 
	Студент за все время обучения в вузе в среднем выполняет 400 задач по математике. Вероятность неверно решить отдельную задачу при условии стопроцентного посещения и активной работы на всех занятиях равна $0{,}0075$, в противном случае --- $0{,}2$. Найти вероятность того, что за время обучения в вузе \\ \indent а) абсолютно прилежный студент решил неверно более чем 3 задачи; \\ \indent б) обычный студент решил правильно ровно 83 задач; \\ \indent в) обычный студент неверно решил от 75 до 82 задач.
 
\end{zkrW}

\newpage\setcounter{zad}{0}



\begin{zkrW}{20}\noindent 
	Среди 15 поступающих в ремонт часов 13 нуждаются в общей чистке механизма. Какова вероятность того, что среди взятых одновременно наудачу 10 часов по крайней мере 2 нуждаются в общей чистке механизма?
 
\end{zkrW}

\begin{zkrW}{20}\noindent 
	Вася и Катя условились встретиться в определенном месте между 6:00 и 7:00. Каждый из них может прийти в любое время в течение указанного промежутка и ждет второго некоторое время. Вася ждет 30 минут, после чего уходит; Катя ждет 40 минут, после чего уходит. В 7:00 любой из них уходит, сколько бы до этого он ни ждал. Чему равна вероятность того, что встреча произойдет не ранее чем без четверти 7:00?
 
\end{zkrW}

\begin{zkrW}{20}\noindent 
	В альбоме 10 чистых и 7 гашеных марок. Из альбома наудачу извлекаются 2 марки и подвергаются гашению, а затем возвращаются в альбом. После этого вновь наудачу извлекаются 5 марки. \\ \indent а) Найти вероятность того, что эти марки гашеные. \\ \indent б) Известно, что эти 5 марки гашеные; найти вероятность того, что первоначально извлеченные 2 марки --- чистые.
 
\end{zkrW}

\begin{zkrW}{20}\noindent 
	В студии находятся 7 телевизионных камер. Для каждой камеры вероятность того, что она включена в данный момент, равна $0{,}9$. Найти вероятность того, что в данный момент оказались выключены хотя бы 2 камеры.
 
\end{zkrW}

\begin{zkrW}{20}\noindent 
	Мастерская за год ремонтирует 3750 мобильных телефонов. Вероятность неисправности в механической части отдельного телефона равна $0{,}6$, в электронной части --- $0{,}0016$. Найти вероятность того, что среди телефонов, отремонтированных за год, \\ \indent а) имели неисправности в механической части от 2182 до 2205 экземпляров; \\ \indent б) имели неисправности в электронной части по меньшей мере 5 телефонов; \\ \indent в) ровно 2565 телефонов имели проблемы в механической части.
 
\end{zkrW}

\newpage\setcounter{zad}{0}



\begin{zkrW}{20}\noindent 
	В коробке смешаны электролампы одинакового размера и формы: по 100 Вт --- 13 штук, по 75 Вт --- 10 штук. Вынуты наудачу три лампы. Какова вероятность того, что хотя бы две из них по 100 Вт?
 
\end{zkrW}

\begin{zkrW}{20}\noindent 
	Катя и Андрей условились встретиться в определенном месте между 19:00 и 23:00. Каждый из них может прийти в любое время в течение указанного промежутка и ждет второго некоторое время. Катя ждет 60 минут, после чего уходит; Андрей ждет 70 минут, после чего уходит. В 23:00 любой из них уходит, сколько бы до этого он ни ждал. Чему равна вероятность того, что встреча состоится в первые двадцать минут?
 
\end{zkrW}

\begin{zkrW}{20}\noindent 
	В альбоме 6 чистых и 12 гашеных марок. Из альбома изымаются 3 наудачу извлеченные марки. После этого из альбома вновь наудачу извлекаются 2 марки. \\ \indent а) Найти вероятность того, что эти марки чистые. \\ \indent б) Известно, что эти 2 марки чистые; найти вероятность того, что первоначально изъятые 3 марки --- гашеные.
 
\end{zkrW}

\begin{zkrW}{20}\noindent 
	В тестовом задании 6 вопросов, на каждый дано 7 варианта ответа, среди которых один правильный. Какова вероятность того, что, выбирая вариант ответа наугад, отвечающий правильно ответит не более чем на 2 вопроса?
 
\end{zkrW}

\begin{zkrW}{20}\noindent 
	Известно, что левши среди населения Уганды составляют в среднем $0{,}36\%$, а люди, одинаково владеющие левой и правой рукой, --- $0{,}9$ (остальные --- правши). Найти вероятность того, что среди 2500 людей \\ \indent а) окажется ровно 5 левшей; \\ \indent б) окажется ровно 2182 амбидекстров\footnote{людей, одинаково владеющих обеими руками}; \\ \indent в) окажется от 2205 до 2295 амбидекстров.
 
\end{zkrW}

\newpage\setcounter{zad}{0}



\begin{zkrW}{20}\noindent 
	В коробке 13 красных, 9 синих и 11 желтых карандашей. Наудачу вынимают три карандаша. Какова вероятность того, что они все разных цветов?
 
\end{zkrW}

\begin{zkrW}{20}\noindent 
	Аня и Надя условились встретиться в определенном месте между 2:00 и 6:00. Каждый из них может прийти в любое время в течение указанного промежутка и ждет второго некоторое время. Аня ждет 80 минут, после чего уходит; Надя ждет 60 минут, после чего уходит. В 6:00 любой из них уходит, сколько бы до этого он ни ждал. Чему равна вероятность того, что Аня опоздает более чем на полчаса?
 
\end{zkrW}

\begin{zkrW}{20}\noindent 
	В каждом из трех ящиков 11 красных и 10 желтых шаров. Из первого ящика в третий перекладывают два наудачу выбранных шара, а из второго ящика в третий перекладывают один наудачу взятый шар. Затем из третьего ящика извлекается один шар. \\ \indent а) Найти вероятность того, что он красный. \\ \indent б) Известно, что этот шар красный; найти вероятность того, что из первого ящика во второй переложили красные шары.
 
\end{zkrW}

\begin{zkrW}{20}\noindent 
	В студии находятся 9 телевизионных камер. Для каждой камеры вероятность того, что она включена в данный момент, равна $1/9$. Найти вероятность того, что в данный момент оказались выключены по меньшей мере 2 камеры.
 
\end{zkrW}

\begin{zkrW}{20}\noindent 
	В лотерее разыгрываются крупные и мелкие выигрыши. Вероятность того, что на лотерейный билет выпадет крупный выигрыш, равна $0{,}001$, а мелкий --- $0{,}2$. Куплено 400 билетов. Найти вероятность того, что \\ \indent а) крупных выигрышей будет по меньшей мере 2; \\ \indent б) мелких выигрышей будет ровно 86; \\ \indent в) мелких выигрышей будет от 74 до 92.
 
\end{zkrW}

\newpage\setcounter{zad}{0}



\begin{zkrW}{20}\noindent 
	В коробке 15 красных, 5 синих и 8 желтых карандашей. Наудачу вынимают три карандаша. Какова вероятность того, что они все разных цветов?
 
\end{zkrW}

\begin{zkrW}{20}\noindent 
	Алексей и Надя условились встретиться в определенном месте между 12:00 и 13:00. Каждый из них может прийти в любое время в течение указанного промежутка и ждет второго некоторое время. Алексей ждет 40 минут, после чего уходит; Надя ждет 50 минут, после чего уходит. В 13:00 любой из них уходит, сколько бы до этого он ни ждал. Чему равна вероятность того, что встреча состоится в первые полчаса?
 
\end{zkrW}

\begin{zkrW}{20}\noindent 
	В альбоме 8 чистых и 6 гашеных марок. Из альбома наудачу извлекаются 2 марки и подвергаются гашению, а затем возвращаются в альбом. После этого вновь наудачу извлекаются 2 марки. \\ \indent а) Найти вероятность того, что эти марки чистые. \\ \indent б) Известно, что эти 2 марки чистые; найти вероятность того, что первоначально извлеченные 2 марки --- гашеные.
 
\end{zkrW}

\begin{zkrW}{20}\noindent 
	Рабочий обслуживает 9 однотипных станков. Вероятность того, что станок потребует внимания рабочего в течение дня, равна $1/2$. Найти вероятность того, что в течение дня этих требований будет по крайней мере 3.
 
\end{zkrW}

\begin{zkrW}{20}\noindent 
	Вероятность выиграть отдельному игроку 1000 рублей в игре <<Кто хочет стать миллионером>> равна $0{,}2$, а 32000 рублей --- $0{,}008$. За сезон в этой игре принимает участие 100 человек. Найти вероятность того, что за сезон \\ \indent а) 1000 рублей получат ровно 18 человек; \\ \indent б) 1000 рублей получат от 17 до 22 человек; \\ \indent в) ровно 5 человек получат крупный выигрыш в 32000 рублей.
 
\end{zkrW}

\newpage\setcounter{zad}{0}



\begin{zkrW}{20}\noindent 
	Вероятность попадания в цель при одном выстреле равна $4/9$ и с каждым выстрелом уменьшается на одну десятую от первоначальной. Произведено 7 выстрелов. Найдите вероятность поражения цели, если для этого достаточно хотя бы одного попадания.
 
\end{zkrW}

\begin{zkrW}{20}\noindent 
	Иван Гермогенович Енотов и Валя условились встретиться в определенном месте между 20:00 и 23:00. Каждый из них может прийти в любое время в течение указанного промежутка и ждет второго некоторое время. Иван Гермогенович Енотов ждет 50 минут, после чего уходит; Валя ждет 40 минут, после чего уходит. В 23:00 любой из них уходит, сколько бы до этого он ни ждал. Чему равна вероятность того, что Иван Гермогенович Енотов опоздает более чем на полчаса?
 
\end{zkrW}

\begin{zkrW}{20}\noindent 
	В альбоме 9 чистых и 9 гашеных марок. Из альбома наудачу извлекаются 3 марки и заменяются на чистые. После этого вновь наудачу извлекаются 3 марки. \\ \indent а) Найти вероятность того, что эти марки чистые. \\ \indent б) Известно, что эти 3 марки чистые; найти вероятность того, что первоначально извлеченные 3 марки --- гашеные.
 
\end{zkrW}

\begin{zkrW}{20}\noindent 
	Наблюдениями установлено, что в некоторой местности в ноябре бывает в среднем 27 дождливых дней. Какова вероятность того, что в следующем году из 9 первых дней сентября не более чем 3 окажутся дождливыми?
 
\end{zkrW}

\begin{zkrW}{20}\noindent 
	Предполагая рождение ребенка в любой день года равновозможным, найти вероятность того, что в группе из 1200 человек \\ \indent а) менее чем 2 родились 1 марта; \\ \indent б) ровно 282 родились осенью; \\ \indent в) от 264 до 327 родились весной.
 
\end{zkrW}

\newpage\setcounter{zad}{0}



\begin{zkrW}{20}\noindent 
	Во время тренировки три баскетболиста бросают мячи в корзину. Вероятность попадания для первого равна $1/9$, для второго --- $5/6$, для третьего --- $3/4$. Каждый баскетболист делает один бросок. Найдите вероятность хотя бы одного попадания мяча в корзину.
 
\end{zkrW}

\begin{zkrW}{20}\noindent 
	Петя и Вася условились встретиться в определенном месте между 13:00 и 16:00. Каждый из них может прийти в любое время в течение указанного промежутка и ждет второго некоторое время. Петя ждет 40 минут, после чего уходит; Вася ждет 50 минут, после чего уходит. В 16:00 любой из них уходит, сколько бы до этого он ни ждал. Чему равна вероятность того, что Петя и Вася не встретятся?
 
\end{zkrW}

\begin{zkrW}{20}\noindent 
	В альбоме 7 чистых и 7 гашеных марок. Из альбома наудачу извлекаются 3 марки и подвергаются гашению, а затем возвращаются в альбом. После этого вновь наудачу извлекаются 3 марки. \\ \indent а) Найти вероятность того, что эти марки гашеные. \\ \indent б) Известно, что эти 3 марки гашеные; найти вероятность того, что первоначально извлеченные 3 марки --- чистые.
 
\end{zkrW}

\begin{zkrW}{20}\noindent 
	Опрошены 5 человек. Найти вероятность того, что хотя бы 4 из них родились осенью.
 
\end{zkrW}

\begin{zkrW}{20}\noindent 
	Мастерская за год ремонтирует 400 мобильных телефонов. Вероятность неисправности в механической части отдельного телефона равна $0{,}1$, в электронной части --- $0{,}015$. Найти вероятность того, что среди телефонов, отремонтированных за год, \\ \indent а) имели неисправности в механической части от 36 до 45 экземпляров; \\ \indent б) имели неисправности в электронной части менее чем 2 телефонов; \\ \indent в) ровно 42 телефонов имели проблемы в механической части.
 
\end{zkrW}

\newpage\setcounter{zad}{0}



\begin{zkrW}{20}\noindent 
	Среди 25 лампочек 11 стандартные. Одновременно берут наудачу 6 лампочки. Найдите вероятность того, что хотя бы одна из них нестандартная.
 
\end{zkrW}

\begin{zkrW}{20}\noindent 
	Арамис и Атос условились встретиться в определенном месте между 9:00 и 12:00. Каждый из них может прийти в любое время в течение указанного промежутка и ждет второго некоторое время. Арамис ждет 40 минут, после чего уходит; Атос ждет 70 минут, после чего уходит. В 12:00 любой из них уходит, сколько бы до этого он ни ждал. Чему равна вероятность того, что встреча состоится в первые двадцать минут?
 
\end{zkrW}

\begin{zkrW}{20}\noindent 
	В каждом из трех ящиков 5 синих и 8 черных шаров. Из первого ящика в третий перекладывают два наудачу выбранных шара, а из второго ящика в третий перекладывают один наудачу взятый шар. Затем из третьего ящика извлекается один шар. \\ \indent а) Найти вероятность того, что он черный. \\ \indent б) Известно, что этот шар черный; найти вероятность того, что из первого ящика во второй переложили синие шары.
 
\end{zkrW}

\begin{zkrW}{20}\noindent 
	На автобазе 5 машин. Вероятность выхода каждой из них на линию равна $0{,}1$. Найти вероятность того, что на линию по каким-либо причинам не смогут выйти хотя бы 3 машины.
 
\end{zkrW}

\begin{zkrW}{20}\noindent 
	Мастерская за год ремонтирует 100 мобильных телефонов. Вероятность неисправности в механической части отдельного телефона равна $0{,}2$, в электронной части --- $0{,}005$. Найти вероятность того, что среди телефонов, отремонтированных за год, \\ \indent а) имели неисправности в механической части от 18 до 22 экземпляров; \\ \indent б) имели неисправности в электронной части по меньшей мере 2 телефонов; \\ \indent в) ровно 19 телефонов имели проблемы в механической части.
 
\end{zkrW}

\newpage\setcounter{zad}{0}



\begin{zkrW}{20}\noindent 
	Экспедиция издательства отправила газеты в три почтовых отделения. Вероятность своевременной доставки газет в первое отделение равна $1/3$, во второе отделение --- $2/3$ и в третье --- $5/8$. Найдите вероятность того, что хотя бы одно отделение получит газеты с опозданием.
 
\end{zkrW}

\begin{zkrW}{20}\noindent 
	Леонтий и Пафнутий условились встретиться в определенном месте между 10:00 и 13:00. Каждый из них может прийти в любое время в течение указанного промежутка и ждет второго некоторое время. Леонтий ждет 50 минут, после чего уходит; Пафнутий ждет 40 минут, после чего уходит. В 13:00 любой из них уходит, сколько бы до этого он ни ждал. Чему равна вероятность того, что Леонтий опоздает более чем на полчаса?
 
\end{zkrW}

\begin{zkrW}{20}\noindent 
	В первом ящике 8 белых и 8 зеленых шаров, а во втором 12 белых и 12 зеленых. Из первого ящика во второй перекладываются 3 наудачу извлеченных шара. После этого из второго ящика наудачу извлекается один шар. \\ \indent а) Найти вероятность того, что он белый. \\ \indent б) Известно, что этот шар белый; найти вероятность того, что извлеченные из первого ящика шары --- белые.
 
\end{zkrW}

\begin{zkrW}{20}\noindent 
	В телевизоре 6 ламп. Вероятность того, что в течение года лампа останется исправной, равна $1/3$. Найти вероятность того, что в течение года из строя выйдут ровно 3 лампы.При разведочном бурении производится отбор керна. Вероятность успешного отбора керна при каждой попытке равна $0{,}2$. Планируется совершить 7 попыток отбора. Какова вероятность того, что по крайней мере в 4 из них керн будет получен?
 
\end{zkrW}

\begin{zkrW}{20}\noindent 
	В ралли принимает участие 3750 экипажей. Каждый экипаж может сойти с дистанции из-за технических неполадок с вероятностью $0{,}4$, а из-за болезни водителя --- с вероятностью $0{,}0024$. Найти вероятность того, что \\ \indent а) хотя бы 3 экипажей сойдут с дистанции из-за болезни водителя; \\ \indent б) ровно 1365 экипажей не смогут продолжать ралли из-за технических неполадок; \\ \indent в) от 1425 до 1560 экипажей пострадают от технических проблем.
 
\end{zkrW}

\newpage\setcounter{zad}{0}



\begin{zkrW}{20}\noindent 
	Вероятность наступления некоторого случайного события в каждом опыте одинакова и равна $0{,}4$. Опыты проводятся последовательно до наступления этого события. Определить вероятность того, что придется проводить 3-й опыт.
 
\end{zkrW}

\begin{zkrW}{20}\noindent 
	Вася и Полина условились встретиться в определенном месте между 3:00 и 4:00. Каждый из них может прийти в любое время в течение указанного промежутка и ждет второго некоторое время. Вася ждет 30 минут, после чего уходит; Полина ждет 20 минут, после чего уходит. В 4:00 любой из них уходит, сколько бы до этого он ни ждал. Чему равна вероятность того, что Вася и Полина встретятся?
 
\end{zkrW}

\begin{zkrW}{20}\noindent 
	В альбоме 10 чистых и 11 гашеных марок. Из альбома наудачу извлекаются 2 марки и заменяются на чистые. После этого вновь наудачу извлекаются 3 марки. \\ \indent а) Найти вероятность того, что эти марки чистые. \\ \indent б) Известно, что эти 3 марки чистые; найти вероятность того, что первоначально извлеченные 2 марки --- гашеные.
 
\end{zkrW}

\begin{zkrW}{20}\noindent 
	В телевизоре 7 ламп. Вероятность того, что в течение года лампа останется исправной, равна $0{,}1$. Найти вероятность того, что в течение года из строя выйдут менее чем 3 лампы.Опрошены 8 человек. Найти вероятность того, что не более чем 3 из них родились осенью.
 
\end{zkrW}

\begin{zkrW}{20}\noindent 
	Вероятность появления опечатки на отдельной странице книги равна $0{,}0024$, а погрешности верстки --- $0{,}4$. Найти вероятность того, что в книге из 3750 страниц \\ \indent а) более чем 4 страниц будут иметь опечатки; \\ \indent б) от 1440 до 1575 страниц будут иметь погрешности верстки; \\ \indent в) погрешности верстки будут присутствовать ровно на 1380 страницах.
 
\end{zkrW}

\newpage\setcounter{zad}{0}



\begin{zkrW}{20}\noindent 
	Вероятность того, что при первом измерении некоторой физической величины будет допущена ошибка, превышающая заданную точность, равна $1/5$; при последующих измерениях --- $8/9$. Произведены три независимых измерения. Найдите вероятность того, что по крайней мере в одном измерении допущенная ошибка превысит заданную точность.
 
\end{zkrW}

\begin{zkrW}{20}\noindent 
	Андрей и Петя условились встретиться в определенном месте между 9:00 и 11:00. Каждый из них может прийти в любое время в течение указанного промежутка и ждет второго некоторое время. Андрей ждет 40 минут, после чего уходит; Петя ждет 50 минут, после чего уходит. В 11:00 любой из них уходит, сколько бы до этого он ни ждал. Чему равна вероятность того, что встреча состоится в первые двадцать минут?
 
\end{zkrW}

\begin{zkrW}{20}\noindent 
	В альбоме 10 чистых и 5 гашеных марок. Из альбома изымаются 3 наудачу извлеченные марки. После этого из альбома вновь наудачу извлекаются 4 марки. \\ \indent а) Найти вероятность того, что эти марки чистые. \\ \indent б) Известно, что эти 4 марки чистые; найти вероятность того, что первоначально изъятые 3 марки --- гашеные.
 
\end{zkrW}

\begin{zkrW}{20}\noindent 
	Каждый из 6 станков в течение 4 рабочих часов останавливается несколько раз и всего в сумме стоит один час, причем остановка его в любой момент времени равновероятна. Найти вероятность того, что в данный момент времени будут работать не менее чем 4 станка.
 
\end{zkrW}

\begin{zkrW}{20}\noindent 
	В лотерее разыгрываются крупные и мелкие выигрыши. Вероятность того, что на лотерейный билет выпадет крупный выигрыш, равна $0{,}0025$, а мелкий --- $0{,}8$. Куплено 1600 билетов. Найти вероятность того, что \\ \indent а) крупных выигрышей будет хотя бы 2; \\ \indent б) мелких выигрышей будет ровно 1421; \\ \indent в) мелких выигрышей будет от 1242 до 1318.
 
\end{zkrW}

\newpage\setcounter{zad}{0}



\begin{zkrW}{20}\noindent 
	Студент успел подготовить к экзамену 11 вопросов из 22. Какова вероятность того, что из 6 наудачу выбранных вопросов студент знает ровно 2?
 
\end{zkrW}

\begin{zkrW}{20}\noindent 
	Надя и Алексей условились встретиться в определенном месте между 7:00 и 11:00. Каждый из них может прийти в любое время в течение указанного промежутка и ждет второго некоторое время. Надя ждет 80 минут, после чего уходит; Алексей ждет 60 минут, после чего уходит. В 11:00 любой из них уходит, сколько бы до этого он ни ждал. Чему равна вероятность того, что встреча состоится в первые двадцать минут?
 
\end{zkrW}

\begin{zkrW}{20}\noindent 
	В первом ящике 5 черных и 6 желтых шаров, а во втором 11 черных и 10 желтых. Из первого ящика во второй перекладываются 2 наудачу извлеченных шара. После этого из второго ящика наудачу извлекается один шар. \\ \indent а) Найти вероятность того, что он желтый. \\ \indent б) Известно, что этот шар желтый; найти вероятность того, что извлеченные из первого ящика шары --- черные.
 
\end{zkrW}

\begin{zkrW}{20}\noindent 
	В урне находятся 6 шара белого цвета и 5 шаров черного цвета. 6 раз проделывают следующее: наугад вынимают шар, записывают его цвет и кладут его обратно. Найти вероятность того, что записей о появлении шара белого цвета будет по меньшей мере 3.
 
\end{zkrW}

\begin{zkrW}{20}\noindent 
	В ралли принимает участие 600 экипажей. Каждый экипаж может сойти с дистанции из-за технических неполадок с вероятностью $0{,}4$, а из-за болезни водителя --- с вероятностью $0{,}0015$. Найти вероятность того, что \\ \indent а) ровно 3 экипажей сойдут с дистанции из-за болезни водителя; \\ \indent б) ровно 247 экипажей не смогут продолжать ралли из-за технических неполадок; \\ \indent в) от 245 до 269 экипажей пострадают от технических проблем.
 
\end{zkrW}

\newpage\setcounter{zad}{0}



\begin{zkrW}{20}\noindent 
	Станция метрополитена оборудована тремя эскалаторами. Вероятность поломки в течение рабочего дня первого эскалатора равна $1/3$, второго --- $1/2$, третьего ---  $7/8$. Найдите вероятность того, что в течение рабочего дня будет исправен хотя бы один эскалатор.
 
\end{zkrW}

\begin{zkrW}{20}\noindent 
	Архип и Пафнутий условились встретиться в определенном месте между 15:00 и 16:00. Каждый из них может прийти в любое время в течение указанного промежутка и ждет второго некоторое время. Архип ждет 40 минут, после чего уходит; Пафнутий ждет 50 минут, после чего уходит. В 16:00 любой из них уходит, сколько бы до этого он ни ждал. Чему равна вероятность того, что Архип придет раньше, чем Пафнутий?
 
\end{zkrW}

\begin{zkrW}{20}\noindent 
	В каждом из трех ящиков 12 черных и 5 белых шаров. Из первого и второго ящиков наудачу извлекается по одному шару и кладется в третий ящик. Затем из третьего ящика извлекается один шар. \\ \indent а) Найти вероятность того, что он черный. \\ \indent б) Известно, что этот шар черный; найти вероятность того, что шары, извлеченные из первого и второго ящиков, --- черные.
 
\end{zkrW}

\begin{zkrW}{20}\noindent 
	По каналу связи передается кодовая комбинация из 7 символов. Вероятность искажения одного символа при приеме равна $0{,}2$. Найти вероятность того, что при приеме будет искажено не более чем 3 символа.
 
\end{zkrW}

\begin{zkrW}{20}\noindent 
	В лотерее разыгрываются крупные и мелкие выигрыши. Вероятность того, что на лотерейный билет выпадет крупный выигрыш, равна $0{,}0024$, а мелкий --- $0{,}4$. Куплено 3750 билетов. Найти вероятность того, что \\ \indent а) крупных выигрышей будет более чем 5; \\ \indent б) мелких выигрышей будет ровно 1275; \\ \indent в) мелких выигрышей будет от 1425 до 1470.
 
\end{zkrW}

\newpage\setcounter{zad}{0}



\begin{zkrW}{20}\noindent 
	Прибор, работающий в течение времени $t$, состоит из трех узлов, каждый из которых независимо от других может за это время выйти из строя. Неисправность хотя бы одного узла выводит прибор из строя целиком. Вероятность безотказной работы в течение времени $t$ первого узла равна $5/6$, второго --- $5/8$, третьего --- $3/4$. Найдите вероятность того, что в течение времени $t$ прибор выйдет из строя.
 
\end{zkrW}

\begin{zkrW}{20}\noindent 
	Андрей и Саша условились встретиться в определенном месте между 14:00 и 18:00. Каждый из них может прийти в любое время в течение указанного промежутка и ждет второго некоторое время. Андрей ждет 80 минут, после чего уходит; Саша ждет 70 минут, после чего уходит. В 18:00 любой из них уходит, сколько бы до этого он ни ждал. Чему равна вероятность того, что Андрей и Саша не встретятся?
 
\end{zkrW}

\begin{zkrW}{20}\noindent 
	В альбоме 8 чистых и 7 гашеных марок. Из альбома наудачу извлекаются 2 марки и подвергаются гашению, а затем возвращаются в альбом. После этого вновь наудачу извлекаются 3 марки. \\ \indent а) Найти вероятность того, что эти марки гашеные. \\ \indent б) Известно, что эти 3 марки гашеные; найти вероятность того, что первоначально извлеченные 2 марки --- чистые.
 
\end{zkrW}

\begin{zkrW}{20}\noindent 
	В цехе 7 моторов. Для каждого мотора вероятность того, что он в данный момент включен, равна $0{,}9$. Найти вероятность того, что в данный момент включено более чем 2 мотора.
 
\end{zkrW}

\begin{zkrW}{20}\noindent 
	Вероятность выиграть отдельному игроку 1000 рублей в игре <<Кто хочет стать миллионером>> равна $0{,}8$, а 32000 рублей --- $0{,}01$. За сезон в этой игре принимает участие 100 человек. Найти вероятность того, что за сезон \\ \indent а) 1000 рублей получат ровно 70 человек; \\ \indent б) 1000 рублей получат от 84 до 91 человек; \\ \indent в) ровно 2 человек получат крупный выигрыш в 32000 рублей.
 
\end{zkrW}

\newpage\setcounter{zad}{0}



\begin{zkrW}{20}\noindent 
	Среди 20 лампочек 14 стандартные. Одновременно берут наудачу 6 лампочки. Найдите вероятность того, что хотя бы одна из них нестандартная.
 
\end{zkrW}

\begin{zkrW}{20}\noindent 
	Дуня и Пафнутий условились встретиться в определенном месте между 6:00 и 7:00. Каждый из них может прийти в любое время в течение указанного промежутка и ждет второго некоторое время. Дуня ждет 50 минут, после чего уходит; Пафнутий ждет 20 минут, после чего уходит. В 7:00 любой из них уходит, сколько бы до этого он ни ждал. Чему равна вероятность того, что встреча состоится в последние полчаса?
 
\end{zkrW}

\begin{zkrW}{20}\noindent 
	В первом ящике 10 черных и 11 красных шаров, а во втором 12 черных и 7 красных. Из первого ящика во второй перекладываются 2 наудачу извлеченных шара. После этого из второго ящика наудачу извлекается один шар. \\ \indent а) Найти вероятность того, что он красный. \\ \indent б) Известно, что этот шар красный; найти вероятность того, что извлеченные из первого ящика шары --- красные.
 
\end{zkrW}

\begin{zkrW}{20}\noindent 
	В тестовом задании 8 вопросов, на каждый дано 4 варианта ответа, среди которых один правильный. Какова вероятность того, что, выбирая вариант ответа наугад, отвечающий правильно ответит более чем на 2 вопроса?
 
\end{zkrW}

\begin{zkrW}{20}\noindent 
	В ралли принимает участие 150 экипажей. Каждый экипаж может сойти с дистанции из-за технических неполадок с вероятностью $0{,}4$, а из-за болезни водителя --- с вероятностью $0{,}02$. Найти вероятность того, что \\ \indent а) менее чем 5 экипажей сойдут с дистанции из-за болезни водителя; \\ \indent б) ровно 64 экипажей не смогут продолжать ралли из-за технических неполадок; \\ \indent в) от 53 до 58 экипажей пострадают от технических проблем.
 
\end{zkrW}

\newpage\setcounter{zad}{0}



\begin{zkrW}{20}\noindent 
	Три студента пришли сдавать экзамен. Вероятность того, что первый студент сдаст экзамен, равна $4/5$, второй --- $6/7$, третий --- $5/6$. Найдите вероятность того, что хотя бы двое сдадут экзамен.
 
\end{zkrW}

\begin{zkrW}{20}\noindent 
	Архип и Дуня условились встретиться в определенном месте между 18:00 и 19:00. Каждый из них может прийти в любое время в течение указанного промежутка и ждет второго некоторое время. Архип ждет 30 минут, после чего уходит; Дуня ждет 40 минут, после чего уходит. В 19:00 любой из них уходит, сколько бы до этого он ни ждал. Чему равна вероятность того, что встреча состоится в первые полчаса?
 
\end{zkrW}

\begin{zkrW}{20}\noindent 
	В альбоме 5 чистых и 6 гашеных марок. Из альбома наудачу извлекаются 2 марки и заменяются на чистые. После этого вновь наудачу извлекаются 5 марки. \\ \indent а) Найти вероятность того, что эти марки чистые. \\ \indent б) Известно, что эти 5 марки чистые; найти вероятность того, что первоначально извлеченные 2 марки --- гашеные.
 
\end{zkrW}

\begin{zkrW}{20}\noindent 
	Рабочий обслуживает 9 однотипных станков. Вероятность того, что станок потребует внимания рабочего в течение дня, равна $0{,}8$. Найти вероятность того, что в течение дня этих требований будет по крайней мере 4.
 
\end{zkrW}

\begin{zkrW}{20}\noindent 
	Студент за все время обучения в вузе в среднем выполняет 100 задач по математике. Вероятность неверно решить отдельную задачу при условии стопроцентного посещения и активной работы на всех занятиях равна $0{,}004$, в противном случае --- $0{,}5$. Найти вероятность того, что за время обучения в вузе \\ \indent а) абсолютно прилежный студент решил неверно не менее чем 4 задачи; \\ \indent б) обычный студент решил правильно ровно 53 задач; \\ \indent в) обычный студент неверно решил от 52 до 53 задач.
 
\end{zkrW}

\newpage\setcounter{zad}{0}



\begin{zkrW}{20}\noindent 
	Игра проводится до выигрыша одним из двух игроков двух партий подряд (ничьи исключаются). Вероятность выигрыша партии каждым из игроков равна $0{,}5$ и не зависит от исходов предыдущих партий. Найдите вероятность того, что игра окончится до 2-й партии.
 
\end{zkrW}

\begin{zkrW}{20}\noindent 
	Света и Надя условились встретиться в определенном месте между 20:00 и 21:00. Каждый из них может прийти в любое время в течение указанного промежутка и ждет второго некоторое время. Света ждет 50 минут, после чего уходит; Надя ждет 40 минут, после чего уходит. В 21:00 любой из них уходит, сколько бы до этого он ни ждал. Чему равна вероятность того, что встреча состоится в последние полчаса?
 
\end{zkrW}

\begin{zkrW}{20}\noindent 
	В каждом из трех ящиков 5 синих и 11 черных шаров. Из первого ящика в третий перекладывают два наудачу выбранных шара, а из второго ящика в третий перекладывают один наудачу взятый шар. Затем из третьего ящика извлекается один шар. \\ \indent а) Найти вероятность того, что он черный. \\ \indent б) Известно, что этот шар черный; найти вероятность того, что из первого ящика во второй переложили синие шары.
 
\end{zkrW}

\begin{zkrW}{20}\noindent 
	В Машбюро стоит 8 пишущих машин. Вероятность того, что каждая из них в течение года потребует ремонта, равна $1/2$. Найти вероятность того, что в течение года придется отремонтировать по крайней мере 2 машины.
 
\end{zkrW}

\begin{zkrW}{20}\noindent 
	Предполагая рождение ребенка в любой день года равновозможным, найти вероятность того, что в группе из 150 человек \\ \indent а) более чем 5 родились 8 сентября; \\ \indent б) ровно 63 родились осенью; \\ \indent в) от 53 до 63 родились весной.
 
\end{zkrW}

\newpage\setcounter{zad}{0}



\begin{zkrW}{20}\noindent 
	В мастерскую для ремонта поступило 16 телевизоров. Известно, что 8 из них нуждаются в общей регулировке. Мастер берет первые попавшиеся 8 телевизоров. Какова вероятность того, что более чем 2 из них нуждаются в общей настройке?
 
\end{zkrW}

\begin{zkrW}{20}\noindent 
	Джон Фаулз и Марк Твен условились встретиться в определенном месте между 17:00 и 21:00. Каждый из них может прийти в любое время в течение указанного промежутка и ждет второго некоторое время. Джон Фаулз ждет 60 минут, после чего уходит; Марк Твен ждет 70 минут, после чего уходит. В 21:00 любой из них уходит, сколько бы до этого он ни ждал. Чему равна вероятность того, что встреча состоится в первые двадцать минут?
 
\end{zkrW}

\begin{zkrW}{20}\noindent 
	В альбоме 11 чистых и 11 гашеных марок. Из альбома наудачу извлекаются 2 марки и подвергаются гашению, а затем возвращаются в альбом. После этого вновь наудачу извлекаются 5 марки. \\ \indent а) Найти вероятность того, что эти марки чистые. \\ \indent б) Известно, что эти 5 марки чистые; найти вероятность того, что первоначально извлеченные 2 марки --- гашеные.
 
\end{zkrW}

\begin{zkrW}{20}\noindent 
	На автобазе 6 машин. Вероятность выхода каждой из них на линию равна $3/5$. Найти вероятность того, что на линию по каким-либо причинам не смогут выйти ровно 3 машины.
 
\end{zkrW}

\begin{zkrW}{20}\noindent 
	Предполагая рождение ребенка в любой день года равновозможным, найти вероятность того, что в группе из 3750 человек \\ \indent а) хотя бы 3 родились 10 апреля; \\ \indent б) ровно 1650 родились осенью; \\ \indent в) от 1530 до 1545 родились весной.
 
\end{zkrW}

\newpage\setcounter{zad}{0}



\begin{zkrW}{20}\noindent 
	Среди 24 поступающих в ремонт часов 11 нуждаются в общей чистке механизма. Какова вероятность того, что среди взятых одновременно наудачу 7 часов менее чем 3 нуждаются в общей чистке механизма?
 
\end{zkrW}

\begin{zkrW}{20}\noindent 
	Микеланджело и Рафаэль условились встретиться в определенном месте между 11:00 и 12:00. Каждый из них может прийти в любое время в течение указанного промежутка и ждет второго некоторое время. Микеланджело ждет 30 минут, после чего уходит; Рафаэль ждет 40 минут, после чего уходит. В 12:00 любой из них уходит, сколько бы до этого он ни ждал. Чему равна вероятность того, что встреча состоится в последние полчаса?
 
\end{zkrW}

\begin{zkrW}{20}\noindent 
	В каждом из трех ящиков 7 желтых и 5 зеленых шаров. Из первого ящика в третий перекладывают два наудачу выбранных шара, а из второго ящика в третий перекладывают один наудачу взятый шар. Затем из третьего ящика извлекается один шар. \\ \indent а) Найти вероятность того, что он зеленый. \\ \indent б) Известно, что этот шар зеленый; найти вероятность того, что из первого ящика во второй переложили зеленые шары.
 
\end{zkrW}

\begin{zkrW}{20}\noindent 
	Партия изделий содержит $50\%$ брака. Найти вероятность того, что среди взятых наугад 7 изделий окажется не более чем 2 бракованных.
 
\end{zkrW}

\begin{zkrW}{20}\noindent 
	Известно, что левши среди населения Мордора составляют в среднем $0{,}24\%$, а люди, одинаково владеющие левой и правой рукой, --- $0{,}4$ (остальные --- правши). Найти вероятность того, что среди 3750 людей \\ \indent а) окажется менее чем 4 левшей; \\ \indent б) окажется ровно 1395 амбидекстров\footnote{людей, одинаково владеющих обеими руками}; \\ \indent в) окажется от 1455 до 1575 амбидекстров.
 
\end{zkrW}

\newpage\setcounter{zad}{0}



\begin{zkrW}{20}\noindent 
	Два стрелка сделали по одному выстрелу по мишени. Известно, что вероятность попадания в мишень для одного из стрелков равна $0{,}7$, а для другого --- $0{,}5$. Найдите вероятность того, что хотя бы один из стрелков попадет в мишень.
 
\end{zkrW}

\begin{zkrW}{20}\noindent 
	Петя и Катя условились встретиться в определенном месте между 6:00 и 10:00. Каждый из них может прийти в любое время в течение указанного промежутка и ждет второго некоторое время. Петя ждет 80 минут, после чего уходит; Катя ждет 50 минут, после чего уходит. В 10:00 любой из них уходит, сколько бы до этого он ни ждал. Чему равна вероятность того, что встреча состоится в первые двадцать минут?
 
\end{zkrW}

\begin{zkrW}{20}\noindent 
	В каждом из трех ящиков 10 желтых и 9 синих шаров. Из первого ящика в третий перекладывают два наудачу выбранных шара, а из второго ящика в третий перекладывают один наудачу взятый шар. Затем из третьего ящика извлекается один шар. \\ \indent а) Найти вероятность того, что он желтый. \\ \indent б) Известно, что этот шар желтый; найти вероятность того, что из первого ящика во второй переложили синие шары.
 
\end{zkrW}

\begin{zkrW}{20}\noindent 
	Самолет имеет 8 двигателя. Вероятность нормальной работы каждого двигателя равна $0{,}7$. Найти вероятность того, что в полете не более чем в 3 двигателях возникнут неполадки.
 
\end{zkrW}

\begin{zkrW}{20}\noindent 
	Вероятность появления опечатки на отдельной странице книги равна $0{,}0025$, а погрешности верстки --- $0{,}25$. Найти вероятность того, что в книге из 1200 страниц \\ \indent а) более чем 4 страниц будут иметь опечатки; \\ \indent б) от 279 до 312 страниц будут иметь погрешности верстки; \\ \indent в) погрешности верстки будут присутствовать ровно на 261 страницах.
 
\end{zkrW}

\newpage\setcounter{zad}{0}



\begin{zkrW}{20}\noindent 
	Игра проводится до выигрыша одним из двух игроков двух партий подряд (ничьи исключаются). Вероятность выигрыша партии каждым из игроков равна $0{,}5$ и не зависит от исходов предыдущих партий. Найдите вероятность того, что игра окончится до 3-й партии.
 
\end{zkrW}

\begin{zkrW}{20}\noindent 
	Барак Обама и Владимир Путин условились встретиться в определенном месте между 4:00 и 8:00. Каждый из них может прийти в любое время в течение указанного промежутка и ждет второго некоторое время. Барак Обама ждет 60 минут, после чего уходит; Владимир Путин ждет 70 минут, после чего уходит. В 8:00 любой из них уходит, сколько бы до этого он ни ждал. Чему равна вероятность того, что встреча произойдет не ранее чем без четверти 8:00?
 
\end{zkrW}

\begin{zkrW}{20}\noindent 
	В альбоме 5 чистых и 5 гашеных марок. Из альбома наудачу извлекаются 2 марки и подвергаются гашению, а затем возвращаются в альбом. После этого вновь наудачу извлекаются 5 марки. \\ \indent а) Найти вероятность того, что эти марки чистые. \\ \indent б) Известно, что эти 5 марки чистые; найти вероятность того, что первоначально извлеченные 2 марки --- гашеные.
 
\end{zkrW}

\begin{zkrW}{20}\noindent 
	Вероятность попадания в мишень при одном выстреле равна $1/5$. Найти вероятность того, что при 6 выстрелах будет не более чем 2 попадания.
 
\end{zkrW}

\begin{zkrW}{20}\noindent 
	Известно, что левши среди населения Уганды составляют в среднем $0{,}5\%$, а люди, одинаково владеющие левой и правой рукой, --- $0{,}6$ (остальные --- правши). Найти вероятность того, что среди 600 людей \\ \indent а) окажется по меньшей мере 5 левшей; \\ \indent б) окажется ровно 392 амбидекстров\footnote{людей, одинаково владеющих обеими руками}; \\ \indent в) окажется от 371 до 389 амбидекстров.
 
\end{zkrW}

\newpage\setcounter{zad}{0}



\begin{zkrW}{20}\noindent 
	Три лыжника съезжают с горы. Вероятность падения первого лыжника равна $3/8$, второго --- $1/4$, третьего --- $1/5$. Найдите вероятность того, что хотя бы два лыжника не упадут.
 
\end{zkrW}

\begin{zkrW}{20}\noindent 
	Андрей и Петя условились встретиться в определенном месте между 17:00 и 19:00. Каждый из них может прийти в любое время в течение указанного промежутка и ждет второго некоторое время. Андрей ждет 30 минут, после чего уходит; Петя ждет 60 минут, после чего уходит. В 19:00 любой из них уходит, сколько бы до этого он ни ждал. Чему равна вероятность того, что Андрей и Петя не встретятся?
 
\end{zkrW}

\begin{zkrW}{20}\noindent 
	В каждом из трех ящиков 11 черных и 5 зеленых шаров. Из первого ящика в третий перекладывают два наудачу выбранных шара, а из второго ящика в третий перекладывают один наудачу взятый шар. Затем из третьего ящика извлекается один шар. \\ \indent а) Найти вероятность того, что он зеленый. \\ \indent б) Известно, что этот шар зеленый; найти вероятность того, что из первого ящика во второй переложили черные шары.
 
\end{zkrW}

\begin{zkrW}{20}\noindent 
	Вероятность того, что за рабочий день расход электроэнергии не превысит норму, равна $8/9$. Найти вероятность того, что за 6 дней работы норма будет превышена более чем 2 раза.
 
\end{zkrW}

\begin{zkrW}{20}\noindent 
	Вероятность появления опечатки на отдельной странице книги равна $0{,}0015$, а погрешности верстки --- $0{,}1$. Найти вероятность того, что в книге из 400 страниц \\ \indent а) не менее чем 5 страниц будут иметь опечатки; \\ \indent б) от 34 до 42 страниц будут иметь погрешности верстки; \\ \indent в) погрешности верстки будут присутствовать ровно на 36 страницах.
 
\end{zkrW}

\newpage\setcounter{zad}{0}



\begin{zkrW}{20}\noindent 
	Студент успел подготовить к экзамену 5 вопросов из 25. Какова вероятность того, что из 7 наудачу выбранных вопросов студент знает хотя бы 3?
 
\end{zkrW}

\begin{zkrW}{20}\noindent 
	Надя и Полина условились встретиться в определенном месте между 15:00 и 18:00. Каждый из них может прийти в любое время в течение указанного промежутка и ждет второго некоторое время. Надя ждет 60 минут, после чего уходит; Полина ждет 50 минут, после чего уходит. В 18:00 любой из них уходит, сколько бы до этого он ни ждал. Чему равна вероятность того, что Надя и Полина не встретятся?
 
\end{zkrW}

\begin{zkrW}{20}\noindent 
	В каждом из трех ящиков 11 синих и 10 желтых шаров. Из первого и второго ящиков наудачу извлекается по одному шару и кладется в третий ящик. Затем из третьего ящика извлекается один шар. \\ \indent а) Найти вероятность того, что он синий. \\ \indent б) Известно, что этот шар синий; найти вероятность того, что шары, извлеченные из первого и второго ящиков, --- желтые.
 
\end{zkrW}

\begin{zkrW}{20}\noindent 
	Известно, что $40\%$ семян огурцов не всходят при посеве. Какова вероятность того, что из 8 посеянных семян взойдут не менее чем 2?
 
\end{zkrW}

\begin{zkrW}{20}\noindent 
	Стрелок попадает в цель из пистолета с вероятностью $0{,}4$, а из снайперской винтовки --- с вероятностью $0{,}999$. Найти вероятность того, что, сделав 600 выстрелов по цели из каждого оружия, стрелок \\ \indent а) промахнется из пистолета от 223 до 271 раз; \\ \indent б) промахнется из пистолета ровно 252 раз; \\ \indent в) допустит не менее чем 5 промаха из снайперской винтовки.
 
\end{zkrW}

\newpage\setcounter{zad}{0}



\begin{zkrW}{20}\noindent 
	Станция метрополитена оборудована тремя эскалаторами. Вероятность поломки в течение рабочего дня первого эскалатора равна $2/3$, второго --- $5/6$, третьего ---  $1/2$. Найдите вероятность того, что в течение рабочего дня будет исправен хотя бы один эскалатор.
 
\end{zkrW}

\begin{zkrW}{20}\noindent 
	Глаша и Пафнутий условились встретиться в определенном месте между 11:00 и 15:00. Каждый из них может прийти в любое время в течение указанного промежутка и ждет второго некоторое время. Глаша ждет 50 минут, после чего уходит; Пафнутий ждет 80 минут, после чего уходит. В 15:00 любой из них уходит, сколько бы до этого он ни ждал. Чему равна вероятность того, что Глаша и Пафнутий встретятся?
 
\end{zkrW}

\begin{zkrW}{20}\noindent 
	В первом ящике 7 белых и 10 зеленых шаров, а во втором 9 белых и 6 зеленых. Из первого ящика во второй перекладываются 3 наудачу извлеченных шара. После этого из второго ящика наудачу извлекается один шар. \\ \indent а) Найти вероятность того, что он белый. \\ \indent б) Известно, что этот шар белый; найти вероятность того, что извлеченные из первого ящика шары --- зеленые.
 
\end{zkrW}

\begin{zkrW}{20}\noindent 
	В случайно выбранной семье 7 детей. Считая вероятности рождения мальчика и девочки одинаковыми, определить вероятность того, что в выбранной семье окажется по меньшей мере 2 мальчика.
 
\end{zkrW}

\begin{zkrW}{20}\noindent 
	Вероятность выиграть отдельному игроку 1000 рублей в игре <<Кто хочет стать миллионером>> равна $0{,}4$, а 32000 рублей --- $0{,}02$. За сезон в этой игре принимает участие 150 человек. Найти вероятность того, что за сезон \\ \indent а) 1000 рублей получат ровно 67 человек; \\ \indent б) 1000 рублей получат от 62 до 68 человек; \\ \indent в) хотя бы 3 человек получат крупный выигрыш в 32000 рублей.
 
\end{zkrW}

\newpage\setcounter{zad}{0}



\begin{zkrW}{20}\noindent 
	Экзаменационный билет содержит три вопроса. Вероятности того, что студент ответит на первый и второй вопросы билета равны $6/7$; на третий --- $6/7$. Найдите вероятность того, что студент сдаст экзамен, если для этого необходимо ответить хотя бы на два вопроса.
 
\end{zkrW}

\begin{zkrW}{20}\noindent 
	Аня и Надя условились встретиться в определенном месте между 19:00 и 22:00. Каждый из них может прийти в любое время в течение указанного промежутка и ждет второго некоторое время. Аня ждет 70 минут, после чего уходит; Надя ждет 40 минут, после чего уходит. В 22:00 любой из них уходит, сколько бы до этого он ни ждал. Чему равна вероятность того, что встреча произойдет не ранее чем без четверти 22:00?
 
\end{zkrW}

\begin{zkrW}{20}\noindent 
	В каждом из трех ящиков 7 черных и 11 синих шаров. Из первого и второго ящиков наудачу извлекается по одному шару и кладется в третий ящик. Затем из третьего ящика извлекается один шар. \\ \indent а) Найти вероятность того, что он черный. \\ \indent б) Известно, что этот шар черный; найти вероятность того, что шары, извлеченные из первого и второго ящиков, --- черные.
 
\end{zkrW}

\begin{zkrW}{20}\noindent 
	Оптовая база снабжает товаром 5 магазинов. Вероятность того, что в течение дня поступит заявка на товар, равна $0{,}6$ для каждого магазина. Найти вероятность того, что в течение дня поступит хотя бы 3 заявки.
 
\end{zkrW}

\begin{zkrW}{20}\noindent 
	Предполагая рождение ребенка в любой день года равновозможным, найти вероятность того, что в группе из 1200 человек \\ \indent а) не менее чем 2 родились 12 июня; \\ \indent б) ровно 963 родились осенью; \\ \indent в) от 882 до 918 родились весной.
 
\end{zkrW}

\newpage\setcounter{zad}{0}



\begin{zkrW}{20}\noindent 
	Студент успел подготовить к экзамену 8 вопросов из 20. Какова вероятность того, что из 7 наудачу выбранных вопросов студент знает по меньшей мере 4?
 
\end{zkrW}

\begin{zkrW}{20}\noindent 
	Петя и Андрей условились встретиться в определенном месте между 18:00 и 20:00. Каждый из них может прийти в любое время в течение указанного промежутка и ждет второго некоторое время. Петя ждет 50 минут, после чего уходит; Андрей ждет 60 минут, после чего уходит. В 20:00 любой из них уходит, сколько бы до этого он ни ждал. Чему равна вероятность того, что встреча состоится в первые полчаса?
 
\end{zkrW}

\begin{zkrW}{20}\noindent 
	В каждом из трех ящиков 8 зеленых и 11 черных шаров. Из первого и второго ящиков наудачу извлекается по одному шару и кладется в третий ящик. Затем из третьего ящика извлекается один шар. \\ \indent а) Найти вероятность того, что он зеленый. \\ \indent б) Известно, что этот шар зеленый; найти вероятность того, что шары, извлеченные из первого и второго ящиков, --- зеленые.
 
\end{zkrW}

\begin{zkrW}{20}\noindent 
	В тестовом задании 8 вопросов, на каждый дано 5 варианта ответа, среди которых один правильный. Какова вероятность того, что, выбирая вариант ответа наугад, отвечающий правильно ответит по крайней мере на 3 вопроса?
 
\end{zkrW}

\begin{zkrW}{20}\noindent 
	Студент за все время обучения в вузе в среднем выполняет 1200 задач по математике. Вероятность неверно решить отдельную задачу при условии стопроцентного посещения и активной работы на всех занятиях равна $0{,}0075$, в противном случае --- $0{,}25$. Найти вероятность того, что за время обучения в вузе \\ \indent а) абсолютно прилежный студент решил неверно по крайней мере 3 задачи; \\ \indent б) обычный студент решил правильно ровно 276 задач; \\ \indent в) обычный студент неверно решил от 285 до 321 задач.
 
\end{zkrW}

\newpage\setcounter{zad}{0}



\begin{zkrW}{20}\noindent 
	Два стрелка сделали по одному выстрелу по мишени. Известно, что вероятность попадания в мишень для одного из стрелков равна $0{,}3$, а для другого --- $0{,}8$. Найдите вероятность того, что не менее чем один из стрелков не попадет в мишень.
 
\end{zkrW}

\begin{zkrW}{20}\noindent 
	Робин и Бэтмен условились встретиться в определенном месте между 18:00 и 20:00. Каждый из них может прийти в любое время в течение указанного промежутка и ждет второго некоторое время. Робин ждет 30 минут, после чего уходит; Бэтмен ждет 50 минут, после чего уходит. В 20:00 любой из них уходит, сколько бы до этого он ни ждал. Чему равна вероятность того, что Робин опоздает менее чем на полчаса?
 
\end{zkrW}

\begin{zkrW}{20}\noindent 
	В каждом из трех ящиков 5 белых и 12 синих шаров. Из первого и второго ящиков наудачу извлекается по одному шару и кладется в третий ящик. Затем из третьего ящика извлекается один шар. \\ \indent а) Найти вероятность того, что он белый. \\ \indent б) Известно, что этот шар белый; найти вероятность того, что шары, извлеченные из первого и второго ящиков, --- синие.
 
\end{zkrW}

\begin{zkrW}{20}\noindent 
	В Машбюро стоит 6 пишущих машин. Вероятность того, что каждая из них в течение года потребует ремонта, равна $0{,}5$. Найти вероятность того, что в течение года придется отремонтировать более чем 3 машины.
 
\end{zkrW}

\begin{zkrW}{20}\noindent 
	Стрелок попадает в цель из пистолета с вероятностью $0{,}6$, а из снайперской винтовки --- с вероятностью $0{,}9975$. Найти вероятность того, что, сделав 2400 выстрелов по цели из каждого оружия, стрелок \\ \indent а) промахнется из пистолета от 1397 до 1498 раз; \\ \indent б) промахнется из пистолета ровно 1282 раз; \\ \indent в) допустит ровно 2 промаха из снайперской винтовки.
 
\end{zkrW}

\newpage\setcounter{zad}{0}



\begin{zkrW}{20}\noindent 
	Станция метрополитена оборудована тремя эскалаторами. Вероятность поломки в течение рабочего дня первого эскалатора равна $0{,}9$, второго --- $0{,}8$, третьего ---  $0{,}5$. Найдите вероятность того, что в течение рабочего дня будет исправен хотя бы один эскалатор.
 
\end{zkrW}

\begin{zkrW}{20}\noindent 
	Леонардо и Донателло условились встретиться в определенном месте между 1:00 и 3:00. Каждый из них может прийти в любое время в течение указанного промежутка и ждет второго некоторое время. Леонардо ждет 40 минут, после чего уходит; Донателло ждет 30 минут, после чего уходит. В 3:00 любой из них уходит, сколько бы до этого он ни ждал. Чему равна вероятность того, что Леонардо и Донателло не встретятся?
 
\end{zkrW}

\begin{zkrW}{20}\noindent 
	В альбоме 5 чистых и 5 гашеных марок. Из альбома наудачу извлекаются 2 марки и заменяются на чистые. После этого вновь наудачу извлекаются 5 марки. \\ \indent а) Найти вероятность того, что эти марки чистые. \\ \indent б) Известно, что эти 5 марки чистые; найти вероятность того, что первоначально извлеченные 2 марки --- гашеные.
 
\end{zkrW}

\begin{zkrW}{20}\noindent 
	В ячейку памяти ЭВМ записывается двоичное число длиной в 9 разрядов. Значения 0 и 1 в каждом разряде появляются с равной вероятностью. Найти вероятность того, что в этом двоичном числе по крайней мере 2 единицы.
 
\end{zkrW}

\begin{zkrW}{20}\noindent 
	Стрелок попадает в цель из пистолета с вероятностью $0{,}2$, а из снайперской винтовки --- с вероятностью $0{,}995$. Найти вероятность того, что, сделав 1600 выстрелов по цели из каждого оружия, стрелок \\ \indent а) промахнется из пистолета от 282 до 362 раз; \\ \indent б) промахнется из пистолета ровно 304 раз; \\ \indent в) допустит по меньшей мере 2 промаха из снайперской винтовки.
 
\end{zkrW}

\newpage\setcounter{zad}{0}



\begin{zkrW}{20}\noindent 
	В ящике 21 деталей, среди которых 12 окрашенных. Сборщик наудачу достает 8 деталей. Найдите вероятность того, что 2 из них оказались окрашенными. 
 
\end{zkrW}

\begin{zkrW}{20}\noindent 
	Марк Твен и Джон Фаулз условились встретиться в определенном месте между 17:00 и 18:00. Каждый из них может прийти в любое время в течение указанного промежутка и ждет второго некоторое время. Марк Твен ждет 20 минут, после чего уходит; Джон Фаулз ждет 50 минут, после чего уходит. В 18:00 любой из них уходит, сколько бы до этого он ни ждал. Чему равна вероятность того, что встреча состоится в первые полчаса?
 
\end{zkrW}

\begin{zkrW}{20}\noindent 
	В альбоме 6 чистых и 9 гашеных марок. Из альбома изымаются 2 наудачу извлеченные марки. После этого из альбома вновь наудачу извлекаются 4 марки. \\ \indent а) Найти вероятность того, что эти марки гашеные. \\ \indent б) Известно, что эти 4 марки гашеные; найти вероятность того, что первоначально изъятые 2 марки --- чистые.
 
\end{zkrW}

\begin{zkrW}{20}\noindent 
	Всхожесть семян данного сорта растений оценивается с вероятностью, равной $1/3$. Какова вероятность того, что из 7 посеянных семян взойдут хотя бы 3?
 
\end{zkrW}

\begin{zkrW}{20}\noindent 
	Вероятность выиграть отдельному игроку 1000 рублей в игре <<Кто хочет стать миллионером>> равна $0{,}6$, а 32000 рублей --- $0{,}001$. За сезон в этой игре принимает участие 600 человек. Найти вероятность того, что за сезон \\ \indent а) 1000 рублей получат ровно 346 человек; \\ \indent б) 1000 рублей получат от 371 до 389 человек; \\ \indent в) ровно 5 человек получат крупный выигрыш в 32000 рублей.
 
\end{zkrW}

\newpage\setcounter{zad}{0}



\begin{zkrW}{20}\noindent 
	Вероятность наступления некоторого случайного события в каждом опыте одинакова и равна $4/7$. Опыты проводятся последовательно до наступления этого события. Определить вероятность того, что придется проводить 3-й опыт.
 
\end{zkrW}

\begin{zkrW}{20}\noindent 
	Атос и Арамис условились встретиться в определенном месте между 20:00 и 23:00. Каждый из них может прийти в любое время в течение указанного промежутка и ждет второго некоторое время. Атос ждет 70 минут, после чего уходит; Арамис ждет 50 минут, после чего уходит. В 23:00 любой из них уходит, сколько бы до этого он ни ждал. Чему равна вероятность того, что Атос опоздает менее чем на полчаса?
 
\end{zkrW}

\begin{zkrW}{20}\noindent 
	В каждом из трех ящиков 6 черных и 8 желтых шаров. Из первого ящика в третий перекладывают два наудачу выбранных шара, а из второго ящика в третий перекладывают один наудачу взятый шар. Затем из третьего ящика извлекается один шар. \\ \indent а) Найти вероятность того, что он желтый. \\ \indent б) Известно, что этот шар желтый; найти вероятность того, что из первого ящика во второй переложили черные шары.
 
\end{zkrW}

\begin{zkrW}{20}\noindent 
	В студии находятся 6 телевизионных камер. Для каждой камеры вероятность того, что она включена в данный момент, равна $5/9$. Найти вероятность того, что в данный момент оказались выключены не менее чем 4 камеры.
 
\end{zkrW}

\begin{zkrW}{20}\noindent 
	Студент за все время обучения в вузе в среднем выполняет 3600 задач по математике. Вероятность неверно решить отдельную задачу при условии стопроцентного посещения и активной работы на всех занятиях равна $0{,}0025$, в противном случае --- $0{,}2$. Найти вероятность того, что за время обучения в вузе \\ \indent а) абсолютно прилежный студент решил неверно ровно 4 задачи; \\ \indent б) обычный студент решил правильно ровно 814 задач; \\ \indent в) обычный студент неверно решил от 698 до 734 задач.
 
\end{zkrW}

\newpage\setcounter{zad}{0}



\begin{zkrW}{20}\noindent 
	Произведено три выстрела по цели из орудия. Вероятность попадания при первом выстреле равна $5/6$; при втором --- $1/4$; при третьем --- $3/7$. Определить вероятность того, что будет хотя бы одно попадание.
 
\end{zkrW}

\begin{zkrW}{20}\noindent 
	Джон Фаулз и Эрнест Хэмингуэй условились встретиться в определенном месте между 20:00 и 22:00. Каждый из них может прийти в любое время в течение указанного промежутка и ждет второго некоторое время. Джон Фаулз ждет 40 минут, после чего уходит; Эрнест Хэмингуэй ждет 30 минут, после чего уходит. В 22:00 любой из них уходит, сколько бы до этого он ни ждал. Чему равна вероятность того, что Джон Фаулз и Эрнест Хэмингуэй встретятся?
 
\end{zkrW}

\begin{zkrW}{20}\noindent 
	В альбоме 6 чистых и 9 гашеных марок. Из альбома наудачу извлекаются 2 марки и подвергаются гашению, а затем возвращаются в альбом. После этого вновь наудачу извлекаются 5 марки. \\ \indent а) Найти вероятность того, что эти марки гашеные. \\ \indent б) Известно, что эти 5 марки гашеные; найти вероятность того, что первоначально извлеченные 2 марки --- чистые.
 
\end{zkrW}

\begin{zkrW}{20}\noindent 
	Вероятность того, что за рабочий день расход электроэнергии не превысит норму, равна $0{,}1$. Найти вероятность того, что за 7 дней работы норма будет превышена хотя бы 3 раза.
 
\end{zkrW}

\begin{zkrW}{20}\noindent 
	Вероятность выиграть отдельному игроку 1000 рублей в игре <<Кто хочет стать миллионером>> равна $0{,}5$, а 32000 рублей --- $0{,}0025$. За сезон в этой игре принимает участие 3600 человек. Найти вероятность того, что за сезон \\ \indent а) 1000 рублей получат ровно 1746 человек; \\ \indent б) 1000 рублей получат от 1746 до 1836 человек; \\ \indent в) по меньшей мере 2 человек получат крупный выигрыш в 32000 рублей.
 
\end{zkrW}

\newpage\setcounter{zad}{0}



\begin{zkrW}{20}\noindent 
	Экспедиция издательства отправила газеты в три почтовых отделения. Вероятность своевременной доставки газет в первое отделение равна $1/2$, во второе отделение --- $1/3$ и в третье --- $5/7$. Найдите вероятность того, что хотя бы одно отделение получит газеты с опозданием.
 
\end{zkrW}

\begin{zkrW}{20}\noindent 
	Петя и Вася условились встретиться в определенном месте между 20:00 и 21:00. Каждый из них может прийти в любое время в течение указанного промежутка и ждет второго некоторое время. Петя ждет 50 минут, после чего уходит; Вася ждет 20 минут, после чего уходит. В 21:00 любой из них уходит, сколько бы до этого он ни ждал. Чему равна вероятность того, что встреча состоится в первые двадцать минут?
 
\end{zkrW}

\begin{zkrW}{20}\noindent 
	В первом ящике 5 синих и 6 красных шаров, а во втором 6 синих и 11 красных. Из первого ящика во второй перекладываются 2 наудачу извлеченных шара. После этого из второго ящика наудачу извлекается один шар. \\ \indent а) Найти вероятность того, что он красный. \\ \indent б) Известно, что этот шар красный; найти вероятность того, что извлеченные из первого ящика шары --- синие.
 
\end{zkrW}

\begin{zkrW}{20}\noindent 
	При разведочном бурении производится отбор керна. Вероятность успешного отбора керна при каждой попытке равна $1/2$. Планируется совершить 7 попыток отбора. Какова вероятность того, что менее чем в 3 из них керн будет получен?
 
\end{zkrW}

\begin{zkrW}{20}\noindent 
	Студент за все время обучения в вузе в среднем выполняет 2500 задач по математике. Вероятность неверно решить отдельную задачу при условии стопроцентного посещения и активной работы на всех занятиях равна $0{,}0036$, в противном случае --- $0{,}2$. Найти вероятность того, что за время обучения в вузе \\ \indent а) абсолютно прилежный студент решил неверно по крайней мере 3 задачи; \\ \indent б) обычный студент решил правильно ровно 450 задач; \\ \indent в) обычный студент неверно решил от 465 до 520 задач.
 
\end{zkrW}

\newpage\setcounter{zad}{0}



\begin{zkrW}{20}\noindent 
	Экспедиция издательства отправила газеты в три почтовых отделения. Вероятность своевременной доставки газет в первое отделение равна $0{,}7$, во второе отделение --- $0{,}6$ и в третье --- $0{,}8$. Найдите вероятность того, что хотя бы одно отделение получит газеты с опозданием.
 
\end{zkrW}

\begin{zkrW}{20}\noindent 
	Портос и Атос условились встретиться в определенном месте между 11:00 и 15:00. Каждый из них может прийти в любое время в течение указанного промежутка и ждет второго некоторое время. Портос ждет 80 минут, после чего уходит; Атос ждет 70 минут, после чего уходит. В 15:00 любой из них уходит, сколько бы до этого он ни ждал. Чему равна вероятность того, что Портос опоздает более чем на полчаса?
 
\end{zkrW}

\begin{zkrW}{20}\noindent 
	В альбоме 6 чистых и 12 гашеных марок. Из альбома наудачу извлекаются 3 марки и подвергаются гашению, а затем возвращаются в альбом. После этого вновь наудачу извлекаются 2 марки. \\ \indent а) Найти вероятность того, что эти марки чистые. \\ \indent б) Известно, что эти 2 марки чистые; найти вероятность того, что первоначально извлеченные 3 марки --- гашеные.
 
\end{zkrW}

\begin{zkrW}{20}\noindent 
	$70\%$ изделий данного предприятия — это продукция высшего сорта. Некто приобрел 6 изделий, изготовленных на этом предприятии. Чему равна вероятность того, что по меньшей мере 2 из них — высшего сорта?
 
\end{zkrW}

\begin{zkrW}{20}\noindent 
	В лотерее разыгрываются крупные и мелкие выигрыши. Вероятность того, что на лотерейный билет выпадет крупный выигрыш, равна $0{,}0025$, а мелкий --- $0{,}2$. Куплено 3600 билетов. Найти вероятность того, что \\ \indent а) крупных выигрышей будет не более чем 5; \\ \indent б) мелких выигрышей будет ровно 828; \\ \indent в) мелких выигрышей будет от 655 до 662.
 
\end{zkrW}

\newpage\setcounter{zad}{0}



\begin{zkrW}{20}\noindent 
	Вероятность наступления некоторого случайного события в каждом опыте одинакова и равна $1/7$. Опыты проводятся последовательно до наступления этого события. Определить вероятность того, что придется проводить 4-й опыт.
 
\end{zkrW}

\begin{zkrW}{20}\noindent 
	Арамис и д'Артаньян условились встретиться в определенном месте между 10:00 и 14:00. Каждый из них может прийти в любое время в течение указанного промежутка и ждет второго некоторое время. Арамис ждет 70 минут, после чего уходит; д'Артаньян ждет 60 минут, после чего уходит. В 14:00 любой из них уходит, сколько бы до этого он ни ждал. Чему равна вероятность того, что Арамис опоздает более чем на полчаса?
 
\end{zkrW}

\begin{zkrW}{20}\noindent 
	В каждом из трех ящиков 8 белых и 12 желтых шаров. Из первого ящика в третий перекладывают два наудачу выбранных шара, а из второго ящика в третий перекладывают один наудачу взятый шар. Затем из третьего ящика извлекается один шар. \\ \indent а) Найти вероятность того, что он белый. \\ \indent б) Известно, что этот шар белый; найти вероятность того, что из первого ящика во второй переложили белые шары.
 
\end{zkrW}

\begin{zkrW}{20}\noindent 
	В цехе 9 моторов. Для каждого мотора вероятность того, что он в данный момент включен, равна $0{,}3$. Найти вероятность того, что в данный момент включено более чем 4 мотора.
 
\end{zkrW}

\begin{zkrW}{20}\noindent 
	Мастерская за год ремонтирует 900 мобильных телефонов. Вероятность неисправности в механической части отдельного телефона равна $0{,}8$, в электронной части --- $0{,}01$. Найти вероятность того, что среди телефонов, отремонтированных за год, \\ \indent а) имели неисправности в механической части от 734 до 749 экземпляров; \\ \indent б) имели неисправности в электронной части по крайней мере 5 телефонов; \\ \indent в) ровно 799 телефонов имели проблемы в механической части.
 
\end{zkrW}

\newpage\setcounter{zad}{0}



\begin{zkrW}{20}\noindent 
	Три лыжника съезжают с горы. Вероятность падения первого лыжника равна $0{,}6$, второго --- $0{,}6$, третьего --- $0{,}7$. Найдите вероятность того, что хотя бы два лыжника не упадут.
 
\end{zkrW}

\begin{zkrW}{20}\noindent 
	Барак Обама и Владимир Путин условились встретиться в определенном месте между 11:00 и 13:00. Каждый из них может прийти в любое время в течение указанного промежутка и ждет второго некоторое время. Барак Обама ждет 50 минут, после чего уходит; Владимир Путин ждет 30 минут, после чего уходит. В 13:00 любой из них уходит, сколько бы до этого он ни ждал. Чему равна вероятность того, что встреча произойдет не ранее чем без четверти 13:00?
 
\end{zkrW}

\begin{zkrW}{20}\noindent 
	В первом ящике 9 зеленых и 9 красных шаров, а во втором 12 зеленых и 11 красных. Из первого ящика во второй перекладываются 3 наудачу извлеченных шара. После этого из второго ящика наудачу извлекается один шар. \\ \indent а) Найти вероятность того, что он красный. \\ \indent б) Известно, что этот шар красный; найти вероятность того, что извлеченные из первого ящика шары --- зеленые.
 
\end{zkrW}

\begin{zkrW}{20}\noindent 
	Вероятность правильного оформления доверенности у нотариуса Иванова-Ежова равна $0{,}8$. В течение одного часа нотариус Иванов-Ежов оформил 8 доверенности. Какова вероятность, что ровно 4 из них оказались оформлены неправильно?
 
\end{zkrW}

\begin{zkrW}{20}\noindent 
	Предполагая рождение ребенка в любой день года равновозможным, найти вероятность того, что в группе из 100 человек \\ \indent а) более чем 5 родились 12 апреля; \\ \indent б) ровно 18 родились осенью; \\ \indent в) от 22 до 23 родились весной.
 
\end{zkrW}

\newpage\setcounter{zad}{0}



\begin{zkrW}{20}\noindent 
	Произведено три выстрела по цели из орудия. Вероятность попадания при первом выстреле равна $0{,}6$; при втором --- $0{,}2$; при третьем --- $0{,}7$. Определить вероятность того, что будет хотя бы одно попадание.
 
\end{zkrW}

\begin{zkrW}{20}\noindent 
	Леонардо и Микеланджело условились встретиться в определенном месте между 13:00 и 16:00. Каждый из них может прийти в любое время в течение указанного промежутка и ждет второго некоторое время. Леонардо ждет 50 минут, после чего уходит; Микеланджело ждет 40 минут, после чего уходит. В 16:00 любой из них уходит, сколько бы до этого он ни ждал. Чему равна вероятность того, что Леонардо и Микеланджело не встретятся?
 
\end{zkrW}

\begin{zkrW}{20}\noindent 
	В каждом из трех ящиков 10 красных и 7 черных шаров. Из первого и второго ящиков наудачу извлекается по одному шару и кладется в третий ящик. Затем из третьего ящика извлекается один шар. \\ \indent а) Найти вероятность того, что он черный. \\ \indent б) Известно, что этот шар черный; найти вероятность того, что шары, извлеченные из первого и второго ящиков, --- черные.
 
\end{zkrW}

\begin{zkrW}{20}\noindent 
	Эксплуатируется устройство, состоящее из 6 независимо работающих элементов. Вероятность отказа каждого из них за время работы устройства равна $3/8$. Найти вероятность того, что за время работы устройства откажут по меньшей мере 3 элемента.
 
\end{zkrW}

\begin{zkrW}{20}\noindent 
	Предполагая рождение ребенка в любой день года равновозможным, найти вероятность того, что в группе из 600 человек \\ \indent а) не менее чем 2 родились 8 августа; \\ \indent б) ровно 317 родились осенью; \\ \indent в) от 349 до 378 родились весной.
 
\end{zkrW}

\newpage\setcounter{zad}{0}



\begin{zkrW}{20}\noindent 
	В коробке 15 красных, 14 синих и 8 желтых карандашей. Наудачу вынимают три карандаша. Какова вероятность того, что они все разных цветов?
 
\end{zkrW}

\begin{zkrW}{20}\noindent 
	Андрей и Света условились встретиться в определенном месте между 13:00 и 16:00. Каждый из них может прийти в любое время в течение указанного промежутка и ждет второго некоторое время. Андрей ждет 50 минут, после чего уходит; Света ждет 60 минут, после чего уходит. В 16:00 любой из них уходит, сколько бы до этого он ни ждал. Чему равна вероятность того, что Андрей опоздает менее чем на полчаса?
 
\end{zkrW}

\begin{zkrW}{20}\noindent 
	В альбоме 9 чистых и 8 гашеных марок. Из альбома наудачу извлекаются 2 марки и подвергаются гашению, а затем возвращаются в альбом. После этого вновь наудачу извлекаются 3 марки. \\ \indent а) Найти вероятность того, что эти марки гашеные. \\ \indent б) Известно, что эти 3 марки гашеные; найти вероятность того, что первоначально извлеченные 2 марки --- чистые.
 
\end{zkrW}

\begin{zkrW}{20}\noindent 
	Всхожесть семян данного сорта растений оценивается с вероятностью, равной $0{,}5$. Какова вероятность того, что из 8 посеянных семян взойдут по крайней мере 3?
 
\end{zkrW}

\begin{zkrW}{20}\noindent 
	Известно, что левши среди населения Уганды составляют в среднем $0{,}25\%$, а люди, одинаково владеющие левой и правой рукой, --- $0{,}1$ (остальные --- правши). Найти вероятность того, что среди 3600 людей \\ \indent а) окажется хотя бы 2 левшей; \\ \indent б) окажется ровно 400 амбидекстров\footnote{людей, одинаково владеющих обеими руками}; \\ \indent в) окажется от 313 до 385 амбидекстров.
 
\end{zkrW}

\newpage\setcounter{zad}{0}



\begin{zkrW}{20}\noindent 
	Студент разыскивает нужную ему формулу в трех справочниках. Вероятность того, что формула содержится в первом, втором и третьем справочниках, равна соответственно $0{,}2$, $0{,}8$ и $0{,}9$. Найдите вероятность того, что эта формула содержится не менее чем в двух спОравочниках.
 
\end{zkrW}

\begin{zkrW}{20}\noindent 
	Арамис и д'Артаньян условились встретиться в определенном месте между 17:00 и 20:00. Каждый из них может прийти в любое время в течение указанного промежутка и ждет второго некоторое время. Арамис ждет 40 минут, после чего уходит; д'Артаньян ждет 70 минут, после чего уходит. В 20:00 любой из них уходит, сколько бы до этого он ни ждал. Чему равна вероятность того, что встреча произойдет не ранее чем без четверти 20:00?
 
\end{zkrW}

\begin{zkrW}{20}\noindent 
	В каждом из трех ящиков 10 зеленых и 7 голубых шаров. Из первого и второго ящиков наудачу извлекается по одному шару и кладется в третий ящик. Затем из третьего ящика извлекается один шар. \\ \indent а) Найти вероятность того, что он голубой. \\ \indent б) Известно, что этот шар голубой; найти вероятность того, что шары, извлеченные из первого и второго ящиков, --- голубые.
 
\end{zkrW}

\begin{zkrW}{20}\noindent 
	Вероятность попадания в мишень при одном выстреле равна $0{,}3$. Найти вероятность того, что при 7 выстрелах будет ровно 3 попадания.
 
\end{zkrW}

\begin{zkrW}{20}\noindent 
	В ралли принимает участие 1600 экипажей. Каждый экипаж может сойти с дистанции из-за технических неполадок с вероятностью $0{,}8$, а из-за болезни водителя --- с вероятностью $0{,}0025$. Найти вероятность того, что \\ \indent а) не менее чем 2 экипажей сойдут с дистанции из-за болезни водителя; \\ \indent б) ровно 1203 экипажей не смогут продолжать ралли из-за технических неполадок; \\ \indent в) от 1254 до 1306 экипажей пострадают от технических проблем.
 
\end{zkrW}

\newpage\setcounter{zad}{0}



\begin{zkrW}{20}\noindent 
	Контролер ОТК, проверив качество сшитых 23 пальто, установил, что 9 из них --- первого сорта, а остальные --- второго. Найдите вероятность того, что среди взятых наудачу из этой партии 7 пальто ровно 4 будут второго сорта.
 
\end{zkrW}

\begin{zkrW}{20}\noindent 
	Владимир Путин и Барак Обама условились встретиться в определенном месте между 17:00 и 19:00. Каждый из них может прийти в любое время в течение указанного промежутка и ждет второго некоторое время. Владимир Путин ждет 30 минут, после чего уходит; Барак Обама ждет 60 минут, после чего уходит. В 19:00 любой из них уходит, сколько бы до этого он ни ждал. Чему равна вероятность того, что встреча произойдет не ранее чем без четверти 19:00?
 
\end{zkrW}

\begin{zkrW}{20}\noindent 
	В первом ящике 5 красных и 11 зеленых шаров, а во втором 5 красных и 11 зеленых. Из первого ящика во второй перекладываются 2 наудачу извлеченных шара. После этого из второго ящика наудачу извлекается один шар. \\ \indent а) Найти вероятность того, что он красный. \\ \indent б) Известно, что этот шар красный; найти вероятность того, что извлеченные из первого ящика шары --- красные.
 
\end{zkrW}

\begin{zkrW}{20}\noindent 
	Вероятность попадания в мишень при одном выстреле равна $0{,}8$. Найти вероятность того, что при 7 выстрелах будет менее чем 3 попадания.
 
\end{zkrW}

\begin{zkrW}{20}\noindent 
	Вероятность появления опечатки на отдельной странице книги равна $0{,}0025$, а погрешности верстки --- $0{,}4$. Найти вероятность того, что в книге из 2400 страниц \\ \indent а) по меньшей мере 4 страниц будут иметь опечатки; \\ \indent б) от 902 до 979 страниц будут иметь погрешности верстки; \\ \indent в) погрешности верстки будут присутствовать ровно на 854 страницах.
 
\end{zkrW}

\newpage\setcounter{zad}{0}



\begin{zkrW}{20}\noindent 
	Контролер ОТК, проверив качество сшитых 15 пальто, установил, что 14 из них --- первого сорта, а остальные --- второго. Найдите вероятность того, что среди взятых наудачу из этой партии 6 пальто не более чем 4 будут второго сорта.
 
\end{zkrW}

\begin{zkrW}{20}\noindent 
	Света и Аня условились встретиться в определенном месте между 20:00 и 22:00. Каждый из них может прийти в любое время в течение указанного промежутка и ждет второго некоторое время. Света ждет 50 минут, после чего уходит; Аня ждет 60 минут, после чего уходит. В 22:00 любой из них уходит, сколько бы до этого он ни ждал. Чему равна вероятность того, что встреча состоится в первые полчаса?
 
\end{zkrW}

\begin{zkrW}{20}\noindent 
	В каждом из трех ящиков 9 зеленых и 12 черных шаров. Из первого и второго ящиков наудачу извлекается по одному шару и кладется в третий ящик. Затем из третьего ящика извлекается один шар. \\ \indent а) Найти вероятность того, что он зеленый. \\ \indent б) Известно, что этот шар зеленый; найти вероятность того, что шары, извлеченные из первого и второго ящиков, --- черные.
 
\end{zkrW}

\begin{zkrW}{20}\noindent 
	По каналу связи передается кодовая комбинация из 9 символов. Вероятность искажения одного символа при приеме равна $0{,}3$. Найти вероятность того, что при приеме будет искажено ровно 2 символа.
 
\end{zkrW}

\begin{zkrW}{20}\noindent 
	В лотерее разыгрываются крупные и мелкие выигрыши. Вероятность того, что на лотерейный билет выпадет крупный выигрыш, равна $0{,}0025$, а мелкий --- $0{,}6$. Куплено 2400 билетов. Найти вероятность того, что \\ \indent а) крупных выигрышей будет ровно 2; \\ \indent б) мелких выигрышей будет ровно 1411; \\ \indent в) мелких выигрышей будет от 1397 до 1483.
 
\end{zkrW}

\newpage\setcounter{zad}{0}



\begin{zkrW}{20}\noindent 
	Среди 23 лампочек 14 стандартные. Одновременно берут наудачу 7 лампочки. Найдите вероятность того, что хотя бы одна из них нестандартная.
 
\end{zkrW}

\begin{zkrW}{20}\noindent 
	Алексей и Полина условились встретиться в определенном месте между 7:00 и 8:00. Каждый из них может прийти в любое время в течение указанного промежутка и ждет второго некоторое время. Алексей ждет 50 минут, после чего уходит; Полина ждет 20 минут, после чего уходит. В 8:00 любой из них уходит, сколько бы до этого он ни ждал. Чему равна вероятность того, что встреча состоится в первые полчаса?
 
\end{zkrW}

\begin{zkrW}{20}\noindent 
	В каждом из трех ящиков 8 красных и 10 черных шаров. Из первого ящика в третий перекладывают два наудачу выбранных шара, а из второго ящика в третий перекладывают один наудачу взятый шар. Затем из третьего ящика извлекается один шар. \\ \indent а) Найти вероятность того, что он красный. \\ \indent б) Известно, что этот шар красный; найти вероятность того, что из первого ящика во второй переложили красные шары.
 
\end{zkrW}

\begin{zkrW}{20}\noindent 
	На автобазе 5 машин. Вероятность выхода каждой из них на линию равна $0{,}1$. Найти вероятность того, что на линию по каким-либо причинам не смогут выйти хотя бы 2 машины.
 
\end{zkrW}

\begin{zkrW}{20}\noindent 
	В ралли принимает участие 1200 экипажей. Каждый экипаж может сойти с дистанции из-за технических неполадок с вероятностью $0{,}25$, а из-за болезни водителя --- с вероятностью $0{,}005$. Найти вероятность того, что \\ \indent а) менее чем 5 экипажей сойдут с дистанции из-за болезни водителя; \\ \indent б) ровно 330 экипажей не смогут продолжать ралли из-за технических неполадок; \\ \indent в) от 258 до 327 экипажей пострадают от технических проблем.
 
\end{zkrW}

\newpage\setcounter{zad}{0}



\begin{zkrW}{20}\noindent 
	Экзаменационный билет содержит три вопроса. Вероятности того, что студент ответит на первый и второй вопросы билета равны $0{,}7$; на третий --- $0{,}6$. Найдите вероятность того, что студент сдаст экзамен, если для этого необходимо ответить хотя бы на два вопроса.
 
\end{zkrW}

\begin{zkrW}{20}\noindent 
	Джон Фаулз и Марк Твен условились встретиться в определенном месте между 7:00 и 9:00. Каждый из них может прийти в любое время в течение указанного промежутка и ждет второго некоторое время. Джон Фаулз ждет 30 минут, после чего уходит; Марк Твен ждет 50 минут, после чего уходит. В 9:00 любой из них уходит, сколько бы до этого он ни ждал. Чему равна вероятность того, что Джон Фаулз и Марк Твен не встретятся?
 
\end{zkrW}

\begin{zkrW}{20}\noindent 
	В альбоме 8 чистых и 8 гашеных марок. Из альбома наудачу извлекаются 2 марки и заменяются на чистые. После этого вновь наудачу извлекаются 2 марки. \\ \indent а) Найти вероятность того, что эти марки чистые. \\ \indent б) Известно, что эти 2 марки чистые; найти вероятность того, что первоначально извлеченные 2 марки --- гашеные.
 
\end{zkrW}

\begin{zkrW}{20}\noindent 
	Вероятность правильного оформления доверенности у нотариуса Иванова-Ежова равна $7/8$. В течение одного часа нотариус Иванов-Ежов оформил 6 доверенности. Какова вероятность, что по крайней мере 3 из них оказались оформлены неправильно?
 
\end{zkrW}

\begin{zkrW}{20}\noindent 
	Вероятность выиграть отдельному игроку 1000 рублей в игре <<Кто хочет стать миллионером>> равна $0{,}8$, а 32000 рублей --- $0{,}01$. За сезон в этой игре принимает участие 900 человек. Найти вероятность того, что за сезон \\ \indent а) 1000 рублей получат ровно 634 человек; \\ \indent б) 1000 рублей получат от 691 до 756 человек; \\ \indent в) хотя бы 4 человек получат крупный выигрыш в 32000 рублей.
 
\end{zkrW}

\newpage\setcounter{zad}{0}



\begin{zkrW}{20}\noindent 
	Мастер обслуживает четыре станка, работающих независимо друг от друга. Вероятность того, что первыйй станок в течение смены потребует внимания мастера, равна $1/2$, второй --- $3/4$, третий --- $3/4$ и четвертый --- $4/9$. Найдите вероятность того, что в течение смены хотя бы один станок не потребует внимания мастера.
 
\end{zkrW}

\begin{zkrW}{20}\noindent 
	Аня и Саша условились встретиться в определенном месте между 3:00 и 6:00. Каждый из них может прийти в любое время в течение указанного промежутка и ждет второго некоторое время. Аня ждет 60 минут, после чего уходит; Саша ждет 70 минут, после чего уходит. В 6:00 любой из них уходит, сколько бы до этого он ни ждал. Чему равна вероятность того, что встреча состоится в первые полчаса?
 
\end{zkrW}

\begin{zkrW}{20}\noindent 
	В альбоме 8 чистых и 7 гашеных марок. Из альбома наудачу извлекаются 3 марки и подвергаются гашению, а затем возвращаются в альбом. После этого вновь наудачу извлекаются 4 марки. \\ \indent а) Найти вероятность того, что эти марки гашеные. \\ \indent б) Известно, что эти 4 марки гашеные; найти вероятность того, что первоначально извлеченные 3 марки --- чистые.
 
\end{zkrW}

\begin{zkrW}{20}\noindent 
	Предполагается, что $55\%$ открывающихся малых предприятий прекращает свою деятельность в течение года. Какова вероятность того, что из 7 малых предприятий по прошествии года продолжат работать менее чем 4?
 
\end{zkrW}

\begin{zkrW}{20}\noindent 
	В ралли принимает участие 2400 экипажей. Каждый экипаж может сойти с дистанции из-за технических неполадок с вероятностью $0{,}6$, а из-за болезни водителя --- с вероятностью $0{,}0025$. Найти вероятность того, что \\ \indent а) не менее чем 3 экипажей сойдут с дистанции из-за болезни водителя; \\ \indent б) ровно 1253 экипажей не смогут продолжать ралли из-за технических неполадок; \\ \indent в) от 1469 до 1498 экипажей пострадают от технических проблем.
 
\end{zkrW}

\newpage\setcounter{zad}{0}



\begin{zkrW}{20}\noindent 
	Детали проходят три операции обработки. Вероятность получения брака на первой операции равна $1/2$; на второй --- $7/8$; на третьей --- $3/8$. Найдите вероятность получения детали без брака после 5 операций, предполагая, что получения брака на отдельных операциях являются независимыми событиями.
 
\end{zkrW}

\begin{zkrW}{20}\noindent 
	Робин и Бэтмен условились встретиться в определенном месте между 16:00 и 20:00. Каждый из них может прийти в любое время в течение указанного промежутка и ждет второго некоторое время. Робин ждет 50 минут, после чего уходит; Бэтмен ждет 70 минут, после чего уходит. В 20:00 любой из них уходит, сколько бы до этого он ни ждал. Чему равна вероятность того, что Робин и Бэтмен встретятся?
 
\end{zkrW}

\begin{zkrW}{20}\noindent 
	В первом ящике 8 красных и 9 белых шаров, а во втором 10 красных и 6 белых. Из первого ящика во второй перекладываются 2 наудачу извлеченных шара. После этого из второго ящика наудачу извлекается один шар. \\ \indent а) Найти вероятность того, что он белый. \\ \indent б) Известно, что этот шар белый; найти вероятность того, что извлеченные из первого ящика шары --- красные.
 
\end{zkrW}

\begin{zkrW}{20}\noindent 
	В среднем $60\%$ акций на аукционах продаются по первоначально заявленной цене. Найти вероятность того, что из 6 пакетов акций в результате торгов по первоначально заявленной цене останутся непроданными по меньшей мере 3.
 
\end{zkrW}

\begin{zkrW}{20}\noindent 
	Мастерская за год ремонтирует 2400 мобильных телефонов. Вероятность неисправности в механической части отдельного телефона равна $0{,}6$, в электронной части --- $0{,}0025$. Найти вероятность того, что среди телефонов, отремонтированных за год, \\ \indent а) имели неисправности в механической части от 1382 до 1397 экземпляров; \\ \indent б) имели неисправности в электронной части менее чем 3 телефонов; \\ \indent в) ровно 1555 телефонов имели проблемы в механической части.
 
\end{zkrW}

\newpage\setcounter{zad}{0}



\begin{zkrW}{20}\noindent 
	Мастер обслуживает четыре станка, работающих независимо друг от друга. Вероятность того, что первыйй станок в течение смены потребует внимания мастера, равна $1/3$, второй --- $2/5$, третий --- $5/8$ и четвертый --- $4/5$. Найдите вероятность того, что в течение смены хотя бы один станок не потребует внимания мастера.
 
\end{zkrW}

\begin{zkrW}{20}\noindent 
	Надя и Аня условились встретиться в определенном месте между 11:00 и 12:00. Каждый из них может прийти в любое время в течение указанного промежутка и ждет второго некоторое время. Надя ждет 20 минут, после чего уходит; Аня ждет 50 минут, после чего уходит. В 12:00 любой из них уходит, сколько бы до этого он ни ждал. Чему равна вероятность того, что встреча состоится в первые полчаса?
 
\end{zkrW}

\begin{zkrW}{20}\noindent 
	В каждом из трех ящиков 7 белых и 5 зеленых шаров. Из первого и второго ящиков наудачу извлекается по одному шару и кладется в третий ящик. Затем из третьего ящика извлекается один шар. \\ \indent а) Найти вероятность того, что он зеленый. \\ \indent б) Известно, что этот шар зеленый; найти вероятность того, что шары, извлеченные из первого и второго ящиков, --- зеленые.
 
\end{zkrW}

\begin{zkrW}{20}\noindent 
	Вероятность правильного оформления доверенности у нотариуса Иванова-Ежова равна $6/7$. В течение одного часа нотариус Иванов-Ежов оформил 9 доверенности. Какова вероятность, что не менее чем 2 из них оказались оформлены неправильно?
 
\end{zkrW}

\begin{zkrW}{20}\noindent 
	Магазин закупил 600 телевизоров и столько же магнитол. Вероятность того, что отдельный телевизор окажется бракованным, равна $0{,}01$, а вероятность того, что магнитола окажется бракованной, --- $0{,}6$. Найти вероятность того, что в этой закупке \\ \indent а) ровно 4 телевизора окажутся бракованными; \\ \indent б) ровно 353 магнитол окажутся нерабочими; \\ \indent в) от 338 до 342 магнитол будут бракованными.
 
\end{zkrW}

\newpage\setcounter{zad}{0}



\begin{zkrW}{20}\noindent 
	Студент успел подготовить к экзамену 6 вопросов из 16. Какова вероятность того, что из 8 наудачу выбранных вопросов студент знает по крайней мере 2?
 
\end{zkrW}

\begin{zkrW}{20}\noindent 
	Аня и Света условились встретиться в определенном месте между 6:00 и 8:00. Каждый из них может прийти в любое время в течение указанного промежутка и ждет второго некоторое время. Аня ждет 40 минут, после чего уходит; Света ждет 60 минут, после чего уходит. В 8:00 любой из них уходит, сколько бы до этого он ни ждал. Чему равна вероятность того, что встреча состоится в первые двадцать минут?
 
\end{zkrW}

\begin{zkrW}{20}\noindent 
	В альбоме 8 чистых и 5 гашеных марок. Из альбома наудачу извлекаются 2 марки и подвергаются гашению, а затем возвращаются в альбом. После этого вновь наудачу извлекаются 3 марки. \\ \indent а) Найти вероятность того, что эти марки чистые. \\ \indent б) Известно, что эти 3 марки чистые; найти вероятность того, что первоначально извлеченные 2 марки --- гашеные.
 
\end{zkrW}

\begin{zkrW}{20}\noindent 
	В среднем $5\%$ акций на аукционах продаются по первоначально заявленной цене. Найти вероятность того, что из 6 пакетов акций в результате торгов по первоначально заявленной цене останутся непроданными по крайней мере 3.
 
\end{zkrW}

\begin{zkrW}{20}\noindent 
	В лотерее разыгрываются крупные и мелкие выигрыши. Вероятность того, что на лотерейный билет выпадет крупный выигрыш, равна $0{,}0028$, а мелкий --- $0{,}9$. Куплено 2500 билетов. Найти вероятность того, что \\ \indent а) крупных выигрышей будет хотя бы 5; \\ \indent б) мелких выигрышей будет ровно 2205; \\ \indent в) мелких выигрышей будет от 2205 до 2295.
 
\end{zkrW}

\newpage\setcounter{zad}{0}



\begin{zkrW}{20}\noindent 
	Экспедиция издательства отправила газеты в три почтовых отделения. Вероятность своевременной доставки газет в первое отделение равна $2/3$, во второе отделение --- $3/4$ и в третье --- $2/3$. Найдите вероятность того, что хотя бы одно отделение получит газеты с опозданием.
 
\end{zkrW}

\begin{zkrW}{20}\noindent 
	Дэйв Гилмор и Ян Пэйс условились встретиться в определенном месте между 8:00 и 10:00. Каждый из них может прийти в любое время в течение указанного промежутка и ждет второго некоторое время. Дэйв Гилмор ждет 40 минут, после чего уходит; Ян Пэйс ждет 60 минут, после чего уходит. В 10:00 любой из них уходит, сколько бы до этого он ни ждал. Чему равна вероятность того, что встреча состоится в первые двадцать минут?
 
\end{zkrW}

\begin{zkrW}{20}\noindent 
	В каждом из трех ящиков 7 желтых и 12 красных шаров. Из первого ящика в третий перекладывают два наудачу выбранных шара, а из второго ящика в третий перекладывают один наудачу взятый шар. Затем из третьего ящика извлекается один шар. \\ \indent а) Найти вероятность того, что он желтый. \\ \indent б) Известно, что этот шар желтый; найти вероятность того, что из первого ящика во второй переложили красные шары.
 
\end{zkrW}

\begin{zkrW}{20}\noindent 
	Опрошены 6 человек. Найти вероятность того, что не менее чем 4 из них родились осенью.
 
\end{zkrW}

\begin{zkrW}{20}\noindent 
	В ралли принимает участие 2400 экипажей. Каждый экипаж может сойти с дистанции из-за технических неполадок с вероятностью $0{,}4$, а из-за болезни водителя --- с вероятностью $0{,}0025$. Найти вероятность того, что \\ \indent а) менее чем 3 экипажей сойдут с дистанции из-за болезни водителя; \\ \indent б) ровно 845 экипажей не смогут продолжать ралли из-за технических неполадок; \\ \indent в) от 998 до 1027 экипажей пострадают от технических проблем.
 
\end{zkrW}

\newpage\setcounter{zad}{0}



\begin{zkrW}{20}\noindent 
	Произведено три выстрела по цели из орудия. Вероятность попадания при первом выстреле равна $0{,}7$; при втором --- $0{,}5$; при третьем --- $0{,}9$. Определить вероятность того, что будет хотя бы одно попадание.
 
\end{zkrW}

\begin{zkrW}{20}\noindent 
	Бэтмен и Робин условились встретиться в определенном месте между 4:00 и 6:00. Каждый из них может прийти в любое время в течение указанного промежутка и ждет второго некоторое время. Бэтмен ждет 60 минут, после чего уходит; Робин ждет 40 минут, после чего уходит. В 6:00 любой из них уходит, сколько бы до этого он ни ждал. Чему равна вероятность того, что встреча произойдет не ранее чем без четверти 6:00?
 
\end{zkrW}

\begin{zkrW}{20}\noindent 
	В каждом из трех ящиков 10 синих и 12 красных шаров. Из первого ящика в третий перекладывают два наудачу выбранных шара, а из второго ящика в третий перекладывают один наудачу взятый шар. Затем из третьего ящика извлекается один шар. \\ \indent а) Найти вероятность того, что он синий. \\ \indent б) Известно, что этот шар синий; найти вероятность того, что из первого ящика во второй переложили красные шары.
 
\end{zkrW}

\begin{zkrW}{20}\noindent 
	В Машбюро стоит 6 пишущих машин. Вероятность того, что каждая из них в течение года потребует ремонта, равна $0{,}2$. Найти вероятность того, что в течение года придется отремонтировать ровно 3 машины.
 
\end{zkrW}

\begin{zkrW}{20}\noindent 
	Вероятность выиграть отдельному игроку 1000 рублей в игре <<Кто хочет стать миллионером>> равна $0{,}75$, а 32000 рублей --- $0{,}005$. За сезон в этой игре принимает участие 1200 человек. Найти вероятность того, что за сезон \\ \indent а) 1000 рублей получат ровно 774 человек; \\ \indent б) 1000 рублей получат от 918 до 936 человек; \\ \indent в) более чем 4 человек получат крупный выигрыш в 32000 рублей.
 
\end{zkrW}

\newpage\setcounter{zad}{0}



\begin{zkrW}{20}\noindent 
	Детали проходят три операции обработки. Вероятность получения брака на первой операции равна $0{,}7$; на второй --- $0{,}4$; на третьей --- $0{,}3$. Найдите вероятность получения детали без брака после 6 операций, предполагая, что получения брака на отдельных операциях являются независимыми событиями.
 
\end{zkrW}

\begin{zkrW}{20}\noindent 
	Барак Обама и Владимир Путин условились встретиться в определенном месте между 14:00 и 15:00. Каждый из них может прийти в любое время в течение указанного промежутка и ждет второго некоторое время. Барак Обама ждет 50 минут, после чего уходит; Владимир Путин ждет 20 минут, после чего уходит. В 15:00 любой из них уходит, сколько бы до этого он ни ждал. Чему равна вероятность того, что Барак Обама и Владимир Путин встретятся?
 
\end{zkrW}

\begin{zkrW}{20}\noindent 
	В первом ящике 11 белых и 6 черных шаров, а во втором 12 белых и 10 черных. Из первого ящика во второй перекладываются 2 наудачу извлеченных шара. После этого из второго ящика наудачу извлекается один шар. \\ \indent а) Найти вероятность того, что он белый. \\ \indent б) Известно, что этот шар белый; найти вероятность того, что извлеченные из первого ящика шары --- черные.
 
\end{zkrW}

\begin{zkrW}{20}\noindent 
	Партия изделий содержит $75\%$ брака. Найти вероятность того, что среди взятых наугад 9 изделий окажется более чем 4 бракованных.
 
\end{zkrW}

\begin{zkrW}{20}\noindent 
	Предполагая рождение ребенка в любой день года равновозможным, найти вероятность того, что в группе из 400 человек \\ \indent а) по крайней мере 2 родились 18 сентября; \\ \indent б) ровно 342 родились осенью; \\ \indent в) от 367 до 374 родились весной.
 
\end{zkrW}

\newpage\setcounter{zad}{0}



\begin{zkrW}{20}\noindent 
	Абонент забыл последнюю цифру номера телефона и поэтому набирает ее наугад. Определить вероятность того, что ему придется звонить не более чем в 3 места.
 
\end{zkrW}

\begin{zkrW}{20}\noindent 
	Света и Саша условились встретиться в определенном месте между 9:00 и 12:00. Каждый из них может прийти в любое время в течение указанного промежутка и ждет второго некоторое время. Света ждет 50 минут, после чего уходит; Саша ждет 60 минут, после чего уходит. В 12:00 любой из них уходит, сколько бы до этого он ни ждал. Чему равна вероятность того, что Света и Саша встретятся?
 
\end{zkrW}

\begin{zkrW}{20}\noindent 
	В альбоме 9 чистых и 8 гашеных марок. Из альбома изымаются 3 наудачу извлеченные марки. После этого из альбома вновь наудачу извлекаются 3 марки. \\ \indent а) Найти вероятность того, что эти марки чистые. \\ \indent б) Известно, что эти 3 марки чистые; найти вероятность того, что первоначально изъятые 3 марки --- гашеные.
 
\end{zkrW}

\begin{zkrW}{20}\noindent 
	Для баскетболиста дяди Стёпы вероятность забросить мяч в корзину равна $0{,}7$. Он выполняет 8 бросков. Какова вероятность, что в корзину попадут ровно 3 мяча?
 
\end{zkrW}

\begin{zkrW}{20}\noindent 
	Стрелок попадает в цель из пистолета с вероятностью $0{,}4$, а из снайперской винтовки --- с вероятностью $0{,}9984$. Найти вероятность того, что, сделав 3750 выстрелов по цели из каждого оружия, стрелок \\ \indent а) промахнется из пистолета от 1425 до 1575 раз; \\ \indent б) промахнется из пистолета ровно 1695 раз; \\ \indent в) допустит более чем 4 промаха из снайперской винтовки.
 
\end{zkrW}

\newpage\setcounter{zad}{0}



\begin{zkrW}{20}\noindent 
	Абонент забыл последнюю цифру номера телефона и поэтому набирает ее наугад. Определить вероятность того, что ему придется звонить по меньшей мере в 4 места.
 
\end{zkrW}

\begin{zkrW}{20}\noindent 
	Саша и Вася условились встретиться в определенном месте между 12:00 и 15:00. Каждый из них может прийти в любое время в течение указанного промежутка и ждет второго некоторое время. Саша ждет 50 минут, после чего уходит; Вася ждет 40 минут, после чего уходит. В 15:00 любой из них уходит, сколько бы до этого он ни ждал. Чему равна вероятность того, что встреча состоится в первые двадцать минут?
 
\end{zkrW}

\begin{zkrW}{20}\noindent 
	В первом ящике 5 красных и 10 черных шаров, а во втором 10 красных и 8 черных. Из первого ящика во второй перекладываются 3 наудачу извлеченных шара. После этого из второго ящика наудачу извлекается один шар. \\ \indent а) Найти вероятность того, что он красный. \\ \indent б) Известно, что этот шар красный; найти вероятность того, что извлеченные из первого ящика шары --- черные.
 
\end{zkrW}

\begin{zkrW}{20}\noindent 
	В магазин вошли 7 покупателей. Найти вероятность того, что ровно 2 из них совершат покупки, если вероятность совершить покупку для каждого из них одинакова и равна $2/3$.
 
\end{zkrW}

\begin{zkrW}{20}\noindent 
	Мастерская за год ремонтирует 3750 мобильных телефонов. Вероятность неисправности в механической части отдельного телефона равна $0{,}6$, в электронной части --- $0{,}0024$. Найти вероятность того, что среди телефонов, отремонтированных за год, \\ \indent а) имели неисправности в механической части от 2205 до 2295 экземпляров; \\ \indent б) имели неисправности в электронной части не более чем 3 телефонов; \\ \indent в) ровно 2520 телефонов имели проблемы в механической части.
 
\end{zkrW}

\newpage\setcounter{zad}{0}

>>>>>>> 29810b41e83a98906e8370d188012efc29b7ef7f
