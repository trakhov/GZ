

\begin{zkrW}{20}\noindent 
	Три студента пришли сдавать экзамен. Вероятность того, что первый студент сдаст экзамен, равна $0{,}1$, второй --- $0{,}8$, третий --- $0{,}4$. Найдите вероятность того, что хотя бы двое сдадут экзамен.
 
\end{zkrW}

\begin{zkrW}{20}\noindent 
	Леонтий и Пелагия условились встретиться в определенном месте между 16:00 и 18:00. Каждый из них может прийти в любое время в течение указанного промежутка и ждет второго некоторое время. Леонтий ждет 30 минут, после чего уходит; Пелагия ждет 60 минут, после чего уходит. В 18:00 любой из них уходит, сколько бы до этого он ни ждал. Чему равна вероятность того, что Леонтий придет раньше, чем Пелагия?
 
\end{zkrW}

\begin{zkrW}{20}\noindent 
	В альбоме 10 чистых и 6 гашеных марок. Из альбома изымаются 3 наудачу извлеченные марки. После этого из альбома вновь наудачу извлекаются 4 марки. \\ \indent а) Найти вероятность того, что эти марки гашеные. \\ \indent б) Известно, что эти 4 марки гашеные; найти вероятность того, что первоначально изъятые 3 марки --- чистые.
 
\end{zkrW}

\begin{zkrW}{20}\noindent 
	При данном технологическом процессе $55\%$ всех сходящих с конвейера автозавода автомобилей имеют цвет «металлик». Найти вероятность того, что из 6 случайно отобранных автомобилей менее чем 4 будут иметь этот цвет.
 
\end{zkrW}

\begin{zkrW}{20}\noindent 
	Вероятность выиграть отдельному игроку 1000 рублей в игре <<Кто хочет стать миллионером>> равна $0{,}1$, а 32000 рублей --- $0{,}0025$. За сезон в этой игре принимает участие 1600 человек. Найти вероятность того, что за сезон \\ \indent а) 1000 рублей получат ровно 168 человек; \\ \indent б) 1000 рублей получат от 141 до 166 человек; \\ \indent в) не более чем 5 человек получат крупный выигрыш в 32000 рублей.
 
\end{zkrW}

\newpage\setcounter{zad}{0}



\begin{zkrW}{20}\noindent 
	В коробке 5 красных, 6 синих и 8 желтых карандашей. Наудачу вынимают три карандаша. Какова вероятность того, что они все разных цветов?
 
\end{zkrW}

\begin{zkrW}{20}\noindent 
	Света и Андрей условились встретиться в определенном месте между 20:00 и 24:00. Каждый из них может прийти в любое время в течение указанного промежутка и ждет второго некоторое время. Света ждет 70 минут, после чего уходит; Андрей ждет 60 минут, после чего уходит. В 24:00 любой из них уходит, сколько бы до этого он ни ждал. Чему равна вероятность того, что встреча состоится в первые двадцать минут?
 
\end{zkrW}

\begin{zkrW}{20}\noindent 
	В альбоме 11 чистых и 8 гашеных марок. Из альбома наудачу извлекаются 3 марки и подвергаются гашению, а затем возвращаются в альбом. После этого вновь наудачу извлекаются 4 марки. \\ \indent а) Найти вероятность того, что эти марки гашеные. \\ \indent б) Известно, что эти 4 марки гашеные; найти вероятность того, что первоначально извлеченные 3 марки --- чистые.
 
\end{zkrW}

\begin{zkrW}{20}\noindent 
	В Машбюро стоит 7 пишущих машин. Вероятность того, что каждая из них в течение года потребует ремонта, равна $2/9$. Найти вероятность того, что в течение года придется отремонтировать ровно 3 машины.
 
\end{zkrW}

\begin{zkrW}{20}\noindent 
	Магазин закупил 1600 телевизоров и столько же магнитол. Вероятность того, что отдельный телевизор окажется бракованным, равна $0{,}0025$, а вероятность того, что магнитола окажется бракованной, --- $0{,}2$. Найти вероятность того, что в этой закупке \\ \indent а) менее чем 2 телевизора окажутся бракованными; \\ \indent б) ровно 355 магнитол окажутся нерабочими; \\ \indent в) от 285 до 301 магнитол будут бракованными.
 
\end{zkrW}

\newpage\setcounter{zad}{0}



\begin{zkrW}{20}\noindent 
	Игра проводится до выигрыша одним из двух игроков двух партий подряд (ничьи исключаются). Вероятность выигрыша партии каждым из игроков равна $0{,}5$ и не зависит от исходов предыдущих партий. Найдите вероятность того, что игра окончится до 4-й партии.
 
\end{zkrW}

\begin{zkrW}{20}\noindent 
	Карик и Валя условились встретиться в определенном месте между 04:00 и 08:00. Каждый из них может прийти в любое время в течение указанного промежутка и ждет второго некоторое время. Карик ждет 60 минут, после чего уходит; Валя ждет 80 минут, после чего уходит. В 08:00 любой из них уходит, сколько бы до этого он ни ждал. Чему равна вероятность того, что встреча произойдет не ранее чем без четверти 08:00?
 
\end{zkrW}

\begin{zkrW}{20}\noindent 
	В первом ящике 8 красных и 11 зеленых шаров, а во втором 11 красных и 12 зеленых. Из первого ящика во второй перекладываются 2 наудачу извлеченных шара. После этого из второго ящика наудачу извлекается один шар. \\ \indent а) Найти вероятность того, что он красный. \\ \indent б) Известно, что этот шар красный; найти вероятность того, что извлеченные из первого ящика шары --- красные.
 
\end{zkrW}

\begin{zkrW}{20}\noindent 
	При массовом производстве полупроводниковых диодов вероятность брака при формовке равна $8/9$. Какова вероятность того, что из 5 взятых диодов будет не менее чем 4 бракованных.
 
\end{zkrW}

\begin{zkrW}{20}\noindent 
	Магазин закупил 2400 телевизоров и столько же магнитол. Вероятность того, что отдельный телевизор окажется бракованным, равна $0{,}0025$, а вероятность того, что магнитола окажется бракованной, --- $0{,}6$. Найти вероятность того, что в этой закупке \\ \indent а) ровно 5 телевизора окажутся бракованными; \\ \indent б) ровно 1325 магнитол окажутся нерабочими; \\ \indent в) от 1397 до 1469 магнитол будут бракованными.
 
\end{zkrW}

\newpage\setcounter{zad}{0}



\begin{zkrW}{20}\noindent 
	Во время тренировки три баскетболиста бросают мячи в корзину. Вероятность попадания первого равна $1/2$, второго --- $5/9$, третьего --- $1/2$. Каждый баскетболист делает один бросок. Найдите вероятность хотя бы одного попадания мяча в корзину.
 
\end{zkrW}

\begin{zkrW}{20}\noindent 
	Петя и Аня условились встретиться в определенном месте между 01:00 и 05:00. Каждый из них может прийти в любое время в течение указанного промежутка и ждет второго некоторое время. Петя ждет 80 минут, после чего уходит; Аня ждет 50 минут, после чего уходит. В 05:00 любой из них уходит, сколько бы до этого он ни ждал. Чему равна вероятность того, что Петя придет раньше, чем Аня?
 
\end{zkrW}

\begin{zkrW}{20}\noindent 
	В каждом из трех ящиков 8 зеленых и 5 черных шаров. Из первого и второго ящиков наудачу извлекается по одному шару и кладется в третий ящик. Затем из третьего ящика извлекается один шар. \\ \indent а) Найти вероятность того, что он черный. \\ \indent б) Известно, что этот шар черный; найти вероятность того, что шары, извлеченные из первого и второго ящиков, --- черные.
 
\end{zkrW}

\begin{zkrW}{20}\noindent 
	При данном технологическом процессе $85\%$ всех сходящих с конвейера автозавода автомобилей имеют цвет «металлик». Найти вероятность того, что из 7 случайно отобранных автомобилей ровно 3 будут иметь этот цвет.
 
\end{zkrW}

\begin{zkrW}{20}\noindent 
	Студент за все время обучения в вузе в среднем выполняет 2500 задач по математике. Вероятность неверно решить отдельную задачу при условии стопроцентного посещения и активной работы на всех занятиях равна $0{,}0036$, в противном случае --- $0{,}9$. Найти вероятность того, что за время обучения в вузе \\ \indent а) абсолютно прилежный студент решил неверно по меньшей мере 5 задачи; \\ \indent б) обычный студент решил правильно ровно 2047 задач; \\ \indent в) обычный студент неверно решил от 2205 до 2295 задач.
 
\end{zkrW}

\newpage\setcounter{zad}{0}



\begin{zkrW}{20}\noindent 
	Два стрелка сделали по одному выстрелу по мишени. Известно, что вероятность попадания в мишень для одного из стрелков равна $3/5$, а для другого --- $1/8$. Найдите вероятность того, что хотя бы один из стрелков попадет в мишень.
 
\end{zkrW}

\begin{zkrW}{20}\noindent 
	Кит Ричардс и Ян Пэйс условились встретиться в определенном месте между 03:00 и 06:00. Каждый из них может прийти в любое время в течение указанного промежутка и ждет второго некоторое время. Кит Ричардс ждет 40 минут, после чего уходит; Ян Пэйс ждет 60 минут, после чего уходит. В 06:00 любой из них уходит, сколько бы до этого он ни ждал. Чему равна вероятность того, что встреча состоится в первые двадцать минут?
 
\end{zkrW}

\begin{zkrW}{20}\noindent 
	В альбоме 5 чистых и 10 гашеных марок. Из альбома наудачу извлекаются 2 марки и заменяются на чистые. После этого вновь наудачу извлекаются 5 марки. \\ \indent а) Найти вероятность того, что эти марки гашеные. \\ \indent б) Известно, что эти 5 марки гашеные; найти вероятность того, что первоначально извлеченные 2 марки --- чистые.
 
\end{zkrW}

\begin{zkrW}{20}\noindent 
	Всхожесть семян данного сорта растений оценивается с вероятностью, равной $5/7$. Какова вероятность того, что из 8 посеянных семян взойдут не менее чем 2?
 
\end{zkrW}

\begin{zkrW}{20}\noindent 
	Стрелок попадает в цель из пистолета с вероятностью $0{,}6$, а из снайперской винтовки --- с вероятностью $0{,}99$. Найти вероятность того, что, сделав 600 выстрелов по цели из каждого оружия, стрелок \\ \indent а) промахнется из пистолета от 342 до 371 раз; \\ \indent б) промахнется из пистолета ровно 338 раз; \\ \indent в) допустит менее чем 5 промаха из снайперской винтовки.
 
\end{zkrW}

\newpage\setcounter{zad}{0}



\begin{zkrW}{20}\noindent 
	В коробке 13 красных, 8 синих и 7 желтых карандашей. Наудачу вынимают три карандаша. Какова вероятность того, что они все разных цветов?
 
\end{zkrW}

\begin{zkrW}{20}\noindent 
	Пелагия и Архип условились встретиться в определенном месте между 06:00 и 07:00. Каждый из них может прийти в любое время в течение указанного промежутка и ждет второго некоторое время. Пелагия ждет 50 минут, после чего уходит; Архип ждет 20 минут, после чего уходит. В 07:00 любой из них уходит, сколько бы до этого он ни ждал. Чему равна вероятность того, что встреча состоится в первые двадцать минут?
 
\end{zkrW}

\begin{zkrW}{20}\noindent 
	В каждом из трех ящиков 9 зеленых и 7 синих шаров. Из первого и второго ящиков наудачу извлекается по одному шару и кладется в третий ящик. Затем из третьего ящика извлекается один шар. \\ \indent а) Найти вероятность того, что он синий. \\ \indent б) Известно, что этот шар синий; найти вероятность того, что шары, извлеченные из первого и второго ящиков, --- синие.
 
\end{zkrW}

\begin{zkrW}{20}\noindent 
	В ячейку памяти ЭВМ записывается двоичное число длиной в 5 разрядов. Значения 0 и 1 в каждом разряде появляются с равной вероятностью. Найти вероятность того, что в этом двоичном числе не менее чем 4 единицы.
 
\end{zkrW}

\begin{zkrW}{20}\noindent 
	Магазин закупил 100 телевизоров и столько же магнитол. Вероятность того, что отдельный телевизор окажется бракованным, равна $0{,}03$, а вероятность того, что магнитола окажется бракованной, --- $0{,}2$. Найти вероятность того, что в этой закупке \\ \indent а) ровно 4 телевизора окажутся бракованными; \\ \indent б) ровно 18 магнитол окажутся нерабочими; \\ \indent в) от 17 до 21 магнитол будут бракованными.
 
\end{zkrW}

\newpage\setcounter{zad}{0}



\begin{zkrW}{20}\noindent 
	Экзаменационный билет содержит три вопроса. Вероятности того, что студент ответит на первый и второй вопросы билета равны $0{,}6$; на третий --- $0{,}3$. Найдите вероятность того, что студент сдаст экзамен, если для этого необходимо ответить хотя бы на два вопроса.
 
\end{zkrW}

\begin{zkrW}{20}\noindent 
	Леонардо и Микеланджело условились встретиться в определенном месте между 01:00 и 02:00. Каждый из них может прийти в любое время в течение указанного промежутка и ждет второго некоторое время. Леонардо ждет 20 минут, после чего уходит; Микеланджело ждет 50 минут, после чего уходит. В 02:00 любой из них уходит, сколько бы до этого он ни ждал. Чему равна вероятность того, что Леонардо опоздает более чем на полчаса?
 
\end{zkrW}

\begin{zkrW}{20}\noindent 
	В альбоме 8 чистых и 11 гашеных марок. Из альбома наудачу извлекаются 3 марки и подвергаются гашению, а затем возвращаются в альбом. После этого вновь наудачу извлекаются 5 марки. \\ \indent а) Найти вероятность того, что эти марки гашеные. \\ \indent б) Известно, что эти 5 марки гашеные; найти вероятность того, что первоначально извлеченные 3 марки --- чистые.
 
\end{zkrW}

\begin{zkrW}{20}\noindent 
	По новым правилам в волейбольном матче игра происходит до тех пор, пока одна из команд не выиграет 4 партии. Вероятность победы российской сборной в каждой партии равна $1/6$. Определить вероятность того, что в ближайшем матче сборная России победит со счетом 4:2.
 
\end{zkrW}

\begin{zkrW}{20}\noindent 
	Магазин закупил 2500 телевизоров и столько же магнитол. Вероятность того, что отдельный телевизор окажется бракованным, равна $0{,}002$, а вероятность того, что магнитола окажется бракованной, --- $0{,}9$. Найти вероятность того, что в этой закупке \\ \indent а) менее чем 3 телевизора окажутся бракованными; \\ \indent б) ровно 2295 магнитол окажутся нерабочими; \\ \indent в) от 2205 до 2295 магнитол будут бракованными.
 
\end{zkrW}

\newpage\setcounter{zad}{0}



\begin{zkrW}{20}\noindent 
	Студент успел подготовить к экзамену 7 вопросов из 15 пяти. Какова вероятность того, что из 7 наудачу выбранных вопросов студент знает не более чем 4.
 
\end{zkrW}

\begin{zkrW}{20}\noindent 
	Бэтмен и Робин условились встретиться в определенном месте между 07:00 и 10:00. Каждый из них может прийти в любое время в течение указанного промежутка и ждет второго некоторое время. Бэтмен ждет 40 минут, после чего уходит; Робин ждет 60 минут, после чего уходит. В 10:00 любой из них уходит, сколько бы до этого он ни ждал. Чему равна вероятность того, что Бэтмен опоздает менее чем на полчаса?
 
\end{zkrW}

\begin{zkrW}{20}\noindent 
	В первом ящике 6 синих и 6 голубых шаров, а во втором 12 синих и 10 голубых. Из первого ящика во второй перекладываются 4 наудачу извлеченных шара. После этого из второго ящика наудачу извлекается один шар. \\ \indent а) Найти вероятность того, что он голубой. \\ \indent б) Известно, что этот шар голубой; найти вероятность того, что извлеченные из первого ящика шары --- голубые.
 
\end{zkrW}

\begin{zkrW}{20}\noindent 
	Вероятность выигрыша по облигации займа за все время его действия равна $0{,}9$. Найти вероятность того, что при покупке 6 облигаций удастся выиграть хотя бы по 4 из них.
 
\end{zkrW}

\begin{zkrW}{20}\noindent 
	Вероятность появления опечатки на отдельной странице книги равна $0{,}0225$, а погрешности верстки --- $0{,}8$. Найти вероятность того, что в книге из 400 страниц \\ \indent а) не менее чем 5 страниц будут иметь опечатки; \\ \indent б) от 326 до 342 страниц будут иметь погрешности верстки; \\ \indent в) погрешности верстки будут присутствовать ровно на 310 страницах.
 
\end{zkrW}

\newpage\setcounter{zad}{0}



\begin{zkrW}{20}\noindent 
	Прибор, работающий в течение времени $t$, состоит из трех узлов, каждый из которых независимо от других может за это время выйти из строя. Неисправность хотя бы одного узла выводит прибор из строя целиком. Вероятность безотказной работы в течение времени $t$ первого узла равна $0{,}3$, второго --- $0{,}4$, третьего --- $0{,}4$. Найдите вероятность того, что в течение времени $t$ прибор выйдет из строя.
 
\end{zkrW}

\begin{zkrW}{20}\noindent 
	Алексей и Вася условились встретиться в определенном месте между 20:00 и 21:00. Каждый из них может прийти в любое время в течение указанного промежутка и ждет второго некоторое время. Алексей ждет 30 минут, после чего уходит; Вася ждет 40 минут, после чего уходит. В 21:00 любой из них уходит, сколько бы до этого он ни ждал. Чему равна вероятность того, что встреча произойдет не ранее чем без четверти 21:00?
 
\end{zkrW}

\begin{zkrW}{20}\noindent 
	В альбоме 11 чистых и 12 гашеных марок. Из альбома наудачу извлекаются 3 марки и заменяются на чистые. После этого вновь наудачу извлекаются 2 марки. \\ \indent а) Найти вероятность того, что эти марки чистые. \\ \indent б) Известно, что эти 2 марки чистые; найти вероятность того, что первоначально извлеченные 3 марки --- гашеные.
 
\end{zkrW}

\begin{zkrW}{20}\noindent 
	По новым правилам в волейбольном матче игра происходит до тех пор, пока одна из команд не выиграет 5 партии. Вероятность победы российской сборной в каждой партии равна $0{,}8$. Определить вероятность того, что в ближайшем матче сборная России победит со счетом 5:1.
 
\end{zkrW}

\begin{zkrW}{20}\noindent 
	Известно, что левши среди населения Островов Крокодилова Клюва составляют в среднем $0{,}5\%$, а люди, одинаково владеющие левой и правой рукой, --- $0{,}9$ (остальные --- правши). Найти вероятность того, что среди 1600 людей \\ \indent а) окажется не менее чем 2 левшей; \\ \indent б) окажется ровно 1253 амбидекстров\footnote{людей, одинаково владеющих обеими руками}; \\ \indent в) окажется от 1411 до 1469 амбидекстров.
 
\end{zkrW}

\newpage\setcounter{zad}{0}



\begin{zkrW}{20}\noindent 
	Вероятность попадания в цель при одном выстреле равна $1/8$ и с каждым выстрелом уменьшается на одну десятую от первоначальной. Произведено 5 выстрелов. Найдите вероятность поражения цели, если для этого достаточно хотя бы одного попадания.
 
\end{zkrW}

\begin{zkrW}{20}\noindent 
	Саша и Света условились встретиться в определенном месте между 15:00 и 18:00. Каждый из них может прийти в любое время в течение указанного промежутка и ждет второго некоторое время. Саша ждет 70 минут, после чего уходит; Света ждет 60 минут, после чего уходит. В 18:00 любой из них уходит, сколько бы до этого он ни ждал. Чему равна вероятность того, что встреча состоится в последние полчаса?
 
\end{zkrW}

\begin{zkrW}{20}\noindent 
	В альбоме 7 чистых и 10 гашеных марок. Из альбома наудачу извлекаются 3 марки и подвергаются гашению, а затем возвращаются в альбом. После этого вновь наудачу извлекаются 4 марки. \\ \indent а) Найти вероятность того, что эти марки чистые. \\ \indent б) Известно, что эти 4 марки чистые; найти вероятность того, что первоначально извлеченные 3 марки --- гашеные.
 
\end{zkrW}

\begin{zkrW}{20}\noindent 
	Всхожесть семян лимона равна $65\%$. Найти вероятность того, что из 5 посеянных семян взойдут по крайней мере 4.
 
\end{zkrW}

\begin{zkrW}{20}\noindent 
	Вероятность появления опечатки на отдельной странице книги равна $0{,}03$, а погрешности верстки --- $0{,}5$. Найти вероятность того, что в книге из 100 страниц \\ \indent а) по меньшей мере 4 страниц будут иметь опечатки; \\ \indent б) от 43 до 56 страниц будут иметь погрешности верстки; \\ \indent в) погрешности верстки будут присутствовать ровно на 56 страницах.
 
\end{zkrW}

\newpage\setcounter{zad}{0}



\begin{zkrW}{20}\noindent 
	Прибор, работающий в течение времени $t$, состоит из трех узлов, каждый из которых независимо от других может за это время выйти из строя. Неисправность хотя бы одного узла выводит прибор из строя целиком. Вероятность безотказной работы в течение времени $t$ первого узла равна $1/5$, второго --- $5/6$, третьего --- $6/7$. Найдите вероятность того, что в течение времени $t$ прибор выйдет из строя.
 
\end{zkrW}

\begin{zkrW}{20}\noindent 
	Джон Фаулз и Габриэль Гарсия Маркес условились встретиться в определенном месте между 17:00 и 19:00. Каждый из них может прийти в любое время в течение указанного промежутка и ждет второго некоторое время. Джон Фаулз ждет 30 минут, после чего уходит; Габриэль Гарсия Маркес ждет 50 минут, после чего уходит. В 19:00 любой из них уходит, сколько бы до этого он ни ждал. Чему равна вероятность того, что Джон Фаулз опоздает более чем на полчаса?
 
\end{zkrW}

\begin{zkrW}{20}\noindent 
	В альбоме 10 чистых и 5 гашеных марок. Из альбома наудачу извлекаются 2 марки и подвергаются гашению, а затем возвращаются в альбом. После этого вновь наудачу извлекаются 3 марки. \\ \indent а) Найти вероятность того, что эти марки гашеные. \\ \indent б) Известно, что эти 3 марки гашеные; найти вероятность того, что первоначально извлеченные 2 марки --- чистые.
 
\end{zkrW}

\begin{zkrW}{20}\noindent 
	По новым правилам в волейбольном матче игра происходит до тех пор, пока одна из команд не выиграет 3 партии. Вероятность победы российской сборной в каждой партии равна $0{,}9$. Определить вероятность того, что в ближайшем матче сборная России победит со счетом 3:2.
 
\end{zkrW}

\begin{zkrW}{20}\noindent 
	Магазин закупил 3600 телевизоров и столько же магнитол. Вероятность того, что отдельный телевизор окажется бракованным, равна $0{,}0025$, а вероятность того, что магнитола окажется бракованной, --- $0{,}8$. Найти вероятность того, что в этой закупке \\ \indent а) не более чем 2 телевизора окажутся бракованными; \\ \indent б) ровно 3168 магнитол окажутся нерабочими; \\ \indent в) от 2822 до 2938 магнитол будут бракованными.
 
\end{zkrW}

\newpage\setcounter{zad}{0}



\begin{zkrW}{20}\noindent 
	Экзаменационный билет содержит три вопроса. Вероятности того, что студент ответит на первый и второй вопросы билета равны $2/9$; на третий --- $1/2$. Найдите вероятность того, что студент сдаст экзамен, если для этого необходимо ответить хотя бы на два вопроса.
 
\end{zkrW}

\begin{zkrW}{20}\noindent 
	Атос и д'Артаньян условились встретиться в определенном месте между 05:00 и 09:00. Каждый из них может прийти в любое время в течение указанного промежутка и ждет второго некоторое время. Атос ждет 70 минут, после чего уходит; д'Артаньян ждет 50 минут, после чего уходит. В 09:00 любой из них уходит, сколько бы до этого он ни ждал. Чему равна вероятность того, что встреча состоится в последние полчаса?
 
\end{zkrW}

\begin{zkrW}{20}\noindent 
	В каждом из трех ящиков 7 белых и 11 красных шаров. Из первого ящика в третий перекладывают два наудачу выбранных шара, а из второго ящика в третий перекладывают один наудачу взятый шар. Затем из третьего ящика извлекается один шар. \\ \indent а) Найти вероятность того, что он белый. \\ \indent б) Известно, что этот шар белый; найти вероятность того, что из первого ящика во второй переложили белые шары.
 
\end{zkrW}

\begin{zkrW}{20}\noindent 
	Вероятность выигрыша по облигации займа за все время его действия равна $0{,}2$. Найти вероятность того, что при покупке 8 облигаций удастся выиграть менее чем по 4 из них.
 
\end{zkrW}

\begin{zkrW}{20}\noindent 
	В ралли принимает участие 2500 экипажей. Каждый экипаж может сойти с дистанции из-за технических неполадок с вероятностью $0{,}9$, а из-за болезни водителя --- с вероятностью $0{,}0032$. Найти вероятность того, что \\ \indent а) не более чем 4 экипажей сойдут с дистанции из-за болезни водителя; \\ \indent б) ровно 2160 экипажей не смогут продолжать ралли из-за технических неполадок; \\ \indent в) от 2205 до 2295 экипажей пострадают от технических проблем.
 
\end{zkrW}

\newpage\setcounter{zad}{0}



\begin{zkrW}{20}\noindent 
	Игра проводится до выигрыша одним из двух игроков двух партий подряд (ничьи исключаются). Вероятность выигрыша партии каждым из игроков равна $0{,}5$ и не зависит от исходов предыдущих партий. Найдите вероятность того, что игра окончится до 3-й партии.
 
\end{zkrW}

\begin{zkrW}{20}\noindent 
	Донателло и Микеланджело условились встретиться в определенном месте между 18:00 и 21:00. Каждый из них может прийти в любое время в течение указанного промежутка и ждет второго некоторое время. Донателло ждет 40 минут, после чего уходит; Микеланджело ждет 60 минут, после чего уходит. В 21:00 любой из них уходит, сколько бы до этого он ни ждал. Чему равна вероятность того, что встреча состоится в первые двадцать минут?
 
\end{zkrW}

\begin{zkrW}{20}\noindent 
	В каждом из трех ящиков 5 синих и 11 желтых шаров. Из первого и второго ящиков наудачу извлекается по одному шару и кладется в третий ящик. Затем из третьего ящика извлекается один шар. \\ \indent а) Найти вероятность того, что он синий. \\ \indent б) Известно, что этот шар синий; найти вероятность того, что шары, извлеченные из первого и второго ящиков, --- синие.
 
\end{zkrW}

\begin{zkrW}{20}\noindent 
	Каждый из 7 станков в течение 7 рабочих часов останавливается несколько раз и всего в сумме стоит один час, причем остановка его в любой момент времени равновероятна. Найти вероятность того, что в данный момент времени будут работать не более чем 3 станка.
 
\end{zkrW}

\begin{zkrW}{20}\noindent 
	Вероятность выиграть отдельному игроку 1000 рублей в игре <<Кто хочет стать миллионером>> равна $0{,}9$, а 32000 рублей --- $0{,}0025$. За сезон в этой игре принимает участие 1600 человек. Найти вероятность того, что за сезон \\ \indent а) 1000 рублей получат ровно 1368 человек; \\ \indent б) 1000 рублей получат от 1411 до 1469 человек; \\ \indent в) ровно 3 человек получат крупный выигрыш в 32000 рублей.
 
\end{zkrW}

\newpage\setcounter{zad}{0}



\begin{zkrW}{20}\noindent 
	Игра проводится до выигрыша одним из двух игроков двух партий подряд (ничьи исключаются). Вероятность выигрыша партии каждым из игроков равна $0{,}5$ и не зависит от исходов предыдущих партий. Найдите вероятность того, что игра окончится до 3-й партии.
 
\end{zkrW}

\begin{zkrW}{20}\noindent 
	Вася и Аня условились встретиться в определенном месте между 20:00 и 24:00. Каждый из них может прийти в любое время в течение указанного промежутка и ждет второго некоторое время. Вася ждет 50 минут, после чего уходит; Аня ждет 80 минут, после чего уходит. В 24:00 любой из них уходит, сколько бы до этого он ни ждал. Чему равна вероятность того, что встреча произойдет не ранее чем без четверти 24:00?
 
\end{zkrW}

\begin{zkrW}{20}\noindent 
	В альбоме 12 чистых и 7 гашеных марок. Из альбома изымаются 4 наудачу извлеченные марки. После этого из альбома вновь наудачу извлекаются 3 марки. \\ \indent а) Найти вероятность того, что эти марки гашеные. \\ \indent б) Известно, что эти 3 марки гашеные; найти вероятность того, что первоначально изъятые 4 марки --- чистые.
 
\end{zkrW}

\begin{zkrW}{20}\noindent 
	При массовом производстве полупроводниковых диодов вероятность брака при формовке равна $0{,}9$. Какова вероятность того, что из 6 взятых диодов будет хотя бы 3 бракованных.
 
\end{zkrW}

\begin{zkrW}{20}\noindent 
	Вероятность появления опечатки на отдельной странице книги равна $0{,}001$, а погрешности верстки --- $0{,}4$. Найти вероятность того, что в книге из 600 страниц \\ \indent а) не более чем 4 страниц будут иметь опечатки; \\ \indent б) от 209 до 264 страниц будут иметь погрешности верстки; \\ \indent в) погрешности верстки будут присутствовать ровно на 223 страницах.
 
\end{zkrW}

\newpage\setcounter{zad}{0}



\begin{zkrW}{20}\noindent 
	Среди 22 поступающих в ремонт часов 5 нуждаются в общей чистке механизма. Какова вероятность того, что среди взятых одновремено наудачу 8 часов более чем 4 нуждаются в общей чистке механизма?
 
\end{zkrW}

\begin{zkrW}{20}\noindent 
	Степан Степаныч Пивораки и Женя Богорад условились встретиться в определенном месте между 13:00 и 16:00. Каждый из них может прийти в любое время в течение указанного промежутка и ждет второго некоторое время. Степан Степаныч Пивораки ждет 40 минут, после чего уходит; Женя Богорад ждет 60 минут, после чего уходит. В 16:00 любой из них уходит, сколько бы до этого он ни ждал. Чему равна вероятность того, что встреча состоится в первые полчаса?
 
\end{zkrW}

\begin{zkrW}{20}\noindent 
	В альбоме 5 чистых и 5 гашеных марок. Из альбома изымаются 3 наудачу извлеченные марки. После этого из альбома вновь наудачу извлекаются 2 марки. \\ \indent а) Найти вероятность того, что эти марки гашеные. \\ \indent б) Известно, что эти 2 марки гашеные; найти вероятность того, что первоначально изъятые 3 марки --- чистые.
 
\end{zkrW}

\begin{zkrW}{20}\noindent 
	Партия изделий содержит $5\%$ брака. Найти вероятность того, что среди взятых наугад 6 изделий окажется не менее чем 2 бракованных.
 
\end{zkrW}

\begin{zkrW}{20}\noindent 
	Магазин закупил 600 телевизоров и столько же магнитол. Вероятность того, что отдельный телевизор окажется бракованным, равна $0{,}001$, а вероятность того, что магнитола окажется бракованной, --- $0{,}4$. Найти вероятность того, что в этой закупке \\ \indent а) более чем 4 телевизора окажутся бракованными; \\ \indent б) ровно 252 магнитол окажутся нерабочими; \\ \indent в) от 259 до 262 магнитол будут бракованными.
 
\end{zkrW}

\newpage\setcounter{zad}{0}



\begin{zkrW}{20}\noindent 
	Три лыжника съезжают с горы. Вероятность падения первого лыжника равна $0{,}8$, второго --- $0{,}3$, третьего --- $0{,}9$. Найдите вероятность того, что хотя бы два лыжника не упадут.
 
\end{zkrW}

\begin{zkrW}{20}\noindent 
	Алексей и Аня условились встретиться в определенном месте между 13:00 и 17:00. Каждый из них может прийти в любое время в течение указанного промежутка и ждет второго некоторое время. Алексей ждет 70 минут, после чего уходит; Аня ждет 80 минут, после чего уходит. В 17:00 любой из них уходит, сколько бы до этого он ни ждал. Чему равна вероятность того, что встреча состоится в первые полчаса?
 
\end{zkrW}

\begin{zkrW}{20}\noindent 
	В первом ящике 11 зеленых и 7 белых шаров, а во втором 11 зеленых и 8 белых. Из первого ящика во второй перекладываются 2 наудачу извлеченных шара. После этого из второго ящика наудачу извлекается один шар. \\ \indent а) Найти вероятность того, что он белый. \\ \indent б) Известно, что этот шар белый; найти вероятность того, что извлеченные из первого ящика шары --- белые.
 
\end{zkrW}

\begin{zkrW}{20}\noindent 
	Партия изделий содержит $10\%$ брака. Найти вероятность того, что среди взятых наугад 7 изделий окажется по меньшей мере 2 бракованных.
 
\end{zkrW}

\begin{zkrW}{20}\noindent 
	Студент за все время обучения в вузе в среднем выполняет 2400 задач по математике. Вероятность неверно решить отдельную задачу при условии стопроцентного посещения и активной работы на всех занятиях равна $0{,}0025$, в противном случае --- $0{,}6$. Найти вероятность того, что за время обучения в вузе \\ \indent а) абсолютно прилежный студент решил неверно по меньшей мере 2 задачи; \\ \indent б) обычный студент решил правильно ровно 1411 задач; \\ \indent в) обычный студент неверно решил от 1411 до 1483 задач.
 
\end{zkrW}

\newpage\setcounter{zad}{0}



\begin{zkrW}{20}\noindent 
	Вероятность того, что при первом измерении некоторой физической величины будет допущена ошибка, превышающая заданную точность, равна $0{,}1$; при последующих измерениях --- $0{,}4$. Произведены три независимых измерения. Найдите вероятность того, что не менее чем в одном измерении допущенная ошибка превысит заданную точность.
 
\end{zkrW}

\begin{zkrW}{20}\noindent 
	Полина и Петя условились встретиться в определенном месте между 15:00 и 17:00. Каждый из них может прийти в любое время в течение указанного промежутка и ждет второго некоторое время. Полина ждет 30 минут, после чего уходит; Петя ждет 60 минут, после чего уходит. В 17:00 любой из них уходит, сколько бы до этого он ни ждал. Чему равна вероятность того, что встреча состоится в первые двадцать минут?
 
\end{zkrW}

\begin{zkrW}{20}\noindent 
	В каждом из трех ящиков 6 красных и 8 белых шаров. Из первого и второго ящиков наудачу извлекается по одному шару и кладется в третий ящик. Затем из третьего ящика извлекается один шар. \\ \indent а) Найти вероятность того, что он красный. \\ \indent б) Известно, что этот шар красный; найти вероятность того, что шары, извлеченные из первого и второго ящиков, --- красные.
 
\end{zkrW}

\begin{zkrW}{20}\noindent 
	Эксплуатируется устройство, состоящее из 7 независимо работающих элементов. Вероятность отказа каждого из них за время работы устройства равна $0{,}2$. Найти вероятность того, что за время работы устройства откажут хотя бы 4 элемента.
 
\end{zkrW}

\begin{zkrW}{20}\noindent 
	В ралли принимает участие 1200 экипажей. Каждый экипаж может сойти с дистанции из-за технических неполадок с вероятностью $0{,}25$, а из-за болезни водителя --- с вероятностью $0{,}005$. Найти вероятность того, что \\ \indent а) по меньшей мере 2 экипажей сойдут с дистанции из-за болезни водителя; \\ \indent б) ровно 291 экипажей не смогут продолжать ралли из-за технических неполадок; \\ \indent в) от 288 до 342 экипажей пострадают от технических проблем.
 
\end{zkrW}

\newpage\setcounter{zad}{0}



\begin{zkrW}{20}\noindent 
	Контролер ОТК, проверив качество сшитых 25 пальто, установил, что 15 из них первого сорта, а остальные --- второго. Найдите вероятность того, что среди взятых наудачу из этой партии 8 пальто ровно 3 будут второго сорта.
 
\end{zkrW}

\begin{zkrW}{20}\noindent 
	Леонардо и Микеланджело условились встретиться в определенном месте между 02:00 и 06:00. Каждый из них может прийти в любое время в течение указанного промежутка и ждет второго некоторое время. Леонардо ждет 50 минут, после чего уходит; Микеланджело ждет 70 минут, после чего уходит. В 06:00 любой из них уходит, сколько бы до этого он ни ждал. Чему равна вероятность того, что Леонардо и Микеланджело не встретятся?
 
\end{zkrW}

\begin{zkrW}{20}\noindent 
	В первом ящике 7 красных и 11 черных шаров, а во втором 8 красных и 11 черных. Из первого ящика во второй перекладываются 2 наудачу извлеченных шара. После этого из второго ящика наудачу извлекается один шар. \\ \indent а) Найти вероятность того, что он красный. \\ \indent б) Известно, что этот шар красный; найти вероятность того, что извлеченные из первого ящика шары --- красные.
 
\end{zkrW}

\begin{zkrW}{20}\noindent 
	В магазин вошли 5 покупателей. Найти вероятность того, что более чем 4 из них совершат покупки, если вероятность совершить покупку для каждого из них одинакова и равна $1/2$.
 
\end{zkrW}

\begin{zkrW}{20}\noindent 
	Предполагая рождение ребенка в любой день года равновозможным, найти вероятность того, что в группе из 150 человек \\ \indent а) по крайней мере 2 родились 14 октября; \\ \indent б) ровно 102 родились осенью; \\ \indent в) от 79 до 83 родились весной.
 
\end{zkrW}

\newpage\setcounter{zad}{0}



\begin{zkrW}{20}\noindent 
	Три студента пришли сдавать экзамен. Вероятность того, что первый студент сдаст экзамен, равна $0{,}5$, второй --- $0{,}2$, третий --- $0{,}4$. Найдите вероятность того, что хотя бы двое сдадут экзамен.
 
\end{zkrW}

\begin{zkrW}{20}\noindent 
	Катя и Петя условились встретиться в определенном месте между 02:00 и 06:00. Каждый из них может прийти в любое время в течение указанного промежутка и ждет второго некоторое время. Катя ждет 70 минут, после чего уходит; Петя ждет 80 минут, после чего уходит. В 06:00 любой из них уходит, сколько бы до этого он ни ждал. Чему равна вероятность того, что Катя опоздает менее чем на полчаса?
 
\end{zkrW}

\begin{zkrW}{20}\noindent 
	В каждом из трех ящиков 7 голубых и 9 желтых шаров. Из первого ящика в третий перекладывают два наудачу выбранных шара, а из второго ящика в третий перекладывают один наудачу взятый шар. Затем из третьего ящика извлекается один шар. \\ \indent а) Найти вероятность того, что он голубой. \\ \indent б) Известно, что этот шар голубой; найти вероятность того, что из первого ящика во второй переложили голубые шары.
 
\end{zkrW}

\begin{zkrW}{20}\noindent 
	5 покупателя приехали на оптовый склад. Вероятность того, что каждому из них потребуется холодильник отечественного производства, равна $2/7$. Найти вероятность того, что такой холодильник потребуется по меньшей мере 3 покупателям.
 
\end{zkrW}

\begin{zkrW}{20}\noindent 
	Мастерская за год ремонтирует 1600 мобильных телефонов. Вероятность неисправности в механической части отдельного телефона равна $0{,}1$, в электронной части --- $0{,}0025$. Найти вероятность того, что среди телефонов, отремонтированных за год, \\ \indent а) имели неисправности в механической части от 165 до 170 экземпляров; \\ \indent б) имели неисправности в электронной части хотя бы 3 телефонов; \\ \indent в) ровно 149 телефонов имели проблемы в механической части.
 
\end{zkrW}

\newpage\setcounter{zad}{0}



\begin{zkrW}{20}\noindent 
	Игра проводится до выигрыша одним из двух игроков двух партий подряд (ничьи исключаются). Вероятность выигрыша партии каждым из игроков равна $0{,}5$ и не зависит от исходов предыдущих партий. Найдите вероятность того, что игра окончится до 4-й партии.
 
\end{zkrW}

\begin{zkrW}{20}\noindent 
	Робин и Бэтмен условились встретиться в определенном месте между 03:00 и 07:00. Каждый из них может прийти в любое время в течение указанного промежутка и ждет второго некоторое время. Робин ждет 70 минут, после чего уходит; Бэтмен ждет 60 минут, после чего уходит. В 07:00 любой из них уходит, сколько бы до этого он ни ждал. Чему равна вероятность того, что встреча состоится в первые полчаса?
 
\end{zkrW}

\begin{zkrW}{20}\noindent 
	В каждом из трех ящиков 7 черных и 5 желтых шаров. Из первого ящика в третий перекладывают два наудачу выбранных шара, а из второго ящика в третий перекладывают один наудачу взятый шар. Затем из третьего ящика извлекается один шар. \\ \indent а) Найти вероятность того, что он желтый. \\ \indent б) Известно, что этот шар желтый; найти вероятность того, что из первого ящика во второй переложили желтые шары.
 
\end{zkrW}

\begin{zkrW}{20}\noindent 
	$85\%$ изделий данного предприятия — это продукция высшего сорта. Некто приобрел 5 изделий, изготовленных на этом предприятии. Чему равна вероятность того, что ровно 3 из них — высшего сорта?
 
\end{zkrW}

\begin{zkrW}{20}\noindent 
	Мастерская за год ремонтирует 900 мобильных телефонов. Вероятность неисправности в механической части отдельного телефона равна $0{,}8$, в электронной части --- $0{,}01$. Найти вероятность того, что среди телефонов, отремонтированных за год, \\ \indent а) имели неисправности в механической части от 698 до 742 экземпляров; \\ \indent б) имели неисправности в электронной части по крайней мере 5 телефонов; \\ \indent в) ровно 698 телефонов имели проблемы в механической части.
 
\end{zkrW}

\newpage\setcounter{zad}{0}



\begin{zkrW}{20}\noindent 
	Три лыжника съезжают с горы. Вероятность падения первого лыжника равна $0{,}8$, второго --- $0{,}2$, третьего --- $0{,}4$. Найдите вероятность того, что хотя бы два лыжника не упадут.
 
\end{zkrW}

\begin{zkrW}{20}\noindent 
	Катя и Света условились встретиться в определенном месте между 03:00 и 04:00. Каждый из них может прийти в любое время в течение указанного промежутка и ждет второго некоторое время. Катя ждет 20 минут, после чего уходит; Света ждет 30 минут, после чего уходит. В 04:00 любой из них уходит, сколько бы до этого он ни ждал. Чему равна вероятность того, что встреча состоится в первые двадцать минут?
 
\end{zkrW}

\begin{zkrW}{20}\noindent 
	В альбоме 8 чистых и 8 гашеных марок. Из альбома изымаются 3 наудачу извлеченные марки. После этого из альбома вновь наудачу извлекаются 5 марки. \\ \indent а) Найти вероятность того, что эти марки гашеные. \\ \indent б) Известно, что эти 5 марки гашеные; найти вероятность того, что первоначально изъятые 3 марки --- чистые.
 
\end{zkrW}

\begin{zkrW}{20}\noindent 
	Всхожесть семян данного сорта растений оценивается с вероятностью, равной $0{,}4$. Какова вероятность того, что из 5 посеянных семян взойдут не менее чем 3?
 
\end{zkrW}

\begin{zkrW}{20}\noindent 
	Вероятность появления опечатки на отдельной странице книги равна $0{,}006$, а погрешности верстки --- $0{,}2$. Найти вероятность того, что в книге из 100 страниц \\ \indent а) не более чем 2 страниц будут иметь опечатки; \\ \indent б) от 19 до 22 страниц будут иметь погрешности верстки; \\ \indent в) погрешности верстки будут присутствовать ровно на 21 страницах.
 
\end{zkrW}

\newpage\setcounter{zad}{0}



\begin{zkrW}{20}\noindent 
	Мастер обслуживает четыре станка, работающих независимо друг от друга. Вероятность того, что первыйй станок в течение смены потребует внимания мастера, равна $1/6$, второй --- $1/9$, третий --- $1/2$ и четвертый --- $3/4$. Найдите вероятность того, что в течение смены хотя бы один станок не потребует внимания мастера.
 
\end{zkrW}

\begin{zkrW}{20}\noindent 
	Андрей и Вася условились встретиться в определенном месте между 04:00 и 07:00. Каждый из них может прийти в любое время в течение указанного промежутка и ждет второго некоторое время. Андрей ждет 40 минут, после чего уходит; Вася ждет 50 минут, после чего уходит. В 07:00 любой из них уходит, сколько бы до этого он ни ждал. Чему равна вероятность того, что Андрей опоздает более чем на полчаса?
 
\end{zkrW}

\begin{zkrW}{20}\noindent 
	В первом ящике 12 голубых и 11 черных шаров, а во втором 8 голубых и 9 черных. Из первого ящика во второй перекладываются 2 наудачу извлеченных шара. После этого из второго ящика наудачу извлекается один шар. \\ \indent а) Найти вероятность того, что он черный. \\ \indent б) Известно, что этот шар черный; найти вероятность того, что извлеченные из первого ящика шары --- черные.
 
\end{zkrW}

\begin{zkrW}{20}\noindent 
	Игрок набрасывает кольца на колышек. Вероятность удачи при этом равна $1/9$. Найти вероятность того, что из 6 колец на колышек попадут более чем 4.
 
\end{zkrW}

\begin{zkrW}{20}\noindent 
	Вероятность появления опечатки на отдельной странице книги равна $0{,}0025$, а погрешности верстки --- $0{,}4$. Найти вероятность того, что в книге из 2400 страниц \\ \indent а) более чем 4 страниц будут иметь опечатки; \\ \indent б) от 893 до 1027 страниц будут иметь погрешности верстки; \\ \indent в) погрешности верстки будут присутствовать ровно на 883 страницах.
 
\end{zkrW}

\newpage\setcounter{zad}{0}



\begin{zkrW}{20}\noindent 
	Два стрелка сделали по одному выстрелу по мишени. Известно, что вероятность попадания в мишень для одного из стрелков равна $0{,}3$, а для другого --- $0{,}1$. Найдите вероятность того, что ровно один из стрелков не попадет в мишень.
 
\end{zkrW}

\begin{zkrW}{20}\noindent 
	Джордж Оруэлл и Джон Фаулз условились встретиться в определенном месте между 16:00 и 17:00. Каждый из них может прийти в любое время в течение указанного промежутка и ждет второго некоторое время. Джордж Оруэлл ждет 20 минут, после чего уходит; Джон Фаулз ждет 50 минут, после чего уходит. В 17:00 любой из них уходит, сколько бы до этого он ни ждал. Чему равна вероятность того, что встреча состоится в первые двадцать минут?
 
\end{zkrW}

\begin{zkrW}{20}\noindent 
	В каждом из трех ящиков 10 красных и 10 синих шаров. Из первого ящика в третий перекладывают два наудачу выбранных шара, а из второго ящика в третий перекладывают один наудачу взятый шар. Затем из третьего ящика извлекается один шар. \\ \indent а) Найти вероятность того, что он красный. \\ \indent б) Известно, что этот шар красный; найти вероятность того, что из первого ящика во второй переложили красные шары.
 
\end{zkrW}

\begin{zkrW}{20}\noindent 
	В тестовом задании 9 вопросов, на каждый дано 6 варианта ответа, среди которых один правильный. Какова вероятность того, что, выбирая вариант ответа наугад, отвечающий правильно ответит не менее чем на 3 вопроса?
 
\end{zkrW}

\begin{zkrW}{20}\noindent 
	Вероятность появления опечатки на отдельной странице книги равна $0{,}0028$, а погрешности верстки --- $0{,}36$. Найти вероятность того, что в книге из 2500 страниц \\ \indent а) по меньшей мере 4 страниц будут иметь опечатки; \\ \indent б) от 873 до 945 страниц будут иметь погрешности верстки; \\ \indent в) погрешности верстки будут присутствовать ровно на 873 страницах.
 
\end{zkrW}

\newpage\setcounter{zad}{0}



\begin{zkrW}{20}\noindent 
	Экзаменационный билет содержит три вопроса. Вероятности того, что студент ответит на первый и второй вопросы билета равны $0{,}3$; на третий --- $0{,}5$. Найдите вероятность того, что студент сдаст экзамен, если для этого необходимо ответить хотя бы на два вопроса.
 
\end{zkrW}

\begin{zkrW}{20}\noindent 
	Алексей и Полина условились встретиться в определенном месте между 13:00 и 16:00. Каждый из них может прийти в любое время в течение указанного промежутка и ждет второго некоторое время. Алексей ждет 40 минут, после чего уходит; Полина ждет 70 минут, после чего уходит. В 16:00 любой из них уходит, сколько бы до этого он ни ждал. Чему равна вероятность того, что Алексей опоздает менее чем на полчаса?
 
\end{zkrW}

\begin{zkrW}{20}\noindent 
	В альбоме 9 чистых и 7 гашеных марок. Из альбома наудачу извлекаются 2 марки и подвергаются гашению, а затем возвращаются в альбом. После этого вновь наудачу извлекаются 3 марки. \\ \indent а) Найти вероятность того, что эти марки гашеные. \\ \indent б) Известно, что эти 3 марки гашеные; найти вероятность того, что первоначально извлеченные 2 марки --- чистые.
 
\end{zkrW}

\begin{zkrW}{20}\noindent 
	В тестовом задании 5 вопросов, на каждый дано 7 варианта ответа, среди которых один правильный. Какова вероятность того, что, выбирая вариант ответа наугад, отвечающий правильно ответит по крайней мере на 2 вопроса?
 
\end{zkrW}

\begin{zkrW}{20}\noindent 
	Известно, что левши среди населения Северной Нарнии составляют в среднем $2{,}0\%$, а люди, одинаково владеющие левой и правой рукой, --- $0{,}9$ (остальные --- правши). Найти вероятность того, что среди 400 людей \\ \indent а) окажется не более чем 3 левшей; \\ \indent б) окажется ровно 346 амбидекстров\footnote{людей, одинаково владеющих обеими руками}; \\ \indent в) окажется от 346 до 374 амбидекстров.
 
\end{zkrW}

\newpage\setcounter{zad}{0}



\begin{zkrW}{20}\noindent 
	Два стрелка сделали по одному выстрелу по мишени. Известно, что вероятность попадания в мишень для одного из стрелков равна $3/4$, а для другого --- $1/4$. Найдите вероятность того, что не менее чем один из стрелков не попадет в мишень.
 
\end{zkrW}

\begin{zkrW}{20}\noindent 
	Дэйв Гилмор и Джим Моррисон условились встретиться в определенном месте между 12:00 и 15:00. Каждый из них может прийти в любое время в течение указанного промежутка и ждет второго некоторое время. Дэйв Гилмор ждет 50 минут, после чего уходит; Джим Моррисон ждет 60 минут, после чего уходит. В 15:00 любой из них уходит, сколько бы до этого он ни ждал. Чему равна вероятность того, что встреча состоится в первые полчаса?
 
\end{zkrW}

\begin{zkrW}{20}\noindent 
	В каждом из трех ящиков 7 желтых и 6 белых шаров. Из первого и второго ящиков наудачу извлекается по одному шару и кладется в третий ящик. Затем из третьего ящика извлекается один шар. \\ \indent а) Найти вероятность того, что он желтый. \\ \indent б) Известно, что этот шар желтый; найти вероятность того, что шары, извлеченные из первого и второго ящиков, --- желтые.
 
\end{zkrW}

\begin{zkrW}{20}\noindent 
	Для баскетболиста дяди Стёпы вероятность забросить мяч в корзину равна $1/3$. Он выполняет 5 бросков. Какова вероятность, что в корзину попадут не менее чем 3 мяча?
 
\end{zkrW}

\begin{zkrW}{20}\noindent 
	Вероятность появления опечатки на отдельной странице книги равна $0{,}005$, а погрешности верстки --- $0{,}2$. Найти вероятность того, что в книге из 1600 страниц \\ \indent а) хотя бы 3 страниц будут иметь опечатки; \\ \indent б) от 285 до 339 страниц будут иметь погрешности верстки; \\ \indent в) погрешности верстки будут присутствовать ровно на 362 страницах.
 
\end{zkrW}

\newpage\setcounter{zad}{0}



\begin{zkrW}{20}\noindent 
	Вероятность того, что при первом измерении некоторой физической величины будет допущена ошибка, превышающая заданную точность, равна $0{,}5$; при последующих измерениях --- $0{,}5$. Произведены три независимых измерения. Найдите вероятность того, что не менее чем в одном измерении допущенная ошибка превысит заданную точность.
 
\end{zkrW}

\begin{zkrW}{20}\noindent 
	Габриэль Гарсия Маркес и Джон Фаулз условились встретиться в определенном месте между 16:00 и 18:00. Каждый из них может прийти в любое время в течение указанного промежутка и ждет второго некоторое время. Габриэль Гарсия Маркес ждет 40 минут, после чего уходит; Джон Фаулз ждет 60 минут, после чего уходит. В 18:00 любой из них уходит, сколько бы до этого он ни ждал. Чему равна вероятность того, что Габриэль Гарсия Маркес придет раньше, чем Джон Фаулз?
 
\end{zkrW}

\begin{zkrW}{20}\noindent 
	В каждом из трех ящиков 7 белых и 11 желтых шаров. Из первого и второго ящиков наудачу извлекается по одному шару и кладется в третий ящик. Затем из третьего ящика извлекается один шар. \\ \indent а) Найти вероятность того, что он желтый. \\ \indent б) Известно, что этот шар желтый; найти вероятность того, что шары, извлеченные из первого и второго ящиков, --- желтые.
 
\end{zkrW}

\begin{zkrW}{20}\noindent 
	В студии находятся 8 телевизионных камер. Для каждой камеры вероятность того, что она включена в данный момент, равна $3/8$. Найти вероятность того, что в данный момент оказались выключены не более чем 3 камеры.
 
\end{zkrW}

\begin{zkrW}{20}\noindent 
	Магазин закупил 150 телевизоров и столько же магнитол. Вероятность того, что отдельный телевизор окажется бракованным, равна $0{,}02$, а вероятность того, что магнитола окажется бракованной, --- $0{,}4$. Найти вероятность того, что в этой закупке \\ \indent а) не менее чем 2 телевизора окажутся бракованными; \\ \indent б) ровно 52 магнитол окажутся нерабочими; \\ \indent в) от 62 до 67 магнитол будут бракованными.
 
\end{zkrW}

\newpage\setcounter{zad}{0}



\begin{zkrW}{20}\noindent 
	Вероятность своевременного выполнения студентом контрольной работы по каждой из трех дисциплин равна соответственно $1/3$, $1/8$ и $1/3$. Найти вероятность своевременного выполнения контрольной работы студентом хотя бы по двум дисциплинам.
 
\end{zkrW}

\begin{zkrW}{20}\noindent 
	Портос и Атос условились встретиться в определенном месте между 10:00 и 11:00. Каждый из них может прийти в любое время в течение указанного промежутка и ждет второго некоторое время. Портос ждет 50 минут, после чего уходит; Атос ждет 40 минут, после чего уходит. В 11:00 любой из них уходит, сколько бы до этого он ни ждал. Чему равна вероятность того, что Портос придет раньше, чем Атос?
 
\end{zkrW}

\begin{zkrW}{20}\noindent 
	В альбоме 12 чистых и 5 гашеных марок. Из альбома наудачу извлекаются 3 марки и подвергаются гашению, а затем возвращаются в альбом. После этого вновь наудачу извлекаются 4 марки. \\ \indent а) Найти вероятность того, что эти марки гашеные. \\ \indent б) Известно, что эти 4 марки гашеные; найти вероятность того, что первоначально извлеченные 3 марки --- чистые.
 
\end{zkrW}

\begin{zkrW}{20}\noindent 
	По новым правилам в волейбольном матче игра происходит до тех пор, пока одна из команд не выиграет 4 партии. Вероятность победы российской сборной в каждой партии равна $0{,}9$. Определить вероятность того, что в ближайшем матче сборная России победит со счетом 4:3.
 
\end{zkrW}

\begin{zkrW}{20}\noindent 
	Студент за все время обучения в вузе в среднем выполняет 600 задач по математике. Вероятность неверно решить отдельную задачу при условии стопроцентного посещения и активной работы на всех занятиях равна $0{,}001$, в противном случае --- $0{,}4$. Найти вероятность того, что за время обучения в вузе \\ \indent а) абсолютно прилежный студент решил неверно не более чем 2 задачи; \\ \indent б) обычный студент решил правильно ровно 228 задач; \\ \indent в) обычный студент неверно решил от 252 до 269 задач.
 
\end{zkrW}

\newpage\setcounter{zad}{0}



\begin{zkrW}{20}\noindent 
	Вероятность наступления некоторого случайного события в каждом опыте одинакова и равна $0{,}9$. Опыты проводятся последовательно до наступления этого события. Определить вероятность того, что придется проводить 2-й опыт.
 
\end{zkrW}

\begin{zkrW}{20}\noindent 
	Петя и Полина условились встретиться в определенном месте между 01:00 и 03:00. Каждый из них может прийти в любое время в течение указанного промежутка и ждет второго некоторое время. Петя ждет 60 минут, после чего уходит; Полина ждет 30 минут, после чего уходит. В 03:00 любой из них уходит, сколько бы до этого он ни ждал. Чему равна вероятность того, что встреча состоится в последние полчаса?
 
\end{zkrW}

\begin{zkrW}{20}\noindent 
	В альбоме 7 чистых и 8 гашеных марок. Из альбома изымаются 4 наудачу извлеченные марки. После этого из альбома вновь наудачу извлекаются 3 марки. \\ \indent а) Найти вероятность того, что эти марки гашеные. \\ \indent б) Известно, что эти 3 марки гашеные; найти вероятность того, что первоначально изъятые 4 марки --- чистые.
 
\end{zkrW}

\begin{zkrW}{20}\noindent 
	Вероятность попадания в мишень при одном выстреле равна $1/3$. Найти вероятность того, что при 9 выстрелах будет менее чем 2 попадания.
 
\end{zkrW}

\begin{zkrW}{20}\noindent 
	Предполагая рождение ребенка в любой день года равновозможным, найти вероятность того, что в группе из 2400 человек \\ \indent а) по крайней мере 2 родились 28 марта; \\ \indent б) ровно 902 родились осенью; \\ \indent в) от 979 до 1008 родились весной.
 
\end{zkrW}

\newpage\setcounter{zad}{0}



\begin{zkrW}{20}\noindent 
	Экспедиция издательства отправила газеты в три почтовых отделения. Вероятность своевременной доставки газет в первое отделение равна $5/9$, во второе отделение --- $1/4$ и в третье --- $5/9$. Найдите вероятность того, что хотя бы одно отделение получит газеты с опозданием.
 
\end{zkrW}

\begin{zkrW}{20}\noindent 
	Рафаэль и Леонардо условились встретиться в определенном месте между 01:00 и 04:00. Каждый из них может прийти в любое время в течение указанного промежутка и ждет второго некоторое время. Рафаэль ждет 50 минут, после чего уходит; Леонардо ждет 70 минут, после чего уходит. В 04:00 любой из них уходит, сколько бы до этого он ни ждал. Чему равна вероятность того, что Рафаэль и Леонардо не встретятся?
 
\end{zkrW}

\begin{zkrW}{20}\noindent 
	В каждом из трех ящиков 10 голубых и 12 синих шаров. Из первого и второго ящиков наудачу извлекается по одному шару и кладется в третий ящик. Затем из третьего ящика извлекается один шар. \\ \indent а) Найти вероятность того, что он синий. \\ \indent б) Известно, что этот шар синий; найти вероятность того, что шары, извлеченные из первого и второго ящиков, --- синие.
 
\end{zkrW}

\begin{zkrW}{20}\noindent 
	Вероятность того, что за рабочий день расход электроэнергии не превысит норму, равна $0{,}3$. Найти вероятность того, что за 9 дней работы норма будет превышена не более чем 3 раза.
 
\end{zkrW}

\begin{zkrW}{20}\noindent 
	В лотерее разыгрываются крупные и мелкие выигрыши. Вероятность того, что на лотерейный билет выпадет крупный выигрыш, равна $0{,}005$, а мелкий --- $0{,}1$. Куплено 400 билетов. Найти вероятность того, что \\ \indent а) крупных выигрышей будет по крайней мере 5; \\ \indent б) мелких выигрышей будет ровно 44; \\ \indent в) мелких выигрышей будет от 43 до 46.
 
\end{zkrW}

\newpage\setcounter{zad}{0}



\begin{zkrW}{20}\noindent 
	Студент разыскивает нужную ему формулу в трех справочниках. Вероятность того, что формула содержится в первом, втором и третьем справочниках, равна соответственно $0{,}2$, $0{,}8$ и $0{,}9$. Найдите вероятность того, что эта формула содержится не менее чем в двух справочниках.
 
\end{zkrW}

\begin{zkrW}{20}\noindent 
	Катя и Полина условились встретиться в определенном месте между 17:00 и 19:00. Каждый из них может прийти в любое время в течение указанного промежутка и ждет второго некоторое время. Катя ждет 60 минут, после чего уходит; Полина ждет 50 минут, после чего уходит. В 19:00 любой из них уходит, сколько бы до этого он ни ждал. Чему равна вероятность того, что Катя опоздает менее чем на полчаса?
 
\end{zkrW}

\begin{zkrW}{20}\noindent 
	В альбоме 5 чистых и 10 гашеных марок. Из альбома наудачу извлекаются 2 марки и заменяются на чистые. После этого вновь наудачу извлекаются 5 марки. \\ \indent а) Найти вероятность того, что эти марки чистые. \\ \indent б) Известно, что эти 5 марки чистые; найти вероятность того, что первоначально извлеченные 2 марки --- гашеные.
 
\end{zkrW}

\begin{zkrW}{20}\noindent 
	В ячейку памяти ЭВМ записывается двоичное число длиной в 7 разрядов. Значения 0 и 1 в каждом разряде появляются с равной вероятностью. Найти вероятность того, что в этом двоичном числе не менее чем 2 единицы.
 
\end{zkrW}

\begin{zkrW}{20}\noindent 
	Магазин закупил 100 телевизоров и столько же магнитол. Вероятность того, что отдельный телевизор окажется бракованным, равна $0{,}008$, а вероятность того, что магнитола окажется бракованной, --- $0{,}2$. Найти вероятность того, что в этой закупке \\ \indent а) ровно 2 телевизора окажутся бракованными; \\ \indent б) ровно 19 магнитол окажутся нерабочими; \\ \indent в) от 22 до 23 магнитол будут бракованными.
 
\end{zkrW}

\newpage\setcounter{zad}{0}



\begin{zkrW}{20}\noindent 
	Три студента пришли сдавать экзамен. Вероятность того, что первый студент сдаст экзамен, равна $0{,}9$, второй --- $0{,}9$, третий --- $0{,}1$. Найдите вероятность того, что хотя бы двое сдадут экзамен.
 
\end{zkrW}

\begin{zkrW}{20}\noindent 
	Света и Петя условились встретиться в определенном месте между 01:00 и 04:00. Каждый из них может прийти в любое время в течение указанного промежутка и ждет второго некоторое время. Света ждет 70 минут, после чего уходит; Петя ждет 50 минут, после чего уходит. В 04:00 любой из них уходит, сколько бы до этого он ни ждал. Чему равна вероятность того, что встреча произойдет не ранее чем без четверти 04:00?
 
\end{zkrW}

\begin{zkrW}{20}\noindent 
	В каждом из трех ящиков 9 голубых и 7 белых шаров. Из первого ящика в третий перекладывают два наудачу выбранных шара, а из второго ящика в третий перекладывают один наудачу взятый шар. Затем из третьего ящика извлекается один шар. \\ \indent а) Найти вероятность того, что он голубой. \\ \indent б) Известно, что этот шар голубой; найти вероятность того, что из первого ящика во второй переложили голубые шары.
 
\end{zkrW}

\begin{zkrW}{20}\noindent 
	$50\%$ изделий данного предприятия — это продукция высшего сорта. Некто приобрел 6 изделий, изготовленных на этом предприятии. Чему равна вероятность того, что не менее чем 2 из них — высшего сорта?
 
\end{zkrW}

\begin{zkrW}{20}\noindent 
	Мастерская за год ремонтирует 3600 мобильных телефонов. Вероятность неисправности в механической части отдельного телефона равна $0{,}2$, в электронной части --- $0{,}0025$. Найти вероятность того, что среди телефонов, отремонтированных за год, \\ \indent а) имели неисправности в механической части от 655 до 677 экземпляров; \\ \indent б) имели неисправности в электронной части хотя бы 3 телефонов; \\ \indent в) ровно 691 телефонов имели проблемы в механической части.
 
\end{zkrW}

\newpage\setcounter{zad}{0}



\begin{zkrW}{20}\noindent 
	В коробке смешаны электролампы одинакового размера и формы: по 100 Вт --- 10 штук, по 75 Вт --- 9 штук. Вынуты наудачу 2 лампы. Какова вероятность того, что они одинаковой мощности?
 
\end{zkrW}

\begin{zkrW}{20}\noindent 
	д'Артаньян и Атос условились встретиться в определенном месте между 13:00 и 15:00. Каждый из них может прийти в любое время в течение указанного промежутка и ждет второго некоторое время. д'Артаньян ждет 40 минут, после чего уходит; Атос ждет 60 минут, после чего уходит. В 15:00 любой из них уходит, сколько бы до этого он ни ждал. Чему равна вероятность того, что д'Артаньян придет раньше, чем Атос?
 
\end{zkrW}

\begin{zkrW}{20}\noindent 
	В первом ящике 11 черных и 6 белых шаров, а во втором 8 черных и 9 белых. Из первого ящика во второй перекладываются 3 наудачу извлеченных шара. После этого из второго ящика наудачу извлекается один шар. \\ \indent а) Найти вероятность того, что он белый. \\ \indent б) Известно, что этот шар белый; найти вероятность того, что извлеченные из первого ящика шары --- белые.
 
\end{zkrW}

\begin{zkrW}{20}\noindent 
	Всхожесть семян данного сорта растений оценивается с вероятностью, равной $7/9$. Какова вероятность того, что из 9 посеянных семян взойдут не более чем 2?
 
\end{zkrW}

\begin{zkrW}{20}\noindent 
	Студент за все время обучения в вузе в среднем выполняет 100 задач по математике. Вероятность неверно решить отдельную задачу при условии стопроцентного посещения и активной работы на всех занятиях равна $0{,}008$, в противном случае --- $0{,}2$. Найти вероятность того, что за время обучения в вузе \\ \indent а) абсолютно прилежный студент решил неверно по меньшей мере 3 задачи; \\ \indent б) обычный студент решил правильно ровно 17 задач; \\ \indent в) обычный студент неверно решил от 18 до 20 задач.
 
\end{zkrW}

\newpage\setcounter{zad}{0}



\begin{zkrW}{20}\noindent 
	Два стрелка сделали по одному выстрелу по мишени. Известно, что вероятность попадания в мишень для одного из стрелков равна $4/7$, а для другого --- $1/9$. Найдите вероятность того, что хотя бы один из стрелков попадет в мишень.
 
\end{zkrW}

\begin{zkrW}{20}\noindent 
	Микеланджело и Рафаэль условились встретиться в определенном месте между 06:00 и 09:00. Каждый из них может прийти в любое время в течение указанного промежутка и ждет второго некоторое время. Микеланджело ждет 60 минут, после чего уходит; Рафаэль ждет 70 минут, после чего уходит. В 09:00 любой из них уходит, сколько бы до этого он ни ждал. Чему равна вероятность того, что встреча произойдет не ранее чем без четверти 09:00?
 
\end{zkrW}

\begin{zkrW}{20}\noindent 
	В альбоме 5 чистых и 11 гашеных марок. Из альбома наудачу извлекаются 2 марки и заменяются на чистые. После этого вновь наудачу извлекаются 2 марки. \\ \indent а) Найти вероятность того, что эти марки гашеные. \\ \indent б) Известно, что эти 2 марки гашеные; найти вероятность того, что первоначально извлеченные 2 марки --- чистые.
 
\end{zkrW}

\begin{zkrW}{20}\noindent 
	В случайно выбранной семье 8 детей. Считая вероятности рождения мальчика и девочки одинаковыми, определить вероятность того, что в выбранной семье окажется хотя бы 2 мальчика.
 
\end{zkrW}

\begin{zkrW}{20}\noindent 
	Вероятность выиграть отдельному игроку 1000 рублей в игре <<Кто хочет стать миллионером>> равна $0{,}75$, а 32000 рублей --- $0{,}0075$. За сезон в этой игре принимает участие 1200 человек. Найти вероятность того, что за сезон \\ \indent а) 1000 рублей получат ровно 873 человек; \\ \indent б) 1000 рублей получат от 864 до 918 человек; \\ \indent в) хотя бы 2 человек получат крупный выигрыш в 32000 рублей.
 
\end{zkrW}

\newpage\setcounter{zad}{0}



\begin{zkrW}{20}\noindent 
	В коробке 13 красных, 8 синих и 6 желтых карандашей. Наудачу вынимают три карандаша. Какова вероятность того, что они все разных цветов?
 
\end{zkrW}

\begin{zkrW}{20}\noindent 
	Хоттабыч и Женя Богорад условились встретиться в определенном месте между 12:00 и 15:00. Каждый из них может прийти в любое время в течение указанного промежутка и ждет второго некоторое время. Хоттабыч ждет 50 минут, после чего уходит; Женя Богорад ждет 70 минут, после чего уходит. В 15:00 любой из них уходит, сколько бы до этого он ни ждал. Чему равна вероятность того, что встреча состоится в последние полчаса?
 
\end{zkrW}

\begin{zkrW}{20}\noindent 
	В альбоме 6 чистых и 10 гашеных марок. Из альбома изымаются 4 наудачу извлеченные марки. После этого из альбома вновь наудачу извлекаются 4 марки. \\ \indent а) Найти вероятность того, что эти марки гашеные. \\ \indent б) Известно, что эти 4 марки гашеные; найти вероятность того, что первоначально изъятые 4 марки --- чистые.
 
\end{zkrW}

\begin{zkrW}{20}\noindent 
	При данном технологическом процессе $45\%$ всех сходящих с конвейера автозавода автомобилей имеют цвет «металлик». Найти вероятность того, что из 8 случайно отобранных автомобилей более чем 3 будут иметь этот цвет.
 
\end{zkrW}

\begin{zkrW}{20}\noindent 
	Студент за все время обучения в вузе в среднем выполняет 1600 задач по математике. Вероятность неверно решить отдельную задачу при условии стопроцентного посещения и активной работы на всех занятиях равна $0{,}005$, в противном случае --- $0{,}5$. Найти вероятность того, что за время обучения в вузе \\ \indent а) абсолютно прилежный студент решил неверно хотя бы 5 задачи; \\ \indent б) обычный студент решил правильно ровно 688 задач; \\ \indent в) обычный студент неверно решил от 784 до 856 задач.
 
\end{zkrW}

\newpage\setcounter{zad}{0}



\begin{zkrW}{20}\noindent 
	Два стрелка сделали по одному выстрелу по мишени. Известно, что вероятность попадания в мишень для одного из стрелков равна $1/6$, а для другого --- $2/9$. Найдите вероятность того, что хотя бы один из стрелков попадет в мишень.
 
\end{zkrW}

\begin{zkrW}{20}\noindent 
	Женя Богорад и Хоттабыч условились встретиться в определенном месте между 19:00 и 22:00. Каждый из них может прийти в любое время в течение указанного промежутка и ждет второго некоторое время. Женя Богорад ждет 50 минут, после чего уходит; Хоттабыч ждет 40 минут, после чего уходит. В 22:00 любой из них уходит, сколько бы до этого он ни ждал. Чему равна вероятность того, что встреча состоится в последние полчаса?
 
\end{zkrW}

\begin{zkrW}{20}\noindent 
	В альбоме 11 чистых и 11 гашеных марок. Из альбома наудачу извлекаются 4 марки и подвергаются гашению, а затем возвращаются в альбом. После этого вновь наудачу извлекаются 5 марки. \\ \indent а) Найти вероятность того, что эти марки гашеные. \\ \indent б) Известно, что эти 5 марки гашеные; найти вероятность того, что первоначально извлеченные 4 марки --- чистые.
 
\end{zkrW}

\begin{zkrW}{20}\noindent 
	Известно, что $15\%$ семян огурцов не всходят при посеве. Какова вероятность того, что из 7 посеянных семян взойдут ровно 2?
 
\end{zkrW}

\begin{zkrW}{20}\noindent 
	В лотерее разыгрываются крупные и мелкие выигрыши. Вероятность того, что на лотерейный билет выпадет крупный выигрыш, равна $0{,}0025$, а мелкий --- $0{,}1$. Куплено 3600 билетов. Найти вероятность того, что \\ \indent а) крупных выигрышей будет более чем 3; \\ \indent б) мелких выигрышей будет ровно 353; \\ \indent в) мелких выигрышей будет от 353 до 374.
 
\end{zkrW}

\newpage\setcounter{zad}{0}



\begin{zkrW}{20}\noindent 
	Ящик содержит 17 годных и 11 дефектных деталей. Сборщик последовательно без возвращения достает из ящика 5 деталей. Найдите вероятность того, что среди взятых деталей хотя бы одна дефектная.
 
\end{zkrW}

\begin{zkrW}{20}\noindent 
	Полина и Алексей условились встретиться в определенном месте между 15:00 и 18:00. Каждый из них может прийти в любое время в течение указанного промежутка и ждет второго некоторое время. Полина ждет 40 минут, после чего уходит; Алексей ждет 60 минут, после чего уходит. В 18:00 любой из них уходит, сколько бы до этого он ни ждал. Чему равна вероятность того, что Полина и Алексей встретятся?
 
\end{zkrW}

\begin{zkrW}{20}\noindent 
	В первом ящике 6 голубых и 9 желтых шаров, а во втором 11 голубых и 6 желтых. Из первого ящика во второй перекладываются 3 наудачу извлеченных шара. После этого из второго ящика наудачу извлекается один шар. \\ \indent а) Найти вероятность того, что он желтый. \\ \indent б) Известно, что этот шар желтый; найти вероятность того, что извлеченные из первого ящика шары --- желтые.
 
\end{zkrW}

\begin{zkrW}{20}\noindent 
	В магазин вошли 8 покупателей. Найти вероятность того, что ровно 2 из них совершат покупки, если вероятность совершить покупку для каждого из них одинакова и равна $0{,}5$.
 
\end{zkrW}

\begin{zkrW}{20}\noindent 
	В ралли принимает участие 1200 экипажей. Каждый экипаж может сойти с дистанции из-за технических неполадок с вероятностью $0{,}25$, а из-за болезни водителя --- с вероятностью $0{,}0025$. Найти вероятность того, что \\ \indent а) не менее чем 5 экипажей сойдут с дистанции из-за болезни водителя; \\ \indent б) ровно 306 экипажей не смогут продолжать ралли из-за технических неполадок; \\ \indent в) от 258 до 288 экипажей пострадают от технических проблем.
 
\end{zkrW}

\newpage\setcounter{zad}{0}



\begin{zkrW}{20}\noindent 
	Контролер ОТК, проверив качество сшитых 20 пальто, установил, что 10 из них первого сорта, а остальные --- второго. Найдите вероятность того, что среди взятых наудачу из этой партии 10 пальто ровно 2 будут второго сорта.
 
\end{zkrW}

\begin{zkrW}{20}\noindent 
	Полина и Аня условились встретиться в определенном месте между 14:00 и 18:00. Каждый из них может прийти в любое время в течение указанного промежутка и ждет второго некоторое время. Полина ждет 70 минут, после чего уходит; Аня ждет 50 минут, после чего уходит. В 18:00 любой из них уходит, сколько бы до этого он ни ждал. Чему равна вероятность того, что Полина опоздает более чем на полчаса?
 
\end{zkrW}

\begin{zkrW}{20}\noindent 
	В альбоме 9 чистых и 8 гашеных марок. Из альбома наудачу извлекаются 4 марки и заменяются на чистые. После этого вновь наудачу извлекаются 2 марки. \\ \indent а) Найти вероятность того, что эти марки гашеные. \\ \indent б) Известно, что эти 2 марки гашеные; найти вероятность того, что первоначально извлеченные 4 марки --- чистые.
 
\end{zkrW}

\begin{zkrW}{20}\noindent 
	Вероятность попадания в цель при одном выстреле равна $1/3$. Производится 8 выстрела. Найти вероятность того, что цель будет поражена менее чем 3 раза.
 
\end{zkrW}

\begin{zkrW}{20}\noindent 
	Вероятность появления опечатки на отдельной странице книги равна $0{,}0025$, а погрешности верстки --- $0{,}5$. Найти вероятность того, что в книге из 3600 страниц \\ \indent а) ровно 5 страниц будут иметь опечатки; \\ \indent б) от 1728 до 1746 страниц будут иметь погрешности верстки; \\ \indent в) погрешности верстки будут присутствовать ровно на 1620 страницах.
 
\end{zkrW}

\newpage\setcounter{zad}{0}



\begin{zkrW}{20}\noindent 
	Во время тренировки три баскетболиста бросают мячи в корзину. Вероятность попадания первого равна $1/3$, второго --- $4/5$, третьего --- $5/8$. Каждый баскетболист делает один бросок. Найдите вероятность хотя бы одного попадания мяча в корзину.
 
\end{zkrW}

\begin{zkrW}{20}\noindent 
	Вася и Саша условились встретиться в определенном месте между 02:00 и 06:00. Каждый из них может прийти в любое время в течение указанного промежутка и ждет второго некоторое время. Вася ждет 80 минут, после чего уходит; Саша ждет 50 минут, после чего уходит. В 06:00 любой из них уходит, сколько бы до этого он ни ждал. Чему равна вероятность того, что Вася и Саша встретятся?
 
\end{zkrW}

\begin{zkrW}{20}\noindent 
	В альбоме 10 чистых и 9 гашеных марок. Из альбома наудачу извлекаются 2 марки и заменяются на чистые. После этого вновь наудачу извлекаются 2 марки. \\ \indent а) Найти вероятность того, что эти марки гашеные. \\ \indent б) Известно, что эти 2 марки гашеные; найти вероятность того, что первоначально извлеченные 2 марки --- чистые.
 
\end{zkrW}

\begin{zkrW}{20}\noindent 
	В среднем $30\%$ акций на аукционах продаются по первоначально заявленной цене. Найти вероятность того, что из 8 пакетов акций в результате торгов по первоначально заявленной цене останутся непроданными менее чем 2.
 
\end{zkrW}

\begin{zkrW}{20}\noindent 
	Магазин закупил 2500 телевизоров и столько же магнитол. Вероятность того, что отдельный телевизор окажется бракованным, равна $0{,}0036$, а вероятность того, что магнитола окажется бракованной, --- $0{,}8$. Найти вероятность того, что в этой закупке \\ \indent а) не более чем 2 телевизора окажутся бракованными; \\ \indent б) ровно 2220 магнитол окажутся нерабочими; \\ \indent в) от 1960 до 2060 магнитол будут бракованными.
 
\end{zkrW}

\newpage\setcounter{zad}{0}



\begin{zkrW}{20}\noindent 
	Контролер ОТК, проверив качество сшитых 22 пальто, установил, что 14 из них первого сорта, а остальные --- второго. Найдите вероятность того, что среди взятых наудачу из этой партии 7 пальто по меньшей мере 4 будут второго сорта.
 
\end{zkrW}

\begin{zkrW}{20}\noindent 
	Полина и Андрей условились встретиться в определенном месте между 17:00 и 19:00. Каждый из них может прийти в любое время в течение указанного промежутка и ждет второго некоторое время. Полина ждет 50 минут, после чего уходит; Андрей ждет 40 минут, после чего уходит. В 19:00 любой из них уходит, сколько бы до этого он ни ждал. Чему равна вероятность того, что Полина опоздает менее чем на полчаса?
 
\end{zkrW}

\begin{zkrW}{20}\noindent 
	В каждом из трех ящиков 8 синих и 9 белых шаров. Из первого ящика в третий перекладывают два наудачу выбранных шара, а из второго ящика в третий перекладывают один наудачу взятый шар. Затем из третьего ящика извлекается один шар. \\ \indent а) Найти вероятность того, что он синий. \\ \indent б) Известно, что этот шар синий; найти вероятность того, что из первого ящика во второй переложили синие шары.
 
\end{zkrW}

\begin{zkrW}{20}\noindent 
	Оптовая база снабжает товаром 9 магазинов. Вероятность того, что в течение дня поступит заявка на товар, равна $0{,}3$ для каждого магазина. Найти вероятность того, что в течение дня поступит не менее чем 3 заявок.
 
\end{zkrW}

\begin{zkrW}{20}\noindent 
	Предполагая рождение ребенка в любой день года равновозможным, найти вероятность того, что в группе из 900 человек \\ \indent а) не более чем 4 родились 17 июня; \\ \indent б) ровно 184 родились осенью; \\ \indent в) от 175 до 185 родились весной.
 
\end{zkrW}

\newpage\setcounter{zad}{0}



\begin{zkrW}{20}\noindent 
	Экзаменационный билет содержит три вопроса. Вероятности того, что студент ответит на первый и второй вопросы билета равны $1/4$; на третий --- $8/9$. Найдите вероятность того, что студент сдаст экзамен, если для этого необходимо ответить хотя бы на два вопроса.
 
\end{zkrW}

\begin{zkrW}{20}\noindent 
	Андрей и Надя условились встретиться в определенном месте между 16:00 и 20:00. Каждый из них может прийти в любое время в течение указанного промежутка и ждет второго некоторое время. Андрей ждет 80 минут, после чего уходит; Надя ждет 70 минут, после чего уходит. В 20:00 любой из них уходит, сколько бы до этого он ни ждал. Чему равна вероятность того, что Андрей опоздает более чем на полчаса?
 
\end{zkrW}

\begin{zkrW}{20}\noindent 
	В альбоме 7 чистых и 8 гашеных марок. Из альбома наудачу извлекаются 4 марки и заменяются на чистые. После этого вновь наудачу извлекаются 3 марки. \\ \indent а) Найти вероятность того, что эти марки гашеные. \\ \indent б) Известно, что эти 3 марки гашеные; найти вероятность того, что первоначально извлеченные 4 марки --- чистые.
 
\end{zkrW}

\begin{zkrW}{20}\noindent 
	В магазин вошли 7 покупателей. Найти вероятность того, что хотя бы 3 из них совершат покупки, если вероятность совершить покупку для каждого из них одинакова и равна $0{,}8$.
 
\end{zkrW}

\begin{zkrW}{20}\noindent 
	В ралли принимает участие 600 экипажей. Каждый экипаж может сойти с дистанции из-за технических неполадок с вероятностью $0{,}6$, а из-за болезни водителя --- с вероятностью $0{,}01$. Найти вероятность того, что \\ \indent а) ровно 4 экипажей сойдут с дистанции из-за болезни водителя; \\ \indent б) ровно 328 экипажей не смогут продолжать ралли из-за технических неполадок; \\ \indent в) от 349 до 389 экипажей пострадают от технических проблем.
 
\end{zkrW}

\newpage\setcounter{zad}{0}



\begin{zkrW}{20}\noindent 
	В ящике 18 деталей, среди которых 8 окрашенных. Сборщик наудачу достает 7 деталей. Найдите вероятность того, что 4 из них оказались окрашенными. 
 
\end{zkrW}

\begin{zkrW}{20}\noindent 
	Надя и Алексей условились встретиться в определенном месте между 08:00 и 10:00. Каждый из них может прийти в любое время в течение указанного промежутка и ждет второго некоторое время. Надя ждет 50 минут, после чего уходит; Алексей ждет 40 минут, после чего уходит. В 10:00 любой из них уходит, сколько бы до этого он ни ждал. Чему равна вероятность того, что встреча состоится в первые двадцать минут?
 
\end{zkrW}

\begin{zkrW}{20}\noindent 
	В первом ящике 12 желтых и 7 зеленых шаров, а во втором 7 желтых и 8 зеленых. Из первого ящика во второй перекладываются 4 наудачу извлеченных шара. После этого из второго ящика наудачу извлекается один шар. \\ \indent а) Найти вероятность того, что он желтый. \\ \indent б) Известно, что этот шар желтый; найти вероятность того, что извлеченные из первого ящика шары --- желтые.
 
\end{zkrW}

\begin{zkrW}{20}\noindent 
	Всхожесть семян данного сорта растений оценивается с вероятностью, равной $2/9$. Какова вероятность того, что из 5 посеянных семян взойдут ровно 2?
 
\end{zkrW}

\begin{zkrW}{20}\noindent 
	Известно, что левши среди населения Северной Нарнии составляют в среднем $0{,}25\%$, а люди, одинаково владеющие левой и правой рукой, --- $0{,}4$ (остальные --- правши). Найти вероятность того, что среди 2400 людей \\ \indent а) окажется хотя бы 4 левшей; \\ \indent б) окажется ровно 854 амбидекстров\footnote{людей, одинаково владеющих обеими руками}; \\ \indent в) окажется от 989 до 1008 амбидекстров.
 
\end{zkrW}

\newpage\setcounter{zad}{0}



\begin{zkrW}{20}\noindent 
	Студент успел подготовить к экзамену 15 вопросов из 15 пяти. Какова вероятность того, что из 10 наудачу выбранных вопросов студент знает по крайней мере 4.
 
\end{zkrW}

\begin{zkrW}{20}\noindent 
	Портос и Арамис условились встретиться в определенном месте между 17:00 и 19:00. Каждый из них может прийти в любое время в течение указанного промежутка и ждет второго некоторое время. Портос ждет 60 минут, после чего уходит; Арамис ждет 40 минут, после чего уходит. В 19:00 любой из них уходит, сколько бы до этого он ни ждал. Чему равна вероятность того, что Портос опоздает более чем на полчаса?
 
\end{zkrW}

\begin{zkrW}{20}\noindent 
	В альбоме 11 чистых и 7 гашеных марок. Из альбома наудачу извлекаются 3 марки и заменяются на чистые. После этого вновь наудачу извлекаются 5 марки. \\ \indent а) Найти вероятность того, что эти марки гашеные. \\ \indent б) Известно, что эти 5 марки гашеные; найти вероятность того, что первоначально извлеченные 3 марки --- чистые.
 
\end{zkrW}

\begin{zkrW}{20}\noindent 
	Всхожесть семян данного сорта растений оценивается с вероятностью, равной $0{,}7$. Какова вероятность того, что из 9 посеянных семян взойдут по меньшей мере 4?
 
\end{zkrW}

\begin{zkrW}{20}\noindent 
	Студент за все время обучения в вузе в среднем выполняет 2500 задач по математике. Вероятность неверно решить отдельную задачу при условии стопроцентного посещения и активной работы на всех занятиях равна $0{,}0032$, в противном случае --- $0{,}8$. Найти вероятность того, что за время обучения в вузе \\ \indent а) абсолютно прилежный студент решил неверно не менее чем 5 задачи; \\ \indent б) обычный студент решил правильно ровно 2180 задач; \\ \indent в) обычный студент неверно решил от 1940 до 1960 задач.
 
\end{zkrW}

\newpage\setcounter{zad}{0}



\begin{zkrW}{20}\noindent 
	Станция метрополитена оборудована тремя эскалаторами. Вероятность поломки в течение рабочего дня первого эскалатора равна $0{,}3$, второго --- $0{,}1$, третьего ---  $0{,}4$. Найдите вероятность того, что в течение рабочего дня будет исправен хотя бы один эскалатор.
 
\end{zkrW}

\begin{zkrW}{20}\noindent 
	Пелагия и Архип условились встретиться в определенном месте между 12:00 и 14:00. Каждый из них может прийти в любое время в течение указанного промежутка и ждет второго некоторое время. Пелагия ждет 50 минут, после чего уходит; Архип ждет 30 минут, после чего уходит. В 14:00 любой из них уходит, сколько бы до этого он ни ждал. Чему равна вероятность того, что встреча состоится в последние полчаса?
 
\end{zkrW}

\begin{zkrW}{20}\noindent 
	В альбоме 10 чистых и 7 гашеных марок. Из альбома наудачу извлекаются 2 марки и подвергаются гашению, а затем возвращаются в альбом. После этого вновь наудачу извлекаются 4 марки. \\ \indent а) Найти вероятность того, что эти марки гашеные. \\ \indent б) Известно, что эти 4 марки гашеные; найти вероятность того, что первоначально извлеченные 2 марки --- чистые.
 
\end{zkrW}

\begin{zkrW}{20}\noindent 
	В тестовом задании 7 вопросов, на каждый дано 6 варианта ответа, среди которых один правильный. Какова вероятность того, что, выбирая вариант ответа наугад, отвечающий правильно ответит не менее чем на 2 вопроса?
 
\end{zkrW}

\begin{zkrW}{20}\noindent 
	В ралли принимает участие 3750 экипажей. Каждый экипаж может сойти с дистанции из-за технических неполадок с вероятностью $0{,}4$, а из-за болезни водителя --- с вероятностью $0{,}0016$. Найти вероятность того, что \\ \indent а) не менее чем 3 экипажей сойдут с дистанции из-за болезни водителя; \\ \indent б) ровно 1305 экипажей не смогут продолжать ралли из-за технических неполадок; \\ \indent в) от 1425 до 1545 экипажей пострадают от технических проблем.
 
\end{zkrW}

\newpage\setcounter{zad}{0}



\begin{zkrW}{20}\noindent 
	Студент разыскивает нужную ему формулу в трех справочниках. Вероятность того, что формула содержится в первом, втором и третьем справочниках, равна соответственно $2/5$, $1/2$ и $3/7$. Найдите вероятность того, что эта формула содержится не менее чем в двух справочниках.
 
\end{zkrW}

\begin{zkrW}{20}\noindent 
	Вася и Андрей условились встретиться в определенном месте между 07:00 и 09:00. Каждый из них может прийти в любое время в течение указанного промежутка и ждет второго некоторое время. Вася ждет 60 минут, после чего уходит; Андрей ждет 40 минут, после чего уходит. В 09:00 любой из них уходит, сколько бы до этого он ни ждал. Чему равна вероятность того, что встреча состоится в последние полчаса?
 
\end{zkrW}

\begin{zkrW}{20}\noindent 
	В каждом из трех ящиков 9 белых и 12 голубых шаров. Из первого и второго ящиков наудачу извлекается по одному шару и кладется в третий ящик. Затем из третьего ящика извлекается один шар. \\ \indent а) Найти вероятность того, что он голубой. \\ \indent б) Известно, что этот шар голубой; найти вероятность того, что шары, извлеченные из первого и второго ящиков, --- голубые.
 
\end{zkrW}

\begin{zkrW}{20}\noindent 
	Изделия проверяются на стандартность. Вероятность того, что изделие не содержит брака, равна $0{,}1$. Найти вероятность того, что из 5 проверенных изделий бракованными окажутся по крайней мере 4.
 
\end{zkrW}

\begin{zkrW}{20}\noindent 
	Магазин закупил 400 телевизоров и столько же магнитол. Вероятность того, что отдельный телевизор окажется бракованным, равна $0{,}0125$, а вероятность того, что магнитола окажется бракованной, --- $0{,}8$. Найти вероятность того, что в этой закупке \\ \indent а) не более чем 2 телевизора окажутся бракованными; \\ \indent б) ровно 304 магнитол окажутся нерабочими; \\ \indent в) от 307 до 333 магнитол будут бракованными.
 
\end{zkrW}

\newpage\setcounter{zad}{0}



\begin{zkrW}{20}\noindent 
	Три лыжника съезжают с горы. Вероятность падения первого лыжника равна $3/4$, второго --- $1/4$, третьего --- $3/4$. Найдите вероятность того, что хотя бы два лыжника не упадут.
 
\end{zkrW}

\begin{zkrW}{20}\noindent 
	Полина и Петя условились встретиться в определенном месте между 05:00 и 09:00. Каждый из них может прийти в любое время в течение указанного промежутка и ждет второго некоторое время. Полина ждет 60 минут, после чего уходит; Петя ждет 50 минут, после чего уходит. В 09:00 любой из них уходит, сколько бы до этого он ни ждал. Чему равна вероятность того, что встреча состоится в последние полчаса?
 
\end{zkrW}

\begin{zkrW}{20}\noindent 
	В первом ящике 11 желтых и 10 синих шаров, а во втором 5 желтых и 6 синих. Из первого ящика во второй перекладываются 3 наудачу извлеченных шара. После этого из второго ящика наудачу извлекается один шар. \\ \indent а) Найти вероятность того, что он желтый. \\ \indent б) Известно, что этот шар желтый; найти вероятность того, что извлеченные из первого ящика шары --- желтые.
 
\end{zkrW}

\begin{zkrW}{20}\noindent 
	Рабочий обслуживает 7 однотипных станков. Вероятность того, что станок потребует внимания рабочего в течение дня, равна $5/9$. Найти вероятность того, что в течение дня этих требований будет более чем 2.
 
\end{zkrW}

\begin{zkrW}{20}\noindent 
	Мастерская за год ремонтирует 150 мобильных телефонов. Вероятность неисправности в механической части отдельного телефона равна $0{,}6$, в электронной части --- $0{,}006$. Найти вероятность того, что среди телефонов, отремонтированных за год, \\ \indent а) имели неисправности в механической части от 87 до 94 экземпляров; \\ \indent б) имели неисправности в электронной части по меньшей мере 3 телефонов; \\ \indent в) ровно 85 телефонов имели проблемы в механической части.
 
\end{zkrW}

\newpage\setcounter{zad}{0}



\begin{zkrW}{20}\noindent 
	Прибор, работающий в течение времени $t$, состоит из трех узлов, каждый из которых независимо от других может за это время выйти из строя. Неисправность хотя бы одного узла выводит прибор из строя целиком. Вероятность безотказной работы в течение времени $t$ первого узла равна $0{,}3$, второго --- $0{,}7$, третьего --- $0{,}1$. Найдите вероятность того, что в течение времени $t$ прибор выйдет из строя.
 
\end{zkrW}

\begin{zkrW}{20}\noindent 
	Вася и Полина условились встретиться в определенном месте между 09:00 и 13:00. Каждый из них может прийти в любое время в течение указанного промежутка и ждет второго некоторое время. Вася ждет 70 минут, после чего уходит; Полина ждет 60 минут, после чего уходит. В 13:00 любой из них уходит, сколько бы до этого он ни ждал. Чему равна вероятность того, что встреча состоится в первые полчаса?
 
\end{zkrW}

\begin{zkrW}{20}\noindent 
	В каждом из трех ящиков 7 голубых и 8 желтых шаров. Из первого ящика в третий перекладывают два наудачу выбранных шара, а из второго ящика в третий перекладывают один наудачу взятый шар. Затем из третьего ящика извлекается один шар. \\ \indent а) Найти вероятность того, что он желтый. \\ \indent б) Известно, что этот шар желтый; найти вероятность того, что из первого ящика во второй переложили желтые шары.
 
\end{zkrW}

\begin{zkrW}{20}\noindent 
	Эксплуатируется устройство, состоящее из 5 независимо работающих элементов. Вероятность отказа каждого из них за время работы устройства равна $5/7$. Найти вероятность того, что за время работы устройства откажут по меньшей мере 4 элемента.
 
\end{zkrW}

\begin{zkrW}{20}\noindent 
	Предполагая рождение ребенка в любой день года равновозможным, найти вероятность того, что в группе из 3600 человек \\ \indent а) по меньшей мере 3 родились 23 октября; \\ \indent б) ровно 619 родились осенью; \\ \indent в) от 662 до 734 родились весной.
 
\end{zkrW}

\newpage\setcounter{zad}{0}



\begin{zkrW}{20}\noindent 
	Среди 18 поступающих в ремонт часов 9 нуждаются в общей чистке механизма. Какова вероятность того, что среди взятых одновремено наудачу 10 часов по меньшей мере 2 нуждаются в общей чистке механизма?
 
\end{zkrW}

\begin{zkrW}{20}\noindent 
	Алексей и Андрей условились встретиться в определенном месте между 07:00 и 11:00. Каждый из них может прийти в любое время в течение указанного промежутка и ждет второго некоторое время. Алексей ждет 80 минут, после чего уходит; Андрей ждет 50 минут, после чего уходит. В 11:00 любой из них уходит, сколько бы до этого он ни ждал. Чему равна вероятность того, что Алексей опоздает менее чем на полчаса?
 
\end{zkrW}

\begin{zkrW}{20}\noindent 
	В альбоме 7 чистых и 9 гашеных марок. Из альбома наудачу извлекаются 3 марки и заменяются на чистые. После этого вновь наудачу извлекаются 4 марки. \\ \indent а) Найти вероятность того, что эти марки чистые. \\ \indent б) Известно, что эти 4 марки чистые; найти вероятность того, что первоначально извлеченные 3 марки --- гашеные.
 
\end{zkrW}

\begin{zkrW}{20}\noindent 
	Вероятность правильного оформления доверенности у нотариуса Иванова-Ежова равна $5/7$. В течение одного часа нотариус Иванов-Ежов оформил 9 доверенности. Какова вероятность, что по меньшей мере 4 из них оказались оформлены неправильно?
 
\end{zkrW}

\begin{zkrW}{20}\noindent 
	Вероятность выиграть отдельному игроку 1000 рублей в игре <<Кто хочет стать миллионером>> равна $0{,}2$, а 32000 рублей --- $0{,}0025$. За сезон в этой игре принимает участие 1600 человек. Найти вероятность того, что за сезон \\ \indent а) 1000 рублей получат ровно 365 человек; \\ \indent б) 1000 рублей получат от 288 до 301 человек; \\ \indent в) не менее чем 4 человек получат крупный выигрыш в 32000 рублей.
 
\end{zkrW}

\newpage\setcounter{zad}{0}



\begin{zkrW}{20}\noindent 
	Три студента пришли сдавать экзамен. Вероятность того, что первый студент сдаст экзамен, равна $4/9$, второй --- $5/8$, третий --- $7/8$. Найдите вероятность того, что хотя бы двое сдадут экзамен.
 
\end{zkrW}

\begin{zkrW}{20}\noindent 
	Петя и Катя условились встретиться в определенном месте между 02:00 и 03:00. Каждый из них может прийти в любое время в течение указанного промежутка и ждет второго некоторое время. Петя ждет 40 минут, после чего уходит; Катя ждет 20 минут, после чего уходит. В 03:00 любой из них уходит, сколько бы до этого он ни ждал. Чему равна вероятность того, что Петя и Катя встретятся?
 
\end{zkrW}

\begin{zkrW}{20}\noindent 
	В каждом из трех ящиков 12 зеленых и 12 голубых шаров. Из первого ящика в третий перекладывают два наудачу выбранных шара, а из второго ящика в третий перекладывают один наудачу взятый шар. Затем из третьего ящика извлекается один шар. \\ \indent а) Найти вероятность того, что он голубой. \\ \indent б) Известно, что этот шар голубой; найти вероятность того, что из первого ящика во второй переложили голубые шары.
 
\end{zkrW}

\begin{zkrW}{20}\noindent 
	Игрок набрасывает кольца на колышек. Вероятность удачи при этом равна $0{,}7$. Найти вероятность того, что из 5 колец на колышек попадут по крайней мере 2.
 
\end{zkrW}

\begin{zkrW}{20}\noindent 
	Вероятность появления опечатки на отдельной странице книги равна $0{,}002$, а погрешности верстки --- $0{,}4$. Найти вероятность того, что в книге из 150 страниц \\ \indent а) хотя бы 3 страниц будут иметь опечатки; \\ \indent б) от 55 до 65 страниц будут иметь погрешности верстки; \\ \indent в) погрешности верстки будут присутствовать ровно на 68 страницах.
 
\end{zkrW}

\newpage\setcounter{zad}{0}



\begin{zkrW}{20}\noindent 
	Студент успел подготовить к экзамену 11 вопросов из 23 пяти. Какова вероятность того, что из 5 наудачу выбранных вопросов студент знает не менее чем 4.
 
\end{zkrW}

\begin{zkrW}{20}\noindent 
	Катя и Алексей условились встретиться в определенном месте между 17:00 и 19:00. Каждый из них может прийти в любое время в течение указанного промежутка и ждет второго некоторое время. Катя ждет 40 минут, после чего уходит; Алексей ждет 60 минут, после чего уходит. В 19:00 любой из них уходит, сколько бы до этого он ни ждал. Чему равна вероятность того, что Катя опоздает менее чем на полчаса?
 
\end{zkrW}

\begin{zkrW}{20}\noindent 
	В альбоме 10 чистых и 10 гашеных марок. Из альбома изымаются 2 наудачу извлеченные марки. После этого из альбома вновь наудачу извлекаются 4 марки. \\ \indent а) Найти вероятность того, что эти марки чистые. \\ \indent б) Известно, что эти 4 марки чистые; найти вероятность того, что первоначально изъятые 2 марки --- гашеные.
 
\end{zkrW}

\begin{zkrW}{20}\noindent 
	В случайно выбранной семье 5 детей. Считая вероятности рождения мальчика и девочки одинаковыми, определить вероятность того, что в выбранной семье окажется не менее чем 3 мальчика.
 
\end{zkrW}

\begin{zkrW}{20}\noindent 
	В ралли принимает участие 600 экипажей. Каждый экипаж может сойти с дистанции из-за технических неполадок с вероятностью $0{,}4$, а из-за болезни водителя --- с вероятностью $0{,}015$. Найти вероятность того, что \\ \indent а) по меньшей мере 4 экипажей сойдут с дистанции из-за болезни водителя; \\ \indent б) ровно 252 экипажей не смогут продолжать ралли из-за технических неполадок; \\ \indent в) от 211 до 228 экипажей пострадают от технических проблем.
 
\end{zkrW}

\newpage\setcounter{zad}{0}



\begin{zkrW}{20}\noindent 
	Среди 24 поступающих в ремонт часов 7 нуждаются в общей чистке механизма. Какова вероятность того, что среди взятых одновремено наудачу 5 часов более чем 4 нуждаются в общей чистке механизма?
 
\end{zkrW}

\begin{zkrW}{20}\noindent 
	Саша и Аня условились встретиться в определенном месте между 03:00 и 05:00. Каждый из них может прийти в любое время в течение указанного промежутка и ждет второго некоторое время. Саша ждет 30 минут, после чего уходит; Аня ждет 40 минут, после чего уходит. В 05:00 любой из них уходит, сколько бы до этого он ни ждал. Чему равна вероятность того, что Саша опоздает более чем на полчаса?
 
\end{zkrW}

\begin{zkrW}{20}\noindent 
	В альбоме 9 чистых и 11 гашеных марок. Из альбома наудачу извлекаются 4 марки и подвергаются гашению, а затем возвращаются в альбом. После этого вновь наудачу извлекаются 4 марки. \\ \indent а) Найти вероятность того, что эти марки гашеные. \\ \indent б) Известно, что эти 4 марки гашеные; найти вероятность того, что первоначально извлеченные 4 марки --- чистые.
 
\end{zkrW}

\begin{zkrW}{20}\noindent 
	При данном технологическом процессе $50\%$ всех сходящих с конвейера автозавода автомобилей имеют цвет «металлик». Найти вероятность того, что из 5 случайно отобранных автомобилей менее чем 3 будут иметь этот цвет.
 
\end{zkrW}

\begin{zkrW}{20}\noindent 
	Магазин закупил 400 телевизоров и столько же магнитол. Вероятность того, что отдельный телевизор окажется бракованным, равна $0{,}0025$, а вероятность того, что магнитола окажется бракованной, --- $0{,}2$. Найти вероятность того, что в этой закупке \\ \indent а) по крайней мере 4 телевизора окажутся бракованными; \\ \indent б) ровно 78 магнитол окажутся нерабочими; \\ \indent в) от 82 до 91 магнитол будут бракованными.
 
\end{zkrW}

\newpage\setcounter{zad}{0}



\begin{zkrW}{20}\noindent 
	Игра проводится до выигрыша одним из двух игроков двух партий подряд (ничьи исключаются). Вероятность выигрыша партии каждым из игроков равна $0{,}5$ и не зависит от исходов предыдущих партий. Найдите вероятность того, что игра окончится до 3-й партии.
 
\end{zkrW}

\begin{zkrW}{20}\noindent 
	Владимир Путин и Барак Обама условились встретиться в определенном месте между 19:00 и 23:00. Каждый из них может прийти в любое время в течение указанного промежутка и ждет второго некоторое время. Владимир Путин ждет 80 минут, после чего уходит; Барак Обама ждет 60 минут, после чего уходит. В 23:00 любой из них уходит, сколько бы до этого он ни ждал. Чему равна вероятность того, что Владимир Путин придет раньше, чем Барак Обама?
 
\end{zkrW}

\begin{zkrW}{20}\noindent 
	В альбоме 9 чистых и 10 гашеных марок. Из альбома наудачу извлекаются 4 марки и подвергаются гашению, а затем возвращаются в альбом. После этого вновь наудачу извлекаются 3 марки. \\ \indent а) Найти вероятность того, что эти марки гашеные. \\ \indent б) Известно, что эти 3 марки гашеные; найти вероятность того, что первоначально извлеченные 4 марки --- чистые.
 
\end{zkrW}

\begin{zkrW}{20}\noindent 
	В магазин вошли 6 покупателей. Найти вероятность того, что не более чем 4 из них совершат покупки, если вероятность совершить покупку для каждого из них одинакова и равна $1/4$.
 
\end{zkrW}

\begin{zkrW}{20}\noindent 
	В лотерее разыгрываются крупные и мелкие выигрыши. Вероятность того, что на лотерейный билет выпадет крупный выигрыш, равна $0{,}02$, а мелкий --- $0{,}9$. Куплено 400 билетов. Найти вероятность того, что \\ \indent а) крупных выигрышей будет ровно 4; \\ \indent б) мелких выигрышей будет ровно 317; \\ \indent в) мелких выигрышей будет от 346 до 349.
 
\end{zkrW}

\newpage\setcounter{zad}{0}



\begin{zkrW}{20}\noindent 
	В двух урнах находятся шары, отличающиеся только цветом. В первой урне 14 белых, 7 черных и 5 красных шаров. Во второй урне 8 белых, 9 черных и 9 красных. Из каждой урны наудачу извлекаются по одному шару. Какова вероятность того, что извлеченные шары будут одинакового цвета?
 
\end{zkrW}

\begin{zkrW}{20}\noindent 
	Степан Степаныч Пивораки и Хоттабыч условились встретиться в определенном месте между 14:00 и 18:00. Каждый из них может прийти в любое время в течение указанного промежутка и ждет второго некоторое время. Степан Степаныч Пивораки ждет 60 минут, после чего уходит; Хоттабыч ждет 80 минут, после чего уходит. В 18:00 любой из них уходит, сколько бы до этого он ни ждал. Чему равна вероятность того, что встреча состоится в первые полчаса?
 
\end{zkrW}

\begin{zkrW}{20}\noindent 
	В каждом из трех ящиков 5 зеленых и 6 красных шаров. Из первого ящика в третий перекладывают два наудачу выбранных шара, а из второго ящика в третий перекладывают один наудачу взятый шар. Затем из третьего ящика извлекается один шар. \\ \indent а) Найти вероятность того, что он зеленый. \\ \indent б) Известно, что этот шар зеленый; найти вероятность того, что из первого ящика во второй переложили зеленые шары.
 
\end{zkrW}

\begin{zkrW}{20}\noindent 
	В студии находятся 8 телевизионных камер. Для каждой камеры вероятность того, что она включена в данный момент, равна $6/7$. Найти вероятность того, что в данный момент оказались выключены более чем 4 камеры.
 
\end{zkrW}

\begin{zkrW}{20}\noindent 
	В ралли принимает участие 150 экипажей. Каждый экипаж может сойти с дистанции из-за технических неполадок с вероятностью $0{,}6$, а из-за болезни водителя --- с вероятностью $0{,}02$. Найти вероятность того, что \\ \indent а) не менее чем 3 экипажей сойдут с дистанции из-за болезни водителя; \\ \indent б) ровно 92 экипажей не смогут продолжать ралли из-за технических неполадок; \\ \indent в) от 101 до 102 экипажей пострадают от технических проблем.
 
\end{zkrW}

\newpage\setcounter{zad}{0}



\begin{zkrW}{20}\noindent 
	Мастер обслуживает четыре станка, работающих независимо друг от друга. Вероятность того, что первыйй станок в течение смены потребует внимания мастера, равна $0{,}5$, второй --- $0{,}5$, третий --- $0{,}4$ и четвертый --- $0{,}2$. Найдите вероятность того, что в течение смены хотя бы один станок не потребует внимания мастера.
 
\end{zkrW}

\begin{zkrW}{20}\noindent 
	Атос и д'Артаньян условились встретиться в определенном месте между 13:00 и 15:00. Каждый из них может прийти в любое время в течение указанного промежутка и ждет второго некоторое время. Атос ждет 60 минут, после чего уходит; д'Артаньян ждет 40 минут, после чего уходит. В 15:00 любой из них уходит, сколько бы до этого он ни ждал. Чему равна вероятность того, что Атос опоздает менее чем на полчаса?
 
\end{zkrW}

\begin{zkrW}{20}\noindent 
	В каждом из трех ящиков 7 красных и 10 белых шаров. Из первого и второго ящиков наудачу извлекается по одному шару и кладется в третий ящик. Затем из третьего ящика извлекается один шар. \\ \indent а) Найти вероятность того, что он красный. \\ \indent б) Известно, что этот шар красный; найти вероятность того, что шары, извлеченные из первого и второго ящиков, --- красные.
 
\end{zkrW}

\begin{zkrW}{20}\noindent 
	$95\%$ изделий данного предприятия — это продукция высшего сорта. Некто приобрел 8 изделий, изготовленных на этом предприятии. Чему равна вероятность того, что по крайней мере 3 из них — высшего сорта?
 
\end{zkrW}

\begin{zkrW}{20}\noindent 
	Предполагая рождение ребенка в любой день года равновозможным, найти вероятность того, что в группе из 150 человек \\ \indent а) по крайней мере 5 родились 10 августа; \\ \indent б) ровно 54 родились осенью; \\ \indent в) от 54 до 63 родились весной.
 
\end{zkrW}

\newpage\setcounter{zad}{0}



\begin{zkrW}{20}\noindent 
	В двух урнах находятся шары, отличающиеся только цветом. В первой урне 5 белых, 13 черных и 13 красных шаров. Во второй урне 15 белых, 9 черных и 8 красных. Из каждой урны наудачу извлекаются по одному шару. Какова вероятность того, что извлеченные шары будут одинакового цвета?
 
\end{zkrW}

\begin{zkrW}{20}\noindent 
	Света и Надя условились встретиться в определенном месте между 02:00 и 04:00. Каждый из них может прийти в любое время в течение указанного промежутка и ждет второго некоторое время. Света ждет 40 минут, после чего уходит; Надя ждет 60 минут, после чего уходит. В 04:00 любой из них уходит, сколько бы до этого он ни ждал. Чему равна вероятность того, что Света опоздает менее чем на полчаса?
 
\end{zkrW}

\begin{zkrW}{20}\noindent 
	В альбоме 12 чистых и 11 гашеных марок. Из альбома наудачу извлекаются 4 марки и подвергаются гашению, а затем возвращаются в альбом. После этого вновь наудачу извлекаются 4 марки. \\ \indent а) Найти вероятность того, что эти марки гашеные. \\ \indent б) Известно, что эти 4 марки гашеные; найти вероятность того, что первоначально извлеченные 4 марки --- чистые.
 
\end{zkrW}

\begin{zkrW}{20}\noindent 
	В среднем $75\%$ акций на аукционах продаются по первоначально заявленной цене. Найти вероятность того, что из 8 пакетов акций в результате торгов по первоначально заявленной цене останутся непроданными ровно 4.
 
\end{zkrW}

\begin{zkrW}{20}\noindent 
	Предполагая рождение ребенка в любой день года равновозможным, найти вероятность того, что в группе из 3600 человек \\ \indent а) хотя бы 4 родились 30 октября; \\ \indent б) ровно 2948 родились осенью; \\ \indent в) от 3078 до 3078 родились весной.
 
\end{zkrW}

\newpage\setcounter{zad}{0}



\begin{zkrW}{20}\noindent 
	Игра проводится до выигрыша одним из двух игроков двух партий подряд (ничьи исключаются). Вероятность выигрыша партии каждым из игроков равна $0{,}5$ и не зависит от исходов предыдущих партий. Найдите вероятность того, что игра окончится до 2-й партии.
 
\end{zkrW}

\begin{zkrW}{20}\noindent 
	Дэйв Гилмор и Ян Пэйс условились встретиться в определенном месте между 06:00 и 08:00. Каждый из них может прийти в любое время в течение указанного промежутка и ждет второго некоторое время. Дэйв Гилмор ждет 50 минут, после чего уходит; Ян Пэйс ждет 60 минут, после чего уходит. В 08:00 любой из них уходит, сколько бы до этого он ни ждал. Чему равна вероятность того, что встреча состоится в последние полчаса?
 
\end{zkrW}

\begin{zkrW}{20}\noindent 
	В каждом из трех ящиков 12 белых и 12 зеленых шаров. Из первого ящика в третий перекладывают два наудачу выбранных шара, а из второго ящика в третий перекладывают один наудачу взятый шар. Затем из третьего ящика извлекается один шар. \\ \indent а) Найти вероятность того, что он белый. \\ \indent б) Известно, что этот шар белый; найти вероятность того, что из первого ящика во второй переложили белые шары.
 
\end{zkrW}

\begin{zkrW}{20}\noindent 
	При массовом производстве полупроводниковых диодов вероятность брака при формовке равна $5/9$. Какова вероятность того, что из 9 взятых диодов будет хотя бы 3 бракованных.
 
\end{zkrW}

\begin{zkrW}{20}\noindent 
	Вероятность появления опечатки на отдельной странице книги равна $0{,}01$, а погрешности верстки --- $0{,}5$. Найти вероятность того, что в книге из 900 страниц \\ \indent а) по меньшей мере 2 страниц будут иметь опечатки; \\ \indent б) от 414 до 418 страниц будут иметь погрешности верстки; \\ \indent в) погрешности верстки будут присутствовать ровно на 391 страницах.
 
\end{zkrW}

\newpage\setcounter{zad}{0}



\begin{zkrW}{20}\noindent 
	Студент разыскивает нужную ему формулу в трех справочниках. Вероятность того, что формула содержится в первом, втором и третьем справочниках, равна соответственно $3/5$, $4/9$ и $1/2$. Найдите вероятность того, что эта формула содержится не менее чем в двух справочниках.
 
\end{zkrW}

\begin{zkrW}{20}\noindent 
	Бэтмен и Робин условились встретиться в определенном месте между 05:00 и 09:00. Каждый из них может прийти в любое время в течение указанного промежутка и ждет второго некоторое время. Бэтмен ждет 80 минут, после чего уходит; Робин ждет 70 минут, после чего уходит. В 09:00 любой из них уходит, сколько бы до этого он ни ждал. Чему равна вероятность того, что встреча произойдет не ранее чем без четверти 09:00?
 
\end{zkrW}

\begin{zkrW}{20}\noindent 
	В каждом из трех ящиков 11 синих и 9 зеленых шаров. Из первого ящика в третий перекладывают два наудачу выбранных шара, а из второго ящика в третий перекладывают один наудачу взятый шар. Затем из третьего ящика извлекается один шар. \\ \indent а) Найти вероятность того, что он синий. \\ \indent б) Известно, что этот шар синий; найти вероятность того, что из первого ящика во второй переложили синие шары.
 
\end{zkrW}

\begin{zkrW}{20}\noindent 
	Опрошены 8 человек. Найти вероятность того, что менее чем 2 из них родились осенью.
 
\end{zkrW}

\begin{zkrW}{20}\noindent 
	Вероятность выиграть отдельному игроку 1000 рублей в игре <<Кто хочет стать миллионером>> равна $0{,}1$, а 32000 рублей --- $0{,}0016$. За сезон в этой игре принимает участие 2500 человек. Найти вероятность того, что за сезон \\ \indent а) 1000 рублей получат ровно 265 человек; \\ \indent б) 1000 рублей получат от 212 до 275 человек; \\ \indent в) хотя бы 2 человек получат крупный выигрыш в 32000 рублей.
 
\end{zkrW}

\newpage\setcounter{zad}{0}



\begin{zkrW}{20}\noindent 
	Произведено три выстрела по цели из орудия. Вероятность попадания при первом выстреле равна $5/8$; при втором --- $5/8$; при третьем --- $7/9$. Определить вероятность того, что будет хотя бы одно попадание.
 
\end{zkrW}

\begin{zkrW}{20}\noindent 
	Владимир Путин и Барак Обама условились встретиться в определенном месте между 08:00 и 12:00. Каждый из них может прийти в любое время в течение указанного промежутка и ждет второго некоторое время. Владимир Путин ждет 80 минут, после чего уходит; Барак Обама ждет 60 минут, после чего уходит. В 12:00 любой из них уходит, сколько бы до этого он ни ждал. Чему равна вероятность того, что Владимир Путин опоздает более чем на полчаса?
 
\end{zkrW}

\begin{zkrW}{20}\noindent 
	В альбоме 5 чистых и 12 гашеных марок. Из альбома наудачу извлекаются 4 марки и заменяются на чистые. После этого вновь наудачу извлекаются 5 марки. \\ \indent а) Найти вероятность того, что эти марки чистые. \\ \indent б) Известно, что эти 5 марки чистые; найти вероятность того, что первоначально извлеченные 4 марки --- гашеные.
 
\end{zkrW}

\begin{zkrW}{20}\noindent 
	Наблюдениями установлено, что в некоторой местности в ноябре бывает в среднем 21 дождливых дней. Какова вероятность того, что в следующем году из 8 первых дней сентября по меньшей мере 2 окажутся дождливыми?
 
\end{zkrW}

\begin{zkrW}{20}\noindent 
	Предполагая рождение ребенка в любой день года равновозможным, найти вероятность того, что в группе из 100 человек \\ \indent а) по меньшей мере 4 родились 26 февраля; \\ \indent б) ровно 86 родились осенью; \\ \indent в) от 84 до 87 родились весной.
 
\end{zkrW}

\newpage\setcounter{zad}{0}



\begin{zkrW}{20}\noindent 
	В коробке смешаны электролампы одинакового размера и формы: по 100 Вт --- 10 штук, по 75 Вт --- 11 штук. Вынуты наудачу три лампы. Какова вероятность того, что хотя бы две из них по 100 Вт?
 
\end{zkrW}

\begin{zkrW}{20}\noindent 
	Андрей и Алексей условились встретиться в определенном месте между 17:00 и 21:00. Каждый из них может прийти в любое время в течение указанного промежутка и ждет второго некоторое время. Андрей ждет 60 минут, после чего уходит; Алексей ждет 80 минут, после чего уходит. В 21:00 любой из них уходит, сколько бы до этого он ни ждал. Чему равна вероятность того, что встреча состоится в последние полчаса?
 
\end{zkrW}

\begin{zkrW}{20}\noindent 
	В альбоме 12 чистых и 6 гашеных марок. Из альбома изымаются 2 наудачу извлеченные марки. После этого из альбома вновь наудачу извлекаются 5 марки. \\ \indent а) Найти вероятность того, что эти марки гашеные. \\ \indent б) Известно, что эти 5 марки гашеные; найти вероятность того, что первоначально изъятые 2 марки --- чистые.
 
\end{zkrW}

\begin{zkrW}{20}\noindent 
	Всхожесть семян лимона равна $65\%$. Найти вероятность того, что из 5 посеянных семян взойдут по крайней мере 3.
 
\end{zkrW}

\begin{zkrW}{20}\noindent 
	Мастерская за год ремонтирует 100 мобильных телефонов. Вероятность неисправности в механической части отдельного телефона равна $0{,}2$, в электронной части --- $0{,}05$. Найти вероятность того, что среди телефонов, отремонтированных за год, \\ \indent а) имели неисправности в механической части от 17 до 22 экземпляров; \\ \indent б) имели неисправности в электронной части менее чем 5 телефонов; \\ \indent в) ровно 21 телефонов имели проблемы в механической части.
 
\end{zkrW}

\newpage\setcounter{zad}{0}



\begin{zkrW}{20}\noindent 
	В коробке смешаны электролампы одинакового размера и формы: по 100 Вт --- 13 штук, по 75 Вт --- 7 штук. Вынуты наудачу три лампы. Какова вероятность того, что хотя бы две из них по 100 Вт?
 
\end{zkrW}

\begin{zkrW}{20}\noindent 
	Карик и Иван Гермогенович Енотов условились встретиться в определенном месте между 10:00 и 12:00. Каждый из них может прийти в любое время в течение указанного промежутка и ждет второго некоторое время. Карик ждет 50 минут, после чего уходит; Иван Гермогенович Енотов ждет 40 минут, после чего уходит. В 12:00 любой из них уходит, сколько бы до этого он ни ждал. Чему равна вероятность того, что встреча состоится в первые двадцать минут?
 
\end{zkrW}

\begin{zkrW}{20}\noindent 
	В каждом из трех ящиков 12 белых и 5 зеленых шаров. Из первого и второго ящиков наудачу извлекается по одному шару и кладется в третий ящик. Затем из третьего ящика извлекается один шар. \\ \indent а) Найти вероятность того, что он белый. \\ \indent б) Известно, что этот шар белый; найти вероятность того, что шары, извлеченные из первого и второго ящиков, --- белые.
 
\end{zkrW}

\begin{zkrW}{20}\noindent 
	Предполагается, что $50\%$ открывающихся малых предприятий прекращает свою деятельность в течение года. Какова вероятность того, что из 7 малых предприятий по прошествии года продолжат работать не более чем 3?
 
\end{zkrW}

\begin{zkrW}{20}\noindent 
	Магазин закупил 400 телевизоров и столько же магнитол. Вероятность того, что отдельный телевизор окажется бракованным, равна $0{,}0125$, а вероятность того, что магнитола окажется бракованной, --- $0{,}9$. Найти вероятность того, что в этой закупке \\ \indent а) по меньшей мере 2 телевизора окажутся бракованными; \\ \indent б) ровно 414 магнитол окажутся нерабочими; \\ \indent в) от 353 до 374 магнитол будут бракованными.
 
\end{zkrW}

\newpage\setcounter{zad}{0}



\begin{zkrW}{20}\noindent 
	Три студента пришли сдавать экзамен. Вероятность того, что первый студент сдаст экзамен, равна $4/7$, второй --- $1/4$, третий --- $5/9$. Найдите вероятность того, что хотя бы двое сдадут экзамен.
 
\end{zkrW}

\begin{zkrW}{20}\noindent 
	Габриэль Гарсия Маркес и Джон Фаулз условились встретиться в определенном месте между 13:00 и 15:00. Каждый из них может прийти в любое время в течение указанного промежутка и ждет второго некоторое время. Габриэль Гарсия Маркес ждет 50 минут, после чего уходит; Джон Фаулз ждет 40 минут, после чего уходит. В 15:00 любой из них уходит, сколько бы до этого он ни ждал. Чему равна вероятность того, что Габриэль Гарсия Маркес опоздает более чем на полчаса?
 
\end{zkrW}

\begin{zkrW}{20}\noindent 
	В альбоме 12 чистых и 6 гашеных марок. Из альбома наудачу извлекаются 2 марки и подвергаются гашению, а затем возвращаются в альбом. После этого вновь наудачу извлекаются 2 марки. \\ \indent а) Найти вероятность того, что эти марки гашеные. \\ \indent б) Известно, что эти 2 марки гашеные; найти вероятность того, что первоначально извлеченные 2 марки --- чистые.
 
\end{zkrW}

\begin{zkrW}{20}\noindent 
	Самолет имеет 6 двигателя. Вероятность нормальной работы каждого двигателя равна $2/9$. Найти вероятность того, что в полете по меньшей мере в 2 двигателях возникнут неполадки.
 
\end{zkrW}

\begin{zkrW}{20}\noindent 
	Студент за все время обучения в вузе в среднем выполняет 2400 задач по математике. Вероятность неверно решить отдельную задачу при условии стопроцентного посещения и активной работы на всех занятиях равна $0{,}0025$, в противном случае --- $0{,}6$. Найти вероятность того, что за время обучения в вузе \\ \indent а) абсолютно прилежный студент решил неверно более чем 5 задачи; \\ \indent б) обычный студент решил правильно ровно 1469 задач; \\ \indent в) обычный студент неверно решил от 1411 до 1469 задач.
 
\end{zkrW}

\newpage\setcounter{zad}{0}

