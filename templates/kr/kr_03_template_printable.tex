\documentclass[12pt,a4paper,%draft
]{scrartcl}


\usepackage[
	left=2cm,
	right=1.5cm,
	top=2.5cm,	
	bottom=0.5cm,
%	offset=.5cm
]{geometry}

\usepackage{mysty}


\lhead{\textsl{\type\ № \arabic{dzn}}}
\chead{
}
\rhead{ФИО:\phantom{aaaaaaaazzz bbbbbbbb ccccccc}
}
\lfoot{%\page\arabic{pag}
}
\cfoot{}
\rfoot{%\arabic{page}
}




%\newcommand{\group}{ОМ-121}
%\newcommand{\name}{foo}

\newcommand{\type}{Контрольная работа}
\setcounter{dzn}{1}

\begin{document}

%%%%%%%%%%%%%%%%%%%%%%%%%%%%%%%%%%%%%%

\begin{zkrW}{20}\noindent Абонент забыл последнюю цифру номера телефона и поэтому набирает ее наугад. Определить вероятность того, что ему придется звонить ровно в 3 места.
 \end{zkrW}




\begin{zkrW}{20}\noindent Полина и Вася условились встретиться в определенном месте между 01:00 и 02:00. Каждый из них может прийти в любое время в течение указанного промежутка и ждет второго некоторое время. Полина ждет 50 минут, после чего уходит; Вася ждет 20 минут, после чего уходит. В 02:00 любой из них уходит, сколько бы до этого он ни ждал. Чему равна вероятность того, что встреча состоится в первые полчаса?
 \end{zkrW}




\begin{zkrW}{20}\noindent 
В каждом из трех ящиков 12 желтых и 11 черных шаров. Из первого и второго ящиков наудачу извлекается по одному шару и кладется в третий ящик. Затем из третьего ящика извлекается один шар. а) Найти вероятность того, что он желтый. б) Известно, что этот шар желтый; найти вероятность того, что шары, извлеченные из первого и второго ящиков, --- желтые.

\end{zkrW}




\begin{zkrW}{20}\noindent Игрок набрасывает кольца на колышек. Вероятность удачи при этом равна $3/4$. Найти вероятность того, что из 9 колец на колышек попадут менее чем 4.
 \end{zkrW}





\begin{zkrW}{20}\noindent В лотерее разыгрываются крупные и мелкие выигрыши. Вероятность того, что на лотерейный билет выпадет крупный выигрыш, равна $0{,}0225$, а мелкий --- $0{,}2$. Куплено 400 билетов. Найти вероятность того, что \\ \indent а) крупных выигрышей будет не более чем 5; \\ \indent б) мелких выигрышей будет ровно 82; \\ \indent в) мелких выигрышей будет от 68 до 78.
 \end{zkrW}


\newpage\setcounter{zad}{0}




%%%%%%%%%%%%%%%%%%%%%%%%%%%%%%%%%%%%%%

\begin{zkrW}{20}\noindent Среди 20 лампочек 7 стандартные. Одновременно берут наудачу 8 лампочки. Найдите вероятность того, что хотя бы одна из них нестандартная.
 \end{zkrW}

\vfill


\begin{zkrW}{20}\noindent Иван Гермогенович Енотов и Карик условились встретиться в определенном месте между 07:00 и 11:00. Каждый из них может прийти в любое время в течение указанного промежутка и ждет второго некоторое время. Иван Гермогенович Енотов ждет 70 минут, после чего уходит; Карик ждет 80 минут, после чего уходит. В 11:00 любой из них уходит, сколько бы до этого он ни ждал. Чему равна вероятность того, что встреча состоится в первые полчаса?
 \end{zkrW}

\vfill


\begin{zkrW}{20}\noindent В первом ящике 7 голубых и 11 красных шаров, а во втором 6 голубых и 9 красных. Из первого ящика во второй перекладываются 2 наудачу извлеченных шара. После этого из второго ящика наудачу извлекается один шар. а) Найти вероятность того, что он красный. б) Известно, что этот шар красный; найти вероятность того, что извлеченные из первого ящика шары --- красные.
 \end{zkrW}

\vfill



\begin{zkrW}{20}\noindent На автобазе 7 машин. Вероятность выхода каждой из них на линию равна $0{,}6$. Найти вероятность того, что на линию по каким-либо причинам не смогут выйти не менее чем 3 машины.
 \end{zkrW}

\vfill



\begin{zkrW}{20}\noindent Студент за все время обучения в вузе в среднем выполняет 3750 задач по математике. Вероятность неверно решить отдельную задачу при условии стопроцентного посещения и активной работы на всех занятиях равна $0{,}0016$, в противном случае --- $0{,}6$. Найти вероятность того, что за время обучения в вузе \\ \indent а) абсолютно прилежный студент решил неверно менее чем 3 задач; \\ \indent б) обычный студент решил правильно ровно 1957 задач; \\ \indent в) обычный студент неверно решил от 2182 до 2295 задач.
 \end{zkrW}


\vfill\newpage\setcounter{zad}{0}





%%%%%%%%%%%%%%%%%%%%%%%%%%%%%%%%%%%%%%

\begin{zkrW}{20}\noindent Студент разыскивает нужную ему формулу в трех справочниках. Вероятность того, что формула содержится в первом, втором и третьем справочниках, равна соответственно $8/9$, $1/3$ и $3/8$. Найдите вероятность того, что эта формула содержится не менее чем в двух справочниках.
 \end{zkrW}

\vfill


\begin{zkrW}{20}\noindent Вася и Саша условились встретиться в определенном месте между 16:00 и 19:00. Каждый из них может прийти в любое время в течение указанного промежутка и ждет второго некоторое время. Вася ждет 70 минут, после чего уходит; Саша ждет 50 минут, после чего уходит. В 19:00 любой из них уходит, сколько бы до этого он ни ждал. Чему равна вероятность того, что встреча состоится в последние полчаса?
 \end{zkrW}

\vfill


\begin{zkrW}{20}\noindent В каждом из трех ящиков 9 белых и 6 черных шаров. Из первого ящика в третий перекладывают два наудачу выбранных шара, а из второго ящика в третий перекладывают один наудачу взятый шар. Затем из третьего ящика извлекается один шар. а) Найти вероятность того, что он синий. б) Известно, что этот шар синий; найти вероятность того, что из первого ящика во второй переложили синие шары.
 \end{zkrW}

\vfill



\begin{zkrW}{20}\noindent Игральная кость подбрасывается 7 раз. Найти вероятность того, что три очка выпадут хотя бы 4 раза.
 \end{zkrW}

\vfill



\begin{zkrW}{20}\noindent Вероятность появления опечатки на отдельной странице книги равна $0{,}005$, а погрешности верстки --- $0{,}75$. Найти вероятность того, что в книге из 1200 страниц \\ \indent а) ровно 2 страниц будут иметь опечатки; \\ \indent б) от 864 до 936 страниц будут иметь погрешности верстки; \\ в) \indent погрешности верстки будут присутствовать ровно на 999 страницах.
 \end{zkrW}


\vfill\newpage\setcounter{zad}{0}





%%%%%%%%%%%%%%%%%%%%%%%%%%%%%%%%%%%%%%

\begin{zkrW}{20}\noindent Студент разыскивает нужную ему формулу в трех справочниках. Вероятность того, что формула содержится в первом, втором и третьем справочниках, равна соответственно $5/6$, $1/3$ и $1/5$. Найдите вероятность того, что эта формула содержится не менее чем в двух справочниках.
 \end{zkrW}

\vfill


\begin{zkrW}{20}\noindent Катя и Саша условились встретиться в определенном месте между 16:00 и 18:00. Каждый из них может прийти в любое время в течение указанного промежутка и ждет второго некоторое время. Катя ждет 40 минут, после чего уходит; Саша ждет 50 минут, после чего уходит. В 18:00 любой из них уходит, сколько бы до этого он ни ждал. Чему равна вероятность того, что встреча состоится в последние полчаса?
 \end{zkrW}

\vfill


\begin{zkrW}{20}\noindent В альбоме 7 чистых и 11 гашеных марок. Из альбома изымаются 2 наудачу извлеченные марки. После этого из альбома вновь наудачу извлекаются 3 марки. а) Найти вероятность того, что эти марки чистые. б) Известно, что эти 3 марки чистые; найти вероятность того, что первоначально изъятые 2 марки --- гашеные.
 \end{zkrW}

\vfill



\begin{zkrW}{20}\noindent Каждый из 5 станков в течение 6 рабочих часов останавливается несколько раз и всего в сумме стоит один час, причем остановка его в любой момент времени равновероятна. Найти вероятность того, что в данный момент времени будут работать более чем 3 станка.
 \end{zkrW}

\vfill



\begin{zkrW}{20}\noindent Известно, что левши среди населения Атлантиды составляют в среднем $1{,}5\%$, а люди, одинаково владеющие левой и правой рукой, --- $0{,}8$ (остальные --- правши). Найти вероятность того, что среди 400 людей \\ \indent а) окажется менее чем 5 левшей; \\ \indent б) окажется ровно 285 амбидекстров\footnote{людей, одинаково владеющих обеими руками}; \\ \indent в) окажется от 301 до 333 амбидекстров.
 \end{zkrW}


\vfill\newpage\setcounter{zad}{0}





%%%%%%%%%%%%%%%%%%%%%%%%%%%%%%%%%%%%%%

\begin{zkrW}{20}\noindent Вероятность попадания в цель при одном выстреле равна $7/9$ и с каждым выстрелом уменьшается на одну десятую от первоначальной. Произведено 6 выстрелов. Найдите вероятность поражения цели, если для этого достаточно хотя бы одного попадания.
 \end{zkrW}

\vfill


\begin{zkrW}{20}\noindent Андрей и Катя условились встретиться в определенном месте между 06:00 и 09:00. Каждый из них может прийти в любое время в течение указанного промежутка и ждет второго некоторое время. Андрей ждет 50 минут, после чего уходит; Катя ждет 70 минут, после чего уходит. В 09:00 любой из них уходит, сколько бы до этого он ни ждал. Чему равна вероятность того, что Андрей опоздает более чем на полчаса?
 \end{zkrW}

\vfill


\begin{zkrW}{20}\noindent В каждом из трех ящиков 8 белых и 10 черных шаров. Из первого ящика в третий перекладывают два наудачу выбранных шара, а из второго ящика в третий перекладывают один наудачу взятый шар. Затем из третьего ящика извлекается один шар. а) Найти вероятность того, что он зеленый. б) Известно, что этот шар зеленый; найти вероятность того, что из первого ящика во второй переложили зеленые шары.
 \end{zkrW}

\vfill



\begin{zkrW}{20}\noindent В тестовом задании 6 вопросов, на каждый дано 4 варианта ответа, среди которых один правильный. Какова вероятность того, что, выбирая вариант ответа наугад, отвечающий правильно ответит менее чем на 4 вопроса?
 \end{zkrW}

\vfill



\begin{zkrW}{20}\noindent Стрелок попадает в цель из пистолета с вероятностью $0{,}4$, а из снайперской винтовки --- с вероятностью $0{,}99$. Найти вероятность того, что, сделав 600 выстрелов по цели из каждого оружия, стрелок \\ \indent а) промахнется из пистолета от 218 до 247 раз; \\ \indent б) промахнется из пистолета ровно 274 раз; \\ \indent в) допустит по крайней мере 3 промахов из снайперской винтовки.
 \end{zkrW}


\vfill\newpage\setcounter{zad}{0}





%%%%%%%%%%%%%%%%%%%%%%%%%%%%%%%%%%%%%%

\begin{zkrW}{20}\noindent Вероятность попадания в цель при одном выстреле равна $0{,}9$ и с каждым выстрелом уменьшается на одну десятую от первоначальной. Произведено 7 выстрелов. Найдите вероятность поражения цели, если для этого достаточно хотя бы одного попадания.
 \end{zkrW}

\vfill


\begin{zkrW}{20}\noindent Катя и Вася условились встретиться в определенном месте между 10:00 и 12:00. Каждый из них может прийти в любое время в течение указанного промежутка и ждет второго некоторое время. Катя ждет 60 минут, после чего уходит; Вася ждет 50 минут, после чего уходит. В 12:00 любой из них уходит, сколько бы до этого он ни ждал. Чему равна вероятность того, что встреча состоится в первые полчаса?
 \end{zkrW}

\vfill


\begin{zkrW}{20}\noindent В альбоме 9 чистых и 12 гашеных марок. Из альбома наудачу извлекаются 2 марки и подвергаются гашению, а затем возвращаются в альбом. После этого вновь наудачу извлекаются 2 марки. а) Найти вероятность того, что эти марки чистые. б) Известно, что эти 2 марки чистые; найти вероятность того, что первоначально извлеченные 2 марки --- гашеные.
 \end{zkrW}

\vfill



\begin{zkrW}{20}\noindent Для баскетболиста дяди Стёпы вероятность забросить мяч в корзину равна $0{,}7$. Он выполняет 6 бросков. Какова вероятность, что в корзину попадут по меньшей мере 3 мяча?
 \end{zkrW}

\vfill



\begin{zkrW}{20}\noindent В ралли принимает участие 2400 экипажей. Каждый экипаж может сойти с дистанции из-за технических неполадок с вероятностью $0{,}4$, а из-за болезни водителя --- с вероятностью $0{,}0025$. Найти вероятность того, что \\ \indent а) менее чем 5 экипажей сойдут с дистанции из-за болезни водителя; \\ \indent б) ровно 979 экипажей не смогут продолжать ралли из-за технических неполадок; \\ \indent в) от 941 до 1008 экипажей пострадают от технических проблем.
 \end{zkrW}


\vfill\newpage\setcounter{zad}{0}








\end{document}